\documentclass{TIJMUjiaoanSY}
\pagestyle{empty}


\begin{document}


%课程名称
\kecheng{生物信息学}
%实验名称
\shiyan{基于Galaxy的基因组数据处理}
%教师姓名
\jiaoshi{伊现富}
%职称
\zhicheng{讲师}
%教学日期(格式:XXXX年XX月XX日XX时-XX时)
\riqi{2013年9月18日14时-16时}
%授课对象(格式:XXX系XXXX年级XX班(硕/本/专科))
\duixiang{生物医学工程学院2010级生信班(本)}
%实验人数
\renshu{23}
%实验类型
\leixing{验证型}
%实验分组
\fenzu{一人一机}
%学时数
\xueshi{2}
%教材版本
\jiaocai{生物信息学实验讲义(自编教材)}


%教案首页
\firstHeader
\maketitle
\thispagestyle{empty}

\mudi{
\begin{itemize}
  \item 掌握基因组注释中常用的BED格式。
  \item 掌握基因组坐标的逻辑运算模式。
  \item 掌握Galaxy的基本使用方法。
\end{itemize}
}

\fenpei{
\begin{itemize}
  \item (5')BED格式:回顾BED格式使用的坐标系统及其每一列的含义。
  \item (5')逻辑运算模式:回顾交集、减法、联合等逻辑运算模式。
  \item (5')Galaxy简介:简单介绍Galaxy分析平台的主界面、工具集及学习资料。
  \item (85')实验操作:寻找人类基因组中22号染色体上含有SNP数目最多的外显子。
\end{itemize}
}

\cailiao{
\begin{itemize}
  \item 实验材料:人类基因组(hg19)中22号染色体(chr22)上的外显子和SNP。
  \item 主要仪器:联网的计算机。
  \item 分析工具:Galaxy。
\end{itemize}
}

\zhongdian{
\begin{itemize}
  \item 难点:基因组坐标的联合运算;解决策略:通过实例进行讲解。
  \item 重点:Galaxy的使用;解决策略:根据资料进行学习,通过练习熟练掌握。
\end{itemize}
}

\sikao{
\begin{itemize}
  \item BED格式使用的哪一类坐标系统?
  \item BED格式每一列的含义是什么?
  \item 如何进行基因组坐标的联合操作?
\end{itemize}
}

\cankao{
\begin{itemize}
  \item Galaxy
\end{itemize}
}

\firstTail


%教案续页
\newpage
\otherHeader

\noindent
一、BED格式(5分钟)
\begin{enumerate}
  \item 坐标系统:0-based\textcolor{red}{(与1-based的区别)}
  \item 每列含义:chrom, chromStart, chromEnd, name, score, strand
\end{enumerate}

\noindent
二、逻辑运算模式(5分钟)
\begin{enumerate}
  \item intersect,交集:保留重叠的坐标
  \item subtract,减法:去除重叠的坐标
  \item join,联合:根据坐标重叠把两组记录对应起来\textcolor{red}{(与交集的关系)}
\end{enumerate}

\noindent
三、Galaxy简介(5分钟)
\begin{enumerate}
  \item 主界面:顶部是刊头,左侧栏是工具菜单,中间是工作区,右侧栏是历史面板
  \item 工具集: Get Data,Operate on Genomic Intervals,Join,Subtract and Group,Filter and Sort
  \item 学习资料:Galaxy 101,Galaxy Screencasts and Demos,Learn Galaxy,Galaxy Wiki
\end{enumerate}

\noindent
四、实验操作(85分钟)

寻找人类基因组(hg19)中22号染色体(chr22)上含有SNP数目最多的外显子。
\begin{enumerate}
  \item 获取数据\textcolor{red}{(选择正确的格式,把结果导出到Galaxy中)}
    \begin{itemize}
      \item 外显子数据:Get Data,UCSC Main,hg19,chr22,RefSeq Genes,BED格式
      \item SNP数据:Get Data,UCSC Main,hg19,chr22,dbSNP137,BED格式
    \end{itemize}
  \item 提取含有SNP的外显子:Operate on Genomic Intervals,Join\textcolor{red}{(注意数据集的顺序)}
  \item 对外显子上的SNP进行计数:Join,Subtract and Group,Group\textcolor{red}{(注意选择正确的列)}
  \item 对SNP数目进行排序:Filter and Sort,Sort\textcolor{red}{(注意选择正确的列)}
  \item 筛选至少含有10个SNP的外显子:Filter and Sort,Filter\textcolor{red}{(学习编写筛选表达式)}
  \item 附加外显子的原始信息:Join,Subtract and Group,Join two Datasets\textcolor{red}{(注意数据集的选择,同时根据每一列的含义选择正确的列)}
\end{enumerate}


\otherTail


\end{document}

