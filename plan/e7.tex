\documentclass{TIJMUjiaoanSY}
\pagestyle{empty}


\begin{document}


%课程名称
\kecheng{生物信息学}
%实验名称
\shiyan{实验七\ 原核生物和真核生物的基因识别}
%教师姓名
\jiaoshi{伊现富}
%职称
\zhicheng{讲师}
%教学日期(格式:XXXX年XX月XX日XX时-XX时)
\riqi{2018年12月7日8:00-10:45}
%授课对象(格式:XXX系XXXX年级XX班(硕/本/专科))
\duixiang{生物医学工程与技术学院2016级生信班(本)}
%实验人数
\renshu{28}
%实验类型
\leixing{验证型}
%实验分组
\fenzu{一人一机}
%学时数
\xueshi{3}
%教材版本
\jiaocai{生物信息学实验讲义(自编教材)}


%教案首页
\firstHeader
\maketitle
\thispagestyle{empty}

\mudi{
\begin{itemize}
  \item 了解隐马尔科夫模型在基因识别中的应用。
  \item 掌握原核基因和真核基因的结构特征。
  \item 掌握GeneMarkS和GENSCAN的使用方法。
\end{itemize}
}

\fenpei{
\begin{itemize}
  \item (10')基因与基因识别:回顾基因和基因识别的基本概念,总结比较原核生物和真核生物基因的结构特点。
  \item (10')基因识别的方法:回顾基因识别的三大类方法,总结比较原核和真核基因预测中“信号”和“内容”的异同。
  \item (10')基因识别的工具:简单介绍GeneMarkS和GENSCAN,重点强调两者的适用范围。
  \item (120')实验操作:对大肠杆菌基因组序列进行基因识别,对人类CD9基因进行结构分析。
\end{itemize}
}

\cailiao{
\begin{itemize}
  \item 实验材料:大肠杆菌基因组,人类CD9基因。
  \item 主要仪器:联网的计算机。
  \item 分析工具:GeneMarkS,GENSCAN。
\end{itemize}
}

\zhongdian{
\begin{itemize}
  \item 难点:FASTA格式与纯序列的区别;解决策略:通过实例进行讲解。
  \item 重点:GeneMarkS和GENSCAN的使用;解决策略:通过练习熟练掌握。
\end{itemize}
}

\sikao{
\begin{itemize}
  \item 原核基因和真核基因的结构有何异同?
  \item 基因识别的方法主要有哪三大类?
  \item 原核和真核基因预测中的“信号”和“内容”有何异同?
  \item GeneMarkS和GENSCAN的适用范围分别是什么?
  \item GeneMarkS和GENSCAN对输入格式的要求有何差别?
\end{itemize}
}

\cankao{
\begin{itemize}
  \item NCBI
  \item GeneMarkS
  \item GENSCAN
\end{itemize}
}

\firstTail


%教案续页
\newpage
\otherHeader

\noindent
一、基因与基因识别(10分钟)
\begin{enumerate}
  \item 基本概念
  \begin{itemize}
\parpic[fr]{\includegraphics[width=6.5cm,height=2cm]{geneP.jpg}}
    \item 基因:产生一条多肽链或功能RNA所需的全部核苷酸序列\textcolor{red}{(强调既包括编码区,也包括非编码区)}
    \item 基因识别:识别DNA序列上具有生物学特征的片段
  \end{itemize}
  \item 基因结构\textcolor{red}{(基因结构的复杂性直接影响着基因预测的策略及最终的准确度)}
  \begin{itemize}
\parpic[fr]{\includegraphics[width=6.5cm]{geneE.jpg}}
    \item 共同点:都包括编码区和非编码区
    \item 原核基因:连续基因
    \item 真核基因:不连续性
  \end{itemize}
\end{enumerate}

\vspace*{0.2cm}
\noindent
二、基因识别的方法(10分钟)
\begin{enumerate}
  \item 识别方法
  \begin{itemize}
    \item 间接识别法:mRNA/蛋白质序列 $\Rightarrow$ DNA序列
    \item 从头预测法:基因预测,基于“信号”和“内容”两类特征
    \item 比较基因组学的方法:比较相关物种的DNA序列
  \end{itemize}
  \item 基因预测
  \begin{itemize}
    \item “信号”和“内容”
    \begin{itemize}
\parpic[fr]{\includegraphics[width=7cm]{signal.jpg}}
      \item 共同点:都包括编码区和非编码区
      \item 信号:不连续的局部序列模体,一般都有一致性序列;如启动子,剪接供体和受体位点,起始和终止密码子,polyA位点
      \item 内容:不同长度的扩展序列,没有一致性序列,但具有把自己与周围DNA区分开来的保守特征;如密码子使用偏好性,双联密码子出现频率,基因组等值区
    \end{itemize}
    \item 原核基因
    \begin{itemize}
      \item 信号:启动子序列,转录因子结合位点
      \item 内容:连续的开放阅读框,统计学特征
      \item 总结:信号容易识别,内容容易判别,预测能达到相对较高的精度
    \end{itemize}
    \item 真核基因
    \begin{itemize}
\parpic[fr]{\includegraphics[width=8cm]{genefind.png}}
      \item 信号:启动子区特征序列,供体和受体位点,起始和终止密码子,polyA序列;确定外显子的边界,识别编码区域
      \item 内容:密码子使用偏好性,双联密码子出现频率,基因组等值区;区分外显子、内含子和基因间区域
      \item 总结:信号复杂,内容难判别,预测相当有挑战性;联合信号和内容检测以及同源性搜索,提高识别效率
    \end{itemize}
  \end{itemize}
\end{enumerate}

\vspace*{0.2cm}
\noindent
三、基因识别的工具(10分钟)

\textcolor{red}{分析工具都有自己的适用范围。}
\begin{enumerate}
  \item GeneMarkS:迭代隐马尔科夫模型,适用于原核生物的基因预测
  \item GENSCAN:广义隐马尔科夫模型,人类及脊椎动物基因预测软件
\end{enumerate}


\otherTail
\newpage
\otherHeader


\noindent
四、实验操作(120分钟)

\textcolor{red}{基因结构的复杂性直接影响着基因预测的准确度。}
\begin{enumerate}
  \item 大肠杆菌基因组序列的基因识别
    \begin{itemize}
      \item 获取序列:NCBI中的Nucleotide数据库,U00096,FASTA格式和GenBank格式\textcolor{red}{(复习GenBank格式)}
      \item 截取序列:EMBOSS,extractseq,1-10000bp
      \item 基因预测:\textcolor{red}{GeneMarkS},FASTA格式
      \item 结果分析:和GenBank格式中的信息进行比较
    \end{itemize}
\begin{figure}[h]
  \centering
  \includegraphics[width=11cm]{genemark.png}
\end{figure}
  \item 人类CD9基因的结构分析
    \begin{itemize}
\parpic[fr]{\includegraphics[width=8cm]{genscan.jpg}}
      \item 获取序列:NCBI中的Nucleotide数据库,AY422198,FASTA格式和GenBank格式
      \item 基因预测:\textcolor{red}{GENSCAN},纯序列\textcolor{red}{(注意不是FASTA格式)}
      \item 结果解析
        \begin{itemize}
          \item Type
            \begin{itemize}
              \item Init: initial exon
              \item Intr: internal exon
              \item Term: terminal exon
              \item Sngl: single-exon gene
              \item Prom: promoter region
              \item PlyA: polyA signal
            \end{itemize}
          \item P
              \begin{itemize}
                \item P>0.99:可能性极高的外显子
                \item 0.50<P<0.99:中等或高可能性的外显子
                \item P<0.50:低可能性的外显子
              \end{itemize}
        \end{itemize}
      \item 结果分析:和GenBank格式中的信息进行比较
    \end{itemize}
\end{enumerate}

\otherTail


\end{document}

