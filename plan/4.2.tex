\documentclass{TIJMUjiaoanLL}
\pagestyle{empty}


\begin{document}


%课程名称
\kecheng{生物信息学}
%课程内容
\neirong{基因识别与RNA序列分析}
%教师姓名
\jiaoshi{伊现富}
%职称
\zhicheng{讲师}
%教学日期(格式:XXXX年XX月XX日XX时-XX时)
\riqi{2014年10月13日13:30-15:20}
%授课对象(格式:XXX系XXXX年级XX班(硕/本/专科))
\duixiang{生物医学工程学院2011级生医班(本)}
%听课人数
\renshu{55}
%授课方式
\fangshi{理论讲授}
%学时数
\xueshi{2}
%教材版本
\jiaocai{生物信息学(自编教材)}


%教案首页
\firstHeader
\maketitle
\thispagestyle{empty}

\mudi{
\begin{itemize}
  \item 掌握基因识别的方法;mRNA选择性剪接的主要机制;miRNA预测和miRNA靶基因预测的方法。
  \item 熟悉原核基因和真核基因的结构特点;miRNA的特点、生成过程和作用方式;miRNA的相关数据库和分析工具。
  \item 了解基因识别的分析工具;选择性剪接的相关数据库与分析工具。
  \item 自学基因识别分析工具的使用方法。
\end{itemize}
}

\fenpei{
\begin{itemize}
  \item (5')回顾与导入:回顾序列基本信息和特征信息分析的主要内容,引出基因识别的内容;总结RNA的主要类别,引出RNA分析的内容。
  \item (45')基因识别:介绍基因和基因识别的基本概念,介绍原核基因和真核基因的结构特点并进行比较,讲解基因识别的主要方法与策略,介绍基因识别的常用工具。
  \item (20')mRNA选择性剪接:介绍剪接和选择性剪接的基本概念,讲解mRNA选择性剪接的主要机制,介绍相关的数据库和分析工具。
  \item (25')miRNA及其靶基因预测:介绍miRNA的特点、生成过程、作用方式和生物学功能,讲解miRNA预测和miRNA靶基因预测的主要方法,介绍常用的数据库和分析工具。
  \item (5')总结与答疑:总结授课内容中的知识点,解答学生疑问。
\end{itemize}
}

\zhongdian{
\begin{itemize}
  \item 重点:原核基因和真核基因的结构特点;基因识别方法;mRNA选择性剪接的主要机制。
  \item 难点:基因识别中“信号”特征和“内容”特征的区别;mRNA选择性剪接各种机制之间的区别。
  \item 解决策略:通过示意图和实例帮助学生理解,通过对比加深记忆。
\end{itemize}
}

\waiyu{
\vspace*{-10pt}
\begin{multicols}{2}
基因识别(gene prediction/finding)

间接识别法(extrinsic approach)

从头计算法(\textit{ab initio} approach)

选择性剪接(alternative splicing)

微RNA(miRNA,microRNA)

非编码RNA(ncRNA,non-coding RNA)
\end{multicols}
\vspace*{-10pt}
}

\fuzhu{
\begin{itemize}
  \item 多媒体:原核基因和真核基因的结构,基因识别的策略,mRNA选择性剪接的机制和实例,miRNA的生成过程和作用方式。
  \item 板书:基因识别的方法与策略。
\end{itemize}
}

\sikao{
  \vspace*{-10pt}
  \begin{multicols}{2}
  \begin{itemize}
    \item 简述原核基因和真核基因结构的异同。
    \item 简述基因识别的三大类方法。
    \item 简述mRNA选择性剪接的主要机制。
    \item 简述miRNA特点、生成过程和作用方式。
    \item 简述miRNA预测和miRNA靶基因预测的方法。
  \end{itemize}
  \end{multicols}
  \vspace*{-10pt}
}

\cankao{
\begin{itemize}
  \item 朱玉贤,李毅,郑晓峰。现代分子生物学(第3版),高等教育出版社,2007。
  \item 李霞,李亦学,廖飞。生物信息学,人民卫生出版社,2010。
  \item 王明怡,杨益,吴平。生物信息学(中译本,第2版),科学出版社,2004。
  \item 维基百科。
\end{itemize}
}

\firstTail


%教案续页
\newpage
\otherHeader

\noindent
一、回顾与导入(5分钟)
\begin{enumerate}
  \item 序列分析
    \begin{itemize}
      \item DNA序列分析:基本信息,序列特征,基因识别
      \item RNA序列分析:mRNA选择性剪接,miRNA与靶基因
    \end{itemize}
  \item
    RNA的分类\textcolor{red}{(RNA既是携带遗传信息的主要生物大分子,也是重要的功能单位)}
    \begin{itemize}
      \item 编码RNA:mRNA
      \item 非编码RNA:tRNA,rRNA;miRNA,siRNA,lncRNA
    \end{itemize}
  \item ncRNA的分类\textcolor{red}{(转录后不编码蛋白质的RNA分子的统称)}
    \begin{itemize}
      \item 基础结构性ncRNA/看家ncRNA:tRNA,rRNA,snRNA,snoRNA
      \item 调节性ncRNA
      \begin{itemize}
	\item sRNA:\textless 200nt,miRNA、siRNA、piRNA\textcolor{red}{(已经开展了广泛的研究)}
        \item lncRNA:\textgreater 200nt,长链非编码RNA\textcolor{red}{(引起关注,研究正逐步深入)}
      \end{itemize}
    \end{itemize}
\end{enumerate}
\vspace*{-10pt}
\begin{figure}[h]
  \centering
  \includegraphics[width=16cm]{ncrnaC.jpg}
\end{figure}
\vspace*{-10pt}

\vspace*{0.2cm}
\noindent
二、基因识别(45分钟)

\textcolor{red}{在介绍基本概念的基础上,通过比较原核基因和真核基因的异同,讲解基因识别的主要策略及各种方法在原核和真核基因识别中的具体应用。}

\begin{enumerate}
  \item 基本概念
    \begin{itemize}
 \parpic[fr]{\includegraphics[width=6.5cm,height=2cm]{geneP.jpg}}
      \item 基因:产生一条多肽链或功能RNA所需的全部核苷酸序列\textcolor{red}{(强调既包括编码区,也包括非编码区)}
      \item 基因识别:识别DNA序列上具有生物学特征的片段
    \end{itemize}
  \item \textcolor{red}{\textbf{【重点】}}基因结构\textcolor{red}{(通过示意图形象化展示、比较原核和真核的基因结构)}
    \begin{itemize}
 \parpic[fr]{\includegraphics[width=6.5cm]{geneE.jpg}}
      \item 共同点:都包括编码区和非编码区
      \item 原核基因:连续基因
      \item 真核基因:不连续性
    \end{itemize}
  \item \textcolor{red}{\textbf{【重点】}}识别方法
    \begin{itemize}
      \item 间接识别法:mRNA/蛋白质序列 $\Rightarrow$ DNA序列
      \item 从头预测法:基因预测,基于“信号”和“内容”两类特征
      \item 比较基因组学的方法:比较相关物种的DNA序列
    \end{itemize}


\otherTail
\newpage
\otherHeader


  \item \textcolor{red}{\textbf{【难点】}}基因预测
    \begin{itemize}
      \item “信号”和“内容”
	\begin{itemize}
 \parpic[fr]{\includegraphics[width=7cm]{signal.jpg}}
	  \item 信号:不连续的局部序列模体,一般都有一致性序列;如启动子,剪接供体和受体位点,起始和终止密码子,polyA位点
	  \item 内容:不同长度的扩展序列,没有一致性序列,但具有把自己与周围DNA区分开来的保守特征;如密码子使用偏好性,双联密码子出现频率,基因组等值区
	\end{itemize}
      \item 原核基因
	\begin{itemize}
	  \item 信号:启动子序列,转录因子结合位点
	  \item 内容:连续的开放阅读框,统计学特征
	  \item 总结:信号容易识别,内容容易判别,预测能达到相对较高的精度
	\end{itemize}
      \item 真核基因
	\begin{itemize}
 \parpic[fr]{\includegraphics[width=8cm]{genefind.png}}
	  \item 信号:启动子区特征序列,供体和受体位点,起始和终止密码子,polyA序列;确定外显子的边界,识别编码区域
	  \item 内容:密码子使用偏好性,双联密码子出现频率,基因组等值区;区分外显子、内含子和基因间区域
	  \item 总结:信号复杂,内容难判别,预测相当有挑战性;联合信号和内容检测以及同源性搜索,提高识别效率
	\end{itemize}
    \end{itemize}
  \item 识别策略
  \item 识别工具\textcolor{red}{(强调分析工具的适用范围)}
    \begin{itemize}
      \item 识别原核基因:GeneMarkS, Glimmer
      \item 识别真核基因:GENSCAN
    \end{itemize}
\end{enumerate}

\vspace*{0.2cm}
\noindent
三、mRNA选择性剪接(20分钟)
\begin{enumerate}
  \item 基本概念
    \begin{itemize}
      \item 剪接:移除内含子、合并外显子
      \item 选择性剪接:一个mRNA前体 $\Rightarrow$ 不同mRNA剪接异构体
    \end{itemize}
  \item \textcolor{red}{\textbf{【重点、难点】}}主要机制\textcolor{red}{(通过示意图和实例详解每种机制)}
    \begin{itemize}
\parpic[fr]{\includegraphics[width=9cm]{splicingModel.jpg}}
      \item 外显子跳跃:外显子被移除或保留
      \item 互斥外显子:两个外显子只有一个保留下来
      \item 5'选择性剪接:使用不同的5'端的供体位点
      \item 3'选择性剪接:使用不同的3'端的受体位点
      \item 内含子保留:内含子作为外显子保留下来
      \item 选择性起始:在不同的位点起始转录
      \item 选择性终止:使用不同的polyA位点
    \end{itemize}
  \item 相关资源\textcolor{red}{(提醒注意数据库的时效性)}
    \begin{itemize}
      \item 数据库:ASTD,ASAP,ASPicDB
      \item 分析工具:ESEfinder,RESCUE-ESE
    \end{itemize}
\end{enumerate}


\otherTail
\newpage
\otherHeader


%\vspace*{0.2cm}
\noindent
四、miRNA及其靶基因预测(25分钟)
\begin{enumerate}
  \item miRNA简介\textcolor{red}{(通过示意图形象化展示相关内容)}
    \begin{itemize}
%\begin{minipage}{0.9\textwidth}
\parpic[fr]{\includegraphics[width=7.5cm,height=6.5cm]{mirna.png}}
      \item 真核生物中广泛存在的一种长约20到24个核苷酸的内源性非编码单链RNA分子
      \item 生成过程:300~1000nt的双链pri-miRNA $\Rightarrow$ 70~90nt的双链具有茎环结构的pre-miRNA $\Rightarrow$ 20~24nt的单链成熟miRNA
      \item 作用方式:完全互补型——结合在mRNA的编码区中,导致靶基因mRNA降解,在植物中比较常见;不完全互补型——结合在mRNA的3' UTR,导致靶基因mRNA的翻译受到抑制
      %\item 作用方式
      %\begin{itemize}
        %\item 完全互补型:结合在mRNA的编码区中,导致靶基因mRNA降解,在植物中比较常见
        %\item 不完全互补型:结合在mRNA的3' UTR,导致靶基因mRNA的翻译受到抑制
      %\end{itemize}
      \item 生物学功能:调控个体发育、细胞分化、组织发育、肿瘤发生发展、……
%\end{minipage}
    \end{itemize}
  \item miRNA的特征
    \begin{itemize}
      \item 序列:不具有开放阅读框,不编码蛋白质;成熟的miRNA 5'端为单一磷酸基团,3'端为羟基
      \item 表达:具有时序性和组织特异性
      \item 调控:miRNA与靶基因间呈多对多的关系
      \item 物理位置:倾向于成簇地出现在染色体上
      \item 进化:在物种间高度保守
    \end{itemize}
  \item miRNA预测
    \begin{itemize}
\parpic[fr]{\includegraphics[width=10.3cm,height=5.3cm]{mirnaS.png}}
      \item 同源片段搜索
      \item 基于比较基因组学
      \item 基于序列和结构特征打分
      \item 结合作用靶标
      \item 基于机器学习
    \end{itemize}
  \item miRNA靶基因预测
    \begin{itemize}
      \item 基于种子区域互补和保守性
      \item 基于机器学习
    \end{itemize}
  \item 相关资源
    \begin{itemize}
      \item 数据库:miRBase,TarBase,miRGen
      \item 分析工具:MiRscan,MiPred,miRFinder;miRanda,TargetScan,PicTar,miTarget
    \end{itemize}
\end{enumerate}

\vspace*{0.2cm}
\noindent
五、总结与答疑(5分钟)
\begin{itemize}
  \item 基因识别:原核和真核的基因结构,基因识别方法
  \item mRNA选择性剪接:主要机制
  \item miRNA:特征、生成过程和作用方式,miRNA及其靶基因预测方法
\end{itemize}



\otherTail

\end{document}

