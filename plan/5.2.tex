\documentclass{TIJMUjiaoanLL}
\pagestyle{empty}


\begin{document}


%课程名称
\kecheng{生物信息学}
%课程内容
\neirong{第五章(5.3)基因组功能的高级注释}
%教师姓名
\jiaoshi{伊现富}
%职称
\zhicheng{讲师}
%教学日期(格式:XXXX年XX月XX日XX时-XX时)
\riqi{2018年6月1日10:00-12:00}
%授课对象(格式:XXX系XXXX年级XX班(硕/本/专科))
\duixiang{基础医学院2015级基础班(本)}
%听课人数
\renshu{18}
%授课方式
\fangshi{理论讲授}
%学时数
\xueshi{2}
%教材版本
\jiaocai{生物信息学:基础及应用}


%教案首页
\firstHeader
\maketitle
\thispagestyle{empty}

\mudi{
\begin{itemize}
  \item 掌握序列标识的含义和制作工具。
  \item 熟悉变异位点注释结果的解析;基因集的富集分析及其结果解析;box plot的含义及其绘制。
  \item 了解变异位点注释的内容和常用工具;基因集富集分析的常用工具。
  \item 自学变异位点注释、基因集富集分析、序列标识制作等工具的使用方法。
\end{itemize}
}

\fenpei{
\begin{itemize}
  \item (5')回顾与导入:回顾基因组注释的基础知识,介绍功能注释的主要内容。
  \item (25')变异位点的注释:介绍变异位点注释的内容、步骤及相关的注释工具,讲解对注释结果的解析。
  \item (20')基因集富集分析:介绍基因集富集分析的用途,讲解常用的DAVID工具及其结果的解析。
  \item (25')序列标识:讲解序列标识的含义,介绍常用的WebLogo及其使用方法并讲解对结果的解析。
  \item (20')box plot:介绍box plot及相关概念,讲解绘制box plot的主要步骤。
  \item (5')总结与答疑:总结授课内容中的知识点与技能,解答学生疑问。
\end{itemize}
}

\zhongdian{
\begin{itemize}
  \item 重点:序列标识的含义;解决策略:通过制作过程的演示和对结果的解读来加深学生的理解。
  \item 难点:注释分析结果的解析;解决策略:通过对实例的分析帮助学生掌握解析结果的基本原则和主要步骤。
\end{itemize}
}

\waiyu{
\vspace*{-10pt}
\begin{multicols}{2}
单核苷酸变异(SNV)

基因集(gene set)

GO(gene ontology)

富集分析(enrichment analysis)

序列标识(sequence logo)

箱线图(box plot)
\end{multicols}
\vspace*{-10pt}
}

\fuzhu{
\begin{itemize}
  \item 多媒体:变异位点注释、基因集富集分析的实例;序列标识和box plot的示意图;DAVID、WebLogo等工具的界面;绘制box plot的演示视频。
  \item 板书:box plot的主要绘制步骤。
  \item 操作演示:序列标识的制作。
\end{itemize}
}

\sikao{
\begin{itemize}
  \item 以变异位点的注释结果为例,论述如何解析一张表。
  \item 以DAVAD富集分析结果为例,论述如何解析一张表。
  \item 简述序列标识的含义,能解读实际的序列标识图。
  \item 以box plot为例,论述如何解析一张图。
\end{itemize}
}

\cankao{
\begin{itemize}
  \item 李霞,李亦学,廖飞。生物信息学,人民卫生出版社,2010年。
  \item 朱玉贤,李毅,郑晓峰。现代分子生物学(第3版),高等教育出版社,2007。
  \item 维基百科
\end{itemize}
}

\firstTail


%教案续页
\newpage
\otherHeader

\noindent
一、回顾与导入(5分钟)

\textcolor{red}{回顾基因组注释的基础知识,介绍高级注释的内容,强调基础知识在高级注释中无处不在。}
\begin{enumerate}
  \item 基因组注释的基础知识
    \begin{itemize}
  \parpic[fr]{\includegraphics[width=7.5cm]{anno.png}}
      \item 基因组的组装版本:hg19与GRCh37,mm10与GRCm38
      \item 两种坐标系统:0-based,1-based
      \item 四种常用格式:FASTA,BED,GFF,VCF
      \item 逻辑运算模式:intersect,subtract,join,\ldots
    \end{itemize}
  \item 基因组功能的高级注释
    \begin{itemize}
      \item 变异位点的注释:SNVs、非同义多态性的注释
      \item 基因集富集分析:GO,KEGG,DAVID
      \item 序列标识:WebLogo
    \end{itemize}
\end{enumerate}

\vspace*{0.2cm}
\noindent
二、变异位点的注释(25分钟)

\textcolor{red}{重点讲解对注释结果的解析及其在功能注释流程中承上启下的作用。}
\begin{enumerate}
  \item 单核苷酸变异的注释
    \begin{itemize}
  \parpic[fr]{\includegraphics[width=10cm,height=5.5cm]{seattleseqannotation.png}}
      \item 注释内容:附加相关的基因组注释信息(数据库ID,基因名,变异功能类别,……)
      \item 注释工具:SeattleSeq Annotation,variant tools,SnpEff
      \item \textcolor{red}{\textbf{【难点】}}结果解析:SeattleSeq Annotation的注释结果\textcolor{red}{(通过实例解读注释结果;对注释结果过滤筛选后可继续进行非同义多态性的注释)}
    \end{itemize}
  \item 非同义多态性的注释
    \begin{itemize}
  \parpic[fr]{\includegraphics[width=10cm]{siftannotation.png}}
      \item 注释内容:对蛋白质产物结构和功能的影响
      \item 注释工具:SIFT,PolyPhen-2,SNPs3D,PROVEAN
      \item \textcolor{red}{\textbf{【难点】}}结果解析:SIFT的注释结果\textcolor{red}{(通过实例解读注释结果;承接SNVs的注释,对结果过滤筛选后可继续进行基因集的富集分析)}
    \end{itemize}
\end{enumerate}

\vspace*{0.2cm}
\noindent
三、基因集富集分析(20分钟)
\begin{enumerate}
  \item 基因集富集分析\textcolor{red}{(承接变异位点的注释)}
    \begin{itemize}
  \parpic[fr]{\includegraphics[width=10.5cm]{david.png}}
      \item 富集分析:基因集,GO,KEGG
      \item GO(Gene Ontology)
      \begin{itemize}
        \item biological process
        \item molecular function
        \item cellular component
      \end{itemize}
      \item 结果解析\textcolor{red}{(解析使用DAVID进行GO富集分析的结果)}
      \begin{itemize}
        \item 富集显著性
        \item 多重检验校正
      \end{itemize}
    \end{itemize}

\otherTail
\newpage
\otherHeader

  \item DAVID分析工具\textcolor{red}{(根据任务选择工具)}
    \begin{itemize}
  \parpic[fr]{\includegraphics[width=9cm]{david2.png}}
      \item Gene Name Batch Viewer
      \item Gene ID Conversion Tool
      \item Gene Functional Classification Tool
      \item Functional Annotation Tool
      \begin{itemize}
        \item Functional Annotation Clustering:\\ 根据注释信息聚类注释项目
        \item Functional Annotation Chart:\\ 根据注释信息进行富集分析
        \item Functional Annotation Table:\\ 以表格形式呈现注释信息
      \end{itemize}
    \end{itemize}
\end{enumerate}

\vspace*{0.2cm}
\noindent
四、序列标识(25分钟)
\begin{enumerate}
  \item \textcolor{red}{\textbf{【重点】}}图形含义\textcolor{red}{(以“GT-AG规则”为例讲解序列标识图)}
    \begin{itemize}
  \parpic[fr]{\includegraphics[width=9cm]{gtag.png}}
      \item 数据:多序列比对信息
      \item 横轴:序列的坐标位置
      \item 纵轴:比特,计量单位
      \item 字符堆叠的总高度:此位置的保守性
      \item 每个字符的高度:出现的相对频率
    \end{itemize}
  \item 制作工具\textcolor{red}{(演示WebLogo的使用)}
    \begin{itemize}
      \item WebLogo
      \item enoLOGOS
      \item Skylign
    \end{itemize}
\end{enumerate}

\vspace*{0.2cm}
\noindent
五、box plot(20分钟)
\begin{enumerate}
  \item box plot简介\textcolor{red}{(通过实例和示意图讲解其优缺点)}
    \begin{itemize}
  \parpic[fr]{\includegraphics[width=9cm]{bp3.jpg}}
      \item box plot,Box-whisker Plot,箱线图
      \item 1977,美国,约翰\textbullet 图基(John Tukey)
      \item 显示一组数据分散情况的统计图
      \item 可以粗略看出数据分布的离散程度
  \parpic[r]{\includegraphics[width=6cm]{bp0.png}}
      \item 适合用于几个样本的比较
      \item 不能提供数据分布偏态的精确度量
    \end{itemize}
  \item 相关概念\textcolor{red}{(通过实例帮助学生理解记忆)}
  \begin{itemize}
    \item 最小值min,最大值max,中位数median
    \item 下四分位数Q1,上四分位数Q3
    \item 四分位数差$IQR = Q3-Q1$
    \item 内限:$Q3 + 1.5IQR$,$Q1 - 1.5IQR$
    \item 外限:$Q3 + 3IQR$,$Q1 - 3IQR$
    \item 异常值(outliers):处于内限以外的数据
    \item 温和的异常值(mild outliers):在内限与外限之间的异常值
    \item 极端的异常值(extreme outliers):在外限以外的异常值
  \end{itemize}
  \item 图解概念
  \begin{itemize}
  \parpic[fr]{\includegraphics[width=9cm,height=1.8cm]{bp1.png}}
    \item $min = 0.5, max = 10$
    \item $Q1 = 7, Q3 = 9, IQR = 2$
    \item $median = 8.5, mean = 8$
  \end{itemize}

\otherTail
\newpage
\otherHeader

  \item 绘图步骤\textcolor{red}{(通过观看视频学习绘图的具体步骤)}
  \begin{itemize}
  \parpic[fr]{\includegraphics[width=8cm,height=2.5cm,angle=270]{bp6.png}}
  \item 绘制数轴。
  \item 计算上四分位数(Q3),中位数,下四分位数(Q1)。
  \item 计算四分位数差(IQR)。
  \item 绘制箱线图的矩形,上限为Q3,下限为Q1。在矩形内部中位数的位置画一条横线(中位线)。
  \item 在$Q3 + 1.5IQR$和$Q1 - 1.5IQR$处画两条与中位线一样的线段,这两条线段为异常值截断点,称为内限;在$Q3 + 3IQR$和$Q1 - 3IQR$处画两条线段,称为外限。
  \item 在非异常值的数据中,最靠近上边缘和下边缘(即内限)的两个数值处画横线,作为箱线图的触须。
  \item 从矩形的两端向外各画一条线段直到不是异常值的最远点(即上一步的触须),表示该批数据正常值的分布区间。
  \item 温和的异常值用空心圆表示;极端的异常值用实心点(一说用星号*)表示。
  \end{itemize}
  \item 绘图工具:BoxPlotR,ECplot,R,\ldots
\end{enumerate}

\vspace*{0.2cm}
\noindent
六、总结与答疑(5分钟)
\begin{enumerate}
  \item 知识点
    \begin{itemize}
      \item 变异位点的注释:用途,注释工具
      \item 基因集富集分析:功能,分析工具
      \item 序列标识:含义,制作工具
      \item box plot:理解,绘制
    \end{itemize}
  \item 技能
    \begin{itemize}
      \item 解析表格:行列,缩写,数值
      \item 解析图片:数据,横纵轴,图元素,元素大小、颜色
    \end{itemize}
\end{enumerate}

\otherTail


\end{document}

