\documentclass{TIJMUjiaoanLL}
\pagestyle{empty}


\begin{document}


%课程名称
\kecheng{生物信息学}
%课程内容
\neirong{基因组功能的高级注释}
%教师姓名
\jiaoshi{伊现富}
%职称
\zhicheng{讲师}
%教学日期(格式:XXXX年XX月XX日XX时-XX时)
\riqi{2013年9月6日8时-10时}
%授课对象(格式:XXX系XXXX年级XX班(硕/本/专科))
\duixiang{生物医学工程学院2010级生信班(本)}
%听课人数
\renshu{23}
%授课方式
\fangshi{理论讲授}
%学时数
\xueshi{2}
%教材版本
\jiaocai{生物信息学(自编教材)}


%教案首页
\firstHeader
\maketitle
\thispagestyle{empty}

\mudi{
\begin{itemize}
  \item 了解变异位点注释的用途,熟悉注释结果的解析,自学注释工具的使用方法。
  \item 熟悉基因集富集分析的概念、用途以及结果的解析,了解富集分析的常用工具并自学其使用方法。
  \item 掌握序列标识的图形含义,熟悉制作序列标识的工具并自学其使用方法。
  \item 熟悉Galaxy分析平台,掌握其基本使用方法,自学Galaxy的高级使用技巧。
\end{itemize}
}

\fenpei{
\begin{itemize}
  \item (5')回顾与导入:回顾基因组注释的基础知识,介绍功能注释的主要内容及常用分析平台。
  \item (15')变异位点的注释:介绍变异位点注释的内容、步骤及相关的注释工具,讲解对注释结果的解读。
  \item (10')基因集富集分析:介绍基因集富集分析的用途,讲解常用的DAVID工具及其结果的解析。
  \item (20')序列标识:讲解序列标识的含义,介绍制作工具及其使用方法并讲解对结果的解读。
  \item (40')Galaxy分析平台:介绍基因组注释分析平台中常用的Galaxy平台,通过实例演示讲解Galaxy的基本使用方法。
  \item (10')总结与答疑:总结授课内容中的知识点,解答学生疑问。
\end{itemize}
}

\zhongdian{
\begin{itemize}
  \item 重点:序列标识的含义,Galaxy分析平台的使用方法。
  \item 难点:Galaxy分析平台的使用方法。
  \item 解决策略:通过制作过程的演示和对结果的解读来讲解序列标识的含义;从实例入手,通过逐步演示讲解Galaxy的使用方法与技巧。
\end{itemize}
}

\waiyu{
\vspace*{-10pt}
\begin{multicols}{2}
单核苷酸变异(SNV)

基因集(gene set)

GO(gene ontology)

富集分析(enrichment analysis)

序列标识(sequence logo)

工作流(workflow)
\end{multicols}
\vspace*{-10pt}
}

\fuzhu{
\begin{itemize}
  \item 多媒体:变异位点注释、富集分析的结果;序列标识示意图;DAVID、Galaxy等工具的界面。
  \item 板书:基因组功能注释的流程及其对应的工具。
  \item 操作演示:序列标识的制作;Galaxy的使用。
\end{itemize}
}

\sikao{
\vspace*{-10pt}
\begin{multicols}{2}
\begin{itemize}
  \item 变异位点注释结果的解读。
  \item DAVAD富集分析结果的解读。
  \item 序列标识的含义。
  \item Galaxy分析平台的使用。
\end{itemize}
\end{multicols}
\vspace*{-10pt}
}

\cankao{
\begin{itemize}
  \item Galaxy
  \item 维基百科
\end{itemize}
}

\firstTail


%教案续页
\newpage
\otherHeader

\noindent
一、回顾与导入(5分钟)

\textcolor{red}{回顾注释的基础知识,介绍高级注释的内容,并说明所学基础知识在高级注释中无处不在。}
\begin{enumerate}
  \item 基因组注释的基础知识
    \begin{itemize}
      \item 基因组的组装版本
      \item 两种坐标系统
      \item 四种常用格式
      \item 逻辑运算模式
    \end{itemize}
  \item 基因组功能的高级注释
    \begin{itemize}
      \item 变异位点的注释
      \item 基因集富集分析
      \item 序列标识
      \item Galaxy分析平台
    \end{itemize}
\end{enumerate}

\noindent
二、变异位点的注释(15分钟)

\textcolor{red}{重点讲解注释结果的解读及其在功能注释流程中承上启下的作用。}
\begin{enumerate}
  \item 单核苷酸变异的注释
    \begin{itemize}
      \item 注释内容:附加相关的基因组注释信息(数据库ID,基因名,变异功能类别,……)
      \item 注释工具:SeattleSeq Annotation,variant tools,SnpEff
      \item 结果解读:SeattleSeq Annotation的注释结果\textcolor{red}{(通过实例解读注释结果;对注释结果过滤筛选后可继续进行非同义多态性的注释)}
    \end{itemize}
  \item 非同义多态性的注释
    \begin{itemize}
      \item 注释内容:对蛋白质产物结构和功能的影响
      \item 注释工具:SIFT,PolyPhen-2,SNPs3D,PROVEAN
      \item 结果解读:SIFT的注释结果\textcolor{red}{(通过实例解读注释结果;承接SNVs的注释结果;对结果过滤筛选后可继续进行基因集的富集分析)}
    \end{itemize}
\end{enumerate}

\noindent
三、基因集富集分析(10分钟)
\begin{enumerate}
  \item 基因集富集分析\textcolor{red}{(承接变异位点的注释)}
    \begin{itemize}
      \item 富集分析:GO,KEGG
      \item 结果解读:富集显著性及多重检验校正\textcolor{red}{(通过实例解读富集分析的结果)}
    \end{itemize}
  \item DAVID分析工具\textcolor{red}{(以DAVID为例介绍查找工具使用protocol的方法与资源)}
    \begin{itemize}
      \item Gene Name Batch Viewer:把基因ID转换成基因名
      \item Gene ID Conversion Tool:转换数据库间的基因ID
      \item Gene Functional Classification Tool:根据注释信息聚类基因
      \item Functional Annotation Tool
	\begin{itemize}
          \item Functional Annotation Clustering:根据注释信息聚类注释项目
          \item Functional Annotation Chart:根据注释信息进行富集分析\textcolor{red}{(重点介绍)}
          \item Functional Annotation Table:以表格形式呈现注释信息
	\end{itemize}
    \end{itemize}
\end{enumerate}

\noindent
四、序列标识(20分钟)
\begin{enumerate}
  \item 图形含义\textcolor{red}{(以GT-AG规则为例,展示示意图)}
    \begin{itemize}
      \item 横轴:序列的位置
      \item 纵轴:保守性
      \item 字符堆叠的总高度:此位置的保守性
      \item 每个字符的高度:出现的相对频率
    \end{itemize}
  \item 制作工具:WebLogo,enoLOGOs\textcolor{red}{(演示WebLogo制作GT-AG规则的操作并解读结果)}
\end{enumerate}

\otherTail
\newpage
\otherHeader

\noindent
五、Galaxy分析平台(40分钟)
\begin{enumerate}
  \item 生物信息学数据分析平台:Galaxy,GenePattern
  \item Galaxy分析平台\textcolor{red}{(展示Galaxy的主界面)}
    \begin{itemize}
      \item 主界面:顶部是刊头,左侧栏是工具菜单,中间是工作区,右侧栏是历史面板
      \item 工具集:Get Data,Text Manipulation,Convert Formats,Operate on Genomic Intervals,……
      \item 学习资料:Galaxy 101,Galaxy Screencasts and Demos,Learn
	Galaxy,Galaxy Wiki\textcolor{red}{(以Galaxy为例介绍学习新工具的方法与步骤:官网手册,由浅入深)}
    \end{itemize}
  \item Galaxy的使用:\textcolor{red}{以寻找Y染色体上含有SNP数目最多的外显子为例进行操作演示}\textcolor{red}{(介绍“工作流”的思想及其优势,以及Galaxy中工作流的提取、制作、使用和分享)}
\end{enumerate}

\noindent
六、总结与答疑(10分钟)
\begin{enumerate}
  \item 知识点
    \begin{itemize}
      \item 变异位点的注释:注释工具,结果解读
      \item 基因集富集分析:DAVID,结果解读
      \item 序列标识:图形含义,制作方法,结果解读
      \item Galaxy分析平台:使用方法
    \end{itemize}
  \item 技能
    \begin{itemize}
      \item 查找工具使用的protocol
      \item 学习新工具的方法与步骤
      \item 数据处理的“工作流”思想
    \end{itemize}
\end{enumerate}

\otherTail

\end{document}

