\documentclass{TIJMUjiaoanLL}
\pagestyle{empty}


\begin{document}


%课程名称
\kecheng{生物信息学}
%课程内容
\neirong{DNA序列及其特征分析}
%教师姓名
\jiaoshi{伊现富}
%职称
\zhicheng{讲师}
%教学日期(格式:XXXX年XX月XX日XX时-XX时)
\riqi{2013年8月28日14时-16时}
%授课对象(格式:XXX系XXXX年级XX班(硕/本/专科))
\duixiang{生物医学工程学院2010级生信班(本)}
%听课人数
\renshu{23}
%授课方式
\fangshi{理论讲授}
%学时数
\xueshi{2}
%教材版本
\jiaocai{生物信息学(自编教材)}


%教案首页
\firstHeader
\maketitle
\thispagestyle{empty}

\mudi{
\begin{itemize}
	\item 掌握DNA序列基本信息的分析内容,了解相关的分析工具并自学其使用方法。

	\item 掌握限制性核酸内切酶的命名规则及特点,了解相关的数据库与分析工具并自学使用方法。
	\item 掌握ORF和CDS的定义与区别,熟悉ORF分析的常用工具及其使用方法。
	\item 掌握原核基因和真核基因启动子的结构,了解启动子和转录因子相关的数据库与分析工具并自学其使用方法。
	\item 掌握CpG岛的概念及其识别依据与标准,熟悉CpG岛识别的工具及其使用的标准。
	\item 掌握重复序列的概念及分类,了解相关数据库与分析工具并自学其使用方法。
\end{itemize}
}

\fenpei{
\begin{itemize}
	\item (5')课堂导入:回顾中心法则,阐释核酸序列携带的两类遗传信息。
	\item (25')DNA序列转换与组分分析:回顾Chargaff规则,讲解基本的序列转换、主要的组分分析、GC含量的计算,总结解决问题的常用方法。

	\item (10')限制性核酸内切酶位点分析:讲解限制性核酸内切酶的概念、命名规则及其特征,介绍常用的数据库与分析工具。
	\item (10')开放阅读框分析:讲解ORF与CDS的定义与区别,介绍常用的ORF分析工具。
	\item (10')启动子分析:讲解启动子与转录因子的基本概念,回顾原核基因和真核基因启动子的结构,介绍常用数据库与工具。
	\item (15')CpG岛识别:讲解CpG岛的概念、判别依据和标准,介绍识别CpG岛的计算工具。
	\item (15')重复序列分析:讲解重复序列的概念、分类及特点,介绍常用数据库与工具。
	\item (10')总结与答疑:回顾授课内容中的知识点,总结分析解决问题的方法。
\end{itemize}
}

\zhongdian{
\begin{itemize}
	\item 重点:CpG岛识别的依据及标准,重复序列的分类;解决策略:通过实例加深学生的理解。
	\item 难点:分析解决一个全新问题的思路、步骤与具体方法;解决策略:以简单的计算GC含量为例进行分析。
\end{itemize}
}

\waiyu{
	\vspace*{-10pt}
	\begin{multicols}{2}
	中心法则(central dogma)

	GC含量(GC content)

	限制性核酸内切酶(restriction endonuclease)

	开放阅读框(Open Reading Frame,ORF)

	编码序列(Coding Sequence,CDS)

	启动子(promoter)

	CpG岛(CpG island)

	重复序列(repetitive/repeated sequence)
	\end{multicols}
	\vspace*{-10pt}
}

\fuzhu{
\begin{itemize}
	\item 多媒体:展示中心法则、开放阅读框、启动子结构等示意图。
	\item 板书:课程结构和分析解决问题的思路。
\end{itemize}
}

\sikao{
	\vspace*{-10pt}
	\begin{multicols}{2}
	\begin{itemize}
		\item DNA携带的两类遗传信息。
		\item 分析解决新问题的思路与方法。
		\item 限制性核酸内切酶的命名规则及特点。
		\item ORF与CDS的定义和区别。
		\item 判别CpG岛的依据及其标准。
		\item 不同标准下重复序列的分类。
	\end{itemize}
	\end{multicols}
	\vspace*{-10pt}
}

\cankao{
\begin{itemize}
	\item 朱玉贤,李毅,郑晓峰。现代分子生物学(第3版),高等教育出版社,2007。
	\item 维基百科。
\end{itemize}
}

\firstTail


%教案续页
\newpage
\otherHeader

\noindent
一、课堂导入(5分钟)
\begin{enumerate}
	\item 分子生物学的中心法则:DNA转录成RNA,RNA翻译成蛋白质。
		\begin{itemize}
			\item DNA:携带最原始的决定个体性状的遗传信息
			\item RNA:参与遗传信息的表达和调控
			\item 蛋白质:执行特定的生物功能从而决定最终的表型
		\end{itemize}
	\item DNA携带两类遗传信息
		\begin{itemize}
			\item 功能序列:具有功能活性的DNA序列,遗传的基本单位
			\item 调控信息:特定的DNA区域,能被功能性蛋白质分子特异地识别结合
		\end{itemize}
	\item DNA序列分析
		\begin{itemize}
			\item 基本信息:碱基组份,序列转换,限制性核酸内切酶位点,……
			\item 特征信息:开放阅读框,启动子,转录因子结合位点,CpG岛,……
		\end{itemize}
\end{enumerate}

\noindent
二、DNA序列转换与组份分析(25分钟)

\textcolor{red}{以查戈夫法则引申出序列组份分析、序列转换的内容与原理。}
\begin{enumerate}
	\item 查戈夫法则
		\begin{itemize}
			\item $A=T, G=C \Rightarrow$ 序列长度,碱基数目及比例,序列转换
			\item $AT/GC$的比值因生物种类不同而异 $\Rightarrow$ GC含量
		\end{itemize}
	\item GC含量
		\begin{itemize}
			\item GC content: $\frac{G+C}{A+T+G+C} \times 100$
			\item GC ratio: $\frac{A+T}{G+C}$
		\end{itemize}
	\item 序列转换
		\begin{itemize}
			\item 反向序列
			\item 互补序列
			\item 反向互补序列 $\Rightarrow$ 序列书写惯例
			\item 显示DNA双链
			\item 转换为RNA序列
		\end{itemize}
	\item 生物信息学技能——分析解决问题的策略
		\begin{itemize}
			\item 以简单的计算GC含量为例\textcolor{red}{(使用简单例子易于学生理解)}
			\item 任务属性决定解决策略\textcolor{red}{(使用序列长短、数目多少的实例)}
		\end{itemize}
\end{enumerate}

\noindent
三、限制性核酸内切酶位点分析(10分钟)
\begin{enumerate}
	\item 限制性核酸内切酶
		\begin{itemize}
			\item 定义:识别DNA特异序列、并在识别位点或其周围切割双链DNA的内切酶
			\item 命名规则:\textcolor{red}{以\textit{Eco}RI为例}
			\item II型特点:识别序列,切割位点,切割末端\textcolor{red}{(以\textit{Eco}RI、\textit{Alu}I等实例加深学生的印象)}
		\end{itemize}
	\item 相关资源
		\begin{itemize}
			\item 数据库:REBASE
			\item 分析工具:NEBCutter V2.0
		\end{itemize}
\end{enumerate}

\noindent
四、开放阅读框分析(10分钟)
\begin{enumerate}
	\item ORF:定义,相位\textcolor{red}{(图示ORF的六个相位,加深理解)}
	\item CDS:定义
	\item ORF VS. CDS:理论预测 VS. 实验证实
	\item 分析工具:ORF Finder
\end{enumerate}

\otherTail
\newpage

\noindent
五、启动子分析(10分钟)
\begin{enumerate}
	\item 转录调控
	  \begin{itemize}
	    \item 顺式作用元件:核酸序列 $\Rightarrow$ 启动子
	    \item 反式作用因子:蛋白质
	  \end{itemize}
	\item 启动子
	  \begin{itemize}
	    \item 概念:启动子,转录起始位点 $\Rightarrow$ 书写规则与坐标含义
	    \item 结构:原核基因 VS. 真核基因\textcolor{red}{(图示、对比原核基因和真核基因的启动子结构,加深理解)}
	  \end{itemize}
	\item 转录因子
	  \begin{itemize}
	    \item 转录因子:蛋白质
	    \item 转录因子结合位点:DNA序列
	  \end{itemize}
	\item 相关资源
	  \begin{itemize}
	    \item 数据库:EPD;TRANSFAC
	    \item 分析工具:Promoter Scan,Promoter 2.0;Tfblast
	  \end{itemize}
\end{enumerate}

\noindent
六、CpG岛识别(15分钟)

\textcolor{red}{以使用EMBOSS识别CpG岛的实例操作加深学生对CpG岛识别依据和标准的理解,同时熟悉CpG岛分析工具的使用方法。}
\begin{enumerate}
  \item CpG岛:概念,特点,功能
  \item 识别依据与标准
    \begin{itemize}
      \item GC含量:50\% $\rightarrow$ 55\%
      \item CpG岛的长度:200bp $\rightarrow$ 500bp
      \item CpG二核苷酸的出现频率:60\% $\rightarrow$ 65\%\\
	(计算公式:$\frac{Num\ of\ CpG}{Num\ of\ C \times Num\ of\ G} \times Total\ number\ of\ nucleotides\ in\ the\ sequence$)
    \end{itemize}
  \item 分析工具:EMBOSS(CpGPlot/CpGReport/Isochore)
\end{enumerate}

\noindent
七、重复序列分析(15分钟)
\begin{enumerate}
  \item 重复序列:定义
  \item 分类\textcolor{red}{(对每一分类都给出实例帮助学生记忆)}
    \begin{itemize}
      \item 重复次数:低度重复序列,中度重复序列,高度重复序列
      \item 组织形式:串联重复序列(卫星DNA,小卫星,微卫星),散在重复序列(短散在重复序列,长散在重复序列)
    \end{itemize}
  \item 相关资源
    \begin{itemize}
      \item 数据库:Repbase,L1Base,STRBase
      \item 分析工具:RepeatMasker(四个搜索引擎)
    \end{itemize}
\end{enumerate}

\noindent
八、总结与答疑(10分钟)
\begin{enumerate}
  \item 知识点
    \begin{itemize}
      \item DNA序列基本信息分析:查戈夫法则,GC含量,序列转换
      \item 限制性核酸内切酶位点分析:命名规则,II型核酸酶的特点
      \item 开放阅读框分析:ORF和CDS的区别
      \item 启动子分析:原核基因和真核基因的启动子结构
      \item CpG岛识别:判别依据及其标准
      \item 重复序列分析:不同标准下的分类
    \end{itemize}
  \item 技能
    \begin{itemize}
      \item 解决新问题的思路
      \item 寻找最合适的方法
    \end{itemize}
\end{enumerate}

\otherTail


\end{document}

