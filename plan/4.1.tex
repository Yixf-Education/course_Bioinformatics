\documentclass{TIJMUjiaoanLL}
\pagestyle{empty}


\begin{document}


%课程名称
\kecheng{生物信息学}
%课程内容
\neirong{第四章(4.1)DNA序列信息分析}
%教师姓名
\jiaoshi{伊现富}
%职称
\zhicheng{讲师}
%教学日期(格式:XXXX年XX月XX日XX时-XX时)
\riqi{2021年3月31日18:00-19:40}
%授课对象(格式:XXX系XXXX年级XX班(硕/本/专科))
\duixiang{全校公选课}
%听课人数
\renshu{33}
%授课方式
\fangshi{理论讲授}
%学时数
\xueshi{2}
%教材版本
\jiaocai{生物信息学:基础及应用}


%教案首页
\firstHeader
\maketitle
\thispagestyle{empty}

\mudi{
\begin{itemize}
  \item 掌握限制性核酸内切酶的命名规则及II型限制酶特点;CpG岛的概念及其识别依据和判别标准。
  \item 熟悉DNA序列分析的常见内容;ORF分析中相位的概念;原核和真核基因启动子的结构。
  \item 了解DNA携带的两类遗传信息;DNA序列分析相关的数据库和工具;ORF和CDS的定义与区别。
  \item 自学DNA序列分析数据库和工具的使用方法。
\end{itemize}
}

\fenpei{
\begin{itemize}
  \item (5')引言与导入:回顾中心法则,阐释核酸序列携带的两类遗传信息。
  \item (25')DNA组份分析与序列转换:回顾Chargaff法则,讲解GC含量的定义与计算,介绍组份分析和序列转换的原理和思路,讨论解决问题的基本策略。
  \item (15')限制性核酸内切酶位点分析:讲解限制性核酸内切酶的概念、命名规则和II型限制酶的特征,介绍常用的数据库与分析工具。
  \item (10')开放阅读框分析:讲解相位的概念以及ORF与CDS的定义和区别,介绍常用的ORF分析工具。
  \item (10')启动子分析:讲解启动子与转录因子的基本概念,回顾原核基因和真核基因启动子的结构,介绍相关数据库与工具。
  \item (10')CpG岛识别:讲解CpG岛的概念、识别依据和判别标准,介绍识别CpG岛的计算工具。
  \item (10')EMBOSS简介:介绍EMBOSS软件包及其中常用的程序。
  \item (5')总结与答疑:总结授课内容中的知识点与技能,解答学生疑问。
\end{itemize}
}

\zhongdian{
\begin{itemize}
  \item 重点:限制酶的命名规则,CpG岛的识别依据和判别标准。
  \item 难点:开放阅读框中相位的概念。
  \item 解决策略:通过示意图和实例帮助学生理解、记忆。
\end{itemize}
}

\waiyu{
  \vspace*{-10pt}
  \begin{multicols}{2}
  中心法则(central dogma)

  GC含量(GC content)

  限制性核酸内切酶(restriction endonuclease)

  开放阅读框(Open Reading Frame,ORF)

  编码序列(Coding Sequence,CDS)

  启动子(promoter)

  转录因子结合位点(TFBS)

  CpG岛(CpG island)
  \end{multicols}
  \vspace*{-10pt}
}

\fuzhu{
\begin{itemize}
  \item 多媒体:展示中心法则、开放阅读框相位、启动子结构等的示意图。
  \item 板书:序列的书写惯例,限制酶的命名规则,CpG岛的识别依据和判别标准。
\end{itemize}
}

\sikao{
  \vspace*{-10pt}
  \begin{multicols}{2}
  \begin{itemize}
    \item 简述DNA携带的两类遗传信息及常见的分析内容。
    \item 简述限制酶的命名规则及II型的主要特点。
    \item 简述ORF与CDS的定义和区别。
    \item 简述CpG岛的概念、识别依据和判别标准。
    \item 论述分析任务属性和解决问题的基本策略。
  \end{itemize}
  \end{multicols}
  \vspace*{-10pt}
}

\cankao{
\begin{itemize}
  \item 朱玉贤,李毅,郑晓峰。现代分子生物学(第3版),高等教育出版社,2007。
  \item 维基百科。
\end{itemize}
}

\firstTail


%教案续页
\newpage
\otherHeader

\noindent
一、引言与导入(5分钟)
\begin{enumerate}
  \item 分子生物学的中心法则:DNA转录成RNA,RNA翻译成蛋白质。
    \begin{itemize}
\parpic[fr]{\includegraphics[width=9cm]{dogma.jpg}}
      \item DNA:携带最原始的决定个体性状的遗传信息
      \item RNA:参与遗传信息的表达和调控
      \item 蛋白质:执行特定的生物功能从而决定最终的表型
      \item 排列顺序蕴含生物信息:类似于二进制中运用一连串的0和1以及英文字母表中运用26个不同的字母来表达信息\textcolor{red}{(通过类比进行说明)}
    \end{itemize}
  \item DNA携带两类遗传信息
    \begin{itemize}
      \item 功能序列:具有功能活性的DNA序列,遗传的基本单位
      \item 调控信息:特定的DNA区域,能被功能性蛋白质分子特异地识别结合
    \end{itemize}
  \item DNA序列分析
    \begin{itemize}
      \item 基本信息:碱基组份,GC含量,序列转换,限制性核酸内切酶位点,……
      \item 特征信息:开放阅读框,启动子,转录因子结合位点,CpG岛,……
    \end{itemize}
\end{enumerate}

\vspace*{0.2cm}
\noindent
二、DNA组份分析与序列转换(25分钟)

\textcolor{red}{以Chargaff法则引申出序列组份分析、序列转换的内容与原理。}
\begin{enumerate}
  \item Chargaff法则
    \begin{itemize}
      \item $A=T, G=C \Rightarrow$ 序列长度,碱基数目及比例,序列转换
      \item $AT/GC$的比值因生物种类不同而异 $\Rightarrow$ GC含量
    \end{itemize}
  \vspace*{-10pt}
  \begin{multicols}{2}
  \item GC含量
    \begin{itemize}
      \item 鸟嘌呤(G)和胞嘧啶(C)所占的比例
      \item GC content: $\frac{G+C}{A+T+G+C} \times 100$
      \item GC ratio: $\frac{A+T}{G+C}$
    \end{itemize}
  \item 序列转换
    \begin{itemize}
      %\item 反向序列
      %\item 互补序列
      \item 反向序列,互补序列
      \item 反向互补序列 $\Rightarrow$ 序列书写惯例
      \item 显示DNA双链
      \item 转换为RNA序列
    \end{itemize}
  \vspace*{-10pt}
  \end{multicols}
  \vspace*{-15pt}
  \item 序列书写惯例
    \begin{itemize}
      \item DNA/RNA:[左] 5' $\Longrightarrow$ 3' [右]
      \item 多肽/蛋白质:[左] N端(氨基端)$\Longrightarrow$ C端(羧基端) [右]
    \end{itemize}
  \item 分析解决问题的策略
    \begin{itemize}
      \item 以计算GC含量为例\textcolor{red}{(使用简单例子易于学生理解)}
      \item 任务属性决定解决策略\textcolor{red}{(使用序列长短、数目多少的实例进行讲解)}
    \end{itemize}
\end{enumerate}

\vspace*{0.2cm}
\noindent
三、限制性核酸内切酶位点分析(15分钟)
\begin{enumerate}
  \item 限制性核酸内切酶
    \begin{itemize}
\parpic[fr]{\includegraphics[width=7cm]{ecori.png}}
      \item 定义:识别DNA特异序列、并在识别位点或其周围切割双链DNA的内切酶
      \item \textcolor{red}{\textbf{【重点】}}命名规则\textcolor{red}{(以\textit{Eco}RI为例)}
        \begin{itemize}
          \item 属名的第一个字母
          \item 种名的前两个字母
          \item 细菌的菌株/品系
          \item 同一品系中的发现顺序
        \end{itemize}
    \end{itemize}


\otherTail
\newpage
\otherHeader


    \begin{itemize}
      \item II型限制酶的特点\textcolor{red}{(以\textit{Eco}RI、\textit{Alu}I等实例加深学生的印象)}
      \begin{itemize}
\parpic[fr]{\includegraphics[width=7.8cm]{enzyme.png}}
        \item 识别、切割位点专一
        \item 识别序列:4-8个碱基,回文对称结构
        \item 切割序列:识别序列,切割位点对称
        \item 切割末端:黏性末端,平滑末端
        \item 黏性末端:切割位点在回文序列的一侧
        \item 平滑末端:切割位点在回文序列的中间
      \end{itemize}
    \end{itemize}
  \item 相关资源
    \begin{itemize}
      \item 数据库:REBASE收录了限制酶的所有信息
      \item 分析工具:NEBCutter V2.0产生DNA序列的酶切位点分析结果
    \end{itemize}
\end{enumerate}

\vspace*{0.2cm}
\noindent
四、开放阅读框分析(10分钟)
\begin{enumerate}
\parpic[fr]{\includegraphics[width=9cm,height=4cm]{orf.png}}
  \item ORF:开放阅读框
  \item \textcolor{red}{\textbf{【难点】}}frame:相位\textcolor{red}{(通过示意图加深理解)}
  \item CDS:编码序列
  \item ORF vs. CDS:理论预测 vs. 实验证实
  \item 分析工具:ORF Finder
\end{enumerate}

\vspace*{0.2cm}
\noindent
五、启动子分析(10分钟)
\begin{enumerate}
  \item 转录调控
    \begin{itemize}
      \item 顺式作用元件:核酸序列 $\Rightarrow$ 启动子
      \item 反式作用因子:蛋白质
      \item 两者相互作用实现转录调控
    \end{itemize}
  \item 启动子
    \begin{itemize}
      \item 基本概念
        \begin{itemize}
          \item 启动子:一段位于转录起始位点5'端上游区的DNA序列
          \item 转录起始位点:与新生RNA链第一个核苷酸相对应DNA链上的碱基\textcolor{red}{(图示TSS附近的坐标)}
        \end{itemize}
      \begin{figure}[h]
        \centering
        \includegraphics[width=13cm]{tss.png}
      \end{figure}
      \item 启动子结构\textcolor{red}{(图示、对比原核和真核基因的启动子结构,帮助学生记忆)}
        \begin{itemize}
\parpic[fr]{\includegraphics[width=8cm]{promoter.jpg}}
          \item 原核基因
            \begin{itemize}
	      \item -10区,-10,TATAAT
	      \item -35区,-35,TTGACA
	    \end{itemize}
          \item 真核基因
            \begin{itemize}
	      \item TATA box,-25~-30,TATAAA
	      \item CAAT box,-70~-80,CCAAT
	    \end{itemize}
        \end{itemize}
    \end{itemize}
  \item 转录因子
    \begin{itemize}
      \item 转录因子:蛋白质
      \item 转录因子结合位点:DNA序列,5~20bp
    \end{itemize}
  \item 相关资源
    \begin{itemize}
      \item 数据库:EPD;TRANSFAC
      \item 分析工具:Promoter Scan,Promoter 2.0;Tfblast
    \end{itemize}
\end{enumerate}


\otherTail
\newpage
\otherHeader


\noindent
六、CpG岛识别(10分钟)
\begin{enumerate}
  \item CpG岛简介
    \begin{itemize}
      \item CpG保持或高于正常概率的基因组区段
      \item 一般位于基因(尤其是看家基因)的5'端区域,长度约300~3000bp;大多数未甲基化
    \end{itemize}
  \item \textcolor{red}{\textbf{【重点】}}识别依据与判别标准\textcolor{red}{(提醒学生判别标准不是唯一的)}
    \begin{itemize}
      \item GC含量:50\% $\rightarrow$ 55\%
      \item CpG岛的长度:200bp $\rightarrow$ 500bp
      \item CpG二核苷酸的出现频率:60\% $\rightarrow$ 65\%\\
	(计算公式:$\frac{Num\ of\ CpG}{Num\ of\ C \times Num\ of\ G} \times Total\ number\ of\ nucleotides\ in\ the\ sequence$)
    \end{itemize}
  \item 分析工具:EMBOSS(CpGPlot/CpGReport/Isochore)
\end{enumerate}

\vspace*{0.2cm}
\noindent
七、EMBOSS简介(10分钟)
\begin{enumerate}
  \item EMBOSS简介
    \begin{itemize}
      \item 开源、免费的序列分析软件包,整合了目前可以获得的大部分序列分析软件
      \item 可以将系列分析工作进行无缝整合,弥补了许多软件功能分散、分析效率低下的缺陷
    \end{itemize}
  \item 使用界面
    \begin{itemize}
      \item 操作系统:Linux,Mac,\textcolor{darkgray}{Windows}
      \item JEMBOSS:java界面
      \item EMBOSS Explorer:web界面
    \end{itemize}
  \item 主要程序
    \begin{itemize}
      \item 最重要的程序。Wossname:根据关键字查找程序;Showdb:显示所有整合的数据库。
      \item 序列编辑。Revseq:将序列反转并互补;Seqret:序列格式转换。
      \item 两个序列相似性图形表达。Dottup:精确匹配;Dotmatcher:近似匹配。
      \item 双序列比对。Needle:全局比对;Water:局部比对。
      \item 多序列比对。Emma:clustalW。
      \item 寻找SNP。Deffseq:仅限于双序列比对中。
      \item 其他。Plotorf,Getorf:翻译;Iep:等电点预测;Tmap:跨膜区预测;Pepinfo:蛋白质性质;Patmatmotifs:Motif搜索。
    \end{itemize}
  \item 使用实例:\textcolor{red}{以使用EMBOSS识别CpG岛的实例操作加深学生对CpG岛识别依据和标准的理解,同时熟悉EMBOSS的使用方法}
\end{enumerate}

\vspace*{0.2cm}
\noindent
八、总结与答疑(5分钟)
\begin{enumerate}
  \item 知识点
    \begin{itemize}
      \item DNA序列基本信息分析:Chargaff法则,GC含量,序列转换
      \item 限制性核酸内切酶位点分析:命名规则,II型核酸酶的特点
      \item 开放阅读框分析:相位,ORF和CDS的区别
      \item 启动子分析:原核基因和真核基因的启动子结构
      \item CpG岛识别:概念、识别依据及判别标准
    \end{itemize}
  \item 技能
    \begin{itemize}
      \item 任务属性决定解决方案
      \item 寻找最合适的方法
      \item 先易后难,由浅入深
    \end{itemize}
\end{enumerate}


\otherTail


\end{document}

