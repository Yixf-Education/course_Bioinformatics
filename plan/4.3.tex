\documentclass{TIJMUjiaoanLL}
\pagestyle{empty}


\begin{document}


%课程名称
\kecheng{生物信息学}
%课程内容
\neirong{第四章(4.3)RNA序列分析}
%教师姓名
\jiaoshi{伊现富}
%职称
\zhicheng{讲师}
%教学日期(格式:XXXX年XX月XX日XX时-XX时)
\riqi{2021年4月7日18:00-19:40}
%授课对象(格式:XXX系XXXX年级XX班(硕/本/专科))
\duixiang{全校公选课}
%听课人数
\renshu{33}
%授课方式
\fangshi{理论讲授}
%学时数
\xueshi{2}
%教材版本
\jiaocai{生物信息学:基础及应用}


%教案首页
\firstHeader
\maketitle
\thispagestyle{empty}

\mudi{
\begin{itemize}
  \item 掌握mRNA选择性剪接的主要机制;miRNA预测和miRNA靶基因预测的方法。
  \item 熟悉miRNA的特点、生成过程和作用方式;miRNA的相关数据库和分析工具。
  \item 了解选择性剪接的相关数据库与分析工具;lncRNA的定义、主要特征及其研究现状。
  \item 自学lncRNA的生物功能和作用方式;lncRNA在疾病发生发展过程中的作用。
\end{itemize}
}

\fenpei{
\begin{itemize}
  \item (5')回顾与导入:回顾序列分析和基因识别的主要内容,总结RNA的主要类别,引出RNA分析的内容。
  \item (30')mRNA选择性剪接:介绍剪接和选择性剪接的基本概念,讲解mRNA选择性剪接的主要机制,介绍相关的数据库和分析工具。
  \item (30')miRNA及其靶基因预测:回顾miRNA的特点、生成过程、作用方式和生物学功能,讲解miRNA预测和miRNA靶基因预测的主要方法,介绍常用的数据库和分析工具。
  \item (10')lncRNA简介:介绍lncRNA的定义、主要特征及其研究进展。
  \item (10')学习数据库与分析工具的使用:讨论学习数据库和分析工具使用方法的主要策略。
  \item (5')总结与答疑:总结授课内容中的知识点,解答学生疑问。
\end{itemize}
}

\zhongdian{
\begin{itemize}
  \item 重点:mRNA选择性剪接的主要机制。
  \item 难点:mRNA选择性剪接各种机制之间的区别。
  \item 解决策略:通过示意图和实例帮助学生理解、记忆。
\end{itemize}
}

\waiyu{
\vspace*{-10pt}
\begin{multicols}{2}
选择性剪接(alternative splicing)

微RNA(miRNA,microRNA)

非编码RNA(ncRNA,non-coding RNA)

长链非编码RNA(lncRNA)
\end{multicols}
\vspace*{-10pt}
}

\fuzhu{
\begin{itemize}
  \item 多媒体:mRNA选择性剪接的机制和实例,miRNA的生成过程和作用方式,lncRNA的生物功能和作用机制。
  \item 板书:学习数据库和分析工具使用方法的主要策略。
\end{itemize}
}

\sikao{
  \begin{itemize}
    \item 简述mRNA选择性剪接的主要机制。
    \item 简述miRNA的特点、生成过程和作用方式。
    \item 简述miRNA预测和miRNA靶基因预测的方法。
    \item 论述学习数据库和分析工具使用方法的主要策略。
  \end{itemize}
}

\cankao{
\begin{itemize}
  \item 朱玉贤,李毅,郑晓峰。现代分子生物学(第3版),高等教育出版社,2007。
  \item 李霞,李亦学,廖飞。生物信息学,人民卫生出版社,2010。
  \item 维基百科。
\end{itemize}
}

\firstTail


%教案续页
\newpage
\otherHeader

\noindent
一、回顾与导入(5分钟)
\begin{enumerate}
  \item 序列分析
    \begin{itemize}
      \item DNA序列分析:基本信息,序列特征,基因识别
      \item RNA序列分析:mRNA选择性剪接,miRNA与靶基因,lncRNA
    \end{itemize}
  \item
    RNA的分类\textcolor{red}{(RNA既是携带遗传信息的主要生物大分子,也是重要的功能单位)}
    \begin{itemize}
      \item 编码RNA:mRNA
      \item 非编码RNA:tRNA,rRNA;miRNA,siRNA,lncRNA
    \end{itemize}
  \item ncRNA的分类\textcolor{red}{(转录后不编码蛋白质的RNA分子的统称)}
    \begin{itemize}
      \item 基础结构性ncRNA/看家ncRNA:tRNA,rRNA,snRNA,snoRNA
      \item 调节性ncRNA
      \begin{itemize}
	\item sRNA:\textless 200nt,miRNA、siRNA、piRNA\textcolor{red}{(已经开展了广泛的研究)}
        \item lncRNA:\textgreater 200nt,长链非编码RNA\textcolor{red}{(引起关注,研究正逐步深入)}
      \end{itemize}
    \end{itemize}
\end{enumerate}
\vspace*{-10pt}
\begin{figure}[h]
  \centering
  \includegraphics[width=16cm]{ncrnaC.jpg}
\end{figure}
\vspace*{-10pt}

\vspace*{0.2cm}
\noindent
二、mRNA选择性剪接(30分钟)
\begin{enumerate}
  \item 基本概念
    \begin{itemize}
      \item 剪接:移除内含子、合并外显子
      \item 选择性剪接:一个mRNA前体 $\Rightarrow$ 不同mRNA剪接异构体
    \end{itemize}
  \item \textcolor{red}{\textbf{【重点、难点】}}主要机制\textcolor{red}{(通过示意图和实例详解每种机制)}
    \begin{itemize}
\parpic[fr]{\includegraphics[width=9cm]{splicingModel.jpg}}
      \item 外显子跳跃:外显子被移除或保留,最常见
      \item 互斥外显子:两个外显子只有一个保留下来,相对较少见
      \item 5'选择性剪接:使用不同的5'端的供体位点
      \item 3'选择性剪接:使用不同的3'端的受体位点
      \item 内含子保留:内含子作为外显子保留下来,最少见
      \item 选择性起始:在不同的位点起始转录
      \item 选择性终止:使用不同的polyA位点
    \end{itemize}
  \item 相关资源\textcolor{red}{(提醒注意数据库的时效性)}
    \begin{itemize}
      \item 数据库:ASTD,ASAP,ASPicDB
      \item 分析工具:ESEfinder,RESCUE-ESE
    \end{itemize}
\end{enumerate}


\otherTail
\newpage
\otherHeader


%\vspace*{0.2cm}
\noindent
三、miRNA及其靶基因预测(30分钟)
\begin{enumerate}
  \item miRNA简介\textcolor{red}{(通过示意图形象化展示相关内容)}
    \begin{itemize}
%\begin{minipage}{0.9\textwidth}
\parpic[fr]{\includegraphics[width=7.5cm,height=6.5cm]{mirna.png}}
      \item 真核生物中广泛存在的一种长约20到24个核苷酸的内源性非编码单链RNA分子
      \item 生成过程:300~1000nt的双链pri-miRNA $\Rightarrow$ 70~90nt的双链具有茎环结构的pre-miRNA $\Rightarrow$ 20~24nt的单链成熟miRNA
      \item 作用方式:完全互补型——结合在mRNA的编码区中,导致靶基因mRNA降解,在植物中比较常见;不完全互补型——结合在mRNA的3' UTR,导致靶基因mRNA的翻译受到抑制
      %\item 作用方式
      %\begin{itemize}
        %\item 完全互补型:结合在mRNA的编码区中,导致靶基因mRNA降解,在植物中比较常见
        %\item 不完全互补型:结合在mRNA的3' UTR,导致靶基因mRNA的翻译受到抑制
      %\end{itemize}
      \item 生物学功能:调控个体发育、细胞分化、组织发育、肿瘤发生发展、……
%\end{minipage}
    \end{itemize}
  \item miRNA的特征
    \begin{itemize}
      \item 序列:不具有开放阅读框,不编码蛋白质;成熟的miRNA 5'端为单一磷酸基团,3'端为羟基
      \item 表达:具有时序性和组织特异性
      \item 调控:miRNA与靶基因间呈多对多的关系
      \item 物理位置:倾向于成簇地出现在染色体上
      \item 进化:在物种间高度保守
    \end{itemize}
  \item miRNA预测
    \begin{itemize}
\parpic[fr]{\includegraphics[width=10.3cm,height=5.3cm]{mirnaS.png}}
      \item 同源片段搜索
      \item 基于比较基因组学
      \item 基于序列和结构特征打分
      \item 结合作用靶标
      \item 基于机器学习
    \end{itemize}
  \item miRNA靶基因预测
    \begin{itemize}
      \item 基于种子区域互补和保守性
      \item 基于机器学习
    \end{itemize}
  \item 相关资源
    \begin{itemize}
      \item 数据库:miRBase,TarBase,miRGen
      \item 分析工具:MiRscan,MiPred,miRFinder;miRanda,TargetScan,PicTar,miTarget
    \end{itemize}
\end{enumerate}

\vspace*{0.2cm}
\noindent
四、lncRNA简介(10分钟)
\begin{enumerate}
  \item lncRNA的特征
    \begin{itemize}
      \item 序列结构特征
	\vspace*{-10pt}
	\begin{multicols}{2}
        \begin{itemize}
	  \item 大多被RNA聚合酶II所转录
	  \item 有5'帽子和3'端的polyA尾巴
	  \item 剪接现象
	  \item 启动子区域和剪接位置具有保守性\\ \\
	  \item 长度偏短、外显子数目偏少
	  \item 不存在较长的ORF
	  \item 密码子偏好性与内含子区域相似
	  \item 二级结构中有丰富的长茎发夹结构
	  \item 在不同物种间的保守性差
	  \item 主要富集在细胞核
	\end{itemize}
      \end{multicols}
      \vspace*{-10pt}
    \end{itemize}

\otherTail
\newpage
\otherHeader

    \begin{itemize}
      \item 生物功能特征
        \begin{itemize}
	  \item 表达具有时空特异性,与特定的生物过程相关
	  \item 具有复杂的调控功能,在染色质改变、转录调控及后转录调控中发挥重要作用
\parpic[fr]{\includegraphics[width=8.5cm,height=2cm]{lncrnaL.jpg}}
	  \item 复杂的代谢机制,大多数lncRNA是稳定的,半衰期的变化范围较大
	  \item 与疾病存在密切关系
	\end{itemize}
    \end{itemize}
  \item lncRNA的研究进展\textcolor{red}{(展示相关内容的示意图)}
    \begin{itemize}
\parpic[fr]{\includegraphics[width=9.5cm]{lncrnaF.jpg}}
      \item 基因数目:13870(人类基因组,GENCODE V19)
      \item 类型:sense, antisense, intronic, intergenic, bidirectional
      \item 作用方式:表观遗传学水平、转录水平和转录后水平
      \item 生物学功能:基因转录、剪接、翻译、修饰和印迹等
      \item 与疾病的关系:肿瘤、阿尔兹海默病、心血管疾病等
    \end{itemize}
\end{enumerate}

\vspace*{0.2cm}
\noindent
五、学习数据库与分析工具的使用(10分钟)
\begin{itemize}
  \item 阅读官方的帮助手册
  \item 请教有使用经验的专家
  \item 查找简单的使用实例,并重复其操作步骤
  \item 使用Google等搜索引擎搜索相关资料
  \item 各种protocols期刊:\textit{Nature protocols, Current Protocols (in Bioinformatics), SpringerProtocols, Methods in Molecular Biology}
\end{itemize}

\vspace*{0.2cm}
\noindent
六、总结与答疑(5分钟)
\begin{enumerate}
  \item 知识点
    \begin{itemize}
      \item mRNA选择性剪接:主要机制
      \item miRNA:特征、生成过程和作用方式,miRNA及其靶基因预测方法
    \end{itemize}
  \item 技能
    \begin{itemize}
      \item 学习使用方法:阅读手册、请教专家、重复实例、搜索网络
      \item 历史资料使用的是历史版本
    \end{itemize}
\end{enumerate}


\otherTail

\end{document}

