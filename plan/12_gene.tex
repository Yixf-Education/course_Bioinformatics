\documentclass{TIJMUjiaoanSY}
\pagestyle{empty}


\begin{document}


%课程名称
\kecheng{生物信息学}
%实验名称
\shiyan{原核生物和真核生物的基因识别}
%教师姓名
\jiaoshi{伊现富}
%职称
\zhicheng{讲师}
%教学日期(格式:XXXX年XX月XX日XX时-XX时)
\riqi{2013年9月11日14时-16时}
%授课对象(格式:XXX系XXXX年级XX班(硕/本/专科))
\duixiang{生物医学工程学院2010级生信班(本)}
%实验人数
\renshu{23}
%实验类型
\leixing{验证型}
%实验分组
\fenzu{一人一机}
%学时数
\xueshi{2}
%教材版本
\jiaocai{生物信息学实验讲义(自编教材)}


%教案首页
\firstHeader
\maketitle
\thispagestyle{empty}

\mudi{
\begin{itemize}
  \item 了解隐马尔科夫模型在基因识别中的应用。
  \item 掌握原核基因和真核基因的结构特征。
  \item 掌握GeneMarkS和GENSCAN的使用方法。
\end{itemize}
}

\fenpei{
\begin{itemize}
  \item (5')基因结构:回顾原核生物和真核生物基因的结构特点。
  \item (5')基因识别的方法:回顾基因识别的三大类方法。
  \item (5')基因识别的工具:介绍GeneMarkS和GENSCAN。
  \item (85')实验操作:对大肠杆菌基因组序列进行基因识别,对人类CD9基因进行结构分析。
\end{itemize}
}

\cailiao{
\begin{itemize}
  \item 实验材料:大肠杆菌基因组,人类CD9基因。
  \item 主要仪器:联网的计算机。
  \item 分析工具:GeneMarkS,GENSCAN。
\end{itemize}
}

\zhongdian{
\begin{itemize}
  \item 难点:FASTA格式与序列的区别;解决策略:通过实例进行讲解。
  \item 重点:GeneMarkS和GENSCAN的使用;解决策略:通过练习熟练掌握。
\end{itemize}

}

\sikao{
\begin{itemize}
  \item 原核基因和真核基因的结构有何异同?
  \item GeneMarkS和GENSCAN的适用范围有何差别?
  \item GeneMarkS和GENSCAN对输入格式的要求有何差别?
\end{itemize}
}

\cankao{
\begin{itemize}
  \item NCBI
  \item GeneMarkS
  \item GENSCAN
\end{itemize}
}

\firstTail


%教案续页
\newpage
\otherHeader

\noindent
一、基因结构(5分钟)

\textcolor{red}{基因结构的复杂性直接影响着基因预测的策略及最终的准确度。}
\begin{enumerate}
  \item 原核基因:连续基因
  \item 真核基因:不连续性
\end{enumerate}

\noindent
二、基因识别的方法(5分钟)
\begin{enumerate}
  \item 间接识别法
  \item 从头预测法:基因预测
  \item 比较基因组学的方法
\end{enumerate}

\noindent
三、基因识别的工具(5分钟)

\textcolor{red}{分析工具有各自的适用范围。}
\begin{enumerate}
  \item GeneMarkS:迭代隐马尔科夫模型,适用于原核生物的基因预测
  \item GENSCAN:广义隐马尔科夫模型,脊椎动物基因预测软件
\end{enumerate}

\noindent
四、实验操作(85分钟)

\textcolor{red}{基因结构的复杂性直接影响着基因预测的准确度。}
\begin{enumerate}
  \item 大肠杆菌基因组序列的基因识别
    \begin{itemize}
      \item 获取序列:NCBI中的Nucleotide数据库,U00096,FASTA格式和GenBank格式\textcolor{red}{(自学GenBank格式)}
      \item 截取序列:EMBOSS,extractseq,1-10000bp
      \item 基因预测:GeneMarkS,FASTA格式
      \item 结果分析:和GenBank格式中的信息进行比较
    \end{itemize}
  \item 人类CD9基因的结构分析
    \begin{itemize}
      \item 获取序列:NCBI中的Nucleotide数据库,AY422198,FASTA格式和GenBank格式
      \item 基因预测:GENSCAN,纯序列\textcolor{red}{(注意不是FASTA格式)}
      \item 结果分析:和GenBank格式中的信息进行比较
    \end{itemize}
\end{enumerate}

\otherTail


\end{document}

