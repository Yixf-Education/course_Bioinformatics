\documentclass{TIJMUjiaoanLL}
\pagestyle{empty}


\begin{document}


%课程名称
\kecheng{生物信息学}
%课程内容
\neirong{第五章(5.3)Galaxy分析平台}
%教师姓名
\jiaoshi{伊现富}
%职称
\zhicheng{讲师}
%教学日期(格式:XXXX年XX月XX日XX时-XX时)
\riqi{2020年5月29日8:00-9:40}
%授课对象(格式:XXX系XXXX年级XX班(硕/本/专科))
\duixiang{2017级基础班(本)}
%听课人数
\renshu{18}
%授课方式
\fangshi{理论讲授}
%学时数
\xueshi{2}
%教材版本
\jiaocai{生物信息学:基础及应用}


%教案首页
\firstHeader
\maketitle
\thispagestyle{empty}

\mudi{
\begin{itemize}
  \item 掌握Galaxy分析平台的基本使用方法。
  \item 熟悉Galaxy分析平台;数据处理的基本策略。
  \item 了解基因组功能注释分析平台。
  \item 自学Galaxy分析平台的高级使用技巧。
\end{itemize}
}

\fenpei{
\begin{itemize}
  \item (5')回顾与导入:回顾基因组注释的基础知识和高级注释,介绍基因组功能注释分析平台。
  \item (10')Galaxy分析平台:介绍Galaxy分析平台、Galaxy中的常用工具集和主界面、Galaxy的相关资料。
  \item (30')Galaxy的基本使用:通过坐标转换、格式转换、坐标逻辑运算等实例演示、讲解Galaxy的基本使用方法。
  \item (40')Galaxy的综合运用:通过实例演示Galaxy在生物信息学工作中的综合运用,介绍Galaxy的高级使用技巧。
  \item (5')总结与答疑:总结授课内容中的知识点与技能,解答学生疑问。
\end{itemize}
}

\zhongdian{
\begin{itemize}
  \item 重点:Galaxy分析平台的使用。
  \item 难点:Galaxy分析平台的使用。
  \item 解决策略:通过实例的逐步演示,详细讲解Galaxy的使用方法与技巧。
\end{itemize}
}

\waiyu{
\vspace*{-10pt}
\begin{multicols}{2}
工作区(work area)

历史面板(history panel)

属性(attribute)

工作流(workflow)
\end{multicols}
\vspace*{-10pt}
}

\fuzhu{
\begin{itemize}
  \item 多媒体:Galaxy分析平台的界面。
  \item 板书:数据处理的主要步骤。
  \item 操作演示:Galaxy分析平台的使用。
\end{itemize}
}

\sikao{
\begin{itemize}
  \item Galaxy分析平台的基本使用方法。
  \item 以坐标转换为例,论述“输入-加工-输出”的工作流程。
\end{itemize}
}

\cankao{
\begin{itemize}
  \item Galaxy
  \item 维基百科
\end{itemize}
}

\firstTail


%教案续页
\newpage
\otherHeader

\noindent
一、回顾与导入(5分钟)
\begin{enumerate}
  \item 基因组注释
    \begin{itemize}
      \item 基础知识:基因组组装版本、坐标系统、常用格式、坐标的逻辑运算
      \item 高级注释:变异位点的注释、基因集的富集分析、序列标识
    \end{itemize}
  \item 生物信息学分析平台:Galaxy,GenePattern,\ldots
\end{enumerate}

\vspace*{0.2cm}
\noindent
二、Galaxy分析平台(10分钟)
\begin{enumerate}
  \item 主界面\textcolor{red}{(通过讲解每部分的具体功能加深学生的理解)}
      \begin{itemize}
        \item 顶部是刊头:切换“分析数据”、“工作流”和“帐号”等主界面
        \item 左侧栏是工具菜单:以工具集的形式组织罗列着各种工具
        \item 中间是工作区:工具参数设置、使用说明和数据内容、属性等信息的输出位置
        \item 右侧栏是历史面板:以历史记录的形式记录存储着每一步操作
      \end{itemize}
    \item 工具集\textcolor{red}{(展示工具集中的具体工具,加深学生的记忆)}
      \begin{itemize}
  \parpic[fr]{\includegraphics[width=9.5cm]{galaxy.png}}
        \item Get Data:从公共数据库提取数据
        \item Text Manipulation:处理文本数据
        \item Convert Formats:数据格式转换
        \item Operate on Genomic Intervals:坐标的逻辑运算
        \item Statistics和Graph/Display Data:统计绘图
        \item NGS Toolbox:分析第二代测序数据
        \item ……
      \end{itemize}
    \item 学习资料\textcolor{red}{(先易后难,由浅入深)}
      \begin{itemize}
  \parpic[fr]{\includegraphics[width=9.5cm]{learnGalaxy.png}}
        \item Galaxy 101
        \item Galaxy Screencasts and Demos
        \item Shared Pages, Histories \& Workflows
        \item Learn Galaxy
        \item Galaxy Wiki
      \end{itemize}
\end{enumerate}

\vspace*{0.2cm}
\noindent
三、\textcolor{red}{\textbf{【重点、难点】}}Galaxy的基本使用(30分钟)
\begin{enumerate}
  \item 坐标转换:\textcolor{red}{使用集成到Galaxy中的liftOver把人类的基因坐标从hg19转换到hg18}
    \begin{itemize}
  \parpic[fr]{\includegraphics[width=8cm]{coordinateConvert.png}}
      \item 获取输入。输入文件:hg19的基因坐标
      \item 数据处理。设置参数:hg19 $\Rightarrow$ hg18
      \item 保存输出。过滤结果:MAPPED vs. UNMAPPED
      \item 坐标转换的常用工具
        \begin{itemize}
          \item \href{http://genome.ucsc.edu/cgi-bin/hgLiftOver}{liftOver}:支持BED和“chrN:start-end”格式的输入
          \item \href{https://usegalaxy.org/}{Galaxy中的liftOver}:支持BED、GFF和GTF格式的输入
          \item \href{http://www.ncbi.nlm.nih.gov/genome/tools/remap}{NCBI Remap}:支持BED、GFF、GTF和VCF等格式的输入
          \item \href{http://asia.ensembl.org/Homo\_sapiens/UserData/SelectFeatures}{Ensembl assembly converter}:支持BED、GFF、GFT和PSL格式的输入,但输出都是GFF格式的
          \item \href{https://pypi.python.org/pypi/pyliftover}{pyliftover}:仅支持点坐标(point coordinates)的转换,无法对区段(ranges)坐标进行转换
        \end{itemize}
    \end{itemize}


\otherTail
\newpage
\otherHeader


  \item 格式转换:\textcolor{red}{使用Galaxy实现BED与GFF格式的互转}
    \begin{itemize}
  \parpic[fr]{\includegraphics[width=9cm]{formatConvert.png}}
      \item 获取输入。输入文件:BED
      \item 数据处理。格式互转:BED $\Rightarrow$ GFF;GFF $\Rightarrow$ BED
      \item 保存输出。查看结果:互相比较
    \end{itemize}
  \item 逻辑运算:\textcolor{red}{使用Galaxy进行外显子与SNP的比较}
    \begin{itemize}
  \parpic[r]{\includegraphics[width=8.6cm,height=3.5cm]{intervalOperation.png}}
      \item 获取输入。输入文件:exon,SNP
      \item 数据处理。不含SNP的exon:subtract;含有SNP的exon:join
      \item 保存输出。解析结果\textcolor{red}{(一个exon上可能含有多个SNP)}
      \item 逻辑运算的常用工具
        \begin{itemize}
          \item \href{https://usegalaxy.org/}{Galaxy} 中的“Operate on Genomic Intervals”工具集
          \item \href{http://bedtools.readthedocs.org/en/latest/}{bedtools}: a powerful toolset for genome arithmetic
          \item \href{https://bedops.readthedocs.org/en/latest/}{BEDOPS}: the fast, highly scalable and easily-parallelizable genome analysis toolkit
        \end{itemize}
    \end{itemize}
\end{enumerate}

\vspace*{0.2cm}
\noindent
四、\textcolor{red}{\textbf{【重点、难点】}}Galaxy的综合运用(40分钟)

\textcolor{red}{以寻找Y染色体上含有SNP数目最多的外显子为例进行操作演示:}
\begin{enumerate}
  \item Input: Getting exons, SNPs; UCSC Table Browser
  \parpic[fr]{\includegraphics[width=9cm,height=5cm]{exonSNP.png}}
  \item Join[Operate on Genomic Intervals]: Joining exons with SNPs
  \item Group: Counting the number of SNPs per exon
  \item Filter: Filtering exons that have ten or more SNPs
  \item Compare two Datasets: Recovering exon info
  \item Visualize: Display data in genome browser
  \setcounter{enumi}{3}
\item \textcolor{gray}{Sort: Sorting exons by SNPs count}
  \setcounter{enumi}{3}
\item \textcolor{gray}{Select first: Selecting top ten}
  \setcounter{enumi}{4}
\item \textcolor{gray}{Join[Join two Datasets]: Recovering exon info}
\end{enumerate}

\textcolor{red}{扩展介绍“工作流”的思想及其优势,以及Galaxy中工作流的提取、制作、使用和分享:}
\begin{enumerate}
  \item Save: rename the history as ''Exons and SNPs''
  \item Workflow: extract workflow from history
  \item Modify: open workflow editor and modify the parameter
  \item Rerun: run workflow on whole genome data
  \item Share: share or publish workflow
  \item Create: create workflows from scratch (e.g. Find the 50 longest exons)
\end{enumerate}

\vspace*{0.2cm}
\noindent
五、总结与答疑(5分钟)
\begin{enumerate}
  \parpic[fr]{\includegraphics[width=9cm,height=4cm]{io3.png}}
  \item 知识点
    \begin{itemize}
      \item Galaxy分析平台:界面,学习,使用
    \end{itemize}
  \item 技能\textcolor{red}{(数据处理的“输入-处理-输出”三段论)}
    \begin{itemize}
      \item 获取输入:格式、来源、过滤
      \item 数据处理:工具、版本、参数
      \item 解析输出:格式、注释、解析
    \end{itemize}
\end{enumerate}


\otherTail

\end{document}

