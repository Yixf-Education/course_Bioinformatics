\documentclass{TIJMUjiaoanLL}
\pagestyle{empty}


\begin{document}


%课程名称
\kecheng{生物信息学}
%课程内容
\neirong{基因组功能注释分析}
%教师姓名
\jiaoshi{伊现富}
%职称
\zhicheng{讲师}
%教学日期(格式:XXXX年XX月XX日XX时-XX时)
\riqi{2013年9月2日14时-16时}
%授课对象(格式:XXX系XXXX年级XX班(硕/本/专科))
\duixiang{生物医学工程学院2010级生信班(本)}
%听课人数
\renshu{23}
%授课方式
\fangshi{理论讲授}
%学时数
\xueshi{2}
%教材版本
\jiaocai{生物信息学(自编教材)}


%教案首页
\firstHeader
\maketitle
\thispagestyle{empty}

\mudi{
\begin{itemize}
  \item 掌握基因组的组装版本与坐标系统,熟悉组装版本间坐标转换的工具并了解其用法。
  \item 掌握基因组注释常用格式中的FASTA和BED格式,熟悉GFF格式,了解VCF格式,了解基因组注释格式间的转换工具并自学其使用方法。
  \item 掌握基因组坐标的逻辑运算模式,熟悉坐标逻辑运算的适用范围,了解进行逻辑运算的常用工具并自学其使用方法。
\end{itemize}
}

\fenpei{
\begin{itemize}
  \item (5')回顾与导入:回顾序列分析中结构注释的内容,介绍功能注释的基础知识及主要内容。
  \item (20')基因组组装版本与坐标系统:介绍基因组的组装版本、不同数据库中基因组组装版本的命名及其对应关系,举例讲解0-based和1-based两种不同的基因组坐标系统及其各自的适用范围。
  \item (20')基因组注释常用格式:通过实例详细讲解FASTA、BED、GFF和VCF格式,介绍GFF和VCF格式的解读方法。
  \item (25')基因组坐标的逻辑运算:讲解交集、减法、补集等常用的逻辑运算模式,介绍各种逻辑运算的适用范围及实例。
  \item (20')操作演示:介绍坐标转换、格式转换、逻辑运算的常用工具,并通过实例演示各种工具的使用方法。
  \item (10')总结与答疑:回顾授课内容中的知识点,解答学生疑问。
\end{itemize}
}

\zhongdian{
\begin{itemize}
  \item 重点:基因组的两种坐标系统,基因组注释中常用的BED格式,基因组坐标的逻辑运算模式。
  \item 难点:基因组坐标中的0-based坐标系统,基因组坐标的逻辑运算模式。
  \item 解决策略:通过形象化的图示、与集合运算等的类比解释基本概念,通过实例帮助学生理解记忆。
\end{itemize}
}

\waiyu{
\vspace*{-10pt}
\begin{multicols}{2}
基因组注释(genome annotation)

功能注释(functional annotation)

基因组组装版本(genome build)

坐标系统(coordinate system)

坐标转换(coordinate transform)

单核苷酸多态性(SNP)
\end{multicols}
\vspace*{-10pt}
}

\fuzhu{
\begin{itemize}
  \item 多媒体:两种坐标系统、注释常用格式、逻辑运算模式的示意图。
  \item 板书:两种坐标系统、FASTA和BED格式、逻辑运算模式的简单示例。
  \item 演示:hg19和hg18间坐标转换、BED和GFF格式转换、减法和联合运算的操作实例。
\end{itemize}
}

\sikao{
\vspace*{-10pt}
\begin{multicols}{2}
\begin{itemize}
  \item 不同数据库间基因组组装版本的对应关系。
  \item 基因组坐标的两种表示方法。
  \item BED格式中每一列的含义。
  \item 常见的基因组坐标逻辑运算模式。
\end{itemize}
\end{multicols}
\vspace*{-10pt}
}

\cankao{
\begin{itemize}
  \item UCSC FAQ(Frequently Asked Questions)
  \item Galaxy Wiki
  \item File formats on Wikipedia
\end{itemize}
}

\firstTail


%教案续页
\newpage
\otherHeader

\noindent
一、回顾与导入(5分钟)

基因组注释:\textcolor{red}{(回顾结构注释的相关内容,引出功能注释的主要工作)}
\begin{itemize}
  \item 基因组结构注释:序列基本信息分析、寻找限制酶切位点、开放阅读框的预测、启动子和转录因子结合位点的分析、CpG 岛的识别、屏蔽重复序列、基因识别、……
  \item 基因组功能注释\textcolor{red}{(结合第二代测序数据的生物信息学处理过程逐步引出功能注释的内容)}
  \begin{itemize}
    \item 基础工作:坐标转换、格式转换、坐标的逻辑运算、……
    \item 高级注释:变异位点的注释、富集分析、序列标识、……
  \end{itemize}
\end{itemize}

\noindent
二、基因组组装版本与坐标系统(20分钟)
\begin{enumerate}
  \item 基因组组装版本
    \begin{itemize}
      \item 基因组组装版本在不断变化\textcolor{red}{(与操作系统、软件的版本变化进行类比)}
      \item 不同数据库采用不同的命名规则\textcolor{red}{(板书对应关系)}
	\begin{itemize}
	  \item NCBI:Build X
	  \item Ensembl:NCBIX
	  \item UCSC:hgX、mmX、……\textcolor{red}{(解释hg、mm等缩写的含义)}
	\end{itemize}
    \end{itemize}
  \item 基因组的坐标系统\textcolor{red}{(以数学中的坐标引出基因组的坐标,并通过实例予以讲解;类比英式和美式英语的first floor以及编程语言中的计数方式;引导学生思考两种坐标系统的优缺点)}
    \begin{itemize}
      \item 1-based(one-based, fully-closed):[start, end]
      \item 0-based(zero-based, half-closed-half-open):[start, end)
    \end{itemize}
  \item 坐标系统的适用范围
    \begin{itemize}
      \item 1-based:主要给研究人员肉眼查看的数据,如:GFF、VCF、SAM和Wiggle等格式以及DAS和UCSC的Genome Browser等工具
      \item 0-based:主要用于计算机程序处理的数据,如:BED、BAM和PSL等格式以及NCBI的dbSNP和UCSC的Table Browser等数据库与工具
    \end{itemize}
\end{enumerate}

\noindent
三、基因组注释常用格式(20分钟)

基因组数据类型的多样性导致了数据格式的多样性。\textcolor{red}{通过实例详细介绍FASTA、BED、GFF和VCF四种常用格式,讲解格式中每一列的含义,引导学生学会解析特定格式中的信息。}
\begin{enumerate}
  \item FASTA格式
    \begin{itemize}
      \item 首行:起始标识符“>”,ID,描述信息;其余行:具体的序列。
      \item IUB/IUPAC核酸代码:“N”代表任意一种核酸,“-”代表空位。
      \item IUB/IUPAC氨基酸代码:“X”代表任意一种氨基酸,“*”代表翻译终止,“-” 代表空位。
    \end{itemize}
  \item BED格式:3+9=12列(BED12)。\textcolor{red}{引申出简化版的BED3,BED4,BED5和BED6。}
    \begin{itemize}
      \item BED3:chrom, chromStart, chromEnd
      \item BED4:chrom, chromStart, chromEnd, name
      \item BED5:chrom, chromStart, chromEnd, name, score
      \item BED6:chrom, chromStart, chromEnd, name, score, strand
    \end{itemize}
  \item GFF格式
    \begin{itemize}
      \item 注释信息:以“\#\#”开头
      \item 特征信息:9列
    \end{itemize}
  \item VCF格式
    \begin{itemize}
      \item 元信息:以“\#\#”起始
      \item 标题行:以“\#”起始
      \item 数据行:8+1+N列
    \end{itemize}
\end{enumerate}

强调它们都以纯文本形式进行存储,简单介绍常用的文本编辑器(Notepad++,Vim,Emacs)。

\otherTail
\newpage
\otherHeader

\noindent
四、基因组坐标的逻辑运算(25分钟)
\begin{enumerate}
  \item 集合运算与逻辑运算
    \begin{itemize}
      \item \textcolor{red}{类比数学中的集合运算}
      \item 以交集为例,板书由集合运算转换到逻辑运算的过程
    \end{itemize}
  \item 基因组坐标逻辑运算\textcolor{red}{(先用示意图予以讲解,再辅以实例帮助记忆,最终理解其适用情况)}
    \begin{itemize}
      \item intersect,交集:保留重叠的坐标
      \item subtract,减法:去除重叠的坐标
      \item merge,合并:合并重叠的坐标
      \item concatenate,串联:合并多组坐标
      \item complement,补集:取坐标的补集
      \item cluster,聚类:聚合符合要求的坐标
      \item join,联合:根据坐标重叠把两组记录对应起来
    \end{itemize}
\end{enumerate}

\noindent
五、操作演示(20分钟)
\begin{enumerate}
  \item 坐标转换
    \begin{itemize}
      \item 工具:liftOver
      \item 实例:把人类的基因坐标从hg19转换到hg18
    \end{itemize}
  \item 格式转换
    \begin{itemize}
      \item 工具:Galaxy
      \item 实例:BED与GFF格式的互转
    \end{itemize}
  \item 逻辑运算
    \begin{itemize}
      \item 工具:Galaxy,BEDTools
      \item 实例:外显子与SNP的比较
    \end{itemize}
  \item 处理问题的基本步骤\textcolor{red}{(将“三步走”的思想贯穿在每个操作实例中)}
    \begin{itemize}
      \item 获取输入:数据来源,文件格式,……
      \item 数据处理:程序选择,参数调整,……
      \item 解析输出:文件格式,数据校验,……
    \end{itemize}
\end{enumerate}

\noindent
六、总结与答疑(10分钟)
\begin{enumerate}
  \item 知识点
    \begin{itemize}
      \item 基因组组装版本:命名规则,对应关系
      \item 两种基因组坐标系统:1-based,0-based
      \item 四种注释常用格式:FASTA,BED,GFF,VCF
      \item 逻辑运算模式:intersect,subtract,merge,concatenate,complement,cluster,join
      \item 坐标转换、格式转换、逻辑运算的工具:liftOver,Galaxy,BEDTools
    \end{itemize}
  \item 技能
    \begin{itemize}
      \item “输入 $\Rightarrow$ 加工 $\Rightarrow$ 输出”三步走
      \item 获取输入:数据来源,文件格式,……
      \item 数据处理:程序选择,参数调整,……
      \item 解析输出:文件格式,数据校验,……
    \end{itemize}
\end{enumerate}


\otherTail


\end{document}

