\documentclass{TIJMUjiaoanSY}
\pagestyle{empty}


\begin{document}


%课程名称
\kecheng{生物信息学}
%实验名称
\shiyan{DNA序列的基本信息及特征分析}
%教师姓名
\jiaoshi{伊现富}
%职称
\zhicheng{讲师}
%教学日期(格式:XXXX年XX月XX日XX时-XX时)
\riqi{2013年9月4日14时-16时}
%授课对象(格式:XXX系XXXX年级XX班(硕/本/专科))
\duixiang{生物医学工程学院2010级生信班(本)}
%实验人数
\renshu{23}
%实验类型
\leixing{验证型}
%实验分组
\fenzu{一人一机}
%学时数
\xueshi{2}
%教材版本
\jiaocai{生物信息学实验讲义(自编教材)}


%教案首页
\firstHeader
\maketitle
\thispagestyle{empty}

\mudi{
\begin{itemize}
  \item 掌握使用NCBI查询核酸序列的方法。
  \item 掌握EMBOSS的基本使用方法。
  \item 掌握ORF的性质及其分析方法。
\end{itemize}
}

\fenpei{
\begin{itemize}
  \item (5')查询核酸序列:简单介绍NCBI数据库,讲解Nucleotide数据库的使用。
  \item (5')EMBOSS简介:介绍EMBOSS及其参考资料,讲解compseq等工具的使用方法。
  \item (5')序列组分分析:回顾序列组分分析的主要内容。
  \item (5')开放阅读框分析:回顾ORF的定义、相位的概念和最长ORF法。
  \item (80')实验操作:对人类CD9基因序列进行组分分析,对大肠杆菌基因组序列进行ORF分析。
\end{itemize}
}

\cailiao{
\begin{itemize}
  \item 实验材料:人类CD9基因,大肠杆菌基因组。
  \item 主要仪器:联网的计算机。
  \item 分析工具:NCBI,EMBOSS。
\end{itemize}
}

\zhongdian{
\begin{itemize}
  \item 重点难点:NCBI和EMBOSS的使用。
  \item 解决策略:通过演示进行学习,通过练习熟练掌握。
\end{itemize}
}

\sikao{
\begin{itemize}
  \item 如何使用NCBI查询获取核酸序列?
  \item EMBOSS中进行序列组分分析的程序有哪些?
  \item getorf和ORF Finder的分析结果有何异同?
\end{itemize}
}

\cankao{
\begin{itemize}
  \item NCBI
  \item EMBOSS
\end{itemize}
}

\firstTail


%教案续页
\newpage
\otherHeader

\noindent
一、查询核酸序列(5分钟)
\begin{enumerate}
  \item NCBI:Nucleotide,Gene\textcolor{red}{(多种数据库)}
  \item Nucleotide:查询\textcolor{red}{(ID,基因名)},下载\textcolor{red}{(选择合适的格式)}
\end{enumerate}

\noindent
二、EMBOSS简介(5分钟)
\begin{enumerate}
  \item EMBOSS:EMBOSS Explorer,Jemboss\textcolor{red}{(开源、免费)}
  \item EMBOSS Explorer:NUCLEIC COMPOSITION,NUCLEIC GENE FINDING\textcolor{red}{(模块化,功能强大)}
\end{enumerate}

\noindent
三、序列组分分析(5分钟)
\begin{enumerate}
  \item 碱基组成分析:长度,碱基数目及其比例,GC含量
  \item 序列转换:反向序列,互补序列,反向互补序列
\end{enumerate}

\noindent
四、开放阅读框分析(5分钟)
\begin{enumerate}
  \item ORF:在给定的阅读框架中不包含终止密码子的一串序列
  \item 相位:六相位(+1, +2, +3, -1, -2, -3)
  \item 预测方法:最长ORF法\textcolor{red}{(适用于原核生物)}
\end{enumerate}

\noindent
五、实验操作(80分钟)
\begin{enumerate}
  \item 人类CD9基因的序列组份分析
    \begin{itemize}
      \item 获取序列:NCBI的Nucleotide数据库,AY422198,FASTA格式
      \item 打开EMBOSS:EMBOSS Explorer
      \item 碱基组分分析:compseq\textcolor{red}{(注意修改默认参数)}
      \item 计算GC含量:geecee
      \item 序列转换:revseq\textcolor{red}{(调整参数即可分别获得反向序列、互补序列或反向互补序列)}
    \end{itemize}
  \item 大肠杆菌基因组序列的ORF分析
    \begin{itemize}
      \item 获取序列:NCBI的Nucleotide数据库,U00096,FASTA格式
      \item 截取序列:extractseq,1-3000bp\textcolor{red}{(仅使用部分序列进行练习)}
      \item ORF预测:getorf\textcolor{red}{(注意选择合适的参数)}
      \item 结果分析:和ORF Finder的结果进行比较
    \end{itemize}
\end{enumerate}

\otherTail


\end{document}

