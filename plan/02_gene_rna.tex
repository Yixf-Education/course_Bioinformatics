\documentclass{TIJMUjiaoanLL}
\pagestyle{empty}


\begin{document}


%课程名称
\kecheng{生物信息学}
%课程内容
\neirong{基因识别与RNA序列分析}
%教师姓名
\jiaoshi{伊现富}
%职称
\zhicheng{讲师}
%教学日期(格式:XXXX年XX月XX日XX时-XX时)
\riqi{2013年8月30日8时-10时}
%授课对象(格式:XXX系XXXX年级XX班(硕/本/专科))
\duixiang{生物医学工程学院2010级生信班(本)}
%听课人数
\renshu{23}
%授课方式
\fangshi{理论讲授}
%学时数
\xueshi{2}
%教材版本
\jiaocai{生物信息学(自编教材)}


%教案首页
\firstHeader
\maketitle
\thispagestyle{empty}

\mudi{
\begin{itemize}
  \item 熟悉原核基因和真核基因的结构特点,掌握基因识别的方法和工具并自学其使用方法。
  \item 掌握mRNA选择性剪接的主要机制,了解可变剪接的数据库与分析工具并自学其使用方法。
  \item 熟悉miRNA的特点、生成过程和作用方式,掌握miRNA预测和miRNA靶基因预测的方法,熟悉miRNA的相关数据库和分析工具并自学其使用方法。
  \item 了解lncRNA的定义及其主要特点,自学lncRNA的相关数据库,以及lncRNA在疾病发生发展过程中的作用。
\end{itemize}
}

\fenpei{
\begin{itemize}
  \item (5')回顾与导入:回顾序列基本信息和特征信息分析的主要内容,引出基因识别与RNA分析的内容。
  \item (25')基因识别:介绍基因和基因识别的基本概念,回顾原核基因和真核基因的结构特点并进行比较,讲解基因识别的主要方法,介绍基因识别的常用工具。
  \item (25')mRNA选择性剪接:介绍剪接和选择性剪接的基本概念,讲解mRNA选择性剪接的主要机制,介绍相关的数据库和分析工具。
  \item
    (25')miRNA及其靶基因预测:回顾miRNA的特点、生成过程、作用方式和生物学功能,讲解miRNA预测和miRNA靶基因预测的主要方法,介绍常用的数据库和分析工具。
  \item (10')lncRNA:介绍lncRNA的定义、主要特点及其研究进展,总结查找数据库和分析工具的主要策略。
  \item (10')总结与答疑:回顾授课内容中的知识点,解答学生疑问。
\end{itemize}
}

\zhongdian{
\begin{itemize}
  \item 重点:原核基因和真核基因的结构特点,mRNA选择性剪接的主要机制;解决策略:通过示意图和实例帮助学生理解、记忆。
  \item 难点:查找数据库和分析工具的主要策略;解决策略:以miRNA和lncRNA为例进行分析与讲解。
\end{itemize}
}

\waiyu{
\vspace*{-10pt}
\begin{multicols}{2}
选择性剪接(alternative splicing)

微RNA(miRNA,microRNA)

非编码RNA(ncRNA,non-coding RNA)

长链非编码RNA(lncRNA)
\end{multicols}
\vspace*{-10pt}
}

\fuzhu{
\begin{itemize}
  \item 多媒体:原核基因和真核基因的结构,mRNA选择性剪接的机制,miRNA的结构、生成过程和作用方式等。
  \item 板书:课程结构,查找数据库和分析工具的主要策略。
\end{itemize}
}

\sikao{
\vspace*{-10pt}
\begin{multicols}{2}
  \begin{itemize}
    \item 原核基因和真核基因结构的异同。
    \item 基因识别的主要方法。
    \item mRNA选择性剪接的主要机制。
    \item miRNA的特点、生成过程和作用方式。
    \item miRNA预测和miRNA靶基因预测的方法。
    \item 查找数据库和分析工具的主要策略。
  \end{itemize}
\end{multicols}
\vspace*{-10pt}
}

\cankao{
\begin{itemize}
  \item 朱玉贤,李毅,郑晓峰。现代分子生物学(第3版),高等教育出版社,2007。
  \item 维基百科。
\end{itemize}
}

\firstTail


%教案续页
\newpage
\otherHeader

\noindent
一、回顾与导入(5分钟)
\begin{enumerate}
  \item 序列分析\textcolor{red}{(上期回顾)}
    \begin{itemize}
      \item 基本信息:碱基比例、GC含量、序列转换、限制性核酸内切酶位点、……
      \item 特征信息:开放阅读框、启动子、转录因子结合位点、CpG岛、……
    \end{itemize}
  \item 基因识别、mRNA分析、miRNA分析、……\textcolor{red}{(本期导读)}
\end{enumerate}

\noindent
二、基因识别(25分钟)

\textcolor{red}{在介绍基本概念的基础上,通过比较原核基因和真核基因的异同,讲解基因识别方法在原核和真核基因识别中的具体应用。}
\begin{enumerate}
  \item 基本概念
    \begin{itemize}
      \item 基因:产生一条多肽链或功能RNA所需的全部核苷酸序列
      \item 基因识别:识别DNA序列上具有生物学特征的片段
    \end{itemize}
  \item 基因结构\textcolor{red}{(通过示意图形象化展示不同的基因结构)}
    \begin{itemize}
      \item 原核基因:连续基因
      \item 真核基因:不连续性
    \end{itemize}
  \item 基因识别的方法
    \begin{itemize}
      \item 间接识别法:mRNA/蛋白质序列 $\Rightarrow$ DNA序列
      \item 从头预测法:基因预测,基于“信号”和“内容”两类特征
      \item 比较基因组学的方法:比较相关物种的DNA序列
    \end{itemize}
  \item 基因预测
    \begin{itemize}
      \item 原核基因 
	\begin{itemize}
	  \item 信号:启动子序列
	  \item 内容:连续的开放阅读框
	\end{itemize}
      \item 真核基因
	\begin{itemize}
	  \item 信号:启动子区特征序列,供体位点和受体位点,终止密码子,polyA序列,……
	  \item 内容:密码子使用偏好性,双联密码子出现频率,基因组等值区,……
	\end{itemize}
    \end{itemize}
  \item 识别工具\textcolor{red}{(提醒:分析工具的适用范围)}
    \begin{itemize}
      \item 识别原核基因:GeneMarkS, Glimmer
      \item 识别真核基因:GENSCAN
    \end{itemize}
\end{enumerate}

\noindent
三、mRNA选择性剪接(25分钟)
\begin{enumerate}
  \item 基本概念
    \begin{itemize}
      \item 剪接:移除内含子、合并外显子
      \item 选择性剪接:一个mRNA前体 $\Rightarrow$ 不同mRNA剪接异构体
    \end{itemize}
  \item 主要机制\textcolor{red}{(展示示意图并给出实例)}
    \begin{itemize}
      \item 外显子跳跃:外显子被移除或保留,最常见
      \item 互斥外显子:两个外显子只有一个保留下来,相对较少见
      \item 5'选择性剪接:使用不同的5'端的供体位点
      \item 3'选择性剪接:使用不同的3'端的受体位点
      \item 内含子保留:内含子作为外显子保留下来,最少见
      \item 选择性起始:在不同的位点起始转录
      \item 选择性终止:使用不同的polyA位点
    \end{itemize}
  \item 相关资源\textcolor{red}{(提醒:数据库具有时效性)}
    \begin{itemize}
      \item 数据库:ASTD,ASAP,ASPicDB
      \item 分析工具:ESEfinder,RESCUE-ESE
    \end{itemize}
\end{enumerate}

\otherTail
\newpage
\otherHeader

\noindent
四、miRNA及其靶基因预测(25分钟)
\begin{enumerate}
  \item miRNA\textcolor{red}{(通过示意图将相关内容形象化展示出来)}
    \begin{itemize}
      \item 特点:20~24nt,单链,没有开放阅读框,不编码蛋白质,具有时序性和组织特异性,进化上高度保守
      \item 生成过程:300~1000nt的双链pri-miRNA $\Rightarrow$ 70~90nt的双链具有茎环结构的pre-miRNA $\Rightarrow$ 20~24nt的单链成熟miRNA
      \item 作用方式:完全互补型(导致靶基因mRNA降解)和不完全互补型(导致靶基因mRNA的翻译受到抑制)
      \item 生物学功能:调控个体发育、细胞分化、组织发育、肿瘤发生发展、……
    \end{itemize}
  \item miRNA预测
    \begin{itemize}
      \item 同源片段搜索
      \item 基于比较基因组学
      \item 基于序列和结构特征打分
      \item 结合作用靶标
      \item 基于机器学习
    \end{itemize}
  \item miRNA靶基因预测
    \begin{itemize}
      \item 基于种子区域互补和保守性
      \item 基于机器学习
    \end{itemize}
  \item 相关资源
    \begin{itemize}
      \item 数据库:miRBase,TarBase,miRGen
      \item 分析工具:MiRscan,MiPred,miRFinder;miRanda,TargetScan,PicTar,miTarget
    \end{itemize}
\end{enumerate}

\noindent
五、lncRNA(10分钟)
\begin{enumerate}
  \item RNA的分类
    \begin{itemize}
      \item mRNA
      \item 非编码RNA
	\begin{itemize}
	  \item 基础结构性ncRNA:tRNA,rRNA,snRNA,snoRNA
	  \item 调节性ncRNA
	    \begin{itemize}
	      \item sRNA:$\textless$ 200nt,miRNA、siRNA、piRNA
	      \item lncRNA:$\textgreater$ 200nt
	    \end{itemize}
	\end{itemize}
    \end{itemize}
  \item lncRNA的特点
    \begin{itemize}
      \item RNA聚合酶II所转录
      \item 有5'帽子和3'端的polyA尾巴
      \item 主要富集在细胞核
      \item 长度偏短、外显子数目偏少
      \item 保守性差,稳定性低
      \item 表达水平很低,表达具有细胞、组织、发育、疾病等时空特异性
    \end{itemize}
  \item lncRNA的研究进展
    \begin{itemize}
      \item 作用方式:表观遗传学水平、转录水平和转录后水平
      \item 生物学功能:基因转录、剪接、翻译、修饰和印迹等
      \item 与疾病的关系:肿瘤、阿尔兹海默病、心血管疾病等
    \end{itemize}
  \item 查找数据库和分析工具的策略\textcolor{red}{(以lncRNA和miRNA为例)}
    \begin{itemize}
      \item 专辑资料:NAR,相关文献,维基百科
      \item 请教搜索:领域专家,搜索引擎
    \end{itemize}
\end{enumerate}

\otherTail
\newpage
\otherHeader

\noindent
六、总结与答疑(10分钟)
\begin{enumerate}
  \item 知识点
    \begin{itemize}
      \item 基因识别:原核和真核的基因结构,基因识别的方法
      \item mRNA选择性剪接:主要机制
      \item miRNA及其靶基因预测:特点、生成过程和作用方式,miRNA及其靶基因预测方法
      \item lncRNA:特点
    \end{itemize}
  \item 技能
    \begin{itemize}
      \item 查找数据库:注意时效性
      \item 查找分析工具:注意适用范围
    \end{itemize}
\end{enumerate}


\otherTail

\end{document}

