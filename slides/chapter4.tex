\documentclass[table]{beamer}
%\documentclass[table]{beamer}
%[]中可以使用draft、handout、screen、transparency、trancompress、compress等参数

%指定beamer的模式与主题
\mode<presentation>
{
  \usetheme{Madrid}
%\usetheme{Boadilla}
%\usecolortheme{default}
%\usecolortheme{orchid}
%\usecolortheme{whale}
%\usefonttheme{professionalfonts}
}

%\usetheme{Madrid}
%这里还可以选择别的主题:Bergen, Boadilla, Madrid, AnnArbor, CambridgeUS, Pittsburgh, Rochester, Warsaw, ...
%有导航栏的Antibes, JuanLesPins, Montpellier, ...
%有内容的Berkeley, PaloAlto, Goettingen, Marburg, Hannover, ...
%有最小导航栏的Berlin, Ilmenau, Dresden, Darmstadt, Frankfurt, Singapore, Szeged, ...
%有章和节表单的Copenhagen, Luebeck, Malmoe, Warsaw, ...

%\usecolortheme{default}
%设置内部颜色主题(这些主题一般改变block里的颜色);这个主题一般选择动物来命名
%这里还可以选择别的颜色主题,如默认的和有特别目的的颜色主题default,structure,sidebartab,全颜色主题albatross,beetle,crane,dove,fly,seagull,wolverine,beaver

%\usecolortheme{orchid}
%设置外部颜色主题(这些主题一般改变title里的颜色);这个主题一般选择植物来命名
%这里还可以选择别的颜色主题,如默认的和有特别目的的颜色主题lily,orchid,rose

%\usecolortheme{whale}
%设置字体主题;这个主题一般选择海洋动物来命名
%这里还可以选择别的颜色主题,如默认的和有特别目的的颜色主题whale,seahorse,dolphin

%\usefonttheme{professionalfonts}
%类似的还可以定义structurebold,structuresmallcapsserif,professionalfonts


% 控制 beamer 的风格,可以根据自己的爱好修改
%\usepackage{beamerthemesplit} %使用 split 风格
%\usepackage{beamerthemeshadow} %使用 shadow 风格
%\usepackage[width=2cm,dark,tab]{beamerthemesidebar}


% 设定英文字体
%\usepackage{fontspec}
\usepackage[no-math]{fontspec}
\setmainfont{Times New Roman}
\setsansfont{Arial}
\setmonofont{Courier New}

% 设定中文字体
\usepackage[BoldFont,SlantFont,CJKchecksingle,CJKnumber]{xeCJK}
%\setCJKmainfont[BoldFont={Adobe Heiti Std},ItalicFont={Adobe Kaiti Std}]{Adobe Song Std}
\setCJKmainfont[BoldFont={Adobe Heiti Std},ItalicFont={Adobe Kaiti Std}]{WenQuanYi Micro Hei}
\setCJKsansfont{Adobe Heiti Std}
\setCJKmonofont{Adobe Fangsong Std}
\punctstyle{hangmobanjiao}

\defaultfontfeatures{Mapping=tex-text}
\usepackage{xunicode}
\usepackage{xltxtra}

\XeTeXlinebreaklocale "zh"
\XeTeXlinebreakskip = 0pt plus 1pt minus 0.1pt

\usepackage{setspace}
\usepackage{colortbl,xcolor}
\usepackage{hyperref}
%\hypersetup{xetex,bookmarksnumbered=true,bookmarksopen=true,pdfborder=1,breaklinks,colorlinks,linkcolor=blue,filecolor=black,urlcolor=cyan,citecolor=green}
\hypersetup{xetex,bookmarksnumbered=true,bookmarksopen=true,pdfborder=1,breaklinks,colorlinks,linkcolor=cyan,filecolor=black,urlcolor=blue,citecolor=green}

% 插入图片
\usepackage{graphicx}
\graphicspath{{figures/}}

% 可能用到的包
\usepackage{amsmath,amssymb}
\usepackage{multimedia}
\usepackage{multicol}
\usepackage{multirow}

% 定义一些自选的模板,包括背景、图标、导航条和页脚等,修改要慎重
% 设置背景渐变由10%的红变成10%的结构颜色
%\beamertemplateshadingbackground{red!10}{structure!10}
%\beamertemplatesolidbackgroundcolor{white!90!blue}
% 使所有隐藏的文本完全透明、动态,而且动态的范围很小
\beamertemplatetransparentcovereddynamic
% 使itemize环境中变成小球,这是一种视觉效果
\beamertemplateballitem
% 为所有已编号的部分设置一个章节目录,并且编号显示成小球
\beamertemplatenumberedballsectiontoc
% 将每一页的要素的要素名设成加粗字体
\beamertemplateboldpartpage

% item逐步显示时,使已经出现的item、正在显示的item、将要出现的item呈现不同颜色
\def\hilite<#1>{
 \temporal<#1>{\color{gray}}{\color{blue}}
    {\color{blue!25}}
}

\renewcommand{\today}{\number\year 年 \number\month 月 \number\day 日}

%五角星
\usepackage{MnSymbol}

%去除图表标题中的figure等
\usepackage{caption}
\captionsetup{labelformat=empty,labelsep=none}

\usepackage{tabu}
\usepackage{multirow}

% 千分号
%\usepackage{textcomp}

%罗马数字
\makeatletter
\newcommand{\rmnum}[1]{\romannumeral #1}
\newcommand{\Rmnum}[1]{\expandafter\@slowromancap\romannumeral #1@}
\makeatother

%分栏
\usepackage{multicol}

%\usepackage{enumitem}
\usepackage{enumerate}


%\setbeamercolor{alerted text}{fg=magenta}

\setbeamercolor{bgcolor}{fg=yellow,bg=cyan}

\begin{document}

%\includeonlyframes{current}

\logo{\includegraphics[height=0.08\textwidth]{tijmu.png}}
\title[核酸序列分析]{核酸序列分析}
\author[Yixf]{伊现富(Yi Xianfu)}
\institute[TIJMU]{天津医科大学(TIJMU)\\ 生物医学工程学院}
\date{2014年3月}


% 在每个Section前都会加入的Frame
\AtBeginSection[]
{
  \begin{frame}<beamer>
    %\frametitle{Outline}
    \frametitle{教学提纲}
    \setcounter{tocdepth}{2}
    \begin{multicols}{2}
      %\tableofcontents[currentsection,currentsubsection]
      \tableofcontents[currentsection]
    \end{multicols}
  \end{frame}
}
% 在每个Subsection前都会加入的Frame
%\AtBeginSubsection[]
%{
  %\begin{frame}<beamer>
%%\begin{frame}<handout:0>
%% handout:0 表示只在手稿中出现
    %\frametitle{Outline}
    %\setcounter{tocdepth}{2}
    %\tableofcontents[currentsection,currentsubsection]
%% 显示在目录中加亮的当前章节
  %\end{frame}
%}

\begin{frame}[plain]
  \begin{center}
    {\Huge 生物信息学\\}
    \vspace{1cm}
    {\LARGE 天津医科大学\\}
    %\vspace{0.2cm}
    {\LARGE 生物医学工程学院\\}
    \vspace{1cm}
    {\large 2013-2014学年下学期(春)\\ 2011级生信班}
  \end{center}
\end{frame}

\begin{frame}
  \titlepage
\end{frame}

\begin{frame}[plain]
  \frametitle{教学提纲}
  \setcounter{tocdepth}{2}
  \begin{multicols}{2}
  \tableofcontents
  \end{multicols}
\end{frame}

\section{引言}
\begin{frame}
  \frametitle{引言 | 大千世界}
  \begin{figure}
    \centering
    \includegraphics[width=6cm]{world.jpg}
  \end{figure}
\end{frame}

\begin{frame}
  \frametitle{引言 | 中心法则}
  \begin{figure}
    \centering
    \includegraphics[width=11cm]{dogma.jpg}
  \end{figure}
\end{frame}

\begin{frame}
  \frametitle{引言 | ACGT}
  \begin{figure}
    \centering
    \includegraphics[width=10cm]{acgt.jpg}
  \end{figure}
\end{frame}

\begin{frame}
  \frametitle{引言 | ACGT$\Rightarrow$生信}
  \begin{figure}
    \centering
    \includegraphics[width=10cm]{matrix.png}
  \end{figure}
\end{frame}

\begin{frame}
  \frametitle{引言 | 遗传信息}
  \begin{figure}
    \centering
    \includegraphics[width=11cm]{info.jpg}
  \end{figure}
\end{frame}

\section{DNA组份分析与序列转换}
\begin{frame}
  \frametitle{DNA序列 | 查戈夫法则}
  \begin{block}{查戈夫法则}
    \begin{description}
      \item[第一法则]$A=T, G=C \Longrightarrow A+C=T+G, A+G=C+T$
      \item[第二法则]AT/GC 的比值因生物种类不同而异
    \end{description}
  \end{block}
\end{frame}

\begin{frame}
  \frametitle{DNA序列 | GC含量}
  \begin{block}{GC含量(GC content)}
    \begin{itemize}
      \item 鸟嘌呤(G)和胞嘧啶(C)所占的比例
      \item GC含量随DNA不同而异
      \item GC含量高的DNA更加稳定
      \item 计算公式:$\frac{G+C}{A+T+G+C}\times100$
      \item GC比(GC-ratio):$\frac{A+T}{G+C}$
    \end{itemize}
  \end{block}
\end{frame}

\begin{frame}
  \frametitle{DNA序列 | GC含量 | 基因组}
  \begin{figure}
    \centering
    \includegraphics[width=9.5cm]{genomeGC.jpg}
  \end{figure}
  %{\tiny Genome evolution in bacterial endosymbionts of insects. Jennifer J. Wernegreen. Nature Reviews Genetics 3, 850-861 (November 2002). doi:10.1038/nrg931}
\end{frame}

\begin{frame}
  \frametitle{DNA序列 | GC含量 | 基因区}
  \begin{figure}
    \centering
    \includegraphics[width=12cm]{exonGC.png}
  \end{figure}
\end{frame}

\begin{frame}
  \frametitle{DNA序列 | GC含量 | 基因 VS. 基因组}
  \begin{figure}
    \centering
    \includegraphics[width=11cm]{geneGC.jpg}
  \end{figure}
\end{frame}

\begin{frame}
  \frametitle{DNA序列 | 实例与策略}
  \pause
  \begin{block}{任务分析}
    \begin{itemize}
      \item 序列长短
      \item 序列数目
      \item 任务数量
      \item 任务频率
      \item 工作时间
      \item \ldots
    \end{itemize}
  \end{block}
\end{frame}

\begin{frame}
  \frametitle{DNA序列 | 序列转换}
  \begin{block}{序列转换}
    \begin{itemize}
      \item 反向序列
      \item 互补序列
      \item 反向互补序列
      \item DNA双链
      \item RNA序列
    \end{itemize}
  \end{block}
  \pause
  \begin{block}{书写惯例}
    \begin{itemize}
      \item DNA/RNA:[左] 5' $\Longrightarrow$ 3' [右]
      \item 多肽/蛋白质:[左] N端(氨基端)$\Longrightarrow$ C端(羧基端) [右]
    \end{itemize}
  \end{block}
\end{frame}

\begin{frame}
  \frametitle{DNA序列 | 分析工具}
  \begin{itemize}
    \item \href{http://yixf.name/2011/06/01/\%E5\%AF\%B9fasta\%E6\%A0\%BC\%E5\%BC\%8F\%E7\%9A\%84\%E7\%AE\%80\%E5\%8D\%95\%E5\%A4\%84\%E7\%90\%86\%E4\%B8\%8E\%E7\%BB\%9F\%E8\%AE\%A1/}{SeqTools.pl}
    \item EMBOSS
    \item \href{http://bioinfx.net/}{bioinfx(Free Online Tools for Bioinformatics)}
    \item \href{http://clasher.myweb.uga.edu/testpages/seqconv.html}{Complementary Sequence Conversion Tool}
    \item \href{http://www.cellbiol.com/scripts/complement/dna\_sequence\_reverse\_complement.php}{DNA Sequence Reverse and Complement Online Tool}
    \item \href{http://www.endmemo.com/bio/gc.php}{DNA/RNA GC Content Calculator}
    \item \href{http://www.sciencelauncher.com/oligocalc.html}{Oligo Calculator}
    \item \ldots
  \end{itemize}
\end{frame}

\section{限制酶位点分析}
\begin{frame}
  \frametitle{限制酶 | 定义}
  \begin{block}{限制酶(restriction enzyme)}
    又称限制内切酶或限制性内切酶(restriction endonuclease),全称限制性核酸内切酶,是可以识别DNA的特异序列、并在识别位点或其周围切割双链DNA的一类内切酶。
  \end{block}
  \pause
  \begin{block}{切割末端}
    \begin{itemize}
      \item 黏性末端
      \item 平滑末端
    \end{itemize}
  \end{block}
\end{frame}

\begin{frame}
  \frametitle{限制酶 | 命名}
  %\begin{block}{\textit{Eco}R\Rmnum{1}}
    %\begin{description}
      %\item[\textit{E}]属名\textit{Escherichia}
      %\item[\textit{co}]种名\textit{coli}
      %\item[R]RY13品系
      %\item[\Rmnum{1}]在此类细菌中的发现顺序
    %\end{description}
  %\end{block}
  \begin{figure}
    \centering
    \includegraphics[width=10cm]{ecori.png}
  \end{figure}
\end{frame}

\begin{frame}
  \frametitle{限制酶 | \Rmnum{2}型}
  \begin{itemize}
    \item 识别、切割位点专一
    \item 识别序列:4-8个碱基,回文对称结构
    \item 切割序列:识别序列,切割位点对称
    \item 切割末端:黏性末端,平滑末端
    \item 黏性末端:切割位点在回文序列的一侧
    \item 平滑末端:切割位点在回文序列的中间
  \end{itemize}
\end{frame}

\begin{frame}
  \frametitle{限制酶 | \Rmnum{2}型 | 回文}
  \begin{block}{《题金山寺》,北宋\textbullet 苏轼}
  \begin{columns}
    \column{0.5\textwidth}
潮随暗浪雪山倾,远浦渔舟钓月明。
桥对寺门松径小,槛当泉眼石波清。
迢迢绿树江天晓,霭霭红霞晚日晴。
遥望四边云接水,雪峰千点数鸥轻。
    \column{0.5\textwidth}
    \pause
轻鸥数点千峰雪,水接云边四望遥。
晴日晚霞红霭霭,晓天江树绿迢迢。
清波石眼泉当槛,小径松门寺对桥。
明月钓舟渔浦远,倾山雪浪暗随潮。
  \end{columns}
  \end{block}
  \pause
  \begin{block}{回文对称(palindrome)}
    特指DNA的一种具有反向重复的结构。具有这种结构的DNA,其一条链从左向右读和另一条链从右向左读的序列是相同的。
  \end{block}
\end{frame}

\begin{frame}
  \frametitle{限制酶 | \Rmnum{2}型 | 黏性末端}
  \begin{figure}
    \centering
    \includegraphics[width=9cm]{enzyme1.png}
  \end{figure}
\end{frame}

\begin{frame}
  \frametitle{限制酶 | \Rmnum{2}型 | 平滑末端}
  \begin{figure}
    \centering
    \includegraphics[width=9cm]{enzyme2.png}
  \end{figure}
\end{frame}

\begin{frame}
  \frametitle{限制酶 | 数据库与分析工具}
  \begin{itemize}
    \item REBASE:收录了限制酶的所有信息
    \item NEBCutter V2.0:产生DNA序列的酶切位点分析结果
  \end{itemize}
\end{frame}

\section{开放阅读框分析}
\begin{frame}
  \frametitle{开放阅读框}
  \begin{block}{开放阅读框(Open Reading Frame,ORF)}
    在给定的阅读框架中,不包含终止密码子的一串序列,是生物个体的基因组中可能作为蛋白质编码序列的部分,包含从5'端翻译起始密码子(AUG)到终止密码子(UAA、UAG、UGA)之间的一段编码蛋白质的碱基序列。
  \end{block}
\end{frame}

\begin{frame}
  \frametitle{开放阅读框 | 相位}
  \begin{figure}
    \centering
    \includegraphics[width=10cm]{orf.png}
  \end{figure}
\end{frame}

\begin{frame}
  \frametitle{开放阅读框 | ORF VS. CDS}
  \pause
  \begin{itemize}
    \item 一个ORF对应一个候选的CDS(编码序列,Coding DNA Sequence)
    \item ORF:理论预测
    \item CDS:实验证实
    \item 分析DNA序列中的ORF是对该序列是否为CDS的初步判断
  \end{itemize}
\end{frame}

\begin{frame}
  \frametitle{开放阅读框 | 分析工具}
  \begin{itemize}
    \item 确定第一个AUG和终止密码子
    \item 原核生物:最长ORF法
    \item 真核生物:特征统计、模式识别、同源比对
    \item ORF Finder:NCBI的在线分析工具
  \end{itemize}
\end{frame}

\section{启动子分析}
\begin{frame}
  \frametitle{启动子 | 转录调控}
  \begin{itemize}
    \item 顺式作用元件(cis-acting element):核酸序列
      \begin{itemize}
        \item 启动子(promoter)
        \item 增强子(enhancer)
        \item \ldots
      \end{itemize}
    \item 反式作用因子(trans-acting factor):蛋白质
    \item 两者相互作用实现转录调控
  \end{itemize}
\end{frame}

\begin{frame}
  \frametitle{启动子 | 定义}
  \begin{block}{启动子(promoter)}
    一段位于转录起始位点5'端上游区的DNA序列,能活化RNA聚合酶,使之与模板DNA准确地结合并具有转录起始的特异性。
  \end{block}
  \pause
  \begin{block}{转录起始位点(Transcription Start Site,TSS)}
    与新生RNA链第一个核苷酸相对应DNA链上的碱基,研究证实通常为一个嘌呤。
  \end{block}
  \visible<3->{
  \begin{figure}
    \centering
    \includegraphics[width=12cm]{tss.png}
  \end{figure}
  }
\end{frame}

\begin{frame}
  \frametitle{启动子 | 结构}
  \begin{figure}
    \centering
    \includegraphics[width=10cm]{promoter.jpg}
  \end{figure}
\end{frame}

\begin{frame}
  \frametitle{启动子 | TF\&TFBS}
  \begin{block}{转录因子(transcription factor)}
    能够结合在某基因上游特异核苷酸序列上的蛋白质,这些蛋白质能调控其基因的转录。
  \end{block}
  \pause
  \begin{block}{转录因子结合位点(Transcription Factor Binding Site,TFBS)}
    与转录因子结合的DNA序列,长度约为5~20bp,它们与转录因子相互作用进行基因的转录调控。
  \end{block}
\end{frame}

\begin{frame}
  \frametitle{启动子 | TFBS}
  \begin{figure}
    \centering
    \includegraphics[width=10cm]{tfbs.jpg}
  \end{figure}
\end{frame}

\begin{frame}
  \frametitle{启动子 | 数据库与分析工具}
  \begin{itemize}
    \item 启动子
      \begin{itemize}
	\item EPD:有注释、非冗余的真核生物RNA聚合酶\Rmnum{2}启动子数据集
        \item Promoter Scan,Promoter 2.0
      \end{itemize}
    \item 转录因子
      \begin{itemize}
        \item TRANSFAC:真核生物顺式作用元件和反式作用因子数据库
        \item Tfblast(TRANSFAC BLAST)
      \end{itemize}
  \end{itemize}
\end{frame}

\section{CpG岛识别}
\begin{frame}
  \frametitle{CpG岛 | 特征}
  \begin{block}{CpG岛}
    在基因组的某些区段,CpG保持或高于正常概率,这些区段被称作CpG岛(CpG island)。
  \end{block}
  \pause
  \begin{block}{特征}
    \begin{itemize}
      \item 几乎看家基因都含有CpG岛
      \item 一般位于基因的5'端区域(转录起始位点附近),长度约300~3000bp
      \item 大多数CpG岛是未甲基化的,未甲基化CpG岛说明基因可能具有潜在活性
      \item CpG岛中的核小体中H1含量低,其他组蛋白被广泛乙酰化,并具有超敏感位点
    \end{itemize}
  \end{block}
\end{frame}

\begin{frame}
  \frametitle{CpG岛 | 预测标准}
  \begin{enumerate}
    \item CpG岛长度:至少200bp
    \item GC含量:超过50\%
    \item CpG的观察值与预测值的比率:高于60\%
      \begin{itemize}
        \item $\frac{Num\ of\ CpG}{Num\ of\ C \times Num\ of\ G} \times Total\ number\ of\ nucleotides\ in\ the\ sequence$
      \end{itemize}
    \pause
    \item 500bp,55\%,65\%
  \end{enumerate}
\end{frame}

\begin{frame}
  \frametitle{CpG岛 | 分析工具}
  \begin{itemize}
    \item EMBOSS中的CpGPlot/CpGReport/Isochore
    \item CpG Island Searcher
    \item CpGcluster2
  \end{itemize}
\end{frame}

\section{EMBOSS}
\begin{frame}
  \frametitle{EMBOSS | 简介}
  \begin{block}{简介}
    EMBOSS(The European Molecular Biology Open Software Suite)是一个开源、免费的序列分析软件包,整合了目前可以获得的大部分序列分析软件。

    使用EMBOSS,可以将系列分析工作进行无缝整合,弥补了许多软件功能分散、分析效率低下的缺陷。
  \end{block}
  \begin{block}{使用}
    \begin{itemize}
      \item 操作系统:Linux,Mac,\textcolor{gray}{Windows}
      \item 界面:JEMBOSS(java),EMBOSS Explorer(web)
    \end{itemize}
  \end{block}
\end{frame}

\begin{frame}
  \frametitle{EMBOSS | 主要程序}
  \begin{itemize}
    \item 最重要的程序。Wossname:根据关键字查找程序;Showdb:显示所有整合的数据库。
    \item 序列编辑。Revseq:将序列反转并互补;Seqret:序列格式转换。
    \item 两个序列相似性图形表达。Dottup:精确匹配;Dotmatcher:近似匹配。
    \item 双序列比对。Needle:全局比对;Water:局部比对。
    \item 多序列比对。Emma:clustalW。
    \item 寻找SNP。Deffseq:仅限于双序列比对中。
    \item 其他。Plotorf,Getorf:翻译;Iep:等电点预测;Tmap:跨膜区预测;Pepinfo:蛋白质性质;Patmatmotifs:Motif搜索。
  \end{itemize}
\end{frame}

\begin{frame}
  \frametitle{EMBOSS | 演示}
  \begin{block}{组份分析}
  \begin{itemize}
    \item compseq: Calculate the composition of unique words in sequences
    \item geecee: Calculate fractional GC content of nucleic acid sequences
    \item revseq: Reverse and complement a nucleotide sequence
  \end{itemize}
  \end{block}
  \begin{block}{CpG岛分析}
  \begin{itemize}
    \item extractseq: Extract regions from a sequence
    \item cpgplot: Identify and plot CpG islands in nucleotide sequence(s)
    \item cpgreport: Identify and report CpG-rich regions in nucleotide sequence(s)
    \item isochore: Plot isochores in DNA sequences
  \end{itemize}
\end{block}
\end{frame}

\section{总结与答疑}
\begin{frame}
  \frametitle{总结与答疑}
  \begin{block}{知识点——DNA序列的基本信息与特征信息分析}
    \begin{itemize}
      \item DNA序列基本信息分析——查戈夫法则,GC含量,序列转换
      \item 限制酶位点分析——命名,\Rmnum{2}型的特点
      \item 开放阅读框分析——相位,ORF与CDS
      \item 启动子与转录因子结合位点分析——启动子结构
      \item CpG岛识别——概念、判别依据及标准
    \end{itemize}
  \end{block}
  \begin{block}{技能——解决问题的思路}
    \begin{itemize}
      \item 首先分析任务的属性
      \item 寻找可能的解决方案
      \item 确定最合适的方法
      \item 先易后难,由浅入深
    \end{itemize}
  \end{block}
\end{frame}

\section{引言}
\begin{frame}
  \frametitle{引言 | 回顾}
  \begin{itemize}[<+-|alert@+>]
    \item 基本信息分析
      \begin{itemize}
        \item 碱基比例
        \item GC含量
        \item 序列转换
        \item 寻找限制酶切位点
      \end{itemize}
    \item 序列特征分析
      \begin{itemize}
        \item 开放阅读框的预测
        \item 启动子和转录因子结合位点的分析
        \item CpG岛的识别
      \end{itemize}
    \item 基因识别
      \begin{itemize}
        \item 屏蔽重复序列
        \item 基因识别
      \end{itemize}
  \end{itemize}
\end{frame}

\section{重复序列分析}
\begin{frame}
  \frametitle{重复序列 | 分类}
  \begin{block}{重复序列(repetitive sequence, repeated sequence)}
    真核生物基因组中重复出现的核苷酸序列,一般不编码多肽,在基因组内可成簇排布,也可散布于基因组。
  \end{block}
  \pause
  \begin{block}{重复次数}
    \begin{itemize}
      \item 低度重复序列(lowly repetitive sequence):2~10个拷贝
      \item 中度重复序列(moderately repetitive sequence):重复几十次到几千次,平均长300bp
      \item 高度重复序列(highly repetitive sequence):重复几百万次,少于10个核苷酸残基组成的短片段
    \end{itemize}
  \end{block}
\end{frame}

\begin{frame}
  \frametitle{重复序列 | 分类}
  \begin{block}{组织形式}
    \begin{itemize}
      \item 串联重复序列:成簇存在于染色体的特定区域
        \begin{itemize}
          \item 卫星DNA(satellite DNA):5~200bp,几百万个拷贝,着丝粒部位
          \item 小卫星(minisatellite,VNTR):10~100bp的基本单位,总长不超过20kb,重复次数高度变异,靠近端粒的位置
          \item 微卫星(microsatellite,SSR,STR):2~10bp,长度50~100bp,STR遗传多态性,内含子
        \end{itemize}
      \item 散在重复序列:分散于染色体的各位点上
        \begin{itemize}
          \item 短散在重复序列(SINE):500bp以下,重复拷贝数达10万以上;非自主转座的反转录转座子;来源于RNA聚合酶\Rmnum{3}的转录产物;Alu(300bp,100万个拷贝)
          \item 长散在重复序列(LINE):1000bp以上,上万份拷贝;可以自主转座的反转录转座子;来源于RNA聚合酶\Rmnum{2}的转录产物;L1(6100bp,3500个拷贝)
        \end{itemize}
    \end{itemize}
  \end{block}
\end{frame}

\begin{frame}
  \frametitle{重复序列 | 分类}
  \begin{figure}
    \centering
    \includegraphics[width=12cm,height=8cm]{repeat.png}
  \end{figure}
\end{frame}

\begin{frame}
  \frametitle{重复序列 | 数据库与分析工具}
  \begin{itemize}
    \item Repbase:真核生物DNA重复序列数据库
    \item L1Base:L1数据库
    \item STRBase:STR数据库
    \item RepeatMasker:识别、分类和屏蔽重复序列
      \begin{itemize}
        \item Cross\_match:速度慢、精度高
        \item ABBlast:速度快、精度略低
        \item RMBlast:NCBI Blast的兼容版
        \item HMMER:只适用于人类基因组序列
      \end{itemize}
  \end{itemize}
\end{frame}

\section{基因识别}
\begin{frame}
  \frametitle{基因识别 | 基因与基因识别}
  \begin{block}{基因(gene)}
    产生一条多肽链或功能RNA所需的全部核苷酸序列。一段具有特定功能和结构的连续的DNA片段,携带着遗传信息,是编码蛋白质或RNA分子遗传信息、控制性状的基本遗传单位。\\
    一个完整的基因,不仅包括编码区,还包括5'末端和3'末端长度不等的特异性序列。
  \end{block}
  \pause
  \begin{block}{基因识别(gene prediction,gene finding)}
    使用生物学实验或计算机等手段识别DNA序列上的具有生物学特征的片段。
  \end{block}
\end{frame}

\begin{frame}
  \frametitle{基因识别 | 基因结构}
  \begin{figure}
    \centering
    \includegraphics[width=10cm]{geneP.jpg}
    \\
    \includegraphics[width=10cm]{geneE.jpg}
  \end{figure}
\end{frame}

\begin{frame}
  \frametitle{基因识别 | 方法}
  \begin{enumerate}
    \item 间接识别法(Extrinsic Approach):利用已知的mRNA或蛋白质序列为线索在DNA序列中搜寻所对应的片段
    \item 从头计算法(\textit{Ab Initio} Approach):基因预测,基于基因的两种类型的特征:
      \begin{itemize}
        \item “信号”:由一些特殊的序列构成,通常预示着周围存在着一个基因
        \item “内容”:蛋白质编码基因所具有的某些统计学特征
      \end{itemize}
    \item 比较基因组学的方法:自然选择的力量使得基因和DNA序列上具有生物学功能的片段较其他部分有较慢的变异速率,在前者的变异更有可能对生物体的生存产生负面影响,因而难以得到保存
  \end{enumerate}
\end{frame}

\begin{frame}
  \frametitle{基因识别 | 基因预测 | 信号 \& 内容}
  \begin{block}{信号}
    \begin{itemize}
      \item 不连续的局部序列模体,一般都有一致性序列(consensus sequence)
      \item 启动子,剪接供体和受体位点,起始和终止密码子,polyA位点
    \end{itemize}
  \end{block}
  \pause
  \begin{block}{内容}
    \begin{itemize}
      \item 不同长度的扩展序列,没有一致性序列,但具有把自己与周围DNA区分开来的保守特征
      \item 密码子使用偏好性(codon usage bias),双联密码子出现频率,基因组等值区(isochore)
    \end{itemize}
  \end{block}
\end{frame}

\begin{frame}
  \frametitle{基因识别 | 基因预测 | 信号}
  \begin{figure}
    \centering
    \includegraphics[width=10cm]{signal.jpg}
  \end{figure}
\end{frame}

\begin{frame}
  \frametitle{基因识别 | 基因预测 | 内容 | 密码子使用偏好性}
  \begin{figure}
    \centering
    \includegraphics[width=12cm]{cu1.png}
  \end{figure}
\end{frame}

\begin{frame}
  \frametitle{基因识别 | 基因预测 | 内容 | 密码子使用偏好性}
  \begin{figure}
    \centering
    \includegraphics[width=8cm]{cu2.jpg}
  \end{figure}
\end{frame}

\begin{frame}
  \frametitle{基因识别 | 原核基因}
  \begin{block}{信号}
    启动子序列(Pribnow盒),转录因子结合位点
  \end{block}
  \begin{block}{内容}
    连续的开放阅读框,统计学特征
  \end{block}
  \pause
  \begin{block}{总结}
    信号容易识别,内容容易判别,预测能达到相对较高的精度
  \end{block}
\end{frame}

\begin{frame}
  \frametitle{基因识别 | 真核基因}
  \begin{block}{信号}
    启动子(TATA box,CAAT box,GC box),供体和受体位点,起始和终止密码子,polyA信号序列
  \end{block}
  \begin{block}{内容}
    密码子使用偏好性,双联密码子出现频率,基因组等值区
  \end{block}
  \pause
  \begin{block}{总结}
    \begin{itemize}
      \item 综合信号信息确定外显子的边界,识别编码区域
      \item 通过内容统计值区分外显子、内含子和基因间区域
      \item 信号复杂,内容难判别,预测相当有挑战性
      \item 联合信号和内容检测以及同源性搜索,提高识别效率
    \end{itemize}
  \end{block}
\end{frame}

\begin{frame}
  \frametitle{基因识别 | 真核基因}
  \begin{figure}
    \centering
    \includegraphics[width=11cm]{genfind.png}
  \end{figure}
\end{frame}

\begin{frame}
  \frametitle{基因识别 | 策略}
  \begin{figure}
    \centering
    \includegraphics[width=11.5cm]{gp.jpg}
  \end{figure}
\end{frame}

\note{
\textbf{Gene-finding strategies.} Given a genome DNA sequence, information on the location of genes and transcripts can be obtained from different sources: conservation with one or more informant genomes (1); intrinsic signals involved in gene specification, such as start and stop codons and splice sites (2); the statistical properties of coding sequences (3); and, most importantly, known transcript sequences (either full-length cDNAs or partial ESTs) and protein sequences (4). Over the past two decades, a plethora of programs and strategies has been developed to combine these sources of information to obtain reliable gene predictions. The 'intrinsic' evidence from sequence signals and statistical bias can be combined (using a variety of frameworks often related to hidden Markov models [59]), to produce gene predictions (6). These programs are often referred to as ab initio or de novo gene finders. They are the programs of choice in the absence of known transcript or protein sequences or phylogenetically related genomes. If related genome sequences are available, the intrinsic information can be combined with patterns of genomic sequence conservation using programs often referred to as comparative (or dual- or multi-genome) gene finders (5). With these programs, maximum resolution is achieved when the compared genomes are at a phylogenetic distance such that there is maximum separation between the conservation in coding and noncoding regions. To increase resolution, programs have been developed that use multiple informant genomes. The most sophisticated use an underlying phylogenetic tree to appropriately weight sequence conservation depending on evolutionary distance. If cDNA and EST sequences are available, these often take priority over other sources of information. The initial map of the transcript or protein sequences onto the genome, which can be obtained using a variety of tools, including sequence-similarity searches, is refined using more sophisticated 'splice alignment' algorithms, whose explicit splice-site models allow more precise alignment across gaps corresponding to introns (8). Alternatively, cDNA and protein information can be fed into an ab initio gene-finder algorithm to give information on the exons included in the prediction (7). Often, cDNA and protein evidence is only partial; in such cases, the initial reliable gene and transcript set may be extended with more hypothetical models derived from ab initio or comparative gene finders, or from the genome mapping of cDNA and protein sequences from other species. Pipelines have been derived that automate this multi-step process (9). More recently, programs have been developed that combine the output of many individual gene finders (10). The underlying assumption in these 'combiners' is that consensus across programs increases the likelihood of the predictions. Thus, predictions are weighted according to the particular features of the program producing them. The most general frameworks allow the integration of a great variety of types of predictions - not only gene predictions, but also predictions of individual sites and exons. Despite all the developments in computational gene finding, the most reliable and complete gene annotations are still obtained after the initial alignments of cDNA and proteins onto the genome sequence are inspected manually to establish the exon boundaries of genes and transcripts (11). This is the task carried out by the HAVANA team at the Sanger Institute. The initial manual annotation can be refined even further by subsequent experimental verification of those transcript models lacking sufficiently strong evidence, as in the GENCODE project (12). Examples of gene-prediction programs (with references and URLs) corresponding to each strategy outlined here are provided in Additional data file 1.
}

\begin{frame}
  \frametitle{基因识别 | 工具列表}
  \begin{figure}
    \centering
    \includegraphics[width=12cm]{gps0.png}
  \end{figure}
\end{frame}

\begin{frame}
  \frametitle{基因识别 | 工具列表}
  \begin{figure}
    \centering
    \includegraphics[width=11cm]{gps1.png}
  \end{figure}
\end{frame}

\begin{frame}
  \frametitle{基因识别 | 工具列表}
  \begin{itemize}
    \item GeneMarkS:迭代隐马尔科夫模型
    \item Glimmer:插入式马尔科夫模型
    \item GENSCAN:广义隐马尔科夫模型
    \item GRAIL:人工神经网路
    \item \href{http://en.wikipedia.org/wiki/List\_of\_gene\_prediction\_software}{List of gene prediction software(Wikipedia)}
    \item \href{http://www.nature.com/nrg/journal/v3/n9/box/nrg890\_BX2.html}{Computational prediction of eukaryotic protein-coding genes, Box 2, Useful internet resources}
  \end{itemize}
\end{frame}

\section{查找数据库与分析工具}
\begin{frame}
  \frametitle{查找数据库与分析工具}
  \pause
  \begin{itemize}
    \item 借鉴相关文献中使用的数据库与工具
    \item 向特定领域的专家请教
    \item \textit{Nucleic Acids Research}每年的第一期为数据库专刊
    \item 维基百科等总结性网站
    \item \href{http://elements.eaglegenomics.com/}{The Elements of Bioinformatics}
    \item 使用 Google 等搜索引擎搜索
  \end{itemize}
\end{frame}

\section{总结与答疑}
\begin{frame}
  \frametitle{总结与答疑}
  \begin{block}{知识点——重复序列和基因识别}
    \begin{itemize}
      \item 重复序列——分类
      \item 基因识别——原核和真核的基因结构,基因识别方法
    \end{itemize}
  \end{block}
  \begin{block}{技能——查找数据库与分析工具}
    \begin{itemize}
      \item 借鉴文献、收集专刊、请教专家、搜索网络
      \item 数据库有其时效性
      \item 分析工具有其适用范围
    \end{itemize}
  \end{block}
\end{frame}

\section{引言}
\begin{frame}
  \frametitle{引言 | 回顾}
  \begin{itemize}[<+-|alert@+>]
    \item DNA序列分析
      \begin{itemize}
        \item 基本信息
        \item 序列特征
        \item 基因识别
      \end{itemize}
    \item RNA序列分析
      \begin{itemize}
        \item mRNA选择性剪接
        \item miRNA与靶基因
        \item lncRNA
      \end{itemize}
  \end{itemize}
\end{frame}

\begin{frame}
  \frametitle{引言 | RNA}
  \begin{enumerate}
    \item 编码RNA
      \begin{itemize}
        \item mRNA
      \end{itemize}
    \item 非编码RNA
      \begin{itemize}
        \item tRNA、rRNA
        \item miRNA、siRNA、lncRNA
      \end{itemize}
  \end{enumerate}
\end{frame}

\begin{frame}
  \frametitle{引言 | RNA}
  \begin{figure}
    \centering
    \includegraphics[width=10cm]{rnaC.jpg}
  \end{figure}
\end{frame}

\begin{frame}
  \frametitle{引言 | RNA | ncRNA}
  \begin{block}{非编码RNA(non-coding RNAs,ncRNA)}
  \begin{itemize}
    \item 基础结构性ncRNA(infrastructural non-coding RNAs),看家ncRNA(housekeeping non-coding RNAs)
      \begin{itemize}
        \item tRNA、rRNA、snRNA、snoRNA
      \end{itemize}
    \item 调节性ncRNA(regulatory non-coding RNAs)
      \begin{itemize}
        \item 小RNA(small RNAs,sRNA):\textless 200nt
          \begin{itemize}
            \item miRNA、siRNA、piRNA
          \end{itemize}
        \item 长链非编码RNA(long ncRNAs,lncRNA):\textgreater 200nt
      \end{itemize}
  \end{itemize}
\end{block}
\end{frame}

\begin{frame}
  \frametitle{引言 | RNA | ncRNA}
  \begin{figure}
    \centering
    \includegraphics[width=12cm]{ncrnaC.jpg}
  \end{figure}
\end{frame}

\section{mRNA选择性剪接}
\begin{frame}
  \frametitle{选择性剪接 | 剪接与选择性剪接}
  \begin{block}{剪接(splicing)}
    又称拼接,指基因信息在转录后的一种修饰,即将内含子移除及合并外显子,是真核生物的信使RNA前体(precursor messenger RNA)变成成熟mRNA的过程之一。
  \end{block}
  \pause
  \begin{block}{选择性剪接(alternative splicing)}
    又称可变剪接,指用不同的剪接方式(选择不同的剪接位点组合)从一个mRNA前体产生不同的mRNA剪接异构体的过程。
  \end{block}
\end{frame}

\begin{frame}
  \frametitle{选择性剪接 | 实例}
  \begin{figure}
    \centering
    \includegraphics[width=11cm]{splicing.png}
  \end{figure}
\end{frame}

\begin{frame}
  \frametitle{选择性剪接 | 机制 | 五种}
  \begin{figure}
    \centering
    \includegraphics[width=10cm]{splicingModel5.jpg}
  \end{figure}
\end{frame}

\begin{frame}
  \frametitle{选择性剪接 | 机制 | 七种}
  \begin{figure}
    \centering
    \includegraphics[width=9cm]{splicingModel7.jpg}
  \end{figure}
\end{frame}

\begin{frame}
  \frametitle{选择性剪接 | 机制 | 复杂实例}
  \begin{figure}
    \centering
    \includegraphics[width=11cm]{splicingExample.png}
  \end{figure}
\end{frame}

\begin{frame}
  \frametitle{选择性剪接 | 数据库与分析工具}
  \begin{itemize}
    \item ASTD = ASD (= AEDB + AltExtron + AltSplice) + ATD
    \item \textcolor{gray}{ASAP}
    \item ESEfinder 
    \item RESCUE-ESE
    \item ASPicDB
  \end{itemize}
\end{frame}

\section{miRNA及其靶基因预测}

\begin{frame}
  \frametitle{miRNA | 简介}
  \begin{block}{微RNA(microRNAs,miRNA,小分子RNA)}
    归属小RNA范畴,是真核生物中广泛存在的一种长约20到24个核苷酸的内源性非编码单链RNA分子。
    miRNA通过RNA诱导沉默复合体(RISC)与靶基因的3'非翻译区(3' UTR)相结合,导致靶基因mRNA降解或者抑制其翻译,从而调节基因转录后的表达。
  \end{block}
\end{frame}

\begin{frame}
  \frametitle{miRNA | 特征}
  \begin{block}{序列}
    不具有开放阅读框,不编码蛋白质;成熟的miRNA 5'端为单一磷酸基团,3'端为羟基
  \end{block}
  \pause
  \begin{block}{表达}
    具有时序性和组织特异性
  \end{block}
  \pause
  \begin{block}{调控}
    miRNA与靶基因间呈多对多的关系
  \end{block}
  \pause
  \begin{block}{物理位置}
    倾向于成簇地出现在染色体上
  \end{block}
  \pause
  \begin{block}{进化}
    在物种间高度保守
  \end{block}
\end{frame}

\begin{frame}
  \frametitle{miRNA | 生成}
  \begin{figure}
    \centering
    \includegraphics[width=7cm]{mirna.png}
  \end{figure}
\end{frame}

\begin{frame}
  \frametitle{miRNA | 作用网络}
  \begin{figure}
    \centering
    \includegraphics[width=10cm]{mirnaN.jpg}
  \end{figure}
\end{frame}

\begin{frame}
  \frametitle{miRNA | 功能}
  \begin{figure}
    \centering
    \includegraphics[width=12cm]{mirnaF.png}
  \end{figure}
\end{frame}

\begin{frame}
  \frametitle{miRNA | 预测}
  \begin{enumerate}
    \item 同源片段搜索方法
    \item 基于比较基因组学的预测方法
    \item 基于序列和结构特征打分的预测方法
    \item 结合作用靶标的预测方法
    \item 基于机器学习的预测方法
  \end{enumerate}
\end{frame}

\begin{frame}
  \frametitle{miRNA | 种子区域}
  \begin{figure}
    \centering
    \includegraphics[width=10cm]{mirnaS.png}
  \end{figure}
\end{frame}

\begin{frame}
  \frametitle{miRNA | 靶基因预测}
  \begin{enumerate}
    \item 基于种子区域互补和保守性的规则预测
      \begin{itemize}
        \item miRanda
        \item TargetScan
      \end{itemize}
    \item 基于机器学习方法训练参数进行靶基因预测
      \begin{itemize}
        \item PicTar
        \item miTarget
      \end{itemize}
  \end{enumerate}
\end{frame}

\begin{frame}
  \frametitle{miRNA | 数据库与分析工具}
  \begin{itemize}
    \item 数据库:miRBase、TarBase、miRGen
    \item miRNA预测:MiRscan、MiPred、miRFinder
    \item miRNA靶基因预测:miRanda、TargetScan、PicTar、miTarget
    \item \href{http://zh.wikipedia.org/wiki/\%E5\%BE\%AERNA\%E4\%B8\%8E\%E5\%BE\%AERNA\%E9\%9D\%B6\%E6\%95\%B0\%E6\%8D\%AE\%E5\%BA\%93}{微RNA与微RNA靶数据库(维基百科)}
  \end{itemize}
\end{frame}

\section{lncRNA}
\begin{frame}
  \frametitle{lncRNA | 序列结构特征}
  \begin{block}{类似于mRNA}
    \begin{itemize}
      \item 大多被RNA聚合酶\Rmnum{2}所转录
      \item 有5'帽子和3'端的poly(A)尾巴
      \item 剪接现象
      \item 启动子区域和剪接位置具有保守性
    \end{itemize}
  \end{block}
  \pause
  \begin{block}{独特性}
  \begin{itemize}
    \item 长度偏短、外显子数目偏少
    \item 不存在较长的ORF
    \item 密码子偏好性与内含子区域相似
    \item 二级结构中有丰富的长茎发夹结构
    \item 在不同物种间的保守性差
    \item 主要富集在细胞核
  \end{itemize}
  \end{block}
\end{frame}

\begin{frame}
  \frametitle{lncRNA | 生物功能特征}
  \begin{itemize}
    \item 其表达具有时空特异性,与特定的生物过程相关
    \item 具有复杂的调控功能,在染色质改变、转录调控及后转录调控中发挥重要作用
    \item 复杂的代谢机制,大多数lncRNA是稳定的,半衰期的变化范围较大
    \item 与疾病存在密切关系,如肿瘤、阿尔兹海默病、心血管疾病等
  \end{itemize}
  \pause
  \begin{block}{数据库}
    \href{http://zh.wikipedia.org/wiki/\%E9\%95\%BF\%E9\%93\%BE\%E9\%9D\%9E\%E7\%BC\%96\%E7\%A0\%81RNA\%E6\%95\%B0\%E6\%8D\%AE\%E5\%BA\%93}{长链非编码RNA数据库(维基百科)}
  \end{block}
\end{frame}

\begin{frame}
  \frametitle{lncRNA | 类型}
  \begin{figure}
    \centering
    \includegraphics[width=11cm]{lncrnaL0.jpg}\\
    \includegraphics[width=11cm]{lncrnaT.png}
  \end{figure}
\end{frame}

\begin{frame}
  \frametitle{lncRNA | 生物功能}
  \begin{figure}
    \centering
    \includegraphics[width=10cm]{lncrnaF.jpg}
  \end{figure}
\end{frame}

\begin{frame}
  \frametitle{lncRNA | 作用机制}
  \begin{figure}
    \centering
    \includegraphics[width=12cm]{lncrnaM.jpg}
  \end{figure}
\end{frame}

\begin{frame}
  \frametitle{lncRNA | lncRNA与疾病}
  \begin{figure}
    \centering
    \includegraphics[width=10cm]{lncrnaD.png}
  \end{figure}
\end{frame}

\section{学习数据库与分析工具的使用}
\begin{frame}
  \frametitle{学习数据库与分析工具的使用}
  \begin{itemize}
    \item 阅读官方的帮助手册
    \item 请教有使用经验的专家
    \item 查找简单的使用实例,并重复其操作步骤
    \item 使用Google等搜索引擎搜索相关资料
    \item 各种protocols期刊:\textit{Nature protocols, Current Protocols (in Bioinformatics), SpringerProtocols, Methods in Molecular Biology}
  \end{itemize}
\end{frame}

\section{总结与答疑}
\begin{frame}
  \frametitle{总结与答疑}
  \begin{block}{知识点——mRNA选择性剪接和miRNA分析}
    \begin{itemize}
      \item mRNA选择性剪接——选择性剪接的主要机制,数据资源
      \item miRNA——miRNA的特点和作用机制,miRNA预测方法与工具,miRNA靶基因预测方法与工具
    \end{itemize}
  \end{block}
  \begin{block}{技能——学习数据库与分析工具的使用}
    \begin{itemize}
      \item 阅读手册、请教专家、重复实例、搜索网络
      \item 历史资料使用的是历史版本
    \end{itemize}
  \end{block}
\end{frame}

\section{复习思考题}
\begin{frame}
  \frametitle{复习思考题}
  \begin{block}{知识点}
  \begin{enumerate}
    \item DNA序列携带哪两类遗传信息?可以对DNA序列进行哪些分析?
    \item 简述限制性核酸内切酶的命名规则及II型限制酶的主要特点。
    \item 简述CpG岛的概念及其识别依据和判别标准。
    \item 简述重复序列依重复次数和组织形式的分类。
    \item 简述基因识别的三大类方法。
    \item 简述选择性剪接的产生机制。
    \item 简述miRNA预测和miRNA靶基因预测的方法。
  \end{enumerate}
\end{block}
\pause
\begin{block}{技能}
  \begin{enumerate}
    \item 以计算GC含量为例,论述解决思路,即如何通过分析问题的属性确定相应的策略从而找到最合适的方法。
    \item 在解决生物信息学问题时,论述找到所需数据库和分析工具并掌握其使用方法的策略。
  \end{enumerate}
\end{block}
\end{frame}

\section*{Acknowledgements}
\begin{frame}
  \frametitle{Powered by}
  \begin{center}
    \includegraphics[width=9cm]{power.png}
  \end{center}
\end{frame}

\end{document}

