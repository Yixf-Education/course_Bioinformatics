\section{引言}
\begin{frame}
  \frametitle{引言 | 回顾}
  \begin{itemize}[<+-|alert@+>]
    \item DNA序列分析
      \begin{itemize}
        \item 基本信息
        \item 序列特征
        \item 基因识别
      \end{itemize}
    \item RNA序列分析
      \begin{itemize}
        \item mRNA选择性剪接
        \item miRNA与靶基因
        \item lncRNA
      \end{itemize}
  \end{itemize}
\end{frame}

\begin{frame}
  \frametitle{引言 | RNA}
  \begin{enumerate}
    \item 编码RNA
      \begin{itemize}
        \item mRNA
      \end{itemize}
    \item 非编码RNA
      \begin{itemize}
        \item tRNA、rRNA
        \item miRNA、siRNA、lncRNA
      \end{itemize}
  \end{enumerate}
\end{frame}

\begin{frame}
  \frametitle{引言 | RNA}
  \begin{figure}
    \centering
    \includegraphics[width=10cm]{rnaC.jpg}
  \end{figure}
\end{frame}

\begin{frame}
  \frametitle{引言 | RNA | ncRNA}
  \begin{block}{非编码RNA(non-coding RNAs,ncRNA)}
  \begin{itemize}
    \item 基础结构性ncRNA(infrastructural non-coding RNAs),看家ncRNA(housekeeping non-coding RNAs)
      \begin{itemize}
        \item tRNA、rRNA、snRNA、snoRNA
      \end{itemize}
    \item 调节性ncRNA(regulatory non-coding RNAs)
      \begin{itemize}
        \item 小RNA(small RNAs,sRNA):\textless 200nt
          \begin{itemize}
            \item miRNA、siRNA、piRNA
          \end{itemize}
        \item 长链非编码RNA(long ncRNAs,lncRNA):\textgreater 200nt
      \end{itemize}
  \end{itemize}
\end{block}
\end{frame}

\begin{frame}
  \frametitle{引言 | RNA | ncRNA}
  \begin{figure}
    \centering
    \includegraphics[width=12cm]{ncrnaC.jpg}
  \end{figure}
\end{frame}

\section{mRNA选择性剪接}
\begin{frame}
  \frametitle{选择性剪接 | 剪接与选择性剪接}
  \begin{block}{剪接(splicing)}
    又称拼接,指基因信息在转录后的一种修饰,即将内含子移除及合并外显子,是真核生物的信使RNA前体(precursor messenger RNA)变成成熟mRNA的过程之一。
  \end{block}
  \pause
  \begin{block}{选择性剪接(alternative splicing)}
    又称可变剪接,指用不同的剪接方式(选择不同的剪接位点组合)从一个mRNA前体产生不同的mRNA剪接异构体的过程。
  \end{block}
\end{frame}

\begin{frame}
  \frametitle{选择性剪接 | 剪接}
  \begin{figure}
    \centering
    \includegraphics[width=11cm]{splicing.10.png}
  \end{figure}
\end{frame}

\begin{frame}
  \frametitle{选择性剪接 | 剪接 | 机制 | 一致性序列}
Within the intron, a donor site (5' end of the intron), a branch site (near the 3' end of the intron) and an acceptor site (3' end of the intron) are required for splicing.
  \begin{figure}
    \centering
    \includegraphics[width=12cm]{splicingM.jpg}
  \end{figure}
A-G-[cut]-G-U-R-A-G-U (donor site) \ldots intron sequence \ldots Y-U-R-A-C (branch sequence 20-50 nucleotides upstream of acceptor site) \ldots Y-rich-N-C-A-G-[cut]-G (acceptor site)
\end{frame}

\begin{frame}
  \frametitle{选择性剪接 | 剪接 | 机制 | 过程 | 概览}
  \begin{figure}
    \centering
    \includegraphics[width=12cm]{splicing.mechanism.10.png}
  \end{figure}
\end{frame}

\begin{frame}
  \frametitle{选择性剪接 | 剪接 | 机制 | 过程 | 详观}
  \begin{figure}
    \centering
    \includegraphics[width=11cm]{splicing.mechanism.11.jpg}
  \end{figure}
\end{frame}


\begin{frame}
  \frametitle{选择性剪接 | 模式图}
  \begin{figure}
    \centering
    \includegraphics[width=8cm]{splicing.01.jpg}
  \end{figure}
\end{frame}

\begin{frame}
  \frametitle{选择性剪接 | 模式图}
  \begin{figure}
    \centering
    \includegraphics[width=7.5cm]{as.model.02.png}
  \end{figure}
\end{frame}

\begin{frame}
  \frametitle{选择性剪接 | 实例}
  \begin{figure}
    \centering
    \includegraphics[width=11cm]{splicing.png}
  \end{figure}
\end{frame}

\begin{frame}
  \frametitle{选择性剪接 | 实例}
  \begin{figure}
    \centering
    \includegraphics[width=9cm]{as.examples.11.png}
  \end{figure}
\end{frame}

\begin{frame}
  \frametitle{选择性剪接 | 实例}
  \begin{figure}
    \centering
    \includegraphics[width=12cm]{as.examples.jpg}
  \end{figure}
\end{frame}

\begin{frame}
  \frametitle{选择性剪接 | 实例 | \textit{Dscam}}
  \begin{figure}
    \centering
    \includegraphics[width=11cm]{as.dscam.01.jpg}
    \\ \Large{$12 \times 48 \times 33 \times 2 = $}
    \pause
    \Large{$38016$}
  \end{figure}
\end{frame}

\begin{frame}
  \frametitle{选择性剪接 | 调控}
  \begin{itemize}
    \item 剪接因子与调节蛋白相互作用
    \item 剪接体的核心部分包括一组小核RNA(snRNA)以及与之结合的蛋白质,它们以严格的程序组装成剪接体
      \begin{itemize}
	\item snRNA成员分别为U1、U2、U4、U5和U6,长度在106(U6)~185(U2)个核苷酸之间
	\item snRNA与蛋白质结合在一起形成小核核糖核蛋白(snRNP)
      \end{itemize}
    \item 剪接因子依据结合在RNA上的位置及作用方式,可以分为
      \begin{itemize}
	\item 外显子增强子(exonic splicing enhancer,ESE)
	\item 外显子抑制子(exonic splicing silencer,ESS)
	\item 内含子增强子(intronic splicing enhancer,ISE)
	\item 内含子抑制子(intronic splicing silencer,ISS)
      \end{itemize}
  \end{itemize}
\end{frame}

\begin{frame}
  \frametitle{选择性剪接 | 调控}
  \begin{figure}
    \centering
    \includegraphics[width=8cm]{splicing.mechanism.20.jpg}
  \end{figure}
\end{frame}
\note{
Splicing is a conserved mechanism controlled by the spliceosome — a complex composed of many proteins and five small nuclear RNAs (U1, U2, U4, U5 and U6) that assemble with proteins to form small nuclear ribonucleoproteins (snRNPs).
a | The four conserved signals that enable recognition of RNA by the spliceosome are: the exon–intron junctions at the 5′ and 3′ ends of introns (the 5′ splice site (5′ SS) and 3′ SS), the branch site sequence located upstream of the 3′ SS and the polypyrimidine tract (PPT) located between the 3′ SS and the branch site.
b | The key steps in splicing are shown. Regulation of splicing can occur at the basic level of splice-site recognition by the spliceosome through the facilitation or interference of the binding of U1 and U2 snRNPs to the splice sites. The unlabelled orange ovals represent other, unspecified components of the spliceosome.
c | Exons and introns contain short, degenerate binding sites for splicing auxiliary proteins. These sites are called exonic splicing enhancers (ESEs), intronic splicing enhancers (ISEs), exonic splicing silencers (ESSs) and intronic silencing silencers (ISSs). Splice-site recognition is mediated by proteins that bind specific regulatory sequences, such as the serine/arginine (SR) proteins, heterogenous nuclear ribonucleoproteins (hnRNPs), polypyrimidine tract-binding (PTB) proteins, the TIA1 RNA-binding protein, Fox proteins, Nova proteins, and more.
Constitutive exons are shown in blue, alternatively spliced regions in purple, and introns are represented by solid lines.
}

\begin{frame}
  \frametitle{选择性剪接 | \alert{机制} | 五种}
  \begin{figure}
    \centering
    \includegraphics[width=10cm]{splicingModel5.jpg}
  \end{figure}
\end{frame}

\begin{frame}
  \frametitle{选择性剪接 | \alert{机制} | 七种}
  \begin{figure}
    \centering
    \includegraphics[width=9cm]{splicingModel7.jpg}
  \end{figure}
\end{frame}

\begin{frame}
  \frametitle{选择性剪接 | \alert{机制} | 七种}
  \begin{figure}
    \centering
    \includegraphics[width=9cm]{splicingModel8.jpg}
  \end{figure}
\end{frame}

\begin{frame}
  \frametitle{选择性剪接 | 机制 | 实例}
  \begin{figure}
    \centering
    \includegraphics[width=10cm]{splicingExample10.png}
  \end{figure}
\end{frame}

\begin{frame}
  \frametitle{选择性剪接 | 机制 | 复杂实例}
  \begin{figure}
    \centering
    \includegraphics[width=11cm]{splicingExample.png}
  \end{figure}
\end{frame}

\begin{frame}
  \frametitle{选择性剪接 | 数据库与分析工具}
  \begin{block}{两大类数据库(依数据来源)}
    \begin{itemize}
      \item 基于文献报道(收集整理实验数据和文献报道)
      \item 基于EST数据(EST与基因组或DNA、mRNA比对)
    \end{itemize}
  \end{block}
  \pause
  \begin{block}{数据库与工具}
  \begin{itemize}
    \item ASTD = ASD (= AEDB + AltExtron + AltSplice) + ATD
    \item \textcolor{gray}{ASAP}
    \item ESEfinder 
    \item RESCUE-ESE
    \item ASPicDB
  \end{itemize}
\end{block}
\end{frame}

\begin{frame}
  \frametitle{选择性剪接 | 透过表象看本质}
  \begin{figure}
    \centering
    \includegraphics[width=8cm]{as.random.forest.png}
  \end{figure}
\end{frame}

\section{miRNA及其靶基因预测}

\begin{frame}
  \frametitle{miRNA | 简介}
  \begin{block}{微RNA(microRNAs,miRNA,小分子RNA)}
    归属小RNA范畴,是真核生物中广泛存在的一种长约20到24个核苷酸的内源性非编码单链RNA分子。
    miRNA通过RNA诱导沉默复合体(RISC)与靶基因的3'非翻译区(3' UTR)相结合,导致靶基因mRNA降解或者抑制其翻译,从而调节基因转录后的表达。
  \end{block}
\end{frame}

\begin{frame}
  \frametitle{miRNA | 特征}
  \begin{block}{序列}
    不具有开放阅读框,不编码蛋白质;成熟的miRNA 5'端为单一磷酸基团,3'端为羟基
  \end{block}
  \pause
  \begin{block}{表达}
    具有时序性和组织特异性
  \end{block}
  \pause
  \begin{block}{调控}
    miRNA与靶基因间呈多对多的关系
  \end{block}
  \pause
  \begin{block}{物理位置}
    倾向于成簇地出现在染色体上
  \end{block}
  \pause
  \begin{block}{进化}
    在物种间高度保守
  \end{block}
\end{frame}

\begin{frame}
  \frametitle{miRNA | 生成}
  \begin{figure}
    \centering
    \includegraphics[width=7cm]{mirna.png}
  \end{figure}
\end{frame}

\begin{frame}
  \frametitle{miRNA | 作用网络}
  \begin{figure}
    \centering
    \includegraphics[width=10cm]{mirnaN.jpg}
  \end{figure}
\end{frame}

\begin{frame}
  \frametitle{miRNA | 功能}
  \begin{figure}
    \centering
    \includegraphics[width=12cm]{mirnaF.png}
  \end{figure}
\end{frame}

\begin{frame}
  \frametitle{miRNA | 数据来源}
  \begin{itemize}
    \item 实验手段(cDNA克隆测序、miRNA-seq),有局限性。
    \item 计算预测,有优势。
  \end{itemize}
\end{frame}

\begin{frame}
  \frametitle{miRNA | \alert{预测}}
  \begin{enumerate}
    \item 同源片段搜索方法。将已知miRNA或pre-miRNA序列在自身或其他相近基因组中用比对算法搜索同源序列,结合序列二级结构特征进行筛选。(局限于:与已知miRNA/pre-miRNA在序列上和结构上同源的miRNA/pre-miRNA)
    \item 基于比较基因组学的预测方法。依据进化过程中的保守性在多物种中搜索潜在的miRNA。(能够找到不与已知miRNA同源的新miRNA;局限于:保守miRNA)
      \begin{itemize}
	\item 方法一:先在一个物种基因组中根据结构和序列特征找出可能的pre-miRNA,而后与其他物种基因组比较,判断其序列和结构是否保守
	\item 方法二:先通过比较两物种的基因组找出保守区域,而后在保守区域中根据结构和序列特征搜索可能的miRNA
      \end{itemize}
  \end{enumerate}
\end{frame}

\begin{frame}
  \frametitle{miRNA | \alert{预测}(续)}
  \begin{enumerate}
    \setcounter{enumi}{2}
    \item 基于序列和结构特征打分的预测方法。根据已知miRNA序列和结构的特征对全基因组范围内能形成茎环结构的片段进行筛选,是发现非同源、物种特异miRNA的方法。(为降低假阳性,用异常严格的序列和结构标准筛选候选片段,可能遗漏大量的miRNA)
    \item 结合作用靶标的预测方法。依据miRNA与其靶基因序列间的碱基互补配对的保守性的特点预测miRNA。
    \item 基于机器学习的预测方法。通过对阳性miRNA和阴性miRNA数据集的训练来构建区分两者的分类器,根据所得分类器对未知序列进行预测。(支持向量机SVM是目前miRNA分类和预测最常用的机器学习方法)
  \end{enumerate}
\end{frame}

\begin{frame}
  \frametitle{miRNA | 种子区域}
  \begin{figure}
    \centering
    \includegraphics[width=10cm]{mirnaS.png}
  \end{figure}
\end{frame}

\begin{frame}
  \frametitle{miRNA | \alert{靶基因预测}}
  \begin{enumerate}
    \item 基于种子区域互补和保守性的规则预测
      \begin{itemize}
        \item miRanda
        \item TargetScan
      \end{itemize}
    \item 基于机器学习方法训练参数进行靶基因预测
      \begin{itemize}
        \item PicTar
        \item miTarget
      \end{itemize}
  \end{enumerate}
\end{frame}

\begin{frame}
  \frametitle{miRNA | 数据库与分析工具}
  \begin{itemize}
    \item 数据库:miRBase、TarBase、miRGen
    \item miRNA预测:MiRscan、MiPred、miRFinder
    \item miRNA靶基因预测:miRanda、TargetScan、PicTar、miTarget
    \item \href{http://zh.wikipedia.org/wiki/\%E5\%BE\%AERNA\%E4\%B8\%8E\%E5\%BE\%AERNA\%E9\%9D\%B6\%E6\%95\%B0\%E6\%8D\%AE\%E5\%BA\%93}{微RNA与微RNA靶数据库(维基百科)}
  \end{itemize}
  \pause
  \begin{block}{强调}
    \begin{itemize}
      \item miRBase是一个集miRNA序列、注释信息以及预测的靶基因数据为一体的数据库。
      \item TarBase是一个存储已被实验证实的真实miRNA与靶基因间关系的数据库。
    \end{itemize}
  \end{block}
\end{frame}

\begin{frame}
  \frametitle{选择性剪接 | 透过表象看本质 | MiRscan}
  \begin{figure}
    \centering
    \includegraphics[width=11cm]{mirscan.01.jpg}
  \end{figure}
\end{frame}

\begin{frame}
  \frametitle{选择性剪接 | 透过表象看本质 | miRFinder}
  \begin{figure}
    \centering
    \includegraphics[width=7cm]{mirfinder.jpg}
  \end{figure}
\end{frame}

\begin{frame}
  \frametitle{选择性剪接 | 透过表象看本质 | HMM}
  \begin{figure}
    \centering
    \includegraphics[width=11cm]{mirna.hmm.1.jpg}
  \end{figure}
\end{frame}

\begin{frame}
  \frametitle{选择性剪接 | 透过表象看本质 | miRanda}
  \begin{figure}
    \centering
    \includegraphics[width=10cm]{miranda.jpg}
  \end{figure}
\end{frame}

\section{lncRNA}
\begin{frame}
  \frametitle{lncRNA | 序列结构特征}
  \begin{block}{类似于mRNA}
    \begin{itemize}
      \item 大多被RNA聚合酶\Rmnum{2}所转录
      \item 有5'帽子和3'端的poly(A)尾巴
      \item 剪接现象
      \item 启动子区域和剪接位置具有保守性
    \end{itemize}
  \end{block}
  \pause
  \begin{block}{独特性}
  \begin{itemize}
    \item 长度偏短、外显子数目偏少
    \item 不存在较长的ORF
    \item 密码子偏好性与内含子区域相似
    \item 二级结构中有丰富的长茎发夹结构
    \item 在不同物种间的保守性差
    \item 主要富集在细胞核
  \end{itemize}
  \end{block}
\end{frame}

\begin{frame}
  \frametitle{lncRNA | 生物功能特征}
  \begin{itemize}
    \item 其表达具有时空特异性,与特定的生物过程相关
    \item 具有复杂的调控功能,在染色质改变、转录调控及后转录调控中发挥重要作用
    \item 复杂的代谢机制,大多数lncRNA是稳定的,半衰期的变化范围较大
    \item 与疾病存在密切关系,如肿瘤、阿尔兹海默病、心血管疾病等
  \end{itemize}
  \pause
  \begin{block}{数据库}
    \href{http://zh.wikipedia.org/wiki/\%E9\%95\%BF\%E9\%93\%BE\%E9\%9D\%9E\%E7\%BC\%96\%E7\%A0\%81RNA\%E6\%95\%B0\%E6\%8D\%AE\%E5\%BA\%93}{长链非编码RNA数据库(维基百科)}
  \end{block}
\end{frame}

\begin{frame}
  \frametitle{lncRNA | 类型}
  \begin{figure}
    \centering
    \includegraphics[width=11cm]{lncrnaL0.jpg}\\
    \includegraphics[width=11cm]{lncrnaT.png}
  \end{figure}
\end{frame}

\begin{frame}
  \frametitle{lncRNA | 生物功能}
  \begin{figure}
    \centering
    \includegraphics[width=10cm]{lncrnaF.jpg}
  \end{figure}
\end{frame}

\begin{frame}
  \frametitle{lncRNA | 作用机制}
  \begin{figure}
    \centering
    \includegraphics[width=12cm]{lncrnaM.jpg}
  \end{figure}
\end{frame}

\begin{frame}
  \frametitle{lncRNA | 作用机制}
  \begin{figure}
    \centering
    \includegraphics[width=12cm]{lncrnaM3.png}
  \end{figure}
\end{frame}
\note{
Schematic illustration of lncRNAs functioning.
LncRNA transcribed from an upstream non-coding promoter can negatively (1) or positively (2)
affect expression of the downstream gene by inhibiting RNA polymerase II recruitment and/or inducing chromatin remodeling, respectively.
LncRNA is able to hybridize to the pre-mRNA and block recognition of the splice sites by the spliceosome, thus resulting in an alternatively spliced transcript (3).
Alternatively, hybridization of the sense and antisense transcripts can allow Dicer to generate endogenous siRNAs (4).
The binding of lncRNA to the miRNA results in the miRNA function silencing (5).
The complex of lncRNA and specific protein partners can modulate the activity of the protein (6),
is involved in structural and organization roles of the cell (7),
alters the protein localizes in the cell (8),
and affects epigenetic processes (9).
Finally, long ncRNAs can be processed to the small RNAs (10).
}

\begin{frame}
  \frametitle{lncRNA | lncRNA与疾病}
  \begin{figure}
    \centering
    \includegraphics[width=10cm]{lncrnaD.png}
  \end{figure}
\end{frame}

\section{学习数据库与分析工具的使用}
\begin{frame}
  \frametitle{学习数据库与分析工具的使用}
  \begin{itemize}
    \item 阅读官方的帮助手册
    \item 请教有使用经验的专家
    \item 查找简单的使用实例,并重复其操作步骤
    \item 使用Google等搜索引擎搜索相关资料
    \item 参考相关的专业书籍
    \item 各种protocols期刊:\textit{Nature protocols, Current Protocols (in Bioinformatics), SpringerProtocols, Methods in Molecular Biology}
  \end{itemize}
\end{frame}

\section{总结与答疑}
\begin{frame}
  \frametitle{总结与答疑}
  \begin{block}{知识点——mRNA选择性剪接和miRNA分析}
    \begin{itemize}
      \item mRNA选择性剪接——选择性剪接的主要机制,数据资源
      \item miRNA——miRNA的特点和作用机制,miRNA预测方法与工具,miRNA靶基因预测方法与工具
    \end{itemize}
  \end{block}
  \begin{block}{技能——学习数据库与分析工具的使用}
    \begin{itemize}
      \item 阅读手册、请教专家、重复实例、搜索网络
      \item 历史资料使用的是历史版本
    \end{itemize}
  \end{block}
\end{frame}

\section{复习思考题}
\begin{frame}
  \frametitle{复习思考题}
  \begin{block}{知识点}
  \begin{enumerate}
    \item DNA序列携带哪两类遗传信息?可以对DNA序列进行哪些分析?
    \item 简述限制性核酸内切酶的命名规则及II型限制酶的主要特点。
    \item 简述CpG岛的概念及其识别依据和判别标准。
    \item 简述重复序列依重复次数和组织形式的分类。
    \item 简述基因识别的三大类方法及主要策略。
    \item 简述选择性剪接的产生机制。
    \item 简述miRNA预测和miRNA靶基因预测的方法。
  \end{enumerate}
\end{block}
\pause
\begin{block}{技能}
  \begin{enumerate}
    \item 以计算GC含量为例,论述解决思路,即如何通过分析问题的属性确定相应的策略从而找到最合适的方法。
    \item 在解决生物信息学问题时,论述找到所需数据库和分析工具并掌握其使用方法的策略。
  \end{enumerate}
\end{block}
\end{frame}
