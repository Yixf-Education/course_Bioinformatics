\section{引言}
\begin{frame}
  \frametitle{引言 | 大千世界}
  \begin{figure}
    \centering
    \includegraphics[width=6cm]{world.png}
  \end{figure}
\end{frame}

\begin{frame}
  \frametitle{引言 | 中心法则}
  \begin{figure}
    \centering
    \includegraphics[width=11cm]{dogma.jpg}
  \end{figure}
\end{frame}

\begin{frame}
  \frametitle{引言 | DNA}
  \begin{figure}
    \centering
    \includegraphics[width=6cm]{dna.acgt.png}
  \end{figure}
\end{frame}

\begin{frame}
  \frametitle{引言 | DNA \& RNA}
  \begin{figure}
    \centering
    \includegraphics[width=9cm]{dna.rna.01.png}
  \end{figure}
\end{frame}

\begin{frame}
  \frametitle{引言 | ACGT}
  \begin{figure}
    \centering
    \includegraphics[width=10.5cm]{acgt.02.png}
  \end{figure}
\end{frame}

\begin{frame}
  \frametitle{引言 | ACGT$\Rightarrow$生信}
  \begin{figure}
    \centering
    \includegraphics[width=10cm]{matrix.png}
  \end{figure}
\end{frame}

\begin{frame}
  \frametitle{引言 | \alert{遗传信息}}
  \begin{figure}
    \centering
    \includegraphics[width=11cm]{info.jpg}
  \end{figure}
\end{frame}

\begin{frame}
  \frametitle{引言 | 遗传信息 | 特点}
  \begin{itemize}
    \item 遗传信息的分布不是随机的
    \item 遗传信息分布的模式可以遗传
    \item 不同物种间对应相同或类似功能或结构的遗传信息的分布模式具有相似性
  \end{itemize}
  \begin{center}
    %\Large{\alert{核酸序列有与生物学功能相对应的规律和特征!}}
    \Large{核酸序列有与生物学功能相对应的规律和特征!}
  \end{center}
\end{frame}

\begin{frame}
  \frametitle{引言 | 序列分析}
  \begin{block}{序列分析}
    通过实验或计算等方式,确定核苷酸或氨基酸序列中可能与特定功能、结构或生化过程相关联的\alert{具有生物学意义的序列特征},或者\alert{序列自身的规律}。
  \end{block}
\end{frame}

\begin{frame}
  \frametitle{引言 | 序列分析}
  \begin{figure}
    \centering
    \includegraphics[width=8cm]{sequence.analysis.jpg}
  \end{figure}
\end{frame}

\begin{frame}
  \frametitle{引言 | 序列分析 | 问题}
  \begin{block}{基本问题}
    \begin{itemize}
      \item 总的GC含量或者其他核苷酸成分是多少?
      \item 有哪些重复的DNA序列,在什么地方?
      \item 一共有多少个基因(编码蛋白质的序列)?
    \end{itemize}
  \end{block}
  \pause
  \begin{block}{深层问题}
    \begin{itemize}
      \item 为什么会有各种特征序列?(物理、化学性质?进化压力?)
      \item 需要从哪些方面分析序列特征?
      \item 怎样描述这些序列特征?
    \end{itemize}
  \end{block}
\end{frame}

\section{DNA组份分析与序列转换}
\begin{frame}
  \frametitle{DNA序列 | 查戈夫法则}
  \begin{block}{查戈夫法则}
    \begin{description}
      \item[第一法则]$A=T, G=C \Longrightarrow A+C=T+G, A+G=C+T$
      \item[第二法则]AT/GC 的比值因生物种类不同而异
    \end{description}
  \end{block}
\end{frame}

\begin{frame}
  \frametitle{DNA序列 | 序列长度}
  \begin{block}{序列长度}
    序列长度是具有独立生物学功能的序列片段(如基因、启动子等)的基本性质。物种的基因组长度也是重要参数之一。 
  \end{block}
  \pause
  \begin{block}{蛋白质编码基因的序列长度}
    \begin{itemize}
      \item 原核:~1000个核苷酸
      \item 脊椎动物:~30000个核苷酸
      \item 人:20000~50000个核苷酸
    \end{itemize}
  \end{block}
\end{frame}

\begin{frame}
  \frametitle{DNA序列 | 序列长度 | 基因组}
  \begin{figure}
    \centering
    \includegraphics[width=10cm]{genome.size.01.png}
  \end{figure}
\end{frame}

\begin{frame}
  \frametitle{DNA序列 | 碱基组成}
  \begin{itemize}
    \item 核酸序列由ACGT四种碱基组成
    \item 不同物种的DNA碱基组成存在差异
    \item 同一基因组内不同区段的碱基组成有差异
    \item 同一基因内部不同片段的碱基组成也有差异
  \end{itemize}
\end{frame}

\begin{frame}
  \frametitle{DNA序列 | 碱基组成 | 分析}
  \begin{itemize}
    \item 对于随机分布的DNA序列,每种核苷酸的出现是均匀分布的,出现频率各为0.25
    \item 真实基因组的核苷酸分布则是非均匀的(酵母基因组核苷酸出现频率:A/T=0.325,G/C=0.175)
    \item 如果同时计算DNA的正反两条链,根据碱基配对原则,A和T、G和C的出现频率相同
    \item 如果仅统计一条链,则虽然A和T、G和C的出现频率不同,但是数值接近(酵母单链核苷酸出现频率:A=0.344,T=0.343,G=0.157,C=0.155)
  \end{itemize}
\end{frame}

\begin{frame}
  \frametitle{DNA序列 | 碱基组成 | 实例}
  \begin{figure}
    \centering
    \includegraphics[width=11cm]{base.ratio.01.png}
  \end{figure}
\end{frame}

\begin{frame}
  \frametitle{DNA序列 | GC含量}
  \begin{block}{GC含量(GC content)}
    \begin{itemize}
      \item 对象:核酸片段、基因、基因组、……
      \item 鸟嘌呤(G)和胞嘧啶(C)所占的比例
      \item GC含量随DNA不同而异
      \item GC含量高的DNA更加稳定
      \item 计算公式:$\frac{G+C}{A+T+G+C}\times100$
      \item GC比(GC-ratio):$\frac{A+T}{G+C}$
      \item 结合滑动窗口进行计算
    \end{itemize}
  \end{block}
\end{frame}

\begin{frame}
  \frametitle{DNA序列 | GC含量 | 分析}
  \begin{block}{特点}
  \begin{itemize}
    \item 不同物种基因组中GC含量不同。(15\%~75\%,两头少中间多。疟原虫为20\%,啤酒酵母为38\%,人约为40\%,天蓝色链霉菌A3为72\%。)
    \item 同一基因组内,GC含量不均匀。
    \item GC含量与多种生物学特征相关,比如基因密度、内含子、外显子等。
  \end{itemize}
\end{block}
\pause
\begin{block}{应用}
  \begin{itemize}
    \item 根据GC含量差异识别细菌种类
    \item 真核基因组具有GC含量较高或较低的近似均匀片段
    \item 不同物种的密码子使用与其GC含量有关 
    \item GC含量与DNA双链的熔解温度有关,是进行核酸杂交的重要参数
  \end{itemize}
\end{block}
\end{frame}

\begin{frame}
  \frametitle{DNA序列 | GC含量 | 基因组}
  \begin{figure}
    \centering
    \includegraphics[width=9.5cm]{genomeGC.jpg}
  \end{figure}
  %{\tiny Genome evolution in bacterial endosymbionts of insects. Jennifer J. Wernegreen. Nature Reviews Genetics 3, 850-861 (November 2002). doi:10.1038/nrg931}
\end{frame}

\begin{frame}
  \frametitle{DNA序列 | GC含量 | 基因区}
  \begin{figure}
    \centering
    \includegraphics[width=12cm]{exonGC.png}
  \end{figure}
\end{frame}

\begin{frame}
  \frametitle{DNA序列 | GC含量 | 基因 vs. 基因组}
  \begin{figure}
    \centering
    \includegraphics[width=11cm]{geneGC.png}
  \end{figure}
\end{frame}

\begin{frame}
  \frametitle{DNA序列 | GC含量 | 滑动窗口}
  \begin{figure}
    \centering
    \includegraphics[width=9cm]{gc.sw.png}
  \end{figure}
\end{frame}

\begin{frame}
  \frametitle{DNA序列 | \alert{实例与策略}}
  \pause
  \begin{block}{任务分析}
    \begin{itemize}
      \item 序列长短
      \item 序列数目
      \item 任务数量
      \item 任务频率
      \item 工作时间
      \item \ldots
    \end{itemize}
  \end{block}
\end{frame}

\begin{frame}
  \frametitle{DNA序列 | 序列转换}
  \begin{block}{序列转换}
    \begin{itemize}
      \item 反向序列
      \item 互补序列
      \item 反向互补序列
      \item DNA双链
      \item RNA序列
    \end{itemize}
  \end{block}
  \pause
  \begin{block}{书写惯例}
    \begin{itemize}
      \item DNA/RNA:[左] 5' $\Longrightarrow$ 3' [右]
      \item 多肽/蛋白质:[左] N端(氨基端)$\Longrightarrow$ C端(羧基端) [右]
    \end{itemize}
  \end{block}
\end{frame}

\begin{frame}
  \frametitle{DNA序列 | N联核苷酸}
  \begin{block}{简介}
    N(2,3,4,……)个连续出现的核苷酸,也叫\textit{k}-mer。
  \end{block}
  \pause
  \begin{block}{常见}
    \begin{itemize}
      \item 二联核苷酸:$4 \times 4 = 16$
      \item 三联核苷酸:$4 \times 4 \times 4 = 64$
      \item 三联核苷酸 $\Longrightarrow$ 密码子
    \end{itemize}
  \end{block}
\end{frame}

\begin{frame}
  \frametitle{DNA序列 | N联核苷酸 | 密码子}
  \begin{figure}
    \centering
    \includegraphics[width=9cm]{codon.04.png}
  \end{figure}
\end{frame}

\begin{frame}
  \frametitle{DNA序列 | N联核苷酸 | 密码子}
  \begin{itemize}
    \item 密码子(codon):编码多肽链中某氨基酸(共20种)的三联核苷酸(共64种)
    \item 密码子的简并(degeneracy):每种氨基酸(M、W除外)都对应2种以上的密码子,最多有6种
    \item 密码子使用偏好性(codon usage bias):不同物种、不同个体、不同基因中,同义密码子用法(如出现频率等)存在差异
    \item 蛋白三级结构、功能与密码子用法有关
    \item 对于同一类型的基因,由物种引起的同义密码子使用偏好性的差异较小
    \item 密码子使用偏好性的分析:Codon Adaptation Index, CAI
  \end{itemize}
  \begin{figure}
    \centering
    \includegraphics[width=4cm]{cai.01.png}
    \\
    \includegraphics[width=7cm]{cai.02.png}
  \end{figure}
\end{frame}

\begin{frame}
  \frametitle{DNA序列 | 分析工具}
  \begin{itemize}
    \item \href{http://lh3lh3.users.sourceforge.net/fasta.shtml}{SeqTools}
    \item \href{http://www.cellbiol.com/scripts/complement/dna\_sequence\_reverse\_complement.php}{DNA Sequence Reverse and Complement Online Tool}
    \item \href{http://www.endmemo.com/bio/gc.php}{DNA/RNA GC Content Calculator}
    \item \href{http://www.sciencelauncher.com/oligocalc.html}{Oligo Calculator}
    \item \href{http://erilllab.umbc.edu/bioword-2/}{BioWord (A Microsoft Word add-in for biological sequence manipulation)}
    \item \href{http://www.bioinformatics.org/sms2/}{SMS2 (Sequence Manipulation Suite)}
    \item \href{http://emboss.sourceforge.net/}{EMBOSS (The European Molecular Biology Open Software Suite)}
    \item \ldots
    % \item \href{http://yixf.name/2011/06/01/\%E5\%AF\%B9fasta\%E6\%A0\%BC\%E5\%BC\%8F\%E7\%9A\%84\%E7\%AE\%80\%E5\%8D\%95\%E5\%A4\%84\%E7\%90\%86\%E4\%B8\%8E\%E7\%BB\%9F\%E8\%AE\%A1/}{SeqTools.pl}
    % \item \href{http://bioinfx.net/}{bioinfx(Free Online Tools for Bioinformatics)}
    % \item \href{http://clasher.myweb.uga.edu/testpages/seqconv.html}{Complementary Sequence Conversion Tool}
  \end{itemize}
\end{frame}

\begin{frame}
  \frametitle{DNA序列 | 透过表象看本质}
  \begin{itemize}
    \item 计数:计算字符出现的次数
    \item 反转:反转字符串
    \item 互补:字符替换
    \item 计算:简单的四则运算
  \end{itemize}
\end{frame}

\section{限制酶位点分析}
\begin{frame}
  \frametitle{限制酶 | 定义}
  \begin{block}{限制酶(restriction enzyme)}
    又称限制内切酶或限制性内切酶(restriction endonuclease),全称限制性核酸内切酶,是可以识别DNA的特异序列、并在识别位点或其周围切割双链DNA的一类内切酶。
  \end{block}
  \pause
  \begin{block}{切割末端}
    \begin{itemize}
      \item 黏性末端 vs. 平滑末端
    \end{itemize}
  \end{block}
  \begin{figure}
    \centering
    \visible<3->{\includegraphics[width=6cm]{ecori.cut.jpg}}
  \end{figure}
\end{frame}

\begin{frame}
  \frametitle{限制酶 | \alert{命名}}
  %\begin{block}{\textit{Eco}R\Rmnum{1}}
    %\begin{description}
      %\item[\textit{E}]属名\textit{Escherichia}
      %\item[\textit{co}]种名\textit{coli}
      %\item[R]RY13品系
      %\item[\Rmnum{1}]在此类细菌中的发现顺序
    %\end{description}
  %\end{block}
  \begin{figure}
    \centering
    \includegraphics[width=10cm]{ecori.png}
  \end{figure}
\end{frame}

\begin{frame}
  \frametitle{限制酶 | \alert{\Rmnum{2}型}}
  \begin{itemize}
    \item 识别、切割位点专一
    \item 识别序列:4-8个碱基,回文对称结构
    \item 切割序列:识别序列,切割位点对称
    \item 切割末端:黏性末端,平滑末端
    \item 黏性末端:切割位点在回文序列的一侧
    \item 平滑末端:切割位点在回文序列的中间
  \end{itemize}
\end{frame}

\begin{frame}
  \frametitle{限制酶 | \Rmnum{2}型 | 回文}
  \begin{block}{《题金山寺》,北宋\textbullet 苏轼}
  \begin{columns}
    \column{0.5\textwidth}
潮随暗浪雪山倾,远浦渔舟钓月明。
桥对寺门松径小,槛当泉眼石波清。
迢迢绿树江天晓,霭霭红霞晚日晴。
遥望四边云接水,雪峰千点数鸥轻。
    \column{0.5\textwidth}
    \pause
轻鸥数点千峰雪,水接云边四望遥。
晴日晚霞红霭霭,晓天江树绿迢迢。
清波石眼泉当槛,小径松门寺对桥。
明月钓舟渔浦远,倾山雪浪暗随潮。
  \end{columns}
  \end{block}
  \pause
  \begin{block}{回文}
    \begin{itemize}
      \item 上海自来水来自海上
      \item 山东落花生花落东山
      \item 画上荷花和尚画
    \end{itemize}
  \end{block}
\end{frame}

\begin{frame}
  \frametitle{限制酶 | \Rmnum{2}型 | 回文对称}
  \begin{block}{回文对称(palindrome)}
    特指DNA的一种具有反向重复的结构。具有这种结构的DNA,其一条链从左向右读和另一条链从右向左读的序列是相同的。
  \end{block}
  \begin{figure}
    \centering
    \includegraphics[width=8cm]{palindrome.jpg}
  \end{figure}
\end{frame}

\begin{frame}
  \frametitle{限制酶 | \Rmnum{2}型 | 末端}
  \begin{figure}
    \centering
    \includegraphics[width=9cm]{ends.jpg}
  \end{figure}
\end{frame}

\begin{frame}
  \frametitle{限制酶 | \Rmnum{2}型 | 末端 | 黏性末端}
  \begin{figure}
    \centering
    \includegraphics[width=9cm]{enzyme1.png}
  \end{figure}
\end{frame}

\begin{frame}
  \frametitle{限制酶 | \Rmnum{2}型 | 末端 | 平滑末端}
  \begin{figure}
    \centering
    \includegraphics[width=9cm]{enzyme2.png}
  \end{figure}
\end{frame}

\begin{frame}
  \frametitle{限制酶 | 数据库与分析工具}
  \begin{itemize}
    \item REBASE:收录了限制酶的所有信息
    \item NEBCutter V2.0:产生DNA序列的酶切位点分析结果
  \end{itemize}
\end{frame}

\begin{frame}
  \frametitle{限制酶 | 透过表象看本质}
  \begin{block}{字符串搜索}
    已知限制酶识别位点的前提下,在DNA序列这个长的字符串中搜索识别位点对应的子序列这个短字符串。
  \end{block}
  \begin{figure}
    \centering
    \includegraphics[width=11cm]{enzyme3.jpg}
  \end{figure}
\end{frame}

\section{开放阅读框分析}
\begin{frame}
  \frametitle{开放阅读框}
  \begin{block}{开放阅读框(Open Reading Frame,ORF)}
    在给定的阅读框架中,不包含终止密码子的一串序列,是生物个体的基因组中可能作为蛋白质编码序列的部分,包含从5'端翻译起始密码子(AUG)到终止密码子(UAA、UAG、UGA)之间的一段编码蛋白质的碱基序列。
  \end{block}
\end{frame}

\begin{frame}
  \frametitle{开放阅读框 | 相位(frame)}
  \begin{figure}
    \centering
    \includegraphics[width=10cm]{orf.png}
  \end{figure}
\end{frame}

\begin{frame}
  \frametitle{开放阅读框 | \alert{ORF VS. CDS}}
  \pause
  \begin{itemize}
    \item 一个ORF对应一个候选的CDS(编码序列,Coding DNA Sequence)
    \item ORF:理论预测
    \item CDS:实验证实
    \item 分析DNA序列中的ORF是对该序列是否为CDS的初步判断
  \end{itemize}
\end{frame}

\begin{frame}
  \frametitle{开放阅读框 | 分析工具}
  \begin{itemize}
    \item 确定第一个AUG和终止密码子
    \item 原核生物:最长ORF法
    \item 真核生物:特征统计、模式识别、同源比对
    \item ORF Finder:NCBI的在线分析工具
  \end{itemize}
\end{frame}

\begin{frame}
  \frametitle{开放阅读框 | 最长ORF法 | 透过表象看本质}
  \begin{block}{字符串搜索}
  根据对应物种的密码子表,在给定的DNA序列中找到起始密码子,依次向后寻找终止密码子,并计算两者之间的距离,保留满足长度要求的或者最长的,即是最终预测的ORF。
  \end{block}
\end{frame}

\section{功能位点分析}
\begin{frame}
  \frametitle{功能位点 | 简介}
  \begin{block}{功能位点(functional site)}
    DNA序列中,除基因外,还包含其它信息,存放这些信息的DNA片段称为功能位点。它们与功能相关,是功能单元。又称功能序列(functional sequence)、序列模式/模体/基元/基序(motif)、信号(signal)等。如,启动子(promoter)、基因终止序列(terminator sequence)、剪切位点(splice site)等。
  \end{block}
\end{frame}

\begin{frame}
  \frametitle{功能位点 | motif}
  \begin{figure}
    \centering
    \includegraphics[width=0.8\textwidth]{motif.jpg}
  \end{figure}
\end{frame}

\begin{frame}
  \frametitle{功能位点 | 共有序列}
  \begin{block}{共有序列(consensus sequence)}
    又称一致性片段,描述了功能位点每个位置上进化的保守性。例如:NTATN。
  \end{block}
  \pause
  \begin{block}{共有序列的局限}
    \begin{itemize}
      \item 关于序列特征的一种定性描述
      \item 能说明每个位置可能出现的碱基类型,但不能准确说明各碱基出现的可能性
    \end{itemize}
  \end{block}
  \pause
  \begin{block}{共有序列与功能位点}
    \begin{enumerate}
      \item 构造共有序列。
      \item 利用共有序列在给定的核酸序列上搜寻功能位点,并计算所找到的功能位点的可靠性。
    \end{enumerate}
  \end{block}
\end{frame}

\begin{frame}
  \frametitle{功能位点 | 共有序列 | 构造}
  \begin{enumerate}
    \item 初始化共有序列为一系列可变位置,以“N”代表。
    \item 在可变位置寻找出现次数最多的核苷酸,并将该位置转化为保守位置。
    \item 对当前所得到的共有序列进行特异性检查,若通过检查,转(5),否则转(4)。
    \item 形成与当前共有序列一致的位点子集,转(2)。
    \item 从原位点集合中删除与当前共有序列一致的位点,若还有剩余位点,则转(1),构造另外的共有序列。
  \end{enumerate}
\end{frame}

\begin{frame}
  \frametitle{功能位点 | 共有序列 | 构造}
  \begin{figure}
    \centering
    \includegraphics[width=11cm]{consensus.png}
  \end{figure}
\end{frame}

\begin{frame}
  \frametitle{功能位点 | 加权矩阵}
  \begin{block}{加权矩阵}
    用权系数(weight coefficient)描述功能位点各位置上每种核苷酸的相对重要性。加权矩阵的大小为4 $\times$ n(碱基种类数目 $\times$ 功能位点长度)。矩阵的每一个元素M(a,n)的值代表第a种核苷酸在功能位点第n个位置上出现的得分。其中,a $\in$ \{A,T,G,C\}。
  \end{block}
  \begin{figure}
    \centering
    \includegraphics[width=8cm]{site.matrix.png}
  \\ 对于ATTGCA来说,得分$W = 1 + 6 + 14 - 5 + 8 + 19 = 43$。
  \end{figure}
\end{frame}

\begin{frame}
  \frametitle{功能位点 | PWM}
  \begin{block}{PWM}
A position weight matrix (PWM), also known as a position-specific weight matrix (PSWM) or position-specific scoring matrix (PSSM), is a commonly used representation of motifs (patterns) in biological sequences.\\
\vspace{0.5em}
PWMs are often derived from a set of aligned sequences that are thought to be functionally related and have become an important part of many software tools for computational motif discovery.
  \end{block}
\end{frame}

\begin{frame}
  \frametitle{功能位点 | PWM}
  \begin{figure}
    \centering
    \includegraphics[width=0.9\textwidth]{motif.pwm.01.jpg}
  \end{figure}
\end{frame}

\section{启动子分析}
\begin{frame}
  \frametitle{启动子 | 转录调控}
  \begin{itemize}
    \item 顺式作用元件(cis-acting element):核酸序列
      \begin{itemize}
        \item 启动子(promoter)
        \item 增强子(enhancer)
        \item \ldots
      \end{itemize}
    \item 反式作用因子(trans-acting factor):蛋白质
    \item 两者相互作用实现转录调控
  \end{itemize}
\end{frame}

\begin{frame}
  \frametitle{启动子 | 定义}
  \begin{block}{启动子(promoter)}
    一段位于转录起始位点5'端上游区的DNA序列,能活化RNA聚合酶,使之与模板DNA准确地结合并具有转录起始的特异性。
  \end{block}
  \pause
  \begin{block}{转录起始位点(Transcription Start Site,TSS)}
    与新生RNA链第一个核苷酸相对应DNA链上的碱基,研究证实通常为一个嘌呤。
  \end{block}
  \visible<3->{
  \begin{figure}
    \centering
    \includegraphics[width=12cm]{tss.png}
  \end{figure}
  }
\end{frame}

\begin{frame}
  \frametitle{启动子 | 结构}
  \begin{figure}
    \centering
    \includegraphics[width=10cm]{promoter.jpg}
  \end{figure}
\end{frame}

\begin{frame}
  \frametitle{启动子 | TF\&TFBS}
  \begin{block}{转录因子(transcription factor)}
    能够结合在某基因上游特异核苷酸序列上的蛋白质,这些蛋白质能调控其基因的转录。
  \end{block}
  \pause
  \begin{block}{转录因子结合位点(Transcription Factor Binding Site,TFBS)}
    与转录因子结合的DNA序列,长度约为5~20bp,它们与转录因子相互作用进行基因的转录调控。
  \end{block}
\end{frame}

\begin{frame}
  \frametitle{启动子 | TFBS}
  \begin{figure}
    \centering
    \includegraphics[width=10cm]{tfbs.png}
  \end{figure}
\end{frame}

\begin{frame}
  \frametitle{启动子 | 数据库与分析工具}
  \begin{itemize}
    \item 启动子
      \begin{itemize}
	\item EPD:有注释、非冗余的真核生物RNA聚合酶\Rmnum{2}启动子数据集
        \item Promoter Scan(同源性分析),Promoter 2.0(人工神经网络技术)
      \end{itemize}
    \item 转录因子
      \begin{itemize}
        \item TRANSFAC:真核生物顺式作用元件和反式作用因子数据库
        \item Tfblast(TRANSFAC BLAST)
        \item JASPAR: The high-quality transcription factor binding profile database
        \item CIS-BP Database: Catalog of Inferred Sequence Binding Preferences
        \item footprintDB
        \item HOCOMOCO: expansion and enhancement of the collection of transcription factor binding sites models
        % \item HOmo sapiens COmprehensive MOdel COllection (HOCOMOCO) v10 provides transcription factor (TF) binding models for 600 human and 395 mouse TFs.
        \item MotifMap: genome-wide maps of regulatory elements.
        \item UniPROBE (Universal PBM Resource for Oligonucleotide Binding Evaluation) database
        \item ENCODE TF ChIP-seq datasets
        \item Human Protein-DNA Interactome (hPDI)
      \end{itemize}
  \end{itemize}
\end{frame}

\begin{frame}
  \frametitle{启动子 | 透过表象看本质}
  \begin{block}{字符串搜索}
    \begin{itemize}
      \item 在特定范围内进行搜索
      \item 子字符串由权重不等的一组字符串构成
    \end{itemize}
  \end{block}
  \pause
  \begin{figure}
    \centering
    \includegraphics[width=9cm]{pwm.png}
  \end{figure}
\end{frame}

%\begin{frame}
  %\frametitle{启动子 | 透过表象看本质}
  %\begin{figure}
    %\centering
    %\includegraphics[width=12cm]{physbinder.jpg}
  %\end{figure}
%\end{frame}

\begin{frame}
  \frametitle{启动子 | 透过表象看本质 | Promoter 2.0}
  \begin{figure}
    \centering
    \includegraphics[width=12cm]{promoter2.png}
  \end{figure}
\end{frame}

\section{CpG岛识别}
\begin{frame}
  \frametitle{CpG岛 | 特征}
  \begin{block}{\alert{CpG岛}}
    在基因组的某些区段,CpG保持或高于正常概率,这些区段被称作CpG岛(CpG island)。
  \end{block}
  \pause
  \begin{block}{特征}
    \begin{itemize}
      \item 几乎看家基因都含有CpG岛(人类和小鼠分别有55.9\%和46.9\%的基因与CpG岛有密切关联)
      \item 一般位于基因的5'端区域(转录起始位点附近,有助于基因的识别),长度约300~3000bp
      \item 大多数CpG岛是未甲基化的,未甲基化CpG岛说明基因可能具有潜在活性(表观遗传学中重要的作用区域,甲基化异常常常伴随着疾病的发生)
      \item CpG岛中的核小体中H1含量低,其他组蛋白被广泛乙酰化,并具有超敏感位点
    \end{itemize}
  \end{block}
\end{frame}

\begin{frame}
  \frametitle{CpG岛 | \alert{识别依据与判别标准}}
  \begin{enumerate}
    \item CpG岛长度:至少200bp
    \item GC含量:超过50\%
    \item CpG的观察值与预测值的比率:高于60\%
      \begin{itemize}
        \item $\frac{Num\ of\ CpG}{Num\ of\ C \times Num\ of\ G} \times Total\ number\ of\ nucleotides\ in\ the\ sequence$
      \end{itemize}
    \pause
    \item 500bp,55\%,65\%
  \end{enumerate}
\end{frame}

\begin{frame}
  \frametitle{CpG岛 | 分析工具}
  \begin{itemize}
    \item EMBOSS中的CpGPlot/CpGReport/Isochore
    \item CpG Island Searcher
    \item CpGcluster2
  \end{itemize}
\end{frame}

\begin{frame}
  \frametitle{CpG岛 | 透过表象看本质}
  \begin{itemize}
    \item 分段:使用滑动窗口将长序列分段
    \item 计数:长度、G和C
    \item 计算:含量、比率
    \item 比较:和标准进行比较
  \end{itemize}
  \begin{figure}
    \centering
    \includegraphics[width=9cm]{sliding.window.seq.png}
  \end{figure}
\end{frame}

\begin{frame}
  \frametitle{CpG岛 | 透过表象看本质}
  \begin{figure}
    \centering
    \includegraphics[width=11cm]{cpg.hmm.png}
  \end{figure}
\end{frame}

\section{EMBOSS}
\begin{frame}
  \frametitle{EMBOSS | 简介}
  \begin{block}{简介}
    EMBOSS(The European Molecular Biology Open Software Suite)是一个开源、免费的序列分析软件包,整合了目前可以获得的大部分序列分析软件。

    使用EMBOSS,可以将系列分析工作进行无缝整合,弥补了许多软件功能分散、分析效率低下的缺陷。
  \end{block}
  \begin{block}{使用}
    \begin{itemize}
      \item 操作系统:Linux,Mac,\textcolor{gray}{Windows}
      \item 界面:JEMBOSS(Java),EMBOSS Explorer(Web)
    \end{itemize}
  \end{block}
\end{frame}

\begin{frame}
  \frametitle{EMBOSS | 主要程序}
  \begin{itemize}
    \item 最重要的程序。wossname:根据关键字查找程序;showdb:显示所有整合的数据库。
    \item 序列编辑。revseq:将序列反转并互补;seqret:序列格式转换。
    \item 两个序列相似性图形表达。dottup:精确匹配;dotmatcher:近似匹配。
    \item 双序列比对。needle:全局比对;water:局部比对。
    \item 多序列比对。emma:clustalW。
    \item 寻找SNP。deffseq:仅限于双序列比对中。
    \item 其他。plotorf,getorf:翻译;iep:等电点预测;tmap:跨膜区预测;pepinfo:蛋白质性质;patmatmotifs:Motif搜索。
  \end{itemize}
\end{frame}

\begin{frame}
  \frametitle{EMBOSS | 演示}
  \begin{block}{组份分析}
  \begin{itemize}
    \item compseq: Calculate the composition of unique words in sequences
    \item geecee: Calculate fractional GC content of nucleic acid sequences
    \item revseq: Reverse and complement a nucleotide sequence
  \end{itemize}
  \end{block}
  \begin{block}{CpG岛分析}
  \begin{itemize}
    \item extractseq: Extract regions from a sequence
    \item cpgplot: Identify and plot CpG islands in nucleotide sequence(s)
    \item cpgreport: Identify and report CpG-rich regions in nucleotide sequence(s)
    \item isochore: Plot isochores in DNA sequences
  \end{itemize}
\end{block}
\end{frame}

\section{序列分析中的算法}
\begin{frame}
  \frametitle{序列分析中的算法 | 滑动窗口(Sliding Window)}
  \begin{block}{参数}
    \begin{itemize}
      \item window size:窗口大小
      \item step:步长
    \end{itemize}
  \end{block}
  \begin{figure}
    \centering
    \includegraphics[width=7cm]{sw.01.jpg}
  \end{figure}
\end{frame}

\begin{frame}
  \frametitle{序列分析中的算法 | 滑动窗口 | 窗口大小}
  \begin{figure}
    \centering
    \includegraphics[width=11cm]{sw.window.size.png}
  \end{figure}
\end{frame}

\begin{frame}
  \frametitle{序列分析中的算法 | 机器学习}
  \begin{block}{机器学习}
    机器学习理论主要是设计和分析一些让计算机可以自动“学习”的算法。机器学习算法是一类从数据中自动分析获得规律,并利用规律对未知数据进行预测的算法。
  \end{block}
  \pause
  \begin{block}{分类}
    监督学习,无监督学习,半监督学习,增强学习
  \end{block}
  \pause
  \begin{block}{算法}
    人工神经网络,决策树,线性判别分析,最近邻居法,支持向量机,……
  \end{block}
\end{frame}

\begin{frame}
  \frametitle{序列分析中的算法 | 机器学习}
  \begin{figure}
    \centering
    \includegraphics[width=\textwidth]{ml.01.jpg}
  \end{figure}
\end{frame}

\begin{frame}
  \frametitle{序列分析中的算法 | 机器学习}
  \begin{block}{machine learning}
    The classical machine learning workflow can be broken down into four steps: data pre-processing, feature extraction, model learning and model evaluation.\\
    \vspace{0.5em}
    Supervised machine learning methods relate input features x to an output label y, whereas unsupervised method learns factors about x without observed labels.\\
    \vspace{0.5em}
    Raw input data are often high-dimensional and related to the corresponding label in a complicated way, which is challenging for many classical machine learning algorithms. Alternatively, higher-level features extracted using a deep model may be able to better discriminate between classes.\\
    \vspace{0.5em}
    Deep networks use a hierarchical structure to learn increasingly abstract feature representations from the raw data.
  \end{block}
\end{frame}

\begin{frame}
  \frametitle{序列分析中的算法 | 机器学习 | 聚类分析}
  \begin{block}{聚类分析(cluster analysis)}
    聚类是把相似的对象通过静态分类的方法分成不同的组别或者更多的子集(subset),这样让在同一个子集中的成员对象都有相似的一些属性。
  \end{block}
  \begin{figure}
    \centering
    \includegraphics[width=7cm]{cluster.01.jpg}
    \includegraphics[width=5cm]{cluster.02.png}
  \end{figure}
\end{frame}

\begin{frame}
  \frametitle{序列分析中的算法 | 机器学习 | 判别分析}
  \begin{block}{判别分析}
    线性判别分析(Linear Discriminant Analysis),简称判别分析,是统计学上的一种分析方法,用于在已知的分类之下遇到有新的样本时,选定一个判别标准,以判定如何将新样本放置于哪一个类别之中。
  \end{block}
  \begin{figure}
    \centering
    \includegraphics[width=5cm]{da.01.jpg}
    \includegraphics[width=5.5cm]{da.03.png}
  \end{figure}
\end{frame}

\begin{frame}
  \frametitle{序列分析中的算法 | 机器学习 | 人工神经网络}
  \begin{block}{人工神经网络(artificial neural network,ANN)}
   简称神经网络(neural network,NN),是一种模仿生物神经网络的结构和功能的数学模型或计算模型。神经网络由大量的人工神经元联结进行计算。大多数情况下人工神经网络能在外界信息的基础上改变内部结构,是一种自适应系统。现代神经网络是一种非线性统计性数据建模工具,常用来对输入和输出间复杂的关系进行建模,或用来探索数据的模式。
  \end{block}
  \begin{figure}
    \centering
    \includegraphics[width=7.5cm]{ann.01.jpg}
  \end{figure}
\end{frame}

\begin{frame}
  \frametitle{序列分析中的算法 | 机器学习 | 深度学习}
  \begin{block}{深度学习}
    深度学习(deep learning)是机器学习拉出的分支,它试图使用包含复杂结构或由多重非线性变换构成的多个处理层对数据进行高层抽象的算法。\\
    \vspace{0.5em}
深度学习是机器学习中一种基于对数据进行表征学习的方法。深度学习的好处是用非监督式或半监督式(Semi-supervised learning)的特征学习和分层特征提取高效算法来替代手工获取特征(Feature)。\\
    \vspace{0.5em}
至今已有数种深度学习框架,如深度神经网络、卷积神经网络和深度置信网络(Deep belief network)和递归神经网络已被应用于计算机视觉、语音识别、自然语言处理、音频识别与生物信息学等领域并获取了极好的效果。
  \end{block}
\end{frame}

\begin{frame}
  \frametitle{序列分析中的算法 | 机器学习 | 深度学习}
  \begin{figure}
    \centering
    \includegraphics[width=\textwidth]{dl.01.png}
  \end{figure}
\end{frame}

\begin{frame}
  \frametitle{序列分析中的算法 | 机器学习 | 隐马尔可夫模型}
  \begin{block}{马尔可夫性质}
    当一个随机过程在给定现在状态及所有过去状态情况下,其未来状态的条件概率分布仅依赖于当前状态;换句话说,在给定现在状态时,它与过去状态(即该过程的历史路径)是条件独立的,那么此随机过程即具有马尔可夫性质。
  \end{block}
  \pause
  \begin{block}{马尔可夫过程(Markov process)}
    一个具备了马尔可夫性质的随机过程。马尔可夫过程是不具备记忆特质的。换言之,马可夫过程的条件概率仅仅与系统的当前状态相关,而与它的过去历史或未来状态,都是独立、不相关的。
  \end{block}
\end{frame}

\begin{frame}
  \frametitle{序列分析中的算法 | 机器学习 | 隐马尔可夫模型}
  \begin{block}{马尔可夫链(Markov chain)}
    具备离散状态的马尔可夫过程,通常被称为马尔可夫链。又称离散时间马尔可夫链(discrete-time Markov chain,DTMC),是马尔可夫过程中的一个特例,为具备马尔可夫性质与离散时间状态的随机过程。该过程中,在给定当前知识或信息的情况下,只有当前的状态用来预测将来,过去(即当前以前的历史状态)对于预测将来(即当前以后的未来状态)是无关的。
  \end{block}
  \pause
  \begin{block}{隐马尔可夫模型(Hidden Markov Model,HMM)}
    统计模型,用来描述一个含有隐含未知参数的马尔可夫过程。其难点是从可观察的参数中确定该过程的隐含参数。然后利用这些参数来作进一步的分析,例如模式识别。
  \end{block}
\end{frame}

\begin{frame}
  \frametitle{序列分析中的算法 | 机器学习 | 隐马尔可夫模型}
  \begin{figure}
    \centering
    \includegraphics[width=5.5cm]{markov.chain.png}
    \quad
    \includegraphics[width=6cm]{hmm.model.png}
  \end{figure}
\end{frame}

\begin{frame}
  \frametitle{序列分析中的算法 | 机器学习 | 隐马尔可夫模型}
  \begin{figure}
    \centering
    \includegraphics[width=10cm]{hmm.01.png}
  \end{figure}
\end{frame}

\begin{frame}
  \frametitle{序列分析中的算法 | 机器学习 | 隐马尔可夫模型}
  \begin{block}{三个典型问题}
    \begin{enumerate}
      \item 已知模型参数,计算某一特定输出序列的概率。通常使用forward算法解决。
      \item 已知模型参数,寻找最可能的能产生某一特定输出序列的隐含状态的序列。通常使用Viterbi算法解决。
      \item 已知输出序列,寻找最可能的状态转移以及输出概率。通常使用Baum-Welch算法以及Reversed Viterbi算法解决。
    \end{enumerate}
  \end{block}
最近的一些方法使用Junction tree算法来解决这三个问题。
\end{frame}

\begin{frame}
  \frametitle{序列分析中的算法 | 机器学习 | 隐马尔可夫模型}
  \begin{figure}
    \centering
    \includegraphics[width=6.5cm]{hmm.03.png}
    \includegraphics[width=5.5cm]{hmm.04.png}
  \end{figure}
\end{frame}

\begin{frame}
  \frametitle{序列分析中的算法 | 机器学习 | 隐马尔可夫模型}
  \begin{figure}
    \centering
    \includegraphics[width=10.5cm]{hmm.04.jpg}
  \end{figure}
\end{frame}

\section{总结与答疑}
\begin{frame}
  \frametitle{总结与答疑}
  \begin{block}{知识点——DNA序列的基本信息与特征信息分析}
    \begin{itemize}
      \item DNA序列基本信息分析——查戈夫法则,GC含量,序列转换
      \item 限制酶位点分析——命名,\Rmnum{2}型的特点
      \item 开放阅读框分析——相位,ORF与CDS
      \item 启动子与转录因子结合位点分析——启动子结构
      \item CpG岛识别——概念、判别依据及标准
    \end{itemize}
  \end{block}
  \begin{block}{技能——解决问题的思路}
    \begin{itemize}
      \item 首先分析任务的属性
      \item 寻找可能的解决方案
      \item 确定最合适的方法
      \item 先易后难,由浅入深
    \end{itemize}
  \end{block}
\end{frame}
