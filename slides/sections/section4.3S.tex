\section{引言}
\begin{frame}
  \frametitle{引言 | 回顾}
  \begin{itemize}[<+-|alert@+>]
    \item DNA序列分析
      \begin{itemize}
        \item 基本信息
        \item 序列特征
        \item 基因识别
      \end{itemize}
    \item RNA序列分析
      \begin{itemize}
        \item mRNA选择性剪接
        \item miRNA与靶基因
      \end{itemize}
  \end{itemize}
\end{frame}

\begin{frame}
  \frametitle{引言 | RNA}
  \begin{enumerate}
    \item 编码RNA
      \begin{itemize}
        \item mRNA
      \end{itemize}
    \item 非编码RNA
      \begin{itemize}
        \item tRNA、rRNA
        \item miRNA、siRNA、lncRNA
      \end{itemize}
  \end{enumerate}
  \begin{figure}
    \centering
    \includegraphics[width=0.8\textwidth]{RNA_class.jpg}
  \end{figure}
\end{frame}

\begin{frame}
  \frametitle{引言 | RNA}
  \begin{figure}
    \centering
    \includegraphics[width=10cm]{rnaC.jpg}
  \end{figure}
\end{frame}

\begin{frame}
  \frametitle{引言 | RNA | ncRNA}
  \begin{figure}
    \centering
    \includegraphics[width=12cm]{ncrnaC.jpg}
  \end{figure}
\end{frame}

\section{mRNA选择性剪接}
\begin{frame}
  \frametitle{选择性剪接 | \alert{剪接与选择性剪接}}
  \begin{block}{剪接(splicing)}
    又称拼接,指基因信息在转录后的一种修饰,即将内含子移除及合并外显子,是真核生物的信使RNA前体(precursor messenger RNA)变成成熟mRNA的过程之一。
  \end{block}
  \pause
  \begin{block}{选择性剪接(alternative splicing)}
    又称可变剪接,指用不同的剪接方式(选择不同的剪接位点组合)从一个mRNA前体产生不同的mRNA剪接异构体的过程。
  \end{block}
\end{frame}

\begin{frame}
  \frametitle{选择性剪接 | 剪接}
  \begin{figure}
    \centering
    \includegraphics[width=11cm]{splicing_10.png}
  \end{figure}
\end{frame}

\begin{frame}
  \frametitle{选择性剪接 | 模式图}
  \begin{figure}
    \centering
    \includegraphics[width=8cm]{splicing_01.jpg}
  \end{figure}
\end{frame}

\begin{frame}
  \frametitle{选择性剪接 | 模式图}
  \begin{figure}
    \centering
    \includegraphics[width=12cm]{splicing_10.png}
  \end{figure}
\end{frame}

\begin{frame}
  \frametitle{选择性剪接 | 实例}
  \begin{figure}
    \centering
    \includegraphics[width=12cm]{as_examples.jpg}
  \end{figure}
\end{frame}

\begin{frame}
  \frametitle{选择性剪接 | 实例 | \textit{Dscam}}
  \begin{figure}
    \centering
    \includegraphics[width=11cm]{as_dscam_01.jpg}
    \\ \Large{$12 \times 48 \times 33 \times 2 = $}
    \pause
    \Large{$38016$}
  \end{figure}
\end{frame}

\begin{frame}
  \frametitle{选择性剪接 | 调控}
  \begin{itemize}
    \item 剪接因子与调节蛋白相互作用
    \item 剪接体的核心部分包括一组小核RNA(snRNA)以及与之结合的蛋白质,它们以严格的程序组装成剪接体
      \begin{itemize}
	\item snRNA成员分别为U1、U2、U4、U5和U6,长度在106(U6)~185(U2)个核苷酸之间
	\item snRNA与蛋白质结合在一起形成小核核糖核蛋白(snRNP)
      \end{itemize}
    \item 剪接因子依据结合在RNA上的位置及作用方式,可以分为
      \begin{itemize}
	\item 外显子增强子(exonic splicing enhancer,ESE)
	\item 外显子抑制子(exonic splicing silencer,ESS)
	\item 内含子增强子(intronic splicing enhancer,ISE)
	\item 内含子抑制子(intronic splicing silencer,ISS)
      \end{itemize}
  \end{itemize}
\end{frame}

\begin{frame}
  \frametitle{选择性剪接 | 调控}
  \begin{figure}
    \centering
    \includegraphics[width=12cm]{splicing_regulation.jpg}
  \end{figure}
\end{frame}

\begin{frame}
  \frametitle{选择性剪接 | \alert{机制} | 五种}
  \begin{figure}
    \centering
    \includegraphics[width=10cm]{splicingModel5.jpg}
  \end{figure}
\end{frame}

\begin{frame}
  \frametitle{选择性剪接 | \alert{机制} | 七种}
  \begin{figure}
    \centering
    \includegraphics[width=9cm]{splicingModel7.jpg}
  \end{figure}
\end{frame}

\begin{frame}
  \frametitle{选择性剪接 | 机制 | 实例}
  \begin{figure}
    \centering
    \includegraphics[width=10cm]{splicingExample10.png}
  \end{figure}
\end{frame}

\begin{frame}
  \frametitle{选择性剪接 | 机制 | 使用频率}
  \begin{figure}
    \centering
    \includegraphics[width=11cm]{splice_types_frequency.jpg}
  \end{figure}
\end{frame}

\begin{frame}
  \frametitle{选择性剪接 | 数据库与分析工具}
  \begin{block}{两大类数据库(依数据来源)}
    \begin{itemize}
      \item 基于文献报道(收集整理实验数据和文献报道)
      \item 基于EST数据(EST与基因组或DNA、mRNA比对)
    \end{itemize}
  \end{block}
  \pause
  \begin{block}{数据库与工具}
  \begin{itemize}
    \item ASTD = ASD (= AEDB + AltExtron + AltSplice) + ATD
    \item \textcolor{gray}{ASAP}
    \item ESEfinder 
    \item RESCUE-ESE
    \item ASPicDB
  \end{itemize}
\end{block}
\end{frame}

\section{miRNA及其靶基因预测}

\begin{frame}
  \frametitle{miRNA | 简介}
  \begin{block}{微RNA(microRNAs,miRNA,小分子RNA)}
    归属小RNA范畴,是真核生物中广泛存在的一种长约20到24个核苷酸的内源性非编码单链RNA分子。
    miRNA通过RNA诱导沉默复合体(RISC)与靶基因的3'非翻译区(3' UTR)相结合,导致靶基因mRNA降解或者抑制其翻译,从而调节基因转录后的表达。
  \end{block}
\end{frame}

\begin{frame}
  \frametitle{miRNA | 特征}
  \begin{block}{序列}
    不具有开放阅读框,不编码蛋白质;成熟的miRNA 5'端为单一磷酸基团,3'端为羟基
  \end{block}
  \pause
  \begin{block}{表达}
    具有时序性和组织特异性
  \end{block}
  \pause
  \begin{block}{调控}
    miRNA与靶基因间呈多对多的关系
  \end{block}
  \pause
  \begin{block}{物理位置}
    倾向于成簇地出现在染色体上
  \end{block}
  \pause
  \begin{block}{进化}
    在物种间高度保守
  \end{block}
\end{frame}

\begin{frame}
  \frametitle{miRNA | 生成}
  \begin{figure}
    \centering
    \includegraphics[width=7cm]{mirna.png}
  \end{figure}
\end{frame}

\begin{frame}
  \frametitle{miRNA | 作用网络}
  \begin{figure}
    \centering
    \includegraphics[width=10cm]{mirnaN.jpg}
  \end{figure}
\end{frame}

\begin{frame}
  \frametitle{miRNA | 功能}
  \begin{figure}
    \centering
    \includegraphics[width=12cm]{mirnaF.png}
  \end{figure}
\end{frame}

\begin{frame}
  \frametitle{miRNA | 数据来源}
  \begin{itemize}
    \item 实验手段(cDNA克隆测序、miRNA-seq),有局限性。
    \item 计算预测,有优势。
  \end{itemize}
\end{frame}

\begin{frame}
  \frametitle{miRNA | \alert{预测}}
  \begin{enumerate}
    \item 同源片段搜索方法。将已知miRNA或pre-miRNA序列在自身或其他相近基因组中用比对算法搜索同源序列,结合序列二级结构特征进行筛选。(局限于:与已知miRNA/pre-miRNA在序列上和结构上同源的miRNA/pre-miRNA)
    \item 基于比较基因组学的预测方法。依据进化过程中的保守性在多物种中搜索潜在的miRNA。(能够找到不与已知miRNA同源的新miRNA;局限于:保守miRNA)
      \begin{itemize}
	\item 方法一:先在一个物种基因组中根据结构和序列特征找出可能的pre-miRNA,而后与其他物种基因组比较,判断其序列和结构是否保守
	\item 方法二:先通过比较两物种的基因组找出保守区域,而后在保守区域中根据结构和序列特征搜索可能的miRNA
      \end{itemize}
  \end{enumerate}
\end{frame}

\begin{frame}
  \frametitle{miRNA | \alert{预测}(续)}
  \begin{enumerate}
    \setcounter{enumi}{2}
    \item 基于序列和结构特征打分的预测方法。根据已知miRNA序列和结构的特征对全基因组范围内能形成茎环结构的片段进行筛选,是发现非同源、物种特异miRNA的方法。(为降低假阳性,用异常严格的序列和结构标准筛选候选片段,可能遗漏大量的miRNA)
    \item 结合作用靶标的预测方法。依据miRNA与其靶基因序列间的碱基互补配对的保守性的特点预测miRNA。
    \item 基于机器学习的预测方法。通过对阳性miRNA和阴性miRNA数据集的训练来构建区分两者的分类器,根据所得分类器对未知序列进行预测。(支持向量机SVM是目前miRNA分类和预测最常用的机器学习方法)
  \end{enumerate}
\end{frame}

\begin{frame}
  \frametitle{miRNA | 种子区域}
  \begin{figure}
    \centering
    \includegraphics[width=10cm]{mirnaS.png}
  \end{figure}
\end{frame}

\begin{frame}
  \frametitle{miRNA | \alert{靶基因预测}}
  \begin{enumerate}
    \item 基于种子区域互补和保守性的规则预测
      \begin{itemize}
        \item miRanda
        \item TargetScan
      \end{itemize}
    \item 基于机器学习方法训练参数进行靶基因预测
      \begin{itemize}
        \item PicTar
        \item miTarget
      \end{itemize}
  \end{enumerate}
\end{frame}

\begin{frame}
  \frametitle{miRNA | 数据库与分析工具}
  \begin{itemize}
    \item 数据库:miRBase、miRTarBase、miRWalk2.0、TarBase、miRGen
    \item miRNA预测:MiRscan、MiPred、miRFinder
    \item miRNA靶基因预测:miRanda、TargetScan、PicTar、miTarget
    \item \href{http://zh.wikipedia.org/wiki/\%E5\%BE\%AERNA\%E4\%B8\%8E\%E5\%BE\%AERNA\%E9\%9D\%B6\%E6\%95\%B0\%E6\%8D\%AE\%E5\%BA\%93}{微RNA与微RNA靶数据库(维基百科)}
  \end{itemize}
  \pause
  \begin{block}{\alert{强调}}
    \begin{itemize}
      \item miRBase是一个集miRNA序列、注释信息以及预测的靶基因数据为一体的数据库。
      \item miRTarBase: the experimentally validated microRNA-target interactions database
      \item miRWalk2.0: a comprehensive atlas of predicted and validated microRNA-target interactions
      % \item \textcolor{gray}{TarBase是一个存储已被实验证实的真实miRNA与靶基因间关系的数据库。}
      \item TarBase是一个存储已被实验证实的真实miRNA与靶基因间关系的数据库。
    \end{itemize}
  \end{block}
\end{frame}

\section{总结与答疑}
\begin{frame}
  \frametitle{总结与答疑}
  \begin{block}{知识点——mRNA选择性剪接和miRNA分析}
    \begin{itemize}
      \item mRNA选择性剪接——选择性剪接的主要机制,数据资源
      \item miRNA——miRNA的特点和作用机制,miRNA预测方法与工具,miRNA靶基因预测方法与工具
    \end{itemize}
  \end{block}
\end{frame}

\section{复习思考题}
\begin{frame}
  \frametitle{复习思考题}
  \begin{block}{知识点}
  \begin{enumerate}
    \item DNA序列携带哪两类遗传信息?可以对DNA序列进行哪些分析?
    \item 简述限制性核酸内切酶的命名规则及II型限制酶的主要特点。
    \item 简述CpG岛的概念及其识别依据和判别标准。
    \item 简述重复序列依重复次数和组织形式的分类。
    \item 简述基因识别的三大类方法及主要策略。
    \item 简述选择性剪接的产生机制。
    \item 简述miRNA预测和miRNA靶基因预测的方法。
  \end{enumerate}
\end{block}
\end{frame}

