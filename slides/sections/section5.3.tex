\section{引言}
\begin{frame}
  \frametitle{引言}
  \begin{block}{基础知识}
    \begin{itemize}
      \item 组装版本和坐标系统
      \item 常用格式
      \item 坐标的逻辑运算
    \end{itemize}
  \end{block}
  \pause
  \begin{block}{高级注释}
    \begin{itemize}
      \item 变异位点的注释
      \item 基因集富集分析
      \item 制作序列标识
    \end{itemize}
  \end{block}
  \pause
  \begin{block}{分析平台}
    \begin{itemize}
      \item Galaxy
      \item GenePattern
    \end{itemize}
  \end{block}
\end{frame}

\section{Galaxy分析平台}
\begin{frame}
  \frametitle{Galaxy | 工具集}
  \begin{itemize}
    \item Get Data
    \item Text Manipulation
    \item Convert Formats
    \item Operate on Genomic Intervals
    \item Phenotype Association
    \item Statistics
    \item Graph/Display Data
    \item NGS Toolbox
    \item \ldots
  \end{itemize}
\end{frame}

\begin{frame}
  \frametitle{Galaxy | 界面}
  \begin{center}
    \includegraphics[width=12cm]{galaxy.png}
  \end{center}
\end{frame}

\section{Galaxy使用演示}
%\begin{frame}
  %\frametitle{Galaxy演示 | \alert{坐标转换}}
  %\begin{itemize}
    %\item \href{http://genome.ucsc.edu/cgi-bin/hgLiftOver}{liftOver}:支持BED和“chrN:start-end”格式的输入
    %\item \href{https://usegalaxy.org/}{Galaxy中的liftOver}:支持BED、GFF和GTF格式的输入
    %\item \href{http://www.ncbi.nlm.nih.gov/genome/tools/remap}{NCBI Remap}:支持BED、GFF、GTF和VCF等格式的输入
    %\item \href{http://asia.ensembl.org/Homo\_sapiens/UserData/SelectFeatures}{Ensembl assembly converter}:支持BED、GFF、GFT和PSL格式的输入,但输出都是GFF格式的
    %\item \href{https://pypi.python.org/pypi/pyliftover}{pyliftover}:仅支持点坐标(point coordinates)的转换,无法对区段(ranges)坐标进行转换
  %\end{itemize}
%\end{frame}

\begin{frame}
  \frametitle{Galaxy演示 | \alert{坐标转换}}
  \begin{itemize}
    \item \href{http://genome.ucsc.edu/cgi-bin/hgLiftOver}{UCSC liftOver tool}:支持BED和“chrN:start-end”格式的输入
    \item \href{https://usegalaxy.org/}{Galaxy(基于UCSC liftOver tool)}:支持BED、GFF和GTF格式的输入
    \item \href{http://crossmap.sourceforge.net/}{CrossMap}:支持SAM/BAM、Wiggle/BigWig、BED、GFF/GTF和VCF格式的输入,输出对应格式
    \item \href{http://www.ncbi.nlm.nih.gov/genome/tools/remap}{NCBI Remap}:支持BED、GFF、GTF和VCF等格式的输入
    \item \href{http://asia.ensembl.org/Homo\_sapiens/UserData/SelectFeatures}{\textcolor{gray}{Ensembl assembly converter(2015年退休,CrossMap继位)}}\textcolor{gray}{:支持BED、GFF、GFT和PSL格式的输入,但输出都是GFF格式的}
    \item \href{https://pypi.python.org/pypi/pyliftover}{pyliftover}:仅支持点坐标(point coordinates)的转换,无法对区段(ranges)坐标进行转换
  \end{itemize}
\end{frame}

\begin{frame}
  \frametitle{Galaxy演示 | 坐标转换 | liftOver}
  \begin{block}{hg19 $\Rightarrow$ hg18}
    \pause
  \begin{enumerate}[<+-|alert@+>]
    \item 获取输入
      \begin{itemize}
        \item 输入文件:hg19坐标
      \end{itemize}
    \item 数据处理
      \begin{itemize}
        \item 设置参数:hg19 $\Rightarrow$ hg18
      \end{itemize}
    \item 保存输出
      \begin{itemize}
        \item 过滤结果:MAPPED VS. UNMAPPED
      \end{itemize}
  \end{enumerate}
\end{block}
\end{frame}

\begin{frame}
  \frametitle{Galaxy演示 | 格式转换 | BED $\Leftrightarrow$ GFF}
  \begin{block}{BED $\Leftrightarrow$ GFF}
    \pause
  \begin{enumerate}[<+-|alert@+>]
    \item 获取输入
      \begin{itemize}
        \item 输入文件:BED
      \end{itemize}
    \item 数据处理
      \begin{enumerate}
        \item BED $\Rightarrow$ GFF
        \item GFF $\Rightarrow$ BED
      \end{enumerate}
    \item 保存输出
      \begin{itemize}
        \item 查看结果:互相比较
      \end{itemize}
  \end{enumerate}
\end{block}
\end{frame}

\begin{frame}
  \frametitle{Galaxy演示 | \alert{逻辑运算}}
  \begin{itemize}
    \item \href{https://usegalaxy.org/}{Galaxy} 中的“Operate on Genomic Intervals”工具集
    \item \href{http://bedtools.readthedocs.org/en/latest/}{BEDTools}: a powerful toolset for genome arithmetic
    \item \href{https://bedops.readthedocs.org/en/latest/}{BEDOPS}: the fast, highly scalable and easily-parallelizable genome analysis toolkit
  \end{itemize}
\end{frame}

\begin{frame}
  \frametitle{Galaxy演示| 逻辑运算 | subtract \& join}
  \begin{block}{外显子 vs. SNP}
    \pause
  \begin{enumerate}[<+-|alert@+>]
    \item 获取输入
      \begin{itemize}
        \item exon
        \item SNP
      \end{itemize}
    \item 数据处理
      \begin{itemize}
        \item subtract
        \item join
      \end{itemize}
    \item 保存输出
      \begin{itemize}
        \item 解析结果
      \end{itemize}
  \end{enumerate}
\end{block}
\end{frame}

\begin{frame}
  \frametitle{Galaxy演示 | 综合应用 | 简单实例}
  \begin{block}{问题}
    \begin{itemize}
      \item 找到含有至少N(2)个SNP的外显子。
    \end{itemize}
  \end{block}
  \pause
  \begin{block}{输入}
    \begin{itemize}
      \item 外显子数据(BED格式)
      \item SNP数据(BED格式)
    \end{itemize}
  \end{block}
  \pause
  \begin{block}{输出}
    \begin{itemize}
      \item 满足要求的外显子信息(BED格式)
    \end{itemize}
  \end{block}
\end{frame}

\begin{frame}
  \frametitle{Galaxy演示 | 综合应用 | 简单实例}
  \begin{block}{外显子数据}
    \begin{enumerate}
      \item chr1 \qquad 10 \qquad 20 \qquad exon1 \qquad 0 \qquad +
      \item chr1 \qquad 30 \qquad 40 \qquad exon2 \qquad 0 \qquad +
      \item chr1 \qquad 50 \qquad 60 \qquad exon3 \qquad 0 \qquad -
      \item chr1 \qquad 65 \qquad 75 \qquad exon4 \qquad 0 \qquad +
      \item chr1 \qquad 85 \qquad 95 \qquad exon5 \qquad 0 \qquad -
    \end{enumerate}
  \end{block}
\end{frame}

\begin{frame}
  \frametitle{Galaxy演示 | 综合应用 | 简单实例}
  \begin{block}{SNP数据}
    \begin{enumerate}
      \item chr1 \qquad 11 \qquad 12 \qquad snp1 \qquad 0 \qquad +
      \item chr1 \qquad 15 \qquad 16 \qquad snp2 \qquad 0 \qquad +
      \item chr1 \qquad 17 \qquad 18 \qquad snp3 \qquad 0 \qquad +
      \item chr1 \qquad 24 \qquad 25 \qquad snp4 \qquad 0 \qquad +
      \item chr1 \qquad 33 \qquad 34 \qquad snp5 \qquad 0 \qquad +
      \item chr1 \qquad 37 \qquad 38 \qquad snp6 \qquad 0 \qquad +
      \item chr1 \qquad 44 \qquad 45 \qquad snp7 \qquad 0 \qquad +
      \item chr1 \qquad 54 \qquad 55 \qquad snp8 \qquad 0 \qquad +
      \item chr1 \qquad 57 \qquad 58 \qquad snp9 \qquad 0 \qquad -
    \end{enumerate}
  \end{block}
\end{frame}

\begin{frame}
  \frametitle{Galaxy演示 | \alert{综合应用} | Galaxy 101}
  \begin{block}{Reference}
  \href{https://galaxyproject.github.io/training-material/topics/introduction/tutorials/galaxy-intro-101/tutorial.html}{Galaxy 101}
  \end{block}
  \pause
  \begin{block}{Question}
    \begin{itemize}
      \item Which coding exon has the highest number of single nucleotide polymorphisms (SNPs) on human chromosome 22?
    \end{itemize}
  \end{block}
  \pause
  \begin{block}{Objectives}
    \begin{itemize}
      \item Familiarize yourself with the basics of Galaxy
      \item Learn how to obtain data from external sources
      \item Learn how to run tools
      \item Learn how histories work
      \item Learn how to create a workflow
      \item Learn how to share your work
    \end{itemize}
  \end{block}
\end{frame}

\begin{frame}
  \frametitle{Galaxy演示 | \alert{综合应用} | 策略一}
  \begin{block}{Finding exons with the highest number ($\geq$ 10) of SNPs}
    \pause
    \begin{block}{Join-Group-Filter-Compare-Sort}
    \pause
  \begin{enumerate}[<+-|alert@+>]
    \item Input: Get exons, SNPs; UCSC Table Browser
    \item Join[Operate on Genomic Intervals]: Join exons with SNPs
    \item Group: Count the number of SNPs per exon 
    \item Filter: Filter exons that have ten or more SNPs
    \item Compare two Datasets: Recover exon information
    \item Sort: Sort the start and end coordinates
    \item Visualize: Display data in genome browser
  \end{enumerate}
\end{block}
  \end{block}
\end{frame}

\begin{frame}
  \frametitle{Galaxy演示 | \alert{综合应用} | 策略二}
  \begin{block}{Finding exons with the highest number ($\geq$ 10) of SNPs}
    \pause
    \begin{block}{Join-Count-Filter-Cut-Sort}
    \pause
  \begin{enumerate}[<+-|alert@+>]
    \item Input: Get exons, SNPs; UCSC Table Browser
    \item Join[Operate on Genomic Intervals]: Join exons with SNPs
    \item Count: Count the number of SNPs per exon 
    \item Filter: Filter exons that have ten or more SNPs
    \item Cut: Cut columns to recover BED format
    \item Sort: Sort the start and end coordinates
    \item Visualize: Display data in genome browser
  \end{enumerate}
\end{block}
  \end{block}
\end{frame}

\begin{frame}
  \frametitle{Galaxy演示 | \alert{综合应用} | 策略三}
  \begin{block}{Finding 10 exons with the highest number of SNPs}
    \pause
    \begin{block}{Join-Group-Sort-SelectFirst-Join-Cut-Sort}
    \pause
  \begin{enumerate}[<+-|alert@+>]
    \item Input: Get exons, SNPs; UCSC Table Browser
    \item Join[Operate on Genomic Intervals]: Join exons with SNPs
    \item Group: Count the number of SNPs per exon 
    \item Sort: Sort exons by SNPs count
    \item Select first: Select top ten
    \item Join[Join two Datasets]: Recover exon information
    \item Cut: Cut columns to recover BED format
    \item Sort: Sort the start and end coordinates
    \item Visualize: Display data in genome browser
  \end{enumerate}
\end{block}
  \end{block}
\end{frame}

\begin{frame}[fragile]
  \frametitle{Galaxy演示 | 综合应用 | 策略四}
  \begin{block}{Finding exons with the highest number ($\geq$ 10) of SNPs}
    \pause
    \begin{block}{BEDTools-shell-BEDTools}
    \pause
    \begin{block}{考虑链性}
      \verb+bedtools intersect -a exon.bed -b snp.bed -c -s |+
      \verb+awk '{if($7>=10) print;}' | cut -f1-6 |+
      \verb+bedtools sort -i stdin+
    \end{block}
    \pause
    \begin{block}{不考虑链性}
      \verb+bedtools intersect -a exon.bed -b snp.bed -c |+
      \verb+awk '{if($7>=10) print;}' | cut -f1-6 |+
      \verb+bedtools sort -i stdin+
    \end{block}
\end{block}
  \end{block}
\end{frame}


\begin{frame}
  \frametitle{Galaxy演示 | 综合应用 | \alert{工作流}}
  \begin{block}{Workflow:create, modify, rerun, share}
    \pause
  \begin{enumerate}[<+-|alert@+>]
    \item Save: Rename the history as ``Exons and SNPs''
    \item Workflow: Extract workflow from history
    \item Modify: Open workflow editor and modify the parameter
    \item Rerun: Run workflow on whole genome data
    \item Share: Share or publish workflow
    \item Create: Create workflows from scratch (e.g. Find the 50 longest exons)
  \end{enumerate}
  \end{block}
\end{frame}

\section{数据处理三段论}
\begin{frame}
  \frametitle{三段论:“输入-加工-输出”}
  \begin{center}
    \includegraphics[width=8cm]{io1.png}
    \vspace{0.5cm}
    \includegraphics[width=6cm]{io2.jpg}
  \end{center}
\end{frame}

\begin{frame}
  \frametitle{三段论:“输入-加工-输出”}
  \begin{columns}
    \column{0.2\textwidth}
    \begin{block}{获取输入}
      数据来源\\ 文件格式\\ 过滤数据\\ \ldots
    \end{block}
    \column{0.2\textwidth}
    \begin{block}{加工处理}
      选择工具\\ 调整参数\\ 记录步骤\\ \ldots
    \end{block}
    \column{0.2\textwidth}
    \begin{block}{解析输出}
      文件格式\\ 解析图表\\ 保存结果\\ \ldots
    \end{block}
  \end{columns}
  \begin{figure}
    \centering
    \includegraphics[width=0.9\textwidth,height=3.8cm]{io3.png}
  \end{figure}
\end{frame}

\section{总结与答疑}
\begin{frame}
  \frametitle{总结与答疑}
  \begin{block}{知识点——Galaxy分析平台}
    \begin{itemize}
      \item Galaxy——界面、学习、使用
    \end{itemize}
  \end{block}
  \begin{block}{技能——“输入-加工-输出”三段论}
    \begin{itemize}
      \item 获取输入——格式、来源、过滤
      \item 数据处理——工具、版本、参数
      \item 解析输出——格式、注释、解析
    \end{itemize}
  \end{block}
\end{frame}

\section{复习思考题}
\begin{frame}
  \frametitle{复习思考题}
  \begin{block}{知识点}
    \begin{enumerate}
      \item hg19和mm10分别代表什么含义?hg19是和GRCh37相对应,还是和GRCm38相对应?
      \item 常见的基因组坐标系统是哪两种,举例进行说明。
      \item 简述BED格式前6列的含义,能解释实际的BED记录。
      \item 基于基因组坐标的常见逻辑运算模式有哪些,画图进行解释。
      \item 简述序列标识的含义,能解释实际的序列标识图。
      \item 简述从多序列到序列标识(ICM)的计算过程。
    \end{enumerate}
  \end{block}
  \begin{block}{技能}
    \begin{enumerate}
      \item 不同操作系统的换行符有何区别?
      \item 以SNP的注释结果为例,论述如何解析一张表。
      \item 以box plot为例,论述如何解析一张图。
      \item 以坐标转换为例,论述“输入-加工-输出”的工作流程。
    \end{enumerate}
  \end{block}
\end{frame}
