\section{引言}

\begin{frame}
  \frametitle{引言 | 中心法则}
  \begin{figure}
    \centering
    \includegraphics[width=11cm]{dogma.jpg}
  \end{figure}
\end{frame}

\begin{frame}
  \frametitle{引言 | DNA}
  \begin{figure}
    \centering
    \includegraphics[width=6cm]{dna_acgt.png}
  \end{figure}
\end{frame}

\begin{frame}
  \frametitle{引言 | DNA \& RNA}
  \begin{figure}
    \centering
    \includegraphics[width=9cm]{dna_rna_01.png}
  \end{figure}
\end{frame}

\begin{frame}
  \frametitle{引言 | \alert{遗传信息}}
  \begin{figure}
    \centering
    \includegraphics[width=11cm]{info.jpg}
  \end{figure}
\end{frame}

\begin{frame}
  \frametitle{引言 | 序列分析}
  \begin{block}{基本问题}
    \begin{itemize}
      \item 总的GC含量或者其他核苷酸成分是多少?
      \item 有哪些重复的DNA序列,在什么地方?
      \item 一共有多少个基因(编码蛋白质的序列)?
    \end{itemize}
  \end{block}
  \vspace{-0.3em}
  \pause
  \begin{block}{深层问题}
    \begin{itemize}
      \item 为什么会有各种特征序列?(物理、化学性质?进化压力?)
      \item 需要从哪些方面分析序列特征?
      \item 怎样描述这些序列特征?
    \end{itemize}
  \end{block}
  \vspace{-0.3em}
  \pause
  \begin{block}{序列分析}
    通过实验或计算等方式,确定核苷酸或氨基酸序列中可能与特定功能、结构或生化过程相关联的\alert{具有生物学意义的序列特征},或者\alert{序列自身的规律}。
  \end{block}
\end{frame}


\section{DNA组份分析与序列转换}
\begin{frame}
  \frametitle{DNA序列 | 查戈夫法则}
  \begin{block}{查戈夫法则}
    \begin{description}
      \item[第一法则]$A=T, G=C \Longrightarrow A+C=T+G, A+G=C+T$
      \item[第二法则]AT/GC 的比值因生物种类不同而异
    \end{description}
  \end{block}
\end{frame}

\begin{frame}
  \frametitle{DNA序列 | 序列长度}
  \begin{block}{序列长度}
    序列长度是具有独立生物学功能的序列片段(如基因、启动子等)的基本性质。物种的基因组长度也是重要参数之一。 
  \end{block}
  \pause
  \begin{block}{蛋白质编码基因的序列长度}
    \begin{itemize}
      \item 原核:~1000个核苷酸
      \item 脊椎动物:~30000个核苷酸
      \item 人:20000~50000个核苷酸
    \end{itemize}
  \end{block}
\end{frame}

\begin{frame}
  \frametitle{DNA序列 | 序列长度 | 基因组}
  \begin{figure}
    \centering
    \includegraphics[width=10cm]{genome_size_01.png}
  \end{figure}
\end{frame}

\begin{frame}
  \frametitle{DNA序列 | 碱基组成}
  \begin{block}{碱基组成}
  \begin{itemize}
    \item 核酸序列由ACGT四种碱基组成
    \item 不同物种的DNA碱基组成存在差异
    \item 同一基因组内不同区段(基因、基因间)的碱基组成有差异
    \item 同一基因内部不同片段(外显子、内含子)的碱基组成也有差异
  \end{itemize}
  \end{block}
  \vspace{-0.3em}
  \pause
  \begin{block}{碱基频率}
  \begin{itemize}
    \item 对于随机分布的DNA序列,每种核苷酸的出现是均匀分布的(出现频率各为0.25);真实基因组的核苷酸分布则是非均匀的(酵母:A/T=0.325,G/C=0.175)
    \item 如果同时计算DNA的正反两条链,A和T、G和C的出现频率相同(碱基配对原则);如果仅统计一条链,则虽然A和T、G和C的出现频率不同,但是数值接近(酵母:A=0.344,T=0.343,G=0.157,C=0.155)
  \end{itemize}
  \end{block}
\end{frame}

\begin{frame}
  \frametitle{DNA序列 | 碱基组成 | 实例}
  \begin{figure}
    \centering
    \includegraphics[width=11cm]{base_ratio_01.png}
  \end{figure}
\end{frame}

\begin{frame}
  \frametitle{DNA序列 | GC含量}
  \begin{block}{GC含量(GC content)}
    \begin{itemize}
      \item 对象:核酸片段、基因、基因组、……
      \item 鸟嘌呤(G)和胞嘧啶(C)所占的比例
      \item GC含量随DNA不同而异
      \item GC含量高的DNA更加稳定
      \item 计算公式:$\frac{G+C}{A+T+G+C}\times100$
      \item GC比(GC-ratio):$\frac{A+T}{G+C}$
      \item 结合滑动窗口进行计算
    \end{itemize}
  \end{block}
\end{frame}

\begin{frame}
  \frametitle{DNA序列 | GC含量 | 分析}
  \begin{block}{特点}
  \begin{itemize}
    \item 不同物种基因组中GC含量不同。(15\%~75\%,两头少中间多。疟原虫为20\%,啤酒酵母为38\%,人约为40\%,天蓝色链霉菌A3为72\%。)
    \item 同一基因组内,GC含量不均匀。
    \item GC含量与多种生物学特征相关,比如基因密度、内含子、外显子等。
  \end{itemize}
\end{block}
\pause
\begin{block}{应用}
  \begin{itemize}
    \item 根据GC含量差异识别细菌种类
    \item 真核基因组具有GC含量较高或较低的近似均匀片段
    \item 不同物种的密码子使用与其GC含量有关 
    \item GC含量与DNA双链的熔解温度有关,是进行核酸杂交的重要参数
  \end{itemize}
\end{block}
\end{frame}

\begin{frame}
  \frametitle{DNA序列 | GC含量 | 基因组}
  \begin{figure}
    \centering
    \includegraphics[width=9.5cm]{genomeGC.jpg}
  \end{figure}
  %{\tiny Genome evolution in bacterial endosymbionts of insects. Jennifer J. Wernegreen. Nature Reviews Genetics 3, 850-861 (November 2002). doi:10.1038/nrg931}
\end{frame}

\begin{frame}
  \frametitle{DNA序列 | GC含量 | 基因区}
  \begin{figure}
    \centering
    \includegraphics[width=12cm]{exonGC.png}
  \end{figure}
\end{frame}

\begin{frame}
  \frametitle{DNA序列 | GC含量 | 基因 vs. 基因组}
  \begin{figure}
    \centering
    \includegraphics[width=11cm]{geneGC.png}
  \end{figure}
\end{frame}

\begin{frame}
  \frametitle{DNA序列 | 序列转换}
  \begin{block}{序列转换}
    \begin{itemize}
      \item 反向序列
      \item 互补序列
      \item 反向互补序列
      \item DNA双链
      \item RNA序列
    \end{itemize}
  \end{block}
  \pause
  \begin{block}{书写惯例}
    \begin{itemize}
      \item DNA/RNA:[左] 5' $\Longrightarrow$ 3' [右]
      \item 多肽/蛋白质:[左] N端(氨基端)$\Longrightarrow$ C端(羧基端) [右]
    \end{itemize}
  \end{block}
\end{frame}

\section{限制酶位点分析}
\begin{frame}
  \frametitle{限制酶 | 定义}
  \begin{block}{限制酶(restriction enzyme)}
    又称限制内切酶或限制性内切酶(restriction endonuclease),全称限制性核酸内切酶,是可以识别DNA的特异序列、并在识别位点或其周围切割双链DNA的一类内切酶。
  \end{block}
  \pause
  \begin{block}{切割末端}
    \begin{itemize}
      \item 黏性末端 vs. 平滑末端
    \end{itemize}
  \end{block}
  \begin{figure}
    \centering
    \visible<3->{\includegraphics[width=6cm]{ecori_cut.jpg}}
  \end{figure}
\end{frame}

\begin{frame}
  \frametitle{限制酶 | \alert{命名}}
  %\begin{block}{\textit{Eco}R\Rmnum{1}}
    %\begin{description}
      %\item[\textit{E}]属名\textit{Escherichia}
      %\item[\textit{co}]种名\textit{coli}
      %\item[R]RY13品系
      %\item[\Rmnum{1}]在此类细菌中的发现顺序
    %\end{description}
  %\end{block}
  \begin{figure}
    \centering
    \includegraphics[width=10cm]{ecori.png}
  \end{figure}
\end{frame}

\begin{frame}
  \frametitle{限制酶 | \alert{\Rmnum{2}型}}
  \begin{block}{\Rmnum{2}型限制酶}
  \begin{itemize}
    \item 识别与切割位点:专一
      \begin{itemize}
        \item 识别序列:4-8个碱基,回文对称结构
        \item 切割序列:识别序列,切割位点对称
      \end{itemize}
    \item 切割末端:黏性末端,平滑末端
      \begin{itemize}
        \item 黏性末端:切割位点在回文序列的一侧
        \item 平滑末端:切割位点在回文序列的中间
      \end{itemize}
  \end{itemize}
  \end{block}
\end{frame}

\begin{frame}
  \frametitle{限制酶 | \Rmnum{2}型 | 回文对称}
  \begin{block}{回文对称(palindrome)}
    特指DNA的一种具有反向重复的结构。具有这种结构的DNA,其一条链从左向右读和另一条链从右向左读的序列是相同的。
  \end{block}
  \begin{figure}
    \centering
    \includegraphics[width=8cm]{palindrome.jpg}
  \end{figure}
\end{frame}

\begin{frame}
  \frametitle{限制酶 | \Rmnum{2}型 | 末端}
  \begin{figure}
    \centering
    \includegraphics[width=9cm]{ends.jpg}
  \end{figure}
\end{frame}

\begin{frame}
  \frametitle{限制酶 | \Rmnum{2}型 | 末端 | 黏性末端}
  \begin{figure}
    \centering
    \includegraphics[width=9cm]{enzyme1.png}
  \end{figure}
\end{frame}

\begin{frame}
  \frametitle{限制酶 | \Rmnum{2}型 | 末端 | 平滑末端}
  \begin{figure}
    \centering
    \includegraphics[width=9cm]{enzyme2.png}
  \end{figure}
\end{frame}

\begin{frame}
  \frametitle{限制酶 | 数据库与分析工具}
  \begin{block}{资源}
  \begin{itemize}
    \item REBASE:收录了限制酶的所有信息
    \item NEBCutter V2.0:产生DNA序列的酶切位点分析结果
  \end{itemize}
  \end{block}
\end{frame}

\section{开放阅读框分析}
\begin{frame}
  \frametitle{开放阅读框}
  \begin{block}{开放阅读框(Open Reading Frame,ORF)}
    在给定的阅读框架中,不包含终止密码子的一串序列,是生物个体的基因组中可能作为蛋白质编码序列的部分,包含从5'端翻译起始密码子(AUG)到终止密码子(UAA、UAG、UGA)之间的一段编码蛋白质的碱基序列。
  \end{block}
\end{frame}

\begin{frame}
  \frametitle{开放阅读框 | 相位(frame)}
  \begin{figure}
    \centering
    \includegraphics[width=10cm]{orf.png}
  \end{figure}
\end{frame}

\begin{frame}
  \frametitle{开放阅读框 | \alert{ORF VS. CDS}}
  \pause
  \begin{itemize}
    \item 一个ORF对应一个候选的CDS(编码序列,Coding DNA Sequence)
    \item ORF:理论预测
    \item CDS:实验证实
    \item 分析DNA序列中的ORF是对该序列是否为CDS的初步判断
  \end{itemize}
\end{frame}

\begin{frame}
  \frametitle{开放阅读框 | 分析工具}
  \begin{itemize}
    \item 确定第一个AUG和终止密码子
    \item 原核生物:最长ORF法
    \item 真核生物:特征统计、模式识别、同源比对
    \item ORF Finder:NCBI的在线分析工具
  \end{itemize}
\end{frame}

\section{启动子分析}
\begin{frame}
  \frametitle{启动子 | 转录调控}
  \begin{itemize}
    \item 顺式作用元件(cis-acting element):核酸序列
      \begin{itemize}
        \item 启动子(promoter)
        \item 增强子(enhancer)
        \item \ldots
      \end{itemize}
    \item 反式作用因子(trans-acting factor):蛋白质
    \item 两者相互作用实现转录调控
  \end{itemize}
\end{frame}

\begin{frame}
  \frametitle{启动子 | 定义}
  \begin{block}{启动子(promoter)}
    一段位于转录起始位点5'端上游区的DNA序列,能活化RNA聚合酶,使之与模板DNA准确地结合并具有转录起始的特异性。
  \end{block}
  \pause
  \begin{block}{转录起始位点(Transcription Start Site,TSS)}
    与新生RNA链第一个核苷酸相对应DNA链上的碱基,研究证实通常为一个嘌呤。
  \end{block}
  \visible<3->{
  \begin{figure}
    \centering
    \includegraphics[width=12cm]{tss.png}
  \end{figure}
  }
\end{frame}

\begin{frame}
  \frametitle{启动子 | 结构}
  \begin{figure}
    \centering
    \includegraphics[width=10cm]{promoter.jpg}
  \end{figure}
\end{frame}

\begin{frame}
  \frametitle{启动子 | TF\&TFBS}
  \begin{block}{转录因子(transcription factor)}
    能够结合在某基因上游特异核苷酸序列上的蛋白质,这些蛋白质能调控其基因的转录。
    \begin{itemize}
      \item 功能区域:DNA结合结构域、效应结构域
      \item 作用方式:与启动子区域结合、与其他TF形成复合体
      \item 调控模式:同时调控多个基因
    \end{itemize}
  \end{block}
  \pause
  \begin{block}{转录因子结合位点(Transcription Factor Binding Site,TFBS)}
    与转录因子结合的DNA序列,长度约为5~20bp,它们与转录因子相互作用进行基因的转录调控。
    \begin{itemize}
      \item 保守性 vs. 可变性
    \end{itemize}
  \end{block}
\end{frame}

\begin{frame}
  \frametitle{启动子 | TFBS}
  \begin{figure}
    \centering
    \includegraphics[width=10cm]{tfbs.png}
  \end{figure}
\end{frame}

\begin{frame}
  \frametitle{启动子 | 数据库与分析工具}
  \begin{itemize}
    \item 启动子
      \begin{itemize}
	\item EPD:有注释、非冗余的真核生物RNA聚合酶\Rmnum{2}启动子数据集
        \item Promoter Scan(同源性分析),Promoter 2.0(人工神经网络技术)
      \end{itemize}
    \item 转录因子
      \begin{itemize}
        \item TRANSFAC:真核生物顺式作用元件和反式作用因子数据库
        \item Tfblast(TRANSFAC BLAST)
        \item JASPAR: The high-quality transcription factor binding profile database
        \item CIS-BP Database: Catalog of Inferred Sequence Binding Preferences
        \item footprintDB
        \item HOCOMOCO: expansion and enhancement of the collection of transcription factor binding sites models
        % \item HOmo sapiens COmprehensive MOdel COllection (HOCOMOCO) v10 provides transcription factor (TF) binding models for 600 human and 395 mouse TFs.
        \item MotifMap: genome-wide maps of regulatory elements.
        \item UniPROBE (Universal PBM Resource for Oligonucleotide Binding Evaluation) database
        \item ENCODE TF ChIP-seq datasets
        \item Human Protein-DNA Interactome (hPDI)
      \end{itemize}
  \end{itemize}
\end{frame}

\section{CpG岛识别}
\begin{frame}
  \frametitle{CpG岛 | 特征}
  \begin{block}{\alert{CpG岛}}
    在基因组的某些区段,CpG保持或高于正常概率,这些区段被称作CpG岛(CpG island)。
  \end{block}
  \pause
  \begin{block}{特征}
    \begin{itemize}
      \item 几乎看家基因都含有CpG岛(人类和小鼠分别有55.9\%和46.9\%的基因与CpG岛有密切关联)
      \item 一般位于基因的5'端区域(转录起始位点附近,有助于基因的识别),长度约300~3000bp
      \item 大多数CpG岛是未甲基化的,未甲基化CpG岛说明基因可能具有潜在活性(表观遗传学中重要的作用区域,甲基化异常常常伴随着疾病的发生)
      \item CpG岛中的核小体中H1含量低,其他组蛋白被广泛乙酰化,并具有超敏感位点
    \end{itemize}
  \end{block}
\end{frame}

\begin{frame}
  \frametitle{CpG岛 | \alert{识别依据与判别标准}}
  \begin{enumerate}
    \item CpG岛长度:至少200bp
    \item GC含量:超过50\%
    \item CpG的观察值与预测值的比率:高于60\%
      \begin{itemize}
        \item 观测值:$Num\ of\ CpG$
          \vspace{0.5em}
        \item 预测值:$\frac{Num\ of\ C \times Num\ of\ G}{Length\ of\ sequence}$
          \vspace{0.5em}
        \item 比率:$\frac{Num\ of\ CpG}{Num\ of\ C \times Num\ of\ G} \times Length\ of\ sequence$
          \vspace{0.5em}
        \item 实例:ACGCGACGCG;$\frac{4}{\frac{4\times4}{10}}=\frac{4}{4\times4}\times10=2.5$
      \end{itemize}
    \pause
    \item 500bp,55\%,65\%
  \end{enumerate}
\end{frame}

\begin{frame}
  \frametitle{CpG岛 | 分析工具}
  \begin{itemize}
    \item EMBOSS中的CpGPlot/CpGReport/Isochore
    \item CpG Island Searcher
    \item CpGcluster2
  \end{itemize}
\end{frame}

\section{总结与答疑}
\begin{frame}
  \frametitle{总结与答疑}
  \begin{block}{知识点——DNA序列的基本信息与特征信息分析}
    \begin{itemize}
      \item DNA序列基本信息分析——查戈夫法则,GC含量,序列转换
      \item 限制酶位点分析——命名,\Rmnum{2}型的特点
      \item 开放阅读框分析——相位,ORF与CDS
      \item 启动子与转录因子结合位点分析——启动子结构
      \item CpG岛识别——概念、判别依据及标准
    \end{itemize}
  \end{block}
\end{frame}
