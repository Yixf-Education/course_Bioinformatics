%\documentclass[table]{beamer}
%[]中可以使用draft、handout、screen、transparency、trancompress、compress等参数

%指定beamer的模式与主题
\mode<presentation>
{
  \usetheme{Madrid}
%\usetheme{Boadilla}
%\usecolortheme{default}
%\usecolortheme{orchid}
%\usecolortheme{whale}
%\usefonttheme{professionalfonts}
}

%\usetheme{Madrid}
%这里还可以选择别的主题:Bergen, Boadilla, Madrid, AnnArbor, CambridgeUS, Pittsburgh, Rochester, Warsaw, ...
%有导航栏的Antibes, JuanLesPins, Montpellier, ...
%有内容的Berkeley, PaloAlto, Goettingen, Marburg, Hannover, ...
%有最小导航栏的Berlin, Ilmenau, Dresden, Darmstadt, Frankfurt, Singapore, Szeged, ...
%有章和节表单的Copenhagen, Luebeck, Malmoe, Warsaw, ...

%\usecolortheme{default}
%设置内部颜色主题(这些主题一般改变block里的颜色);这个主题一般选择动物来命名
%这里还可以选择别的颜色主题,如默认的和有特别目的的颜色主题default,structure,sidebartab,全颜色主题albatross,beetle,crane,dove,fly,seagull,wolverine,beaver

%\usecolortheme{orchid}
%设置外部颜色主题(这些主题一般改变title里的颜色);这个主题一般选择植物来命名
%这里还可以选择别的颜色主题,如默认的和有特别目的的颜色主题lily,orchid,rose

%\usecolortheme{whale}
%设置字体主题;这个主题一般选择海洋动物来命名
%这里还可以选择别的颜色主题,如默认的和有特别目的的颜色主题whale,seahorse,dolphin

%\usefonttheme{professionalfonts}
%类似的还可以定义structurebold,structuresmallcapsserif,professionalfonts


% 控制 beamer 的风格,可以根据自己的爱好修改
%\usepackage{beamerthemesplit} %使用 split 风格
%\usepackage{beamerthemeshadow} %使用 shadow 风格
%\usepackage[width=2cm,dark,tab]{beamerthemesidebar}


% 设定英文字体
%\usepackage{fontspec}
\usepackage[no-math]{fontspec}
\setmainfont{Times New Roman}
\setsansfont{Arial}
\setmonofont{Courier New}

% 设定中文字体
\usepackage[BoldFont,SlantFont,CJKchecksingle,CJKnumber]{xeCJK}
%\setCJKmainfont[BoldFont={Adobe Heiti Std},ItalicFont={Adobe Kaiti Std}]{Adobe Song Std}
\setCJKmainfont[BoldFont={Adobe Heiti Std},ItalicFont={Adobe Kaiti Std}]{WenQuanYi Micro Hei}
\setCJKsansfont{Adobe Heiti Std}
\setCJKmonofont{Adobe Fangsong Std}
\punctstyle{hangmobanjiao}

\defaultfontfeatures{Mapping=tex-text}
\usepackage{xunicode}
\usepackage{xltxtra}

\XeTeXlinebreaklocale "zh"
\XeTeXlinebreakskip = 0pt plus 1pt minus 0.1pt

\usepackage{setspace}
\usepackage{colortbl,xcolor}
\usepackage{hyperref}
%\hypersetup{xetex,bookmarksnumbered=true,bookmarksopen=true,pdfborder=1,breaklinks,colorlinks,linkcolor=blue,filecolor=black,urlcolor=cyan,citecolor=green}
\hypersetup{xetex,bookmarksnumbered=true,bookmarksopen=true,pdfborder=1,breaklinks,colorlinks,linkcolor=cyan,filecolor=black,urlcolor=blue,citecolor=green}

% 插入图片
\usepackage{graphicx}
\graphicspath{{figures/}}

% 可能用到的包
\usepackage{amsmath,amssymb}
\usepackage{multimedia}
\usepackage{multicol}
\usepackage{multirow}

% 定义一些自选的模板,包括背景、图标、导航条和页脚等,修改要慎重
% 设置背景渐变由10%的红变成10%的结构颜色
%\beamertemplateshadingbackground{red!10}{structure!10}
%\beamertemplatesolidbackgroundcolor{white!90!blue}
% 使所有隐藏的文本完全透明、动态,而且动态的范围很小
\beamertemplatetransparentcovereddynamic
% 使itemize环境中变成小球,这是一种视觉效果
\beamertemplateballitem
% 为所有已编号的部分设置一个章节目录,并且编号显示成小球
\beamertemplatenumberedballsectiontoc
% 将每一页的要素的要素名设成加粗字体
\beamertemplateboldpartpage

% item逐步显示时,使已经出现的item、正在显示的item、将要出现的item呈现不同颜色
\def\hilite<#1>{
 \temporal<#1>{\color{gray}}{\color{blue}}
    {\color{blue!25}}
}

\renewcommand{\today}{\number\year 年 \number\month 月 \number\day 日}

%五角星
\usepackage{MnSymbol}

%去除图表标题中的figure等
\usepackage{caption}
\captionsetup{labelformat=empty,labelsep=none}

\usepackage{tabu}
\usepackage{multirow}

% 千分号
%\usepackage{textcomp}

%罗马数字
\makeatletter
\newcommand{\rmnum}[1]{\romannumeral #1}
\newcommand{\Rmnum}[1]{\expandafter\@slowromancap\romannumeral #1@}
\makeatother

%分栏
\usepackage{multicol}

%\usepackage{enumitem}
\usepackage{enumerate}


%\setbeamercolor{alerted text}{fg=magenta}

\setbeamercolor{bgcolor}{fg=yellow,bg=cyan}

\begin{document}

%\includeonlyframes{current}

\logo{\includegraphics[height=0.08\textwidth]{tijmu.png}}
\title[核酸序列分析]{核酸序列分析}
\author[Yixf]{伊现富(Yi Xianfu)}
\institute[TIJMU]{天津医科大学(TIJMU)\\ 生物医学工程学院}
\date{2013年8月}


% 在每个Section前都会加入的Frame
\AtBeginSection[]
{
  \begin{frame}<beamer>
    %\frametitle{Outline}
    \frametitle{教学提纲}
	\setcounter{tocdepth}{2}
	\begin{multicols}{2}
    %\tableofcontents[currentsection,currentsubsection]
    \tableofcontents[currentsection]
	\end{multicols}
  \end{frame}
}
% 在每个Subsection前都会加入的Frame
%\AtBeginSubsection[]
%{
  %\begin{frame}<beamer>
%%\begin{frame}<handout:0>
%% handout:0 表示只在手稿中出现
    %\frametitle{Outline}
	%\setcounter{tocdepth}{2}
    %\tableofcontents[currentsection,currentsubsection]
%% 显示在目录中加亮的当前章节
  %\end{frame}
%}

\begin{frame}[plain]
	\begin{center}
		{\Huge 生物信息学\\}
		\vspace{1cm}
		{\LARGE 天津医科大学\\}
		%\vspace{0.2cm}
		{\LARGE 生物医学工程学院\\}
		\vspace{1cm}
		{\large 2013-2014学年上学期}
	\end{center}
\end{frame}

\begin{frame}
	\frametitle{课堂纪律}
	\begin{itemize}[<+-|alert@+>]
		\item 只有正式上课前的请假有效。
		\item 提前5分钟到教室,严禁迟到。
		\item 上课期间手机关机或调成震动。
		\item 上课期间离开教室先举手示意。
		\item 课上有疑问的话先举手后提问。
		\item 上课期间严禁交头接耳,大声喧哗。
		\item 随机点名,缺勤扣分如下:1、3、6。
		\item 缺勤三次或三次以上者,平时成绩为0。
	\end{itemize}
\end{frame}

%\begin{frame}
	%\frametitle{自我介绍与信息交流}
	%\begin{columns}
	%\column{0.48\textwidth}
	%\begin{block}{自我介绍}
		%\begin{description}
			%\item[姓\qquad 名]伊现富(Yi Xianfu)
			%\item[本\qquad 科]山东大学
			%\item[硕\qquad 博]中国科学院
			%\item[工作邮箱]yixfbio@gmail.com
			%\item[生活邮箱]yixf1986@gmail.com
			%\item[手\qquad 机]15620610763
			%\item[个人博客]http://yixf.name
			%\item[网络昵称]yixf
		%\end{description}
	%\end{block}
	%\pause
	%\column{0.48\textwidth}
	%\begin{block}{邮箱网盘}
		%\begin{enumerate}
			%%%%%\item 126邮箱
				%%%%%\begin{itemize}
					%%%%%\item bioinfo\_TIJMU@126.com
					%%%%%\item \texttt{C\&563f\&nzx!s}
				%%%%%\end{itemize}
			%%%%%\item 百度云网盘
				%%%%%\begin{itemize}
					%%%%%\item bioinfo\_TIJMU@126.com
					%%%%%\item \texttt{566\&Us3Rp6\#C}
				%%%%%\end{itemize}
		%\end{enumerate}
	%\end{block}
	%\end{columns}
%\end{frame}

\begin{frame}
	\frametitle{自我介绍}
		\begin{description}
			\item[姓\qquad 名]伊现富(Yi Xianfu)
			\item[本\qquad 科]山东大学
			\item[硕\qquad 博]中国科学院
			\item[工作邮箱]yixfbio@gmail.com
			\item[生活邮箱]yixf1986@gmail.com
			\item[手\qquad 机]15620610763
			\item[个人博客]http://yixf.name
			\item[网络昵称]yixf
		\end{description}
\end{frame}

\begin{frame}
	\frametitle{邮箱网盘}
		\begin{enumerate}
			\item 126邮箱
				\begin{itemize}
					\item 账号:bioinfo\_TIJMU@126.com
					\item 密码:\texttt{C\&563f\&nzx!s}
				\end{itemize}
			\item 百度云网盘
				\begin{itemize}
					\item 账号:bioinfo\_TIJMU@126.com
					\item 密码:\texttt{566\&Us3Rp6\#C}
				\end{itemize}
		\end{enumerate}
\end{frame}

\begin{frame}
  \titlepage
\end{frame}

\begin{frame}[plain]
  \frametitle{教学提纲}
  \setcounter{tocdepth}{2}
  \begin{multicols}{2}
  \tableofcontents
  \end{multicols}
\end{frame}

\section{引言}
\begin{frame}
	\frametitle{引言 | 大千世界}
	\begin{figure}
		\centering
		\includegraphics[width=6cm]{world.jpg}
	\end{figure}
\end{frame}

\begin{frame}
	\frametitle{引言 | 中心法则}
	\begin{figure}
		\centering
		\includegraphics[width=11cm]{dogma.jpg}
	\end{figure}
\end{frame}

\begin{frame}
	\frametitle{引言 | ACGT}
	\begin{figure}
		\centering
		\includegraphics[width=10cm]{acgt.jpg}
	\end{figure}
\end{frame}

\begin{frame}
	\frametitle{引言 | ACGT$\Rightarrow$生信}
	\begin{figure}
		\centering
		\includegraphics[width=10cm]{matrix.png}
	\end{figure}
\end{frame}

\begin{frame}
	\frametitle{引言 | 遗传信息}
	\begin{figure}
		\centering
		\includegraphics[width=11cm]{info.jpg}
	\end{figure}
\end{frame}

\section{DNA序列转换与组份分析}
\begin{frame}
	\frametitle{DNA序列 | 查戈夫法则}
	\begin{block}{查戈夫法则}
		\begin{description}
			\item[第一法则]$A=T, G=C \Longrightarrow A+C=T+G, A+G=C+T$
			\item[第二法则]AT/GC 的比值因生物种类不同而异
		\end{description}
	\end{block}
\end{frame}

\begin{frame}
	\frametitle{DNA序列 | GC含量}
	\begin{block}{GC含量(GC content)}
		\begin{itemize}
			\item 鸟嘌呤(G)和胞嘧啶(C)所占的比例
			\item GC含量高的DNA更加稳定
			\item GC含量随DNA不同而异
			\item 计算公式:$\frac{G+C}{A+T+G+C}\times100$
			\item GC比(GC-ratio):$\frac{A+T}{G+C}$
		\end{itemize}
	\end{block}
\end{frame}

\begin{frame}
	\frametitle{DNA序列 | GC含量 | 基因组}
	\begin{figure}
		\centering
		\includegraphics[width=9.5cm]{genomeGC.jpg}
	\end{figure}
	%{\tiny Genome evolution in bacterial endosymbionts of insects. Jennifer J. Wernegreen. Nature Reviews Genetics 3, 850-861 (November 2002). doi:10.1038/nrg931}
\end{frame}

\begin{frame}
	\frametitle{DNA序列 | GC含量 | 基因区}
	\begin{figure}
		\centering
		\includegraphics[width=12cm]{exonGC.png}
	\end{figure}
\end{frame}

\begin{frame}
	\frametitle{DNA序列 | GC含量 | 基因 VS. 基因组}
	\begin{figure}
		\centering
		\includegraphics[width=11cm]{geneGC.jpg}
	\end{figure}
\end{frame}

\begin{frame}
	\frametitle{DNA序列 | 序列转换}
	\begin{block}{序列转换}
		\begin{itemize}
			\item 反向序列
			\item 互补序列
			\item 反向互补序列
			\item DNA双链
			\item RNA序列
		\end{itemize}
	\end{block}
	\pause
	\begin{block}{书写惯例}
		\begin{itemize}
			\item DNA/RNA:[左] 5' $\Longrightarrow$ 3' [右]
			\item 多肽/蛋白质:[左] N端(氨基端)$\Longrightarrow$ C端(羧基端) [右]
		\end{itemize}
	\end{block}
\end{frame}

\begin{frame}
	\frametitle{DNA序列 | 实例与策略}
	\begin{block}{任务分析}
		\begin{itemize}
			\item 序列长短
			\item 序列数目
			\item 任务数量
			\item 任务频率
			\item 工作时间
			\item \ldots
		\end{itemize}
	\end{block}
\end{frame}

\begin{frame}
	\frametitle{DNA序列 | 分析工具}
	\begin{itemize}
		\item \href{http://yixf.name/2011/06/01/\%E5\%AF\%B9fasta\%E6\%A0\%BC\%E5\%BC\%8F\%E7\%9A\%84\%E7\%AE\%80\%E5\%8D\%95\%E5\%A4\%84\%E7\%90\%86\%E4\%B8\%8E\%E7\%BB\%9F\%E8\%AE\%A1/}{SeqTools.pl}
		\item EMBOSS
		\item \href{http://bioinfx.net/}{bioinfx(Free Online Tools for Bioinformatics)}
		\item \href{http://clasher.myweb.uga.edu/testpages/seqconv.html}{Complementary Sequence Conversion Tool}
		\item \href{http://www.cellbiol.com/scripts/complement/dna\_sequence\_reverse\_complement.php}{DNA Sequence Reverse and Complement Online Tool}
		\item \href{http://www.endmemo.com/bio/gc.php}{DNA/RNA GC Content Calculator}
		\item \href{http://www.sciencelauncher.com/oligocalc.html}{Oligo Calculator}
		\item \ldots
	\end{itemize}
\end{frame}

\section{限制酶位点分析}
\begin{frame}
	\frametitle{限制酶 | 定义}
	\begin{block}{限制酶(restriction enzyme)}
		又称限制内切酶或限制性内切酶(restriction endonuclease),全称限制性核酸内切酶,是可以识别DNA的特异序列、并在识别位点或其周围切割双链DNA的一类内切酶。
	\end{block}
	\pause
	\begin{block}{切割末端}
		\begin{itemize}
			\item 黏性末端
			\item 平滑末端
		\end{itemize}
	\end{block}
\end{frame}

\begin{frame}
	\frametitle{限制酶 | 命名}
	%\begin{block}{\textit{Eco}R\Rmnum{1}}
		%\begin{description}
			%\item[\textit{E}]属名\textit{Escherichia}
			%\item[\textit{co}]种名\textit{coli}
			%\item[R]RY13品系
			%\item[\Rmnum{1}]在此类细菌中的发现顺序
		%\end{description}
	%\end{block}
	\begin{figure}
		\centering
		\includegraphics[width=10cm]{ecori.png}
	\end{figure}
\end{frame}

\begin{frame}
	\frametitle{限制酶 | \Rmnum{2}型}
	\begin{itemize}
		\item 识别、切割位点专一
		\item 识别序列:4-8个碱基,回文对称结构
		\item 切割序列:识别序列,切割位点对称
		\item 切割末端:黏性末端,平滑末端
		\item 黏性末端:切割位点在回文序列的一侧
		\item 平滑末端:切割位点在回文序列的中间
	\end{itemize}
\end{frame}

\begin{frame}
	\frametitle{限制酶 | \Rmnum{2}型 | 回文}
	\begin{block}{《题金山寺》,北宋\textbullet 苏轼}
	\begin{columns}
		\column{0.5\textwidth}
潮随暗浪雪山倾,远浦渔舟钓月明。
桥对寺门松径小,槛当泉眼石波清。
迢迢绿树江天晓,霭霭红霞晚日晴。
遥望四边云接水,雪峰千点数鸥轻。
		\column{0.5\textwidth}
		\pause
轻鸥数点千峰雪,水接云边四望遥。
晴日晚霞红霭霭,晓天江树绿迢迢。
清波石眼泉当槛,小径松门寺对桥。
明月钓舟渔浦远,倾山雪浪暗随潮。
	\end{columns}
	\end{block}
	\pause
	\begin{block}{回文对称(palindrome)}
		特指DNA的一种具有反向重复的结构。具有这种结构的DNA,其一条链从左向右读和另一条链从右向左读的序列是相同的。
	\end{block}
\end{frame}

\begin{frame}
	\frametitle{限制酶 | \Rmnum{2}型 | 黏性末端}
	\begin{figure}
		\centering
		\includegraphics[width=9cm]{enzyme1.png}
	\end{figure}
\end{frame}

\begin{frame}
	\frametitle{限制酶 | \Rmnum{2}型 | 平滑末端}
	\begin{figure}
		\centering
		\includegraphics[width=9cm]{enzyme2.png}
	\end{figure}
\end{frame}

\begin{frame}
	\frametitle{限制酶 | 数据库与分析工具}
	\begin{itemize}
		\item REBASE
		\item NEBCutter V2.0
	\end{itemize}
\end{frame}

\section{开放阅读框分析}
\begin{frame}
	\frametitle{开放阅读框}
	\begin{block}{开放阅读框(Open Reading Frame,ORF)}
		在给定的阅读框架中,不包含终止密码子的一串序列,是生物个体的基因组中可能作为蛋白质编码序列的部分,包含从5'端翻译起始密码子(ATG)到终止密码子(TAA、TAG、TGA)之间的一段编码蛋白质的碱基序列。
	\end{block}
\end{frame}

\begin{frame}
	\frametitle{开放阅读框 | 相位}
	\begin{figure}
		\centering
		\includegraphics[width=10cm]{orf.png}
	\end{figure}
\end{frame}

\begin{frame}
	\frametitle{开放阅读框 | ORF VS. CDS}
	\pause
	\begin{itemize}
		\item 一个ORF对应一个候选的CDS(编码序列,Coding Sequence)
		\item ORF:理论预测
		\item CDS:实验证实
	\end{itemize}
\end{frame}

\begin{frame}
	\frametitle{开放阅读框 | 分析工具}
	\begin{itemize}
		\item 确定第一个ATG和终止密码子
		\item 最长ORF法(原核生物)
		\item ORF Finder
	\end{itemize}
\end{frame}

\section{启动子分析}
\begin{frame}
	\frametitle{启动子 | 表达调控}
	\begin{itemize}
		\item 顺式作用元件(cis-acting element):核酸序列
			\begin{itemize}
				\item 启动子(promoter)
				\item 增强子(enhancer)
				\item \ldots
			\end{itemize}
		\item 反式作用因子(trans-acting factor):蛋白质
		\item 两者相互作用参与基因表达调控
	\end{itemize}
\end{frame}

\begin{frame}
	\frametitle{启动子 | 定义}
	\begin{block}{启动子(promoter)}
		一段位于转录起始位点5'端上游区的DNA序列,能活化RNA聚合酶,使之与模板DNA准确地结合并具有转录起始的特异性。
	\end{block}
	\pause
	\begin{block}{转录起始位点(Transcription Start Site,TSS)}
		与新生RNA链第一个核苷酸相对应DNA链上的碱基,研究证实通常为一个嘌呤。
	\end{block}
	\visible<3->{
	\begin{figure}
		\centering
		\includegraphics[width=12cm]{tss.png}
	\end{figure}
	}
\end{frame}

\begin{frame}
	\frametitle{启动子 | 结构}
	\begin{figure}
		\centering
		\includegraphics[width=10cm]{promoter.jpg}
	\end{figure}
\end{frame}

\begin{frame}
	\frametitle{启动子 | TF\&TFBS}
	\begin{block}{转录因子(transcription factor)}
		能够结合在某基因上游特异核苷酸序列上的蛋白质,这些蛋白质能调控其基因的转录。
	\end{block}
	\pause
	\begin{block}{转录因子结合位点(Transcription Factor Binding Site,TFBS)}
		与转录因子结合的DNA序列,长度约为5~20bp,它们与转录因子相互作用进行基因的转录调控。
	\end{block}
\end{frame}

\begin{frame}
	\frametitle{启动子 | TFBS}
	\begin{figure}
		\centering
		\includegraphics[width=10cm]{tfbs.jpg}
	\end{figure}
\end{frame}

\begin{frame}
	\frametitle{启动子 | 数据库与分析工具}
	\begin{itemize}
		\item 启动子
			\begin{itemize}
				\item EPD
				\item Promoter Scan,Promoter 2.0
			\end{itemize}
		\item 转录因子
			\begin{itemize}
				\item TRANSFAC 
				\item Tfblast(TRANSFAC BLAST)
			\end{itemize}
	\end{itemize}
\end{frame}

\section{CpG岛识别}
\begin{frame}
	\frametitle{CpG岛 | 特征}
	\begin{block}{CpG岛}
		在基因组的某些区段,CpG保持或高于正常概率,这些区段被称作CpG岛(CpG island)。
	\end{block}
	\pause
	\begin{block}{特征}
		\begin{itemize}
			\item 几乎看家基因都含有CpG岛
			\item 一般位于基因的5'端区域(转录起始位点附近),长度约300~3000bp
			\item 大多数CpG岛是未甲基化的,未甲基化CpG岛说明基因可能具有潜在活性
			\item CpG岛中的核小体中H1含量低,其他组蛋白被广泛乙酰化,并具有超敏感位点
		\end{itemize}
	\end{block}
\end{frame}

\begin{frame}
	\frametitle{CpG岛 | 预测标准}
	\begin{enumerate}
		\item CpG岛长度:至少200bp
		\item GC含量:超过50\%
		\item CpG的观察值与预测值的比率:高于60\%
			\begin{itemize}
				\item $\frac{Num\ of\ CpG}{Num\ of\ C \times Num\ of\ G} \times Total\ number\ of\ nucleotides\ in\ the\ sequence$
			\end{itemize}
		\pause
		\item 500bp,55\%,65\%
	\end{enumerate}
\end{frame}

\begin{frame}
	\frametitle{CpG岛 | 分析工具}
	\begin{itemize}
		\item EMBOSS中的CpGPlot/CpGReport/Isochore
		\item CpG Island Searcher
		\item CpGcluster2
	\end{itemize}
\end{frame}

\section{重复序列分析}
\begin{frame}
	\frametitle{重复序列 | 分类}
	\begin{block}{重复序列(repetitive sequence, repeated sequence)}
		真核生物基因组中重复出现的核苷酸序列,一般不编码多肽,在基因组内可成簇排布,也可散布于基因组。
	\end{block}
	\pause
	\begin{block}{重复次数}
		\begin{itemize}
			\item 低度重复序列(lowly repetitive sequence):在整个基因组中只含有2~10个拷贝
			\item 中度重复序列(moderately repetitive sequence):重复次数为几十次到几千次,重复单元的平均长度约300bp
			\item 高度重复序列(highly repetitive sequence):重复几百万次,一般是少于10个核苷酸残基组成的短片段
		\end{itemize}
	\end{block}
\end{frame}

\begin{frame}
	\frametitle{重复序列 | 分类}
	\begin{block}{组织形式}
		\begin{itemize}
			\item 串联重复序列:成簇存在于染色体的特定区域
				\begin{itemize}
					\item 卫星DNA(satellite DNA):一类高度重复序列
					\item 小卫星(minisatellite,VNTR):由10~100bp的基本单位串联而成,总长通常不超过20kb,重复次数在群体中是高度变异的
					\item 微卫星(microsatellite,SSR,STR):两个或多个核苷酸重复排列,只有2~10bp,串联成簇,长度50~100bp,STR遗传多态性
				\end{itemize}
			\item 散在重复序列:分散于染色体的各位点上
				\begin{itemize}
					\item 短散在重复序列(Short Interspersed Nuclear Element,SINE):长度在500bp以下,在人基因组中的重复拷贝数达10万以上;非自主转座的反转录转座子,来源于RNA聚合酶\Rmnum{3}的转录产物;Alu
					\item 长散在重复序列(Long Interspersed Nuclear Element,LINE):长度在1000bp以上,在人基因组中有上万份拷贝;可以自主转座的一类反转录转座子,来源于RNA聚合酶\Rmnum{2}的转录产物;L1
				\end{itemize}
		\end{itemize}
	\end{block}
\end{frame}

\begin{frame}
	\frametitle{重复序列 | 数据库与分析工具}
	\begin{itemize}
		\item Repbase
		\item L1Base
		\item STRBase
		\item RepeatMasker:四个搜索引擎
			\begin{itemize}
				\item Cross\_match
				\item ABBlast
				\item RMBlast
				\item HMMER
			\end{itemize}
	\end{itemize}
\end{frame}

\section{总结与答疑}
\begin{frame}
	\frametitle{总结与答疑}
	\begin{block}{知识点}
		\begin{itemize}
			\item DNA序列基本信息分析——查戈夫法则,序列转换,GC含量
			\item 限制酶位点分析——命名,\Rmnum{2}型
			\item 开放阅读框分析——ORF与CDS
			\item 启动子与转录因子结合位点分析——启动子结构
			\item CpG岛识别——判别依据及标准
			\item 重复序列分析——分类
		\end{itemize}
	\end{block}
	\begin{block}{技能}
		\begin{itemize}
			\item 解决问题的思路
			\item 搜索、学习软件
		\end{itemize}
	\end{block}
\end{frame}

\section{引言}
\begin{frame}
	\frametitle{引言 | 回顾}
	\begin{itemize}[<+-|alert@+>]
		\item 基本信息分析
			\begin{itemize}
				\item 序列转换
				\item 碱基比例
				\item GC含量
				\item 寻找限制酶切位点
			\end{itemize}
		\item 序列特征分析
			\begin{itemize}
				\item 开放阅读框的预测
				\item 启动子和转录因子结合位点的分析
				\item CpG岛的识别
			\end{itemize}
		\item 基因识别
			\begin{itemize}
				\item 屏蔽重复序列
				\item 基因识别
			\end{itemize}
	\end{itemize}
\end{frame}

\section{基因识别}
\begin{frame}
	\frametitle{基因识别 | 基因与基因识别}
	\begin{block}{基因(gene)}
		产生一条多肽链或功能RNA所需的全部核苷酸序列。一段具有特定功能和结构的连续的DNA片段,携带着遗传信息,是编码蛋白质或RNA分子遗传信息、控制性状的基本遗传单位。\\
		一个完整的基因,不仅包括编码区,还包括5'末端和3'末端长度不等的特异性序列。
	\end{block}
	\pause
	\begin{block}{基因识别(gene prediction,gene finding)}
		使用生物学实验或计算机等手段识别DNA序列上的具有生物学特征的片段。
	\end{block}
\end{frame}

\begin{frame}
	\frametitle{基因识别 | 基因结构}
	\begin{figure}
		\centering
		\includegraphics[width=10cm]{geneP.jpg}
		\\
		\includegraphics[width=10cm]{geneE.jpg}
	\end{figure}
\end{frame}

\begin{frame}
	\frametitle{基因识别 | 方法}
	\begin{enumerate}
		\item 间接识别法(Extrinsic Approach):利用已知的mRNA或蛋白质序列为线索在DNA序列中搜寻所对应的片段
		\item 从头计算法(\textit{Ab Initio} Approach):基因预测,基于基因的两种类型的特征:
			\begin{itemize}
				\item “信号”:由一些特殊的序列构成,通常预示着其周围存在着一个基因;TATA box、CAAT box、供体位点与受体位点、起始密码子、终止密码子、polyA信号序列、\ldots
				\item “内容”:蛋白质编码基因所具有的某些统计学特征;密码子使用偏好性(codon usage bias)、双联密码子出现频率、基因组等值区 (isochore)、\ldots
			\end{itemize}
		\item 比较基因组学的方法:自然选择的力量使得基因和DNA序列上具有生物学功能的片段较其他部分有较慢的变异速率,在前者的变异更有可能对生物体的生存产生负面影响,因而难以得到保存
	\end{enumerate}
\end{frame}

\begin{frame}
	\frametitle{基因识别 | 分析工具}
	\begin{itemize}
		\item GeneMarkS
		\item Glimmer
		\item GENSCAN
		\item GRAIL
		\item \href{http://en.wikipedia.org/wiki/List\_of\_gene\_prediction\_software}{List of gene prediction software(Wikipedia)}
	\end{itemize}
\end{frame}

\section{mRNA选择性剪接}
\begin{frame}
	\frametitle{选择性剪接 | 剪接与选择性剪接}
	\begin{block}{剪接(splicing)}
		又称拼接,指基因信息在转录后的一种修饰,即将内含子移除及合并外显子,是真核生物的信使RNA前体(precursor messenger RNA)变成成熟mRNA的过程之一。
	\end{block}
	\pause
	\begin{block}{选择性剪接(alternative splicing)}
		又称可变剪接,指用不同的剪接方式(选择不同的剪接位点组合)从一个mRNA前体产生不同的mRNA剪接异构体的过程。
	\end{block}
\end{frame}

\begin{frame}
	\frametitle{选择性剪接 | 实例}
	\begin{figure}
		\centering
		\includegraphics[width=11cm]{splicing.png}
	\end{figure}
\end{frame}

\begin{frame}
	\frametitle{选择性剪接 | 机制 | 五种}
	\begin{figure}
		\centering
		\includegraphics[width=10cm]{splicingModel5.jpg}
	\end{figure}
\end{frame}

\begin{frame}
	\frametitle{选择性剪接 | 机制 | 七种}
	\begin{figure}
		\centering
		\includegraphics[width=9cm]{splicingModel7.jpg}
	\end{figure}
\end{frame}

\begin{frame}
	\frametitle{选择性剪接 | 机制 | 复杂实例}
	\begin{figure}
		\centering
		\includegraphics[width=11cm]{splicingExample.png}
	\end{figure}
\end{frame}

\begin{frame}
	\frametitle{选择性剪接 | 数据库与分析工具}
	\begin{itemize}
		\item ASTD = ASD (= AEDB + AltExtron + AltSplice) + ATD
		\item \textcolor{gray}{ASAP}
		\item ESEfinder 
		\item RESCUE-ESE
		\item ASPicDB
	\end{itemize}
\end{frame}

\section{miRNA及其靶基因预测}
\begin{frame}
	\frametitle{miRNA | ncRNA}
	\begin{block}{非编码RNA(non-coding RNAs,ncRNA)}
	\begin{itemize}
		\item 基础结构性ncRNA(infrastructural non-coding RNAs),看家ncRNA(housekeeping non-coding RNAs)
			\begin{itemize}
				\item tRNA、rRNA、snRNA、snoRNA
			\end{itemize}
		\item 调节性ncRNA(regulatory non-coding RNAs)
			\begin{itemize}
				\item 小RNA(small RNAs,sRNA):\textless 200nt
					\begin{itemize}
						\item miRNA、siRNA、piRNA
					\end{itemize}
				\item 长链非编码RNA(long ncRNAs,lncRNA):\textgreater 200nt
			\end{itemize}
	\end{itemize}
\end{block}
\end{frame}

\begin{frame}
	\frametitle{miRNA | 特点}
	\begin{block}{微RNA(microRNAs,miRNA,小分子RNA)}
		归属小RNA范畴,是真核生物中广泛存在的一种长约20到24个核苷酸的内源性非编码单链RNA分子。miRNA通过RNA诱导沉默复合体(RISC)与靶基因的3'非翻译区(3' UTR)相结合,导致靶基因mRNA降解或者抑制其翻译,从而调节基因转录后的表达。
	\end{block}
	\pause
	\begin{block}{特点}
		\begin{itemize}
			\item 20~24nt
			\item 不具有开放阅读框,不编码蛋白质
			\item 表达具有时序性和组织特异性
			\item 进化上具有高度的保守性
		\end{itemize}
	\end{block}
\end{frame}

\begin{frame}
	\frametitle{miRNA | 生成}
	\begin{figure}
		\centering
		\includegraphics[width=7cm]{mirna.png}
	\end{figure}
\end{frame}

\begin{frame}
	\frametitle{miRNA | 作用网络}
	\begin{figure}
		\centering
		\includegraphics[width=10cm]{mirnaN.jpg}
	\end{figure}
\end{frame}

\begin{frame}
	\frametitle{miRNA | 功能}
	\begin{figure}
		\centering
		\includegraphics[width=12cm]{mirnaF.png}
	\end{figure}
\end{frame}

\begin{frame}
	\frametitle{miRNA | 预测}
	\begin{enumerate}
		\item 同源片段搜索方法
		\item 基于比较基因组学的预测方法
		\item 基于序列和结构特征打分的预测方法
		\item 结合作用靶标的预测方法
		\item 基于机器学习的预测方法
	\end{enumerate}
\end{frame}

\begin{frame}
	\frametitle{miRNA | 种子区域}
	\begin{figure}
		\centering
		\includegraphics[width=10cm]{mirnaS.png}
	\end{figure}
\end{frame}

\begin{frame}
	\frametitle{miRNA | 靶基因预测}
	\begin{enumerate}
		\item 基于种子区域互补和保守性的规则预测
			\begin{itemize}
				\item miRanda
				\item TargetScan
			\end{itemize}
		\item 基于机器学习方法训练参数进行靶基因预测
			\begin{itemize}
				\item PicTar
				\item miTarget
			\end{itemize}
	\end{enumerate}
\end{frame}

\begin{frame}
	\frametitle{miRNA | 数据库与分析工具}
	\begin{itemize}
		\item 数据库:miRBase、TarBase、miRGen
		\item miRNA预测:MiRscan、MiPred、miRFinder 
		\item miRNA靶基因预测:miRanda、TargetScan、PicTar、miTarget
		\item \href{http://zh.wikipedia.org/wiki/\%E5\%BE\%AERNA\%E4\%B8\%8E\%E5\%BE\%AERNA\%E9\%9D\%B6\%E6\%95\%B0\%E6\%8D\%AE\%E5\%BA\%93}{微RNA与微RNA靶数据库(维基百科)}
	\end{itemize}
\end{frame}

\section{lncRNA}
\begin{frame}
	\frametitle{lncRNA}
	\begin{itemize}
		\item 大多被RNA聚合酶\Rmnum{2}所转录
		\item 有5'帽子和3'端的poly(A)尾巴
		\item 主要富集在细胞核
		\item 长度偏短、外显子数目偏少
		\item 在不同物种间的保守性差
		\item 稳定性偏低
		\item 表达水平很低,且具有时空特异性
		\item \href{http://zh.wikipedia.org/wiki/\%E9\%95\%BF\%E9\%93\%BE\%E9\%9D\%9E\%E7\%BC\%96\%E7\%A0\%81RNA\%E6\%95\%B0\%E6\%8D\%AE\%E5\%BA\%93}{长链非编码RNA数据库(维基百科)}
	\end{itemize}
\end{frame}

\section{查找数据库与工具}
\begin{frame}
	\frametitle{查找数据库与分析工具}
	\begin{itemize}
	\item 借鉴相关文献中使用的数据库与工具
	\item 向特定领域的专家请教
	\item \textit{Nucleic Acids Research}每年的第一期为数据库专刊
	\item 维基百科等总结性网站
	\item \href{http://elements.eaglegenomics.com/}{The Elements of Bioinformatics}
	\item 使用Google等搜索引擎搜索
	\end{itemize}
\end{frame}

\section{总结与答疑}
\begin{frame}
	\frametitle{总结与答疑}
	\begin{block}{知识点}
		\begin{itemize}
			\item 基因识别——原核和真核的基因结构,基因识别方法
			\item mRNA可变剪接——选择性剪接的类型,数据资源
			\item miRNA——miRNA的特点,miRNA预测方法与工具,miRNA靶基因预测方法与工具
		\end{itemize}
	\end{block}
	\begin{block}{技能}
		\begin{itemize}
			\item 查找数据库——时效性
			\item 查找分析工具——适用范围
		\end{itemize}
	\end{block}
\end{frame}

\section*{Acknowledgements}
\begin{frame}
	\frametitle{Powered by}
	\begin{center}
		\includegraphics[width=9cm]{power.png}
	\end{center}
\end{frame}

\end{document}
