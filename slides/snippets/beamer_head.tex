\documentclass[table]{beamer}
%[]中可以使用draft、handout、screen、transparency、trancompress、compress等参数

%指定beamer的模式与主题
\mode<presentation>
{
  \usetheme{Madrid}
%\usetheme{Boadilla}
%\usecolortheme{default}
%\usecolortheme{orchid}
%\usecolortheme{whale}
%\usefonttheme{professionalfonts}
}

%\usetheme{Madrid}
%这里还可以选择别的主题:Bergen, Boadilla, Madrid, AnnArbor, CambridgeUS, Pittsburgh, Rochester, Warsaw, ...
%有导航栏的Antibes, JuanLesPins, Montpellier, ...
%有内容的Berkeley, PaloAlto, Goettingen, Marburg, Hannover, ...
%有最小导航栏的Berlin, Ilmenau, Dresden, Darmstadt, Frankfurt, Singapore, Szeged, ...
%有章和节表单的Copenhagen, Luebeck, Malmoe, Warsaw, ...

%\usecolortheme{default}
%设置内部颜色主题(这些主题一般改变block里的颜色);这个主题一般选择动物来命名
%这里还可以选择别的颜色主题,如默认的和有特别目的的颜色主题default,structure,sidebartab,全颜色主题albatross,beetle,crane,dove,fly,seagull,wolverine,beaver

%\usecolortheme{orchid}
%设置外部颜色主题(这些主题一般改变title里的颜色);这个主题一般选择植物来命名
%这里还可以选择别的颜色主题,如默认的和有特别目的的颜色主题lily,orchid,rose

%\usecolortheme{whale}
%设置字体主题;这个主题一般选择海洋动物来命名
%这里还可以选择别的颜色主题,如默认的和有特别目的的颜色主题whale,seahorse,dolphin

%\usefonttheme{professionalfonts}
%类似的还可以定义structurebold,structuresmallcapsserif,professionalfonts


% 控制 beamer 的风格,可以根据自己的爱好修改
%\usepackage{beamerthemesplit} %使用 split 风格
%\usepackage{beamerthemeshadow} %使用 shadow 风格
%\usepackage[width=2cm,dark,tab]{beamerthemesidebar}


% 设定英文字体
%\usepackage{fontspec}
\usepackage[no-math]{fontspec}
\setmainfont{Times New Roman}
\setsansfont{Arial}
\setmonofont{Courier New}

% 设定中文字体
\usepackage[BoldFont,SlantFont,CJKchecksingle,CJKnumber]{xeCJK}
%\setCJKmainfont[BoldFont={Adobe Heiti Std},ItalicFont={Adobe Kaiti Std}]{Adobe Song Std}
\setCJKmainfont[BoldFont={Adobe Heiti Std},ItalicFont={Adobe Kaiti Std}]{WenQuanYi Micro Hei}
\setCJKsansfont{Adobe Heiti Std}
\setCJKmonofont{Adobe Fangsong Std}
\punctstyle{hangmobanjiao}

\defaultfontfeatures{Mapping=tex-text}
\usepackage{xunicode}
\usepackage{xltxtra}

\XeTeXlinebreaklocale "zh"
\XeTeXlinebreakskip = 0pt plus 1pt minus 0.1pt

\usepackage{setspace}
\usepackage{colortbl,xcolor}
\usepackage{hyperref}
%\hypersetup{xetex,bookmarksnumbered=true,bookmarksopen=true,pdfborder=1,breaklinks,colorlinks,linkcolor=blue,filecolor=black,urlcolor=cyan,citecolor=green}
\hypersetup{xetex,bookmarksnumbered=true,bookmarksopen=true,pdfborder=1,breaklinks,colorlinks,linkcolor=cyan,filecolor=black,urlcolor=blue,citecolor=green}

% 插入图片
\usepackage{graphicx}
\graphicspath{{figures/}}
% 图文混排
%\usepackage{picins}
\usepackage{floatflt}

% 可能用到的包
\usepackage{amsmath,amssymb}
%插入多媒体
%\usepackage{media9}
%\usepackage{movie15}
\usepackage{multimedia}
\usepackage{multicol}
\usepackage{multirow}

% 定义一些自选的模板,包括背景、图标、导航条和页脚等,修改要慎重
% 设置背景渐变由10%的红变成10%的结构颜色
%\beamertemplateshadingbackground{red!10}{structure!10}
%\beamertemplatesolidbackgroundcolor{white!90!blue}
% 使所有隐藏的文本完全透明、动态,而且动态的范围很小
\beamertemplatetransparentcovereddynamic
% 使itemize环境中变成小球,这是一种视觉效果
\beamertemplateballitem
% 为所有已编号的部分设置一个章节目录,并且编号显示成小球
\beamertemplatenumberedballsectiontoc
% 将每一页的要素的要素名设成加粗字体
\beamertemplateboldpartpage

% item逐步显示时,使已经出现的item、正在显示的item、将要出现的item呈现不同颜色
\def\hilite<#1>{
 \temporal<#1>{\color{gray}}{\color{blue}}
    {\color{blue!25}}
}

\renewcommand{\today}{\number\year 年 \number\month 月 \number\day 日}

%五角星
\usepackage{MnSymbol}

%去除图表标题中的figure等
\usepackage{caption}
\captionsetup{labelformat=empty,labelsep=none}

\usepackage{tabu}
\usepackage{multirow}
%表格自动换行
\usepackage{tabularx} 

% 千分号
%\usepackage{textcomp}

%罗马数字
\makeatletter
\newcommand{\rmnum}[1]{\romannumeral #1}
\newcommand{\Rmnum}[1]{\expandafter\@slowromancap\romannumeral #1@}
\makeatother

%分栏
\usepackage{multicol}

%\usepackage{enumitem}
\usepackage{enumerate}

%键盘
\usepackage{keystroke}

%插入源代码
\usepackage{listings}
\lstset{
  language=bash,                  % 程序语言名称:TeX, Perl, R, sh, bash, Awk
  basicstyle=\normalsize\tt,      %\tt指monospace字体族,程序源代码使用此族字体表示更加美观
  numbers=left,                   % 行号位置(左侧)
  numberstyle=\small,             % 行号字体的字号
  stepnumber=1,                   % 行号的显示步长
  numbersep=5pt,                  % 行号与代码间距
  backgroundcolor=\color{white},  % 背景色;需要 \usepackage{color}
  showspaces=false,               % 不显示空格
  showstringspaces=false,         % 不显示代码字符串中的空格标记
  showtabs=false,                 % 不显示 TAB
  tabsize=4, 
  frame=shadowbox,                % 把代码用带有阴影的框圈起来
  captionpos=b,                   % 标题位置
  breaklines=true,                % 对过长的代码自动断行
  breakatwhitespace=false,        % 断行只在空格处
  extendedchars=false,            % 解决代码跨页时,章节标题,页眉等汉字不显示的问题
  %escapeinside={\%*}{*},         % 跳脱字符,添加注释,暂时离开 listings 
  %escapeinside=``,
  commentstyle=\color{red!50!green!50!blue!50}\tt,  %浅灰色的注释
  rulesepcolor=\color{red!20!green!20!blue!20},     %代码块边框为淡青色
  keywordstyle=\color{blue!70}\bfseries\tt,         %代码关键字的颜色为蓝色,粗体
  identifierstyle=\tt,
  stringstyle=\tt,                % 代码字符串的特殊格式
  keepspaces=true,
  breakindent=1em,
  %breakindent=22pt,
  %breakindent=4em,
  breakautoindent=true,
  flexiblecolumns=true,
  aboveskip=1em,                  %代码块边框
  xleftmargin=2em,
  xrightmargin=2em
}

%\setbeamercolor{alerted text}{fg=magenta}
\setbeamercolor{bgcolor}{fg=yellow,bg=cyan}
%\setbeamercolor{itemize/enumerate body}{fg=green}