\begin{document}

%\includeonlyframes{current}

\logo{\includegraphics[height=0.08\textwidth]{qr.png}}

% 在每个Section前都会加入的Frame
\AtBeginSection[]
{
  \begin{frame}<beamer>
    %\frametitle{Outline}
    \frametitle{教学提纲}
    \setcounter{tocdepth}{2}
    \begin{multicols}{2}
      %\tableofcontents[currentsection,currentsubsection]
      \tableofcontents[currentsection]
    \end{multicols}
  \end{frame}
}
% 在每个Subsection前都会加入的Frame
%\AtBeginSubsection[]
%{
  %\begin{frame}<beamer>
%%\begin{frame}<handout:0>
%% handout:0 表示只在手稿中出现
    %\frametitle{Outline}
    %\setcounter{tocdepth}{2}
    %\tableofcontents[currentsection,currentsubsection]
%% 显示在目录中加亮的当前章节
  %\end{frame}
%}

\begin{frame}[plain]
  \begin{center}
    {\Huge 生物信息学\\}
    \vspace{1cm}
    {\LARGE 天津医科大学\\}
    %\vspace{0.2cm}
    {\LARGE 生物医学工程与技术学院\\}
    \vspace{1cm}
    {\large 2020-2021学年下学期(春)\\ 公共选修课}
  \end{center}
\end{frame}
