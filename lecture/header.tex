\documentclass[11pt,a4paper,landscape,twoside]{book}

\usepackage{fontspec}
\setmainfont{Times New Roman}
\setsansfont{Arial}
\setmonofont{Courier New}

\usepackage[BoldFont,SlantFont,CJKchecksingle,CJKnumber]{xeCJK}
\setCJKmainfont[BoldFont={Adobe Heiti Std},ItalicFont={Adobe Kaiti Std}]{Adobe Song Std}
\setCJKsansfont{Adobe Heiti Std}
\setCJKmonofont{Adobe Fangsong Std}
\punctstyle{hangmobanjiao}

\defaultfontfeatures{Mapping=tex-text}
\usepackage{xunicode}
\usepackage{xltxtra}

\XeTeXlinebreaklocale "zh"
\XeTeXlinebreakskip = 0pt plus 1pt minus 0.1pt

%\usepackage{indentfirst}
\makeatletter
\let\@afterindentfalse\@afterindenttrue
\@afterindenttrue
\makeatother
\setlength{\parindent}{2em}

\linespread{1.2}

\usepackage[top=1.5cm,bottom=1.5cm,left=1.5cm,right=5cm,marginparwidth=4.7cm,marginparsep=0.3cm]{geometry}

\usepackage{titlesec}
\titleformat{\chapter}{\centering\LARGE\bfseries}{第 \thechapter 章}{1em}{}
\titlespacing*{\section}{0pt}{0.2\baselineskip}{0.2\baselineskip}

\usepackage{fancyhdr}
\pagestyle{fancy}
\renewcommand{\chaptermark}[1]{\markboth{\small 第 \thechapter 章\quad #1}{}}
\renewcommand{\sectionmark}[1]{\markright{\small \thesection \quad #1}{}}
\fancyhf{}
\fancyhead[ER]{\leftmark}
\fancyhead[OL]{\rightmark}
\fancyhead[EL,OR]{$\cdot$ \thepage \ $\cdot$}
\renewcommand{\headrulewidth}{0.5pt}

\usepackage{xcolor}
\usepackage{graphicx}
\graphicspath{{figures/}}
\usepackage[xetex,bookmarksnumbered=true,bookmarksopen=true,pdfborder=1,breaklinks,colorlinks,linkcolor=blue,urlcolor=blue,citecolor=blue]{hyperref}

\renewcommand{\today}{\number\year 年 \number\month 月 \number\day 日}
\renewcommand{\contentsname}{目录}
\renewcommand{\listfigurename}{插图目录}
\renewcommand{\listtablename}{表格目录}
\renewcommand{\figurename}{图}
\renewcommand{\tablename}{表}
\renewcommand{\bibname}{参考文献}

\renewcommand{\figureautorefname}{图}
\renewcommand{\tableautorefname}{表}
\renewcommand{\footnoteautorefname}{脚注}

\usepackage{booktabs,tabu}

\newtheorem{example}{例}[chapter]

%调整表格行高
\renewcommand{\arraystretch}{0.8}

%调整列表间及其上下的间距
%\usepackage{mdwlist}
\usepackage{enumitem}
\setlist{nosep}

% auto adjust the marginals
\usepackage{marginfix}


%可以添加多姿多彩的边注
\usepackage{todonotes}
\newcommand{\checkpoint}[1]{\todo[linecolor=green!70!white,backgroundcolor=blue!20!white,bordercolor=red,noline,size=\large]{#1}}
\newcommand{\question}[1]{\todo[inline,backgroundcolor=yellow!50!gray]{\textbf{提问:}#1}}
\newcommand{\slide}[1]{\todo[color=green!40,noline]{#1}}
%\newcommand{\slide}[1]{\todo[color=green!40]{#1}}

%在正文和边注间添加分割线
\usepackage{lipsum}
\usepackage{eso-pic}
\usepackage{ifthen}
\usepackage{tikz}

\def\bottommargin{\paperheight - \topmargin - \textheight - \headheight - \headsep - 1in - \voffset}
\def\toptotalheight{\paperheight - \topmargin - \headheight - \headsep - 1in - \voffset}
\def\leftlength{\evensidemargin - 0.5*\marginparsep + 1in + \hoffset}
\def\rightlength{\paperwidth - \evensidemargin + 0.5*\marginparsep - 1in - \hoffset} 

\makeatletter
\newcommand{\nomarginbar}{\let\ESO@HookIIBG\@empty}
\makeatother

\newcommand{\thisisfullsize}{\path (0,0) --  (\paperwidth,\paperheight);}

\newcommand\LeftBar{%
  \put(0,0){%
    \parbox[b][\paperheight]{\paperwidth}{%
      \vfill
      \centering
      \begin{tikzpicture}
        \thisisfullsize
        \draw[line width=1pt] (\leftlength,\bottommargin) -- (\leftlength,\toptotalheight);
      \end{tikzpicture}
      \vfill
}}}

\newcommand\RightBar{ 
  \put(0,0){
    \parbox[b][\paperheight]{\paperwidth}{
      \vfill
      \centering
      \begin{tikzpicture}
        \thisisfullsize
        \draw[line width=1pt] (\rightlength,\bottommargin) -- (\rightlength,\toptotalheight);
      \end{tikzpicture}
      \vfill
}}}

%%% Use this in two-side documents
\AtBeginShipout{
  \ifthenelse{\isodd{\value{page}}}
  {\AddToShipoutPictureBG*{\LeftBar}
  }
  {\AddToShipoutPictureBG*{\RightBar}
  }
}

% %%% Use this in one-side documents
% \AtBeginShipout{%
%   \AddToShipoutPictureBG*{\RightBar}%
% }

%%% Use this anyway (to take care of the first page of the document)
\AtBeginDocument{
\AddToShipoutPictureBG*{\RightBar}
}


%设置颜色的快捷命令
\newcommand{\red}{\textcolor{red}}
\newcommand{\gray}{\textcolor{gray}}
\newcommand{\black}{\textcolor{black}}

%罗马数字
\makeatletter
\newcommand{\rmnum}[1]{\romannumeral #1}
\newcommand{\Rmnum}[1]{\expandafter\@slowromancap\romannumeral #1@}
\makeatother
