\documentclass[11pt,a4paper,landscape,twoside]{book}

\usepackage{fontspec}
\setmainfont{Times New Roman}
\setsansfont{Arial}
\setmonofont{Courier New}

\usepackage[BoldFont,SlantFont,CJKchecksingle,CJKnumber]{xeCJK}
\setCJKmainfont[BoldFont={Adobe Heiti Std},ItalicFont={Adobe Kaiti Std}]{Adobe Song Std}
\setCJKsansfont{Adobe Heiti Std}
\setCJKmonofont{Adobe Fangsong Std}
\punctstyle{hangmobanjiao}

\defaultfontfeatures{Mapping=tex-text}
\usepackage{xunicode}
\usepackage{xltxtra}

\XeTeXlinebreaklocale "zh"
\XeTeXlinebreakskip = 0pt plus 1pt minus 0.1pt

%\usepackage{indentfirst}
\makeatletter
\let\@afterindentfalse\@afterindenttrue
\@afterindenttrue
\makeatother
\setlength{\parindent}{2em}

\linespread{1.2}

\usepackage[top=1.5cm,bottom=1.5cm,left=1.5cm,right=5cm,marginparwidth=4.7cm,marginparsep=0.3cm]{geometry}

\usepackage{titlesec}
\titleformat{\chapter}{\centering\LARGE\bfseries}{第 \thechapter 章}{1em}{}
\titlespacing*{\section}{0pt}{0.2\baselineskip}{0.2\baselineskip}

\usepackage{fancyhdr}
\pagestyle{fancy}
\renewcommand{\chaptermark}[1]{\markboth{\small 第 \thechapter 章\quad #1}{}}
\renewcommand{\sectionmark}[1]{\markright{\small \thesection \quad #1}{}}
\fancyhf{}
\fancyhead[ER]{\leftmark}
\fancyhead[OL]{\rightmark}
\fancyhead[EL,OR]{$\cdot$ \thepage \ $\cdot$}
\renewcommand{\headrulewidth}{0.5pt}

\usepackage{xcolor}
\usepackage{graphicx}
\graphicspath{{figures/}}
\usepackage[xetex,bookmarksnumbered=true,bookmarksopen=true,pdfborder=1,breaklinks,colorlinks,linkcolor=blue,urlcolor=blue,citecolor=blue]{hyperref}

\renewcommand{\today}{\number\year 年 \number\month 月 \number\day 日}
\renewcommand{\contentsname}{目录}
\renewcommand{\listfigurename}{插图目录}
\renewcommand{\listtablename}{表格目录}
\renewcommand{\figurename}{图}
\renewcommand{\tablename}{表}
\renewcommand{\bibname}{参考文献}

\renewcommand{\figureautorefname}{图}
\renewcommand{\tableautorefname}{表}
\renewcommand{\footnoteautorefname}{脚注}

\usepackage{booktabs,tabu}

\newtheorem{example}{例}[chapter]

%调整表格行高
\renewcommand{\arraystretch}{0.8}

%调整列表间及其上下的间距
%\usepackage{mdwlist}
\usepackage{enumitem}
\setlist{nosep}

% auto adjust the marginals
\usepackage{marginfix}


%可以添加多姿多彩的边注
\usepackage{todonotes}
\newcommand{\checkpoint}[1]{\todo[linecolor=green!70!white,backgroundcolor=blue!20!white,bordercolor=red,noline,size=\large]{#1}}
\newcommand{\question}[1]{\todo[inline,backgroundcolor=yellow!50!gray]{\textbf{提问:}#1}}
\newcommand{\slide}[1]{\todo[color=green!40,noline]{#1}}
%\newcommand{\slide}[1]{\todo[color=green!40]{#1}}

%在正文和边注间添加分割线
\usepackage{lipsum}
\usepackage{eso-pic}
\usepackage{ifthen}
\usepackage{tikz}

\def\bottommargin{\paperheight - \topmargin - \textheight - \headheight - \headsep - 1in - \voffset}
\def\toptotalheight{\paperheight - \topmargin - \headheight - \headsep - 1in - \voffset}
\def\leftlength{\evensidemargin - 0.5*\marginparsep + 1in + \hoffset}
\def\rightlength{\paperwidth - \evensidemargin + 0.5*\marginparsep - 1in - \hoffset} 

\makeatletter
\newcommand{\nomarginbar}{\let\ESO@HookIIBG\@empty}
\makeatother

\newcommand{\thisisfullsize}{\path (0,0) --  (\paperwidth,\paperheight);}

\newcommand\LeftBar{%
  \put(0,0){%
    \parbox[b][\paperheight]{\paperwidth}{%
      \vfill
      \centering
      \begin{tikzpicture}
        \thisisfullsize
        \draw[line width=1pt] (\leftlength,\bottommargin) -- (\leftlength,\toptotalheight);
      \end{tikzpicture}
      \vfill
}}}

\newcommand\RightBar{ 
  \put(0,0){
    \parbox[b][\paperheight]{\paperwidth}{
      \vfill
      \centering
      \begin{tikzpicture}
        \thisisfullsize
        \draw[line width=1pt] (\rightlength,\bottommargin) -- (\rightlength,\toptotalheight);
      \end{tikzpicture}
      \vfill
}}}

%%% Use this in two-side documents
\AtBeginShipout{
  \ifthenelse{\isodd{\value{page}}}
  {\AddToShipoutPictureBG*{\LeftBar}
  }
  {\AddToShipoutPictureBG*{\RightBar}
  }
}

% %%% Use this in one-side documents
% \AtBeginShipout{%
%   \AddToShipoutPictureBG*{\RightBar}%
% }

%%% Use this anyway (to take care of the first page of the document)
\AtBeginDocument{
\AddToShipoutPictureBG*{\RightBar}
}


%设置颜色的快捷命令
\newcommand{\red}{\textcolor{red}}
\newcommand{\gray}{\textcolor{gray}}
\newcommand{\black}{\textcolor{black}}

%罗马数字
\makeatletter
\newcommand{\rmnum}[1]{\romannumeral #1}
\newcommand{\Rmnum}[1]{\expandafter\@slowromancap\romannumeral #1@}
\makeatother


\begin{document}

\setcounter{chapter}{3}
\chapter{核酸序列分析}
\slide{课程标题页}

\section{课前甜点} \checkpoint{5, -00:05--00:00 (\^{}1.1-)} 
播放《飞越五千年-兵圣》中“孙武练兵”的片段。\red{“约束不明,申令不熟,将之罪也;既已明而不如法者,吏士之罪也。”}
%\slide{孙武练兵的图片}
%正式上课之前先讲一个故事,叫做“孙武练兵”\marginpar{《史记·孙子吴起列传》}。

%孙子武者,齐人也。以兵法见於吴王阖庐。阖庐曰:“子之十三篇,吾尽观之矣,可以小试勒兵乎?”对曰:“可。”阖庐曰:“可试以妇人乎?”曰:“可。”於是许之,出宫中美女,得百八十人。孙子分为二队,以王之宠姬二人各为队长,皆令持戟。令之曰:“汝知而心与左右手背乎?”妇人曰:“知之。”孙子曰:“前,则视心;左,视左手;右,视右手;后,即视背。”妇人曰:“诺。”约束既布,乃设鈇钺,即三令五申之。於是鼓之右,妇人大笑。孙子曰:“\red{约束不明,申令不熟,将之罪也。}”复三令五申而鼓之左,妇人复大笑。孙子曰:“\red{约束不明,申令不熟,将之罪也;既已明而不如法者,吏士之罪也。}”乃欲斩左古队长。吴王从台上观,见且斩爱姬,大骇。趣使使下令曰:“寡人已知将军能用兵矣。寡人非此二姬,食不甘味,愿勿斩也。”孙子曰:“臣既已受命为将,将在军,君命有所不受。”遂斩队长二人以徇。用其次为队长,於是复鼓之。妇人左右前后跪起皆中规矩绳墨,无敢出声。於是孙子使使报王曰:“兵既整齐,王可试下观之,唯王所欲用之,虽赴水火犹可也。”吴王曰:“将军罢休就舍,寡人不愿下观。”孙子曰:“王徒好其言,不能用其实。”於是阖庐知孙子能用兵,卒以为将。西破彊楚,入郢,北威齐晋,显名诸侯,孙子与有力焉。

%孙子名武,是齐国人。因为他精通兵法受到吴王阖庐的接见。阖庐说:“您的十三篇兵书我都看过了,可用来小规模地试着指挥军队吗?”孙子回答说:“可以。”阖庐说:“可以用妇女试验吗?”回答说:“可以。”于是阖庐答应他试验,叫出宫中美女,共约百八十人。孙子把她们分为两队,让吴王阖庐最宠爱的两位侍妾分别担任各队队长,让所有的美女都拿一支戟。然后命令她们说:“你们知道自己的心、左右手和背吗?”妇人们回答说:“知道。”孙子说:“我说向前,你们就看心口所对的方向;我说向左,你们就看左手所对的方向;我说向右,你们就看右手所对的方向;我说向后,你们就看背所对的方向。”妇人们答道:“是。”号令宣布完毕,于是摆好斧铖等刑具,旋即又把已经宣布的号令多次重复地交待清楚。就击鼓发令,叫她们向右,妇人们都哈哈大笑。孙子说:“纪律还不清楚,号令不熟悉,这是将领的过错。\marginpar{\red{约束不明,申令不熟,将之罪也。}}”又多次重复地交待清楚,然后击鼓发令让她们向左,妇人们又都哈哈大笑。孙子说:“纪律弄不清楚,号令不熟悉,这是将领的过错;现在既然讲得清清楚楚,却不遵照号令行事,那就是军官和士兵的过错了。\marginpar{\red{约束不明,申令不熟,将之罪也;既已明而不如法者,吏士之罪也。}}”于是就要杀左、右两队的队长。吴王正在台上观看,见孙子将要杀自己的爱妾,大吃一惊。急忙派使臣传达命令说:“我已经知道将军善用兵了,我要没了这两个侍妾,吃起东西来也不香甜,希望你不要杀她们吧。”孙子回答说:“我已经接受命令为将,将在军队里,国君的命令有的可以不接受。”于是杀了两个队长示众。然后按顺序任用两队第二人为队长,于是再击鼓发令,妇人们不论是向左向右、向前向后、跪倒、站起都符合号令、纪律的要求,再没有人敢出声。于是孙子派使臣向吴王报告说:“队伍已经操练整齐,大王可以下台来验察她们的演习,任凭大王怎样使用她们,即使叫她们赴汤蹈火也办得到啊。”吴王回答说:“让将军停止演练,回宾馆休息。我不愿下去察看了。”孙子感叹地说:“大王只是欣赏我的军事理论,却不能让我付诸实践。”从此,吴王阖庐知道孙子果真善于用兵,终于任命他做了将军。后业吴国向西打败了强大的楚国,攻克郢都,向北威震齐国和晋国,在诸侯各国名声赫赫,这其间,孙子不仅参与,而且出了很大的力啊。

给大家看这个故事的目的是要大家明白:一是“国有国法,家有家规”,三百六十行都有自己的的职业操守;二是“无规矩不成方圆”,有规矩就要遵守,这样才能把事情做好。

在正式上课之前先明确一下作为一个学生应该遵守的课堂纪律\marginpar{明确课堂纪律,收集学生反馈}: 
\slide{课堂纪律条目}

\begin{itemize}
	\item 只有正式上课前的请假有效。
	\item 提前5分钟到教室,严禁迟到。
	\item 上课期间手机关机或调成震动。
	\item 上课期间离开教室先举手示意。
	\item 课上有疑问的话先举手后提问。
	\item 上课期间严禁交头接耳,大声喧哗。
	\item 随机点名,缺勤扣分如下:1、3、6。
	\item 缺勤三次或三次以上者,平时成绩为0。
\end{itemize}

对上述规范有异议的话,可以以不记名的方式把自己的想法写在纸上,课间或课后交给我。如果对授课内容、方式等有何建议,可以通过各种方式反馈给我。下面正式开始上课。

\section{自我介绍}\checkpoint{1, 00:00--00:01 (-1.1-)}
\slide{自我介绍与邮箱网盘}
因为是首次上课,我先用一分钟的时间做一下自我介绍。

伊现富(Yi Xianfu),1986年生人,本科毕业于山东大学生命科学学院,学的是生物科学专业;之后保送到中国科学院上海生命科学研究院读研究生,主要从事人类复杂疾病相关的生物信息学研究。

我常用的邮箱有两个:\href{mailto:yixfbio@gmail.com}{yixfbio@gmail.com},主要用于工作中的科研业务交流;\href{mailto:yixf1986@gmail.com}{yixf1986@gmail.com},主要用于生活中的闲杂琐事交流。联系电话:15620610763。个人博客:\href{http://yixf.name}{http://yixf.name},内容五花八门,其中有和生物信息学相关的一些资料,感兴趣的可以去看看。网络上的昵称以“yixf”为主。

为了方便进行信息交流与资源共享,我注册了一个126的邮箱,账号:\textcolor{red}{bioinfo\_TIJMU@126.com},密码:\textcolor{red}{\texttt{C\&563f\&nzx!s}};申请了一个百度云网盘,账号:\textcolor{red}{bioinfo\_TIJMU@126.com},密码:\textcolor{red}{\texttt{566\&Us3Rp6\#C}}。\marginpar{密码复杂度,密码管理软件}授课讲义、幻灯片、视频等资料都会存储在网盘中,有需要的自行登录下载。

\question{1.生物信息学的英文读法与拼写。2.绪论中提到的一种高通量技术。(芯片$\Rightarrow$第二代测序(Next Generation Sequencing,NGS)技术、单分子实时(Single Molecule Real Time,SMRT)DNA测序技术)}
\section{引言}\checkpoint{5, 00:01--00:05 (-1.1-)}
“龙生龙,凤生凤,老鼠的儿子会打洞!”“种瓜得瓜,种豆得豆。”这些都是大家耳熟能详的谚语。不管是天上飞的、地上跑的、水里游的,还是能动的、不能动的,它们的后代都和它们非常相像,但却也会有少许的差异。这些现象大家都已司空见惯,所以可能没有啥感觉。但仔细想想,你就会发现大自然的奇妙所在。当然,对于生物专业的人来说,这个就没什么奇怪的了,因为我们都知道分子生物学的中心法则(The central dogma of molecular biology)\slide{生物大千世界\\ 中心法则示意图}:DNA转录成RNA,RNA翻译成蛋白质。蛋白质执行特定的生物功能从而决定最终的表型,而DNA则携带着最原始的决定个体性状的遗传信息,RNA主要参与遗传信息的表达和调控。在各种生物中,ACGT都是构成DNA和RNA核酸序列的基本组分。仅仅这么四种碱基怎么可能构建出缤纷多彩的大千世界呢?其秘诀就在于四种核苷酸的排列顺序。就像搭积木一样,通过不同的排列组合我们可以构建出不同的形状。核酸序列中不同的碱基排列顺序,蕴含着不同的生物信息,包括遗传信息和进化信息等。如何从海量的核酸序列中挖掘信息\slide{01世界与ACGT世界:\\ 序列示例\\ Matrix图片},并将其与生物性状联系起来,这是生物信息学的主要研究领域之一,也是提取分子生物信息的首要步骤。

在接下来的四次课中,我将介绍DNA和RNA序列分析、基因组结构及功能注释的相关内容;而蛋白质序列与结构的内容将由耿鑫和张涛老师分别进行介绍。

DNA是主要的遗传物质,是携带遗传信息的载体之一。DNA序列是指DNA的一级结构,基本组份由ACGT四种碱基组成,因此又称为碱基序列。不同种属的DNA碱基组份存在差异。DNA主要携带两类遗传信息。\slide{基因及调控元件结构示意图}一类信息储存在具有功能活性的DNA序列中,能够通过转录过程形成RNA(主要有编码RNA和非编码RNA两种形式),其中编码RNA含有编码蛋白质的氨基酸序列信息,这类DNA序列主要是指遗传的基本单位即功能序列。一类信息属于调控信息,主要存在于特定DNA的区域,能被各种功能性蛋白分子特异地识别结合,进而完成各种生物过程,例如启动子和增强子调控基因的表达。遗传信息储存于具有特征信息的DNA序列中,根据这些特征信息设计不同的算法并开发相应的分析工具,能够在海量的序列数据中挖掘出具有生物学功能的特征信息。

本堂课将介绍DNA序列一级结构的基本信息和序列的特征信息分析方法。DNA基本信息中主要包括序列碱基组份(base composition)分析、序列转换、限制性内切酶位点分析;序列的特征信息主要包括开放阅读框(Open Reading Frame,ORF)、启动子及转录因子结合位点的分析和CpG岛(CpG island)的识别。

\section{DNA序列转换与组份分析}\checkpoint{25, 00:05--00:30 (-1.1-)}
美籍奥地利犹太生物学家Erwin Chargaff在分析组成DNA的碱基时,首先注意到了DNA碱基组成的某些规律性。他在1950年发现DNA中的腺嘌呤与胸腺嘧啶数量几乎完全一样,鸟嘌呤与胞嘧啶的数量也是一样。这项发现后来成为查戈夫第一法则。查戈夫的研究帮助克里克及沃森推断出DNA的双螺旋形结构。第二法则则表示不同物种之间的DNA组合是不同的,特别是A、G、T及C之间的相对数量。这两条法则后来都被称为查戈夫法则\slide{查戈夫法则}:
\begin{itemize}
	\item 腺嘌呤和胸腺嘧啶的摩尔数相等,即A=T;鸟嘌呤和胞嘧啶的摩尔数也相等,即G=C。由此可推导出含氨基的碱基(腺嘌呤和胞嘧啶)总数等于含酮基的碱基(鸟嘌呤和胸腺嘧啶)总数,即A+C=T+G;嘌呤的总数等于嘧啶的总数,即A+G=C+T。
	\item 不同生物种属的DNA碱基组成不同,即AT/GC的比值因生物种类不同而异。
\end{itemize}

GC含量(GC content)\slide{GC含量简介}是在所研究的对象的全基因组中,鸟嘌呤(Guanine)和胞嘧啶(Cytosine)所占的比例。一种生物的基因组或特定DNA、RNA片段有特定的GC含量。在DNA链中G和C是以三个氢键相连,而T和A则是两个氢键相连的。氢键的多少体现连接的能量,氢键多的不容易被打断。因此GC含量高的DNA比GC含量低的DNA更加稳定。在双链DNA中,腺嘌呤与胸腺嘧啶(A/T)之比,以及鸟嘌呤与胞嘧啶(G/C)之比都是1。但是,(A+T)/(G+C)之比则随DNA的种类不同而异。

GC含量\slide{GC含量计算公式}通常以百分数的形式进行表示,计算公式如下:$\frac{G+C}{A+T+G+C}\times100$。但有时也以比值的形式表示,叫做GC比(GC-ratio)。AT/GC比值的计算公式如下:$\frac{A+T}{G+C}$。

原核生物中不同种属的GC含量从25\%到75\%不等,这种组分差异可用于识别细菌种类。(恶性疟原虫的GC含量仅有20\%左右,这种情况下一半说它富含AT(AT-rich),而不会说缺少GC(GC-poor)。)真核生物物种间GC含量的差异不如原核生物明显,但真核基因组中不同区域的GC含量存在差异,基因区的GC含量比整个基因组背景上的GC含量要高。GC含量与密码子使用偏性、DNA双链的溶解温度有关,是进行核酸杂交的重要参数。\slide{物种GC含量比较\\ 不同区域GC含量比较\\ 基因与基因组GC含量比较}

DNA序列具有双链性与双链互补性,因此进行序列分析时,经常需要针对DNA序列进行各种转换,例如:反向序列、互补序列、反向互补序列、显示DNA双链、转换为RNA序列等。其中最常用的是获取反向互补序列。
\question{问什么常说“反向互补”,而不是“反向”、“互补”呢?}
序列的书写惯例\slide{序列书写惯例}:
\begin{itemize}
	\item DNA/RNA:[左] 5' $\Longrightarrow$ 3' [右]
	\item 多肽/蛋白质:[左] N端(氨基端)$\Longrightarrow$ C端(羧基端) [右]
\end{itemize}

教材中演示了使用EditSeq获取DNA反向互补序列的操作\slide{短序列互补实例(板书)}。本质上就是A转换成T、C转换成G、G转换成C、T转换成A,获得互补序列,然后将互补序列反向输出得到最终的反向互补序列。当然,先反向后互补结果是一样的的。

而对于序列碱基组成和GC含量的分析,无非就是进行简单的计数与运算。根据具体任务的不同,采用的策略和工具也会有所不同。
\question{你接到一个进行碱基组成和GC含量的分析的任务,该如何入手解决?}
首先分析任务的属性:序列长短、序列数目多少、处理同样任务的频率,然后再决定采用那种策略\slide{实例与策略(板书)}:
\begin{itemize}
	\item 序列短、数目少:一个一个查也用不了多少时间
	\item 序列长、数目少:另辟蹊径,办法绝不止一种(如:巧用word替换功能进行计数)
	\item 序列数目多:找现成工具(他山之石,可以攻玉);请别人帮忙(术业有专攻);自己写程序(自己动手,丰衣足食);……
\end{itemize}

此处仅给大家提供一个解决问题的思路:遇到问题,先思考后动手;办法总会有的,而且不止一种;没有完美的方法,只有合适的方法。\marginpar{There's More Than One Way To Do It.(不只一种方法来做一件事。)\\ Don't Reinvent the Wheel.(不要重复发明轮子)}
\question{[课后思考]如何获取DNA序列的反向、互补及反向互补序列?}

\section{限制性核酸内切酶位点分析}\checkpoint{10, 00:30--00:40 (-1.1-)}
限制酶(restriction enzyme)又称限制内切酶或限制性内切酶(restriction endonuclease),全称限制性核酸内切酶\slide{限制酶定义及别名},是可以识别DNA的特异序列、并在识别位点或其周围切割双链DNA的一类内切酶。限制酶可以将异源性DNA切断并使之失活,限制异源DNA的侵入。但对自身DNA无损伤作用,从而维持细胞原有遗传信息的完整性。限制酶的切割形式有两种,分别是可产生具有突出单股DNA的黏状末端,以及末端平整无凸起的平滑末端。染色体或DNA上断开的不同限制片段可由DNA连接酶黏合,因此限制酶在分子生物学与遗传工程领域有着广泛的应用。
\question{[课后思考]限制酶如何实现只切割外源DNA而不损伤内源DNA?}

\question{\textit{Eco}R\Rmnum{1}的名称是如何确定的?}
限制酶的命名\slide{\textit{Eco}R\Rmnum{1}的命名示例}根据细菌种类而定,一般由微生物属名的第一个字母和种名的前两个字母组成,第四个字母表示菌株(品系)。在同一品系细菌中得到的识别不同碱基顺序的几种不同特异性的酶,可以编成不同的号。以\textit{Eco}R\Rmnum{1}为例:\textit{E}源于属名\textit{Escherichia},\textit{co}源于种名\textit{coli},R代表RY13品系,\Rmnum{1}表明在此类细菌中发现的顺序——首次发现。

限制酶分为\Rmnum{1}、\Rmnum{2}、\Rmnum{3}型三大类。其中,\Rmnum{2}型限制酶能识别专一的、短的DNA序列,并在识别位点或附近切割双链DNA。这类限制酶具有专一的识别和切割位点,是基因工程中实用性较高的限制酶种类。限制酶识别的序列长度一般为4-8个碱基,常见的是6个碱基,且多数为回文对称结构\slide{回文结构};切割的序列通常就是其识别的序列,切割位点在DNA两条链相对称的位置。切割位点在回文的一侧时,可形成黏性末端,如:\textit{Eco}R\Rmnum{1}、\textit{Bam}H\Rmnum{1}、\textit{Hin}d\Rmnum{3}等;另一些酶如\textit{Alu}\Rmnum{1}、\textit{Sma}\Rmnum{1}等,切割位点在回文序列的中间,形成平滑末端。\slide{\Rmnum{2}型限制酶识别及切割序列示例}
\question{什么是回文结构?}
\question{[课后思考]\Rmnum{1}、\Rmnum{3}型有什么特点,和\Rmnum{2}型有什么区别?}

核酸序列中的限制性内切酶位点识别依据限制酶所识别的序列结构信息进行预测分析。常用的限制酶资源是限制酶数据库(The Restriction Enzyme Database,REBASE),它收录了限制酶的所有信息,包括限制酶识别序列和作用位点、甲基化酶、甲基化特异性、酶类产品的商业来源和参考文献等。REBASE提供了限制酶的查询工具、识别位点序列及限制酶酶切双链DNA的三维结构等信息;分析工具提供理论酶切消化图谱、序列比对、酶切位点分析等功能。

常用的限制性核酸内切酶位点分析工具是NEBCutter V2.0,可以产生DNA序列的酶切位点分析结果。它使用的限制酶来源于REBASE数据库,识别位点列表每天根据REBASE数据库数据同步更新。NEBCutter V2.0可提供单一酶切或多选酶切位点识别和模拟消化图谱。此外,很多DNA分析的软件也都含有没切位点分析的功能。
\question{如何找到NEBCutter V2.0?}

\section{开放阅读框分析}\checkpoint{10, 00:40--00:50 (-1.1\$)}
开放阅读框(Open Reading Frame,ORF,开放阅读框架、开放读架等)是指在给定的阅读框架中,不包含终止密码子的一串序列。这段序列是生物个体的基因组中可能作为蛋白质编码序列的部分,包含从5'端翻译起始密码子(ATG)到终止密码子(TAA、TAG、TGA)之间的一段编码蛋白质的碱基序列。由于一段DNA或RNA序列有多种不同读取方式,因此可能同时存在许多不同的开放阅读框架。

对于任何给定的核酸序列,根据密码子的起始位置,可以按照三种方式进行解释,其反向互补序列又含有三种,因此,一条DNA序列可以按六种框架阅读和翻译\slide{六种读框示例}。ORF的识别需要检测这六个阅读框架并决定哪一个包含以启动子和终止子为界限的DNA序列而其内部不包含终止子。

一个ORF存在一个潜在的编码序列(Coding Sequence,CDS),不同的ORF翻译成氨基酸可以得到不同的蛋白质编码。一个ORF对应一个候选的CDS,分析DNA序列中的ORF是对该序列是否为CDS的初步判断,是研究DNA序列片段的方法之一。
\question{[课后思考]ORF与CDS的区别?}

ORF的预测常与第一个ATG和终止密码子的确定相关,但仅凭第一个ATG和终止密码子是不足以确定ORF的。ORF的预测程序主要是对编码区进行特征统计、相关模式的识别或利用同源比对的方法识别。原核生物编码区通常只含有一个单独的ORF,识别方法相对简单,即最长ORF法。而真核生物的编码区被内含子分割成数个不连续的外显子,其编码区序列分析更加复杂。

ORF分析常用的程序是NCBI的在线分析工具ORF Finder。在其他一些集成化的软件中也有ORF分析相关的工具。

\section{启动子分析}\checkpoint{10, 00:00--00:10 (\^{}1.2-)}
DNA序列中储存着调控信息,其中转录调控控制基因的转录活性。基因的转录表达是生命体的基本生物过程,生物个体在不同发育阶段、不同组织、不同生理状态下,基因的转录调控也不同。真核基因调控主要是在转录水平上进行的,受大量特定的顺式作用元件(cis-acting element)和反式作用因子(trans-acting factor,又称跨域作用因子)的调控,真核生物的转录调控大多数是通过顺式作用元件和反式作用因子复杂的相互作用来实现的。顺式作用元件位于基因的旁侧,是能够影响基因表达的核酸序列,包括启动子(promoter) 、增强子(enhancer)、应答元件(responsive elements)等,其活性只影响与其自身同处于一个DNA分子上的基因。顺式作用元件本身并不编码蛋白质,仅仅提供一个作用位点,与反式作用因子相互作用参与基因表达调控。反式作用因子是参与调控靶基因转录效率的蛋白质,可以直接或间接地识别或结合在各类顺式作用元件核心序列上,可对基因表达产生激活或阻遏的作用。

启动子是一段位于转录起始位点5'端上游区的DNA序列,能活化RNA聚合酶,使之与模板DNA准确地结合并具有转录起始的特异性。转录起始位点(Transcription Start Site,TSS)是指与新生RNA链第一个核苷酸相对应DNA链上的碱基,研究证实通常为一个嘌呤。常把起点前面,即5'端的序列称为上游序列(upstream),而把其后面即3'端的序列称为下游序列(downstream)。在描述剪辑的位置时,一般用数字表示,起点为+1,下游方向依次为+2,+3,……,上游方向依次为-1,-2,-3……序列的书写方向通常是固定的,使转录从左(上游)向右(下游)进行,mRNA同样按照5'$\rightarrow$3'方向书写。原核基因启动子\slide{原核基因启动子结构}具有明显共同一致的序列,包含两个短序列,分别位于从转录起结点起计的-10及-35上游位置,是RNA聚合酶与启动子的结合位点。位于-10的序列称为-10元件或-10区(Pribnow区,Pribnow box),通常包含TATAAT6个核苷;位于-35的序列通常包含TTGACA6个核苷。在真核生物基因中\slide{真核基因启动子结构},类似原核基因启动子Pribnow区的Hogness区(Hogness box),是位于转录起始点上游-25~-30bp处的共同序列TATAAA,也称为TATA区。另外,在起始位点上游-70~-78bp处还有另一段共同序列CCAAT,这是与原核生物中-35bp相对应的序列,称为CAAT区(CAAT box)。
\question{[课后思考]各特征序列区的作用。}

转录因子(transcription factor)是指能够结合在某基因上游特异核苷酸序列上的蛋白质,这些蛋白质能调控其基因的转录。转录因子可以调控RNA聚合酶与DNA模板的结合。转录因子一般有不同的功能区域,如DNA结合结构域与效应结构域。转录因子不单与基因上游的启动子区域结合,也可以和其它转录因子形成转录因子复合体来影响基因的转录。转录因子结合位点(Transcription Factor Binding Site,TFBS)是与转录因子结合的DNA序列,长度约为5~20bp\slide{TFBS的序列标识图(留悬念)},它们与转录因子相互作用进行基因的转录调控。同一转录因子能够同时调控多个基因,虽然与不同基因序列的结合位点具有一定的保守性,但又存在一定的可变性。

识别基因的调控区序列特征信息是研究基因功能、基因转录调控规律、识别新基因及解析基因组结构的途径之一。挖掘调控区序列特征信息的方法主要有同源匹配法和模式识别法。TFBS是较短的DNA片段,在整个基因组中会存在大量的重复序列,这些特征给正确识别TFBS带来一定的难度,也使得预测方法普遍存在较高的假阳性率。

随着基因表达调控研究的深入,越来越多的结合位点的调控区域序列信息不断产生。启动子和转录因子结合位点信息存储在相关的数据库中。EPD(Eukaryotic Promoter Database)是一个有注释的非冗余的真核生物RNA聚合酶II(Pol II)启动子数据集,其中的转录起始位点(TSS)都是通过实验获得的。TRANSFAC是真核生物转录调控信息的数据库,包括转录因子、转录因子结合位点及转录调控关系等信息,收录的数据都经过实验验证。

启动子、转录因子结合位点等保守的功能区可通过序列分析获得相应的序列特征信息。分析工具能直接搜索目的DNA序列中是否含有已知位点的序列模式。Promoter Scan根据转录因子结合序列同源性分析预测DNA中的启动子区域;Promoter 2.0基于遗传算法的人工神经网络技术预测脊椎动物启动子区Pol II和其他调控因子结合位点的信息。Tfblast(TRANSFAC BLAST)可以根据比对算法找出目标DNA序列中可能存在的转录因子结合位点。\marginpar{TESS(Transcription Element Search System)因资源限制已被移除。}

\section{CpG岛识别}\checkpoint{10, 00:10--00:20 (-1.2-)}
CpG二核苷酸占哺乳动物基因组的5\%~10\%,其中,70\%~80\%呈甲基化状态,称为甲基化的CpG(mCpG)。但CpG的分布很不均一,在基因组的某些区段,CpG保持或高于正常概率,这些区段被称作CpG岛(CpG island)\slide{CpG岛的特征}。CpG岛主要位于脊椎动物基因,尤其是看家基因(housekeeping gene)的转录起始位点附近,长度约300~3000bp。对于哺乳动物的基因来说,约40\%的启动子(人类约70\%)含有CpG岛。几乎看家基因都含有CpG岛;一般位于基因的5'端区域;大多数CpG岛是未甲基化的;CpG岛中的核小体中H1含量低,其他组蛋白被广泛乙酰化,并具有超敏感位点;未甲基化CpG岛可能说明基因具有潜在活性。对人类21和22号染色体全序列进行的分析表明,GC含量超过55\%、CpG二核苷酸的出现率(观测值与期望值的比率)达到65\%且长度超过500bp的DNA区域更可能是分布在基因5'端区域的真的CpG岛。CpG岛是表观遗传学中重要的作用区域,CpG岛甲基化是基因转录活性的调控因素之一,CpG岛甲基化异常常伴随着疾病的发生。
\question{现在学习的知识一定正确吗?(知识的时效性)}
\question{[课后思考]假阳性、真阳性、假阴性、真阴性的含义及其之间的关系。}

传统的CpG岛识别方法主要依据三个序列特征\slide{CpG岛预测标准}:GC含量、CpG岛长度、CpG二核苷酸的出现频率。确定一个区域为CpG岛的常用标准为:至少长200bp,GC含量超过50\%,CpG的观察值与预测值的比率高于60\%。其中,CpG的观察值与预测值的比率计算公式为:$\frac{Num\ of\ CpG}{Num\ of\ C \times Num\ of\ G} \times Total\ number\ of\ nucleotides\ in\ the\ sequence$。另一类主要方法是基于统计学特征的识别方法,如使用马尔科夫链和隐马尔科夫链识别CpG岛。

EMBL提供的CpG岛的计算工具是EMBOSS的CpGPlot/CpGReport/Isochore,基于传统的窗口滑动法,参数设置默认CpG岛跨度至少为200bp,GC含量\textgreater 50\%,CpG出现频率\textgreater 0.6,满足这些条件的区域都预测为CpG岛\slide{演示使用cpgplot预测CpG的操作}。其他的类似工具还有CpG Island Searcher、CpGcluster2等。

\section{重复序列分析}\checkpoint{10, 00:20--00:30 (-1.2-)}
基因组注释包括结构注释和功能注释,结构注释的核心是基因识别,而为了提高基因识别的效率,首先要寻找并屏蔽重复的、低复杂性的序列。

重复序列(repetitive sequence, repeated sequence)\slide{重复序列的分类}是指真核生物基因组中重复出现的核苷酸序列。这些序列一般不编码多肽,在基因组内可成簇排布,也可散布于基因组。

根据重复次数的多少,可以分成三大类:(1)低度重复序列(lowly repetitive sequence),在整个基因组中只含有2~10个拷贝,如酵母tRNA基因、人和小鼠的珠蛋白基因等;(2)中度重复序列(moderately repetitive sequence),重复次数为几十次到几千次,重复单元的平均长度约300bp,如rRNA和tRNA基因;(3)高度重复序列(highly repetitive sequence),重复几百万次,一般是少于10个核苷酸残基组成的短片段,如异染色质上的卫星DNA。

按照重复序列的组织形式可以分成两大类:串联重复序列和散在重复序列。前一种成簇存在于染色体的特定区域,后一种分散于染色体的各位点上。串联重复(tandem repeat)包括卫星DNA(satellite DNA)、小卫星(minisatellite)和微卫星(microsatellite)。卫星DNA是一类高度重复序列。小卫星DNA (minisatellite DNA )又称可变数目串联重复(variable number tandem repeat,VNTR),由10~100bp的基本单位串联而成,总长通常不超过20kb,重复次数在群体中是高度变异的。在人类基因组中,约90\%的小卫星序列出现在靠近端粒的位置。微卫星又称为简单重复序列(Simple Sequence Repeats,SSRs)或短串联重复序列(Short Tandem Repeats,STRs),指两个或多个核苷酸重复排列、且不同的重复序列相邻的形式,只有2~10bp,串联成簇,长度50~100bp,常见于非编码的内含子中。微卫星是多型性的一种类型,由于重复单位及重复次数不同,使其在不同种族、不同人群之间的分布具有很大差异性,构成了STR遗传多态性。散在重复(Interspersed repeat)一般都是中度重复序列。根据重复序列的长度可以分为短散在重复序列(Short Interspersed Nuclear Element,SINE)和长散在重复序列(Long Interspersed Nuclear Element,LINE)。前者长度在500bp以下,在人基因组中的重复拷贝数达10万以上;后者长度在1000bp以上,在人基因组中有上万份拷贝。LINE是可以自主转座的一类反转录转座子,来源于RNA聚合酶\Rmnum{2}的转录产物;SINE则是非自主转座的反转录转座子,来源于RNA聚合酶\Rmnum{3}的转录产物。在灵长类中,主要的LINE和SINE分别为L1和Alu。 
\question{[课后思考]不同分类标准之间的关系。}

对于真核生物的核酸序列而言,在进行基因识别之前首先应该把简单的大量的重复序列标记出来并去除,目的是为了避免重复序列对预测程序产生干扰,尤其是涉及数据库搜索的程序。

不同重复序列数据库储存了不同类型重复序列的信息:Repbase是常用的真核生物DNA重复序列数据库;L1Base是LINE-1的数据库;STRBase(Short Tandem Repeat DNA Internet DataBase)是存储短串联重复序列的数据库。RepeatMasker是比较常用的重复序列片段分析程序,应用于识别、分类和屏蔽重复元件,包括低复杂性序列和散在重复,通过与已知重复序列数据库比对搜索基因组序列中的相似序列进行识别。RepeatMasker一共提供了四个搜索引擎:Cross\_match速度慢但比其他引擎的精度高;ABBlast(以前叫做WUBlast)速度快精度略低;RMBlast是NCBI Blast工具的兼容版;HMMER使用nhmmer程序搜索Dfam数据库,但它只适用于人类基因组序列。

\section{操作演示}\checkpoint{10, 00:30--00:40 (-1.2-)}
EMBOSS(The European Molecular Biology Open Software Suite)是一个开放源代码的序列分析软件包,它是一组为分子生物学家所设计的公开且免费的软件。该软件能够自动识别处理以不同格式存储的数据,甚至可以通过互联网提取数据,此外同软件包一同提供的还包括大量的程序库,软件包整合了100多个的序列分析程序,可以满足一般实验室的各种各样的序列分析要求。并且,因为该软件包同时提供了一个扩展库,它也是允许其他科学家依据自由软件精神编制、发布软件的一个平台。EMBOSS同时将现在可以得到的一系列序列分析工具整合成一个无缝的整体。EMBOSS遵照GPL协议,打破了向商业软件包发展的传统模式。使用者可以通过三种不同的方式使用EMBOSS软件:第一种是通过命令行的方式;第二种是通过X-Windows的方式使用EMBOSS软件的图形界面;第三种是内联网的方式。使用者可以免费获得这些软件以及相关界面程序。

\begin{enumerate}
	\item 使用EMBOSS中的相关程序对人类CD9基因序列(序列号:AY422198.1)的组份进行分析。
		\begin{itemize}
			\item compseq: Calculate the composition of unique words in sequences
			\item geecee: Calculate fractional GC content of nucleic acid sequences
			\item revseq: Reverse and complement a nucleotide sequence
		\end{itemize}
	\item 使用EMBOSS中的相关程序对人类TERT基因序列(序列号:NG\_009265.1)中4000-5300bp区域进行CpG岛的分析。
		\begin{itemize}
			\item extractseq: Extract regions from a sequence
			\item cpgplot: Identify and plot CpG islands in nucleotide sequence(s)
			\item cpgreport: Identify and report CpG-rich regions in nucleotide sequence(s)
			\item isochore: Plot isochores in DNA sequences
		\end{itemize}
\end{enumerate}

\section{总结与答疑}\checkpoint{10, 00:40--00:50 (-1.2\$)}
本次课需要掌握的知识点与技能:
\begin{itemize}
	\item 知识点:
		\begin{itemize}
			\item DNA序列基本信息分析——查戈夫法则,序列转换,GC含量。
			\item 限制酶位点分析——命名,\Rmnum{2}型。
			\item 开放阅读框分析——ORF与CDS。
			\item 启动子与转录因子结合位点分析——启动子结构。
			\item CpG岛识别——判别依据及标准。
			\item 重复序列分析——分类。
		\end{itemize}
	\item 技能:
		\begin{itemize}
			\item 解决问题的思路。
			\item 寻找最合适的方法。
			\item 搜索及学习新软件。
			\item 先易后难,由浅入深。
		\end{itemize}
\end{itemize}

\section{课前甜点}\checkpoint{10, -00:10--00:00 (\^{}2.1-)} 
播放“中心法则”的动画视频。

\section{回顾与导入}\checkpoint{5, 00:00--00:05 (-2.1-)}
对于一条未知的核苷酸序列,我们首先进行基本信息的分析,包括:序列转换、计算碱基比例和GC含量、寻找限制酶切位点;接下来进一步进行序列特征的分析,包括:开放阅读框的预测、启动子和转录因子结合位点的分析、CpG岛的识别;之后就是比较重要的编码区基因的预测,而在基因识别之前,为了提高识别效率,需要先屏蔽掉重复序列。这些都是上次课介绍的内容,本次课将承接以上内容,首先讲解基因识别的相关内容,之后把分析的对象转换到转录产物mRNA上,介绍选择性剪接的分析,最后对miRNA及其靶基因的预测进行讲解,并用几分钟的时间介绍一下近几年备受瞩目的长链非编码RNA。

\section{基因识别}\checkpoint{25, 00:05--00:30 (-2.1-)}
\question{你对基因概念的理解。}
基因(gene)指的是产生一条多肽链或功能RNA所需的全部核苷酸序列。基因,一段具有特定功能和结构的连续的DNA片段,携带着遗传信息,是编码蛋白质或RNA分子遗传信息、控制性状的基本遗传单位。基因通过指导蛋白质的合成来表达自己所携带的遗传信息,从而控制生物个体的性状表现。一个完整的基因,不仅包括编码区,还包括5'末端和3'末端长度不等的特异性序列,它们虽然不编码氨基酸,却在基因的转录过程中起着重要的调节作用。基因识别(gene prediction,gene finding),是生物信息学的一个重要分支,使用生物学实验或计算机等手段识别DNA序列上的具有生物学特征的片段。基因识别的对象主要是蛋白质编码基因,也包括其他具有一定生物学功能的因子,如RNA基因和调控因子。基因识别是基因组研究的基础。
\question{[课后思考]基因概念的提出与发展。(维基百科)}
\marginpar{课外读物:\\ 薛定谔的《生命是什么?》\\ 伽莫夫的《从一到无穷大》\\ 沃森的《双螺旋》和《基因\textbullet 女郎\textbullet 伽莫夫》\\ 贾德森的《创世纪的第八天》}

\question{原核基因和真核基因最主要的区别是什么?}
原核与真核生物的基因都包括编码区和非编码区,但两者的结构有着很大的差别。原核基因的结构比较简单\slide{原核基因结构},为连续基因,其编码区是一个完整的DNA片段,非编码区位于编码区的上游及下游。所有原核基因都有一个编码区,依基因类型的不同,或是编码一种蛋白质多肽或是编码一种RNA结构,如tRNA和rRNA。在原核基因编码区两侧,还存在着用于控制转录作用的调节区,即启动子和终止子。在DNA链上,由起始密码子开始到终止密码子为止的一个连续编码序列,叫做开放阅读框(Open Reading Frame,ORF),也就是所谓的编码区。启动子(promoter)是位于基因5'末端上游外侧紧挨转录起点的一段长度为20~200bp的非编码的核苷酸序列,其功能是与RNA聚合酶结合形成转录起始复合物。原核生物的启动子大约40~50bp,其中包含有转录的起始点和两个区(-35区和-10区)。起始点是DNA模板链上开始进行转录作用的位点,通常在其互补的编码链对应位点(碱基)标以“+1”。-10区是RNA聚合酶核心酶与DNA分子紧密结合的部位,大多包含有6bp的共有序列,即:TATAAT。-35区是RNA聚合酶因子识别DNA分子的部位,其共有序列为:TTGACA。终止子(terminator)是位于一个基因或一个操纵子的末端,提供转录停止信号的DNA区段。与启动子不同的是终止子仍能被RNA聚合酶转录成mRNA。与原核基因一样,一个完整的真核基因,不仅包括编码区,还包括编码区两侧的调节序列。但真核基因和原核基因在结构上存在着许多基本的区别,其中最重要的一点是其不连续性\slide{真核基因结构},许多真核生物的蛋白质编码基因以及某些tRNA基因的编码序列,都被一种叫做内含子(intron)的非编码序列所间断。在基因的表达过程中,内含子便从初级mRNA分子中被剪接掉,形成成熟的功能mRNA。真核基因的非编码序列包括非编码区的所有序列以及编码区里面的内含子。真核生物编码蛋白质的基因启动子,与原核生物的启动子相似,也具有两个高度保守的共有序列。其一是在-25~-35区含有TATAA序列,称为TATA盒(TATA box)。TATA盒与原核生物启动子的-10区相似,是转录因子与DNA分子的结合部位。其二是在多数启动子中,-70~-80区含有CAAT序列,称为CAAT盒。另外,还有一部分DNA序列能增强或减弱真核基因转录起始的频率,这些区域称为增强子(enhancer)和沉默子(sliencer)。原核生物和真核生物的基因结构不同,所以使用的基因识别方法也不同。
\question{模板链和编码链。(具有转录功能、合成RNA的模板的那条链是模板链、反义链;无转录功能、编码蛋白质的那条链是编码链、有义链,它与RNA的序列相同)}
\question{基因结构不同,识别方法是否也不同?(扩展:任务性质不同,处理策略不同)}

基因识别的方法主要包括三大类\slide{基因识别的方法}:间接识别法(Extrinsic Approach)、从头计算法(\textit{Ab Initio} Approach)和比较基因组学的方法。在基因的间接识别法中,人们利用已知的mRNA或蛋白质序列为线索在DNA序列中搜寻所对应的片段。由给定的mRNA序列确定唯一的作为转录源的DNA序列;而由给定的蛋白质序列,也可以由密码子反转确定一组可能的DNA序列。因此,在线索的提示下搜寻工作相对较为容易,搜寻算法的关键在于提高效率,并能够容忍由于测序不完整或者不精确所带来的误差。BLAST是目前以此为目的最广泛使用的软件之一。若DNA序列的某一片段与mRNA或蛋白质序列具有高度相似性,这说明该DNA片段极有可能是蛋白编码基因。但是,测定mRNA或蛋白质序列的成本高昂,而且在复杂的生物体中,任意确定的时刻往往只有一部分基因得到了表达。这意味着从任何单个细胞的mRNA和蛋白质上都只能获得一小部分基因的信息;要想得到更为完整的信息,不得不对成百上千个不同状态的细胞中的mRNA和蛋白质测序。这是相当困难的。鉴于间接识别法的种种缺陷,仅仅由DNA序列信息预测蛋白质编码基因的从头计算法就显得十分重要了。一般意义上基因具有两种类型的特征,一类特征是“信号”,由一些特殊的序列构成,通常预示着其周围存在着一个基因;另一类特征是“内容”,即蛋白质编码基因所具有的某些统计学特征。使用\textit{Ab Initio}方法识别基因又称为基因预测。通常仍需借助实验证实预测的DNA片段是否具有生物学功能。由于多个物种的基因组序列已完全测出,使得比较基因组学得以发展,并产生了新的基因识别的方法。该方法基于如下原理:自然选择的力量使得基因和DNA序列上具有生物学功能的其他片段较其他部分有较慢的变异速率,在前者的变异更有可能对生物体的生存产生负面影响,因而难以得到保存。因此,通过比较相关的物种的DNA序列,我们能够取得预测基因的新线索。

在原核生物中,基因往往具有特定且容易识别的启动子序列(信号),如Pribnow盒和转录因子结合位点。与此同时,构成蛋白质编码的序列构成一个连续的开放阅读框(内容),其长度约为数百个到数千个碱基对。除此之外,原核生物的蛋白质编码还具有其他一些容易判别的统计学的特征。这使得对原核生物的基因预测能达到相对较高的精度。对真核生物(尤其是复杂的生物如人类)的基因预测则相当有挑战性。一方面,真核生物中的启动子和其他控制信号更为复杂,还未被很好的了解。两个被真核生物基因搜寻器识别到的讯号例子有CpG islands及poly(A) tail的结合点。另一方面,由于真核生物所具有的剪接机制,基因中一个蛋白质编码序列被分为了若干段(外显子),中间由非编码序列连接(内含子)。人类的一个普通蛋白质编码基因可能被分为了十几个外显子,其中每个外显子的长度少于200个碱基对,而某些外显子更可能只有二三十个碱基对长。因而蛋白质编码的一些统计学特征变得难于判别。真核基因预测中可利用的信号有:上游启动子区特征序列(TATA box、CAAT box、GC box);5'端外显子位于核心启动子TATA盒的下游,含有起始密码子;内部的外显子两端的供体位点和受体位点;3'端的外显子下游包含终止密码子和polyA信号序列。综合多个序列信号信息确定外显子的边界,从而达到识别编码区域的目的。可以利用的内容信息主要包括密码子使用偏好性(codon usage bias)、双联密码子出现频率、核苷酸周期性分析(即分析同一个核苷酸在3,6,9,...位置上周期性出现的规律)、基因组等值区(isochore)等。对已知编码区进行统计学分析找出编码规律和特性,通过统计值区分外显子、内含子和基因间区域。在实际应用中常常联合几种方法,以提高识别效率。
\question{供体位点和受体位点在外显子上还是在内含子上?}
\question{[课后思考]什么是密码子使用偏好性,什么是基因组等值区(isochore)?}

高级的基因识别算法\slide{基因预测的工具列表}常使用更加复杂的概率论模型,如隐马尔科夫模型、人工神经网络、决策树方法等。GeneMarkS是采用迭代隐马尔科夫模型(iterative Hidden Markov model)的识别工具;Glimmer是一个广泛应用的高级基因识别程序,基于插入式马尔科夫模型(interpolated Markov models,IMMs),它对原核生物基因的预测已非常精确。相比之下,对真核生物的预测则效果有限。GENSCAN是脊椎动物基因预测软件,使用广义隐马尔科夫模型(GHMM)根据基因的整体结构进行基因预测,包括外显子、内含子、基因间区域、转录信号、翻译信号、剪接信号等信息,能对基因组DNA序列识别完整的外显子-内含子结构,能识别多个基因,具有同时处理正、反两条链的功能。其他常用识别工具还有利用神经网络技术同时组合各种编码度量的GRAIL。

\section{mRNA选择性剪接}\checkpoint{25, 00:30--00:55 (-2.1\$)}
\question{RNA包括哪些种类?}
DNA转录生成RNA。RNA即是携带遗传信息的主要生物大分子,也是重要的功能单位。RNA包括mRNA、tRNA、rRNA三种主要形式,参与蛋白质的生物合成;还包括微小RNA(miRNA)、干扰小RNA(siRNA)、长链非编码RNA(long non-coding RNA,lncRNA)等参与生物调控。mRNA属于编码RNA,tRNA、rRNA、miRNA、siRNA、lncRNA属于非编码RNA。前面介绍了DNA分析的相关主题,接下来重点介绍RNA的相关内容,主要是mRNA和miRNA的生物学特征及其分析方法。

真核细胞的基因序列中,包含了内含子(intron)与外显子(exon),两者交互穿插。其中内含子在基因转录成mRNA前体后会被RNA剪接体移除,剩下的外显子才是能够存在于成熟mRNA(之后再进一步翻译成蛋白质)的片段。剪接(splicing),又称拼接,指基因信息在转录后的一种修饰,即将内含子移除及合并外显子,是真核生物的信使RNA前体(precursor messenger RNA)变成成熟mRNA的过程之一。这些成熟的mRNA会接着进行蛋白质生物合成中的翻译,以产生蛋白质,称翻译作用。剪接也是真核生物与原核生物的区别之一。在很多时候,剪接过程可以通过对同一个基因转录的相同pre-mRNA使用不同的剪接选择,产生不同的mRNA异构体(isoform),最后产生多种相似却又独特的蛋白质,或是产生出稳定性低的mRNA产物以达到调节基因表达的目的。RNA的选择性剪接(alternative splicing),又称可变剪接,就是指这种用不同的剪接方式(选择不同的剪接位点组合)从一个mRNA前体产生不同的mRNA剪接异构体的过程。大多数真核生物的基因都存在选择性剪接的现象。由于选择性剪接的存在而使基因组可以产生比基因数量还多许多倍的基因产物。
\question{[课后思考]剪接的分子生物学过程。}

选择性剪接利用基因的不连续性,一条未经剪接的RNA,含有的多种外显子被剪成的不同组合,可翻译出不同的蛋白质。从而将同一基因中的外显子以不同的组合方式来表现,使一个基因在不同时间、不同环境中能够制造出不同的蛋白质(基因表达调控),这可增加生理状况下系统的复杂性或适应性。Pre-mRNA的剪接也并不是完美的,其中相当一部分的剪接产物 (spliced transcripts) 因为剪接过程的不够精确、或是形成未成熟的终止密码子 (premature termination codon, PTC) 而造成该RNA的降解 (RAN degradation)。选择性剪接受时间和空间的限制,在不同的组织中,在相同组织的不同细胞中,在同一组织的不同发育阶段,在对病理过程的不同反应过程中都会产生不同的剪接异构体。有研究表明,人体细胞中有92\%~94\%的基因会进行选择性剪接。多达50\%的致病突变会影响剪接,选择性剪接的异常改变使得基因在转录后期产生异常的剪接变体,编码出异常的蛋白质,导致人类遗传疾病甚至癌变。

选择性剪接的形式多样,主要有以下五种产生机制\slide{选择性剪接类型的示意图}:
\begin{enumerate}
	\item 外显子跳跃(exon skipping),也叫盒式外显子(cassette exon)、外显子遗漏等,在剪接时外显子会被移除或者保留下来,这是选择性剪接最常见的方式。
	\item 互斥外显子(mutually exclusive exons),即相互排斥性剪接,两个外显子只有一个会保留在剪接后的产物中,两者不会同时存在。
	\item 5'选择性剪接(alternative donor site),选择性使用不同的5'端的剪接连接点(即供体位点),从而改变上游外显子的3'边界。
	\item 3'选择性剪接(alternative acceptor site),选择性使用不同的3'端的剪接连接点(即受体位点),从而改变下游外显子的5'边界。
	\item 内含子保留(intron retention),一段序列在剪接过程中或作为内含子被去除,或作为外显子被保留下来。它和外显子跳跃的区别在于这段序列的两边不是内含子。这是最少见的选择性剪接机制。
		\question{外显子跳跃和内含子保留的区别。}
\end{enumerate}

除了上述五种主要的选择性剪接机制外,还有两种从同一基因生成不同mRNA的主要机制:多启动子(multiple promoters)或选择性起始(alternative initiation),多polyA位点(multiple polyadenylation sites)或选择性终止(alternative termination)。通过在不同的位点起始转录,可以产生含有不同5'端外显子的转录产物;但它常被看做转录调节(transcriptional regulation)的机制而非选择性剪接的方式。与之类似,不同的polyA位点会产生含有不同3'端的转录本。这两种机制与选择性剪接相结合,为从同一基因产生不同mRNA增加了多样性。

在各种选择性剪接的类型中,外显子跳跃最为常见,互斥外显子相对较少见,内含子保留是最少见的。
\question{为什么无法统一各种观点?(生物系统的复杂性)}

以上介绍的只是选择性剪接机制的基本模型,但真实的剪接事件则更为复杂,比如小鼠hyaluronidase 3基因的三个剪接异构体\slide{真实剪接的示例}。前两者(黄色和绿色)的比较表明是内含子保留的机制,但后两者(黄色和蓝色)的比较则表明是外显子跳跃。\marginpar{真实情况远比理论模型复杂}

选择性剪接数据资源根据数据来源分成两大类。一类是基于文献报道的数据库,通过收集、整理已有的实验数据和文献报道而建立。一类是基于EST数据的选择性剪接数据库,主要是通过采用EST序列数据与基因组或DNA、mRNA序列进行比对的方法,发现新的或已经存在的选择性剪接形式后建立的数据库或数据集。常用的选择性剪接数据库有ASTD和ASAP。ASTD(Alternative Splicing and Transcript Diversity database),选择性剪接和转录多样性数据库,由ASD(Alternative Splicing Database)和ATD(Alternate Transcript Diversity Project)合并而来,提供人、小鼠、大鼠、斑马鱼、线虫、果蝇等多个物种的选择性剪接数据,是目前常用的选择性剪接数据库。ASD包含了多种模式生物的选择性剪接数据。ASD由三个子数据库组成:AEDB(Alternative Exon Database),从文献中收集的经实验验证的人类选择性外显子(alternative exons);AltExtron,由EST与全基因组序列比对得到选择性剪接数据及选择性外显子和内含子;AltSplice(Alternative Splicing Database),收集了通过计算方法得到的选择性剪接事件及其模式。ATD试图通过创建人和小鼠的全长选择性剪接转录本,来揭示转录异构体的产生机制。ASAP(Alternative Splicing Annotation Project)\marginpar{ASAP网站失效(注意数据库的更新日期)}是通过全基因组范围内比对EST数据得到的人和小鼠的选择性剪接数据库。它提供基因的外显子、内含子结构、选择性剪接、组织特异性选择性剪接、选择性剪接产生的蛋白质异构体等信息。
\question{英语中ASAP缩写的含义。}

剪接位点的精确定位是确定真核生物基因结构的关键。生物信息学已开发出从头预测法、基于EST/cDNA序列比对法和基于RNA-seq数据识别等多种方法。从头预测法主要采用支持向量机、概率模型、隐马尔科夫模型、神经网络和二次判别分析法等技术预测剪接位点。选择性剪接过程的调控机制具有多样性,主要由剪接调节因子(splicing-regulatory element)和调节蛋白相互作用来进行调节。剪接调节因子主要由外显子剪接增强子(Exonic Splicing Enhancer,ESE)、外显子剪接沉默子(Exonic Splicing Silencer,ESS)、内含子剪接增强子(Intronic Spicing Enhancer,ISE)和内含子剪接沉默子(Intronic Splicing Silencer,ISS)。目前常用工具都结合剪接调节因子预测进行选择性剪接的分析。ESEfinder和RESCUE-ESE等用于外显子剪接增强子的预测。ASPicDB(Alternative Splicing Prediction Database)旨在提供人类基因选择性剪接模式的可靠注释和剪接异构体的功能注释,能够在基因、转录本、外显子、蛋白质或剪接位点水平上进行分析,提供两类蛋白质(球状蛋白和跨膜蛋白)及有关定位、PFAM结构域、信号肽(signal peptides)、跨膜片段和卷曲螺旋片段的信息。

\section{miRNA及其靶基因预测}\checkpoint{25, 00:05--00:30 (\^{}2.2-)} 
\question{ncRNA包括哪些?}
人类基因组中,虽然仅有1.5\%左右的序列可以编码蛋白质,但研究表明,90\%以上的基因组都是可以转录的。这些转录后不编码蛋白质的RNA分子统称为非编码RNA(non-coding RNAs,ncRNA)。ncRNA主要分成基础结构性ncRNA(infrastructural non-coding RNAs)和调节性ncRNA(regulatory non-coding RNAs)大类。基础结构性ncRNA即看家ncRNA(housekeeping non-coding RNAs),主要包括tRNA、rRNA、snRNA和snoRNA。根据转录本的长度,调节性ncRNA分为短于200nt的小RNA(small RNAs,sRNA)和长于200nt的长链非编码RNA(long ncRNAs,lncRNA)。sRNA主要包括已为研究人员所熟知的miRNA、siRNA和piRNA,其产生过程及作用机理都已研究的比较透彻。

微RNA(microRNAs,miRNA,小分子RNA)归属小RNA范畴,是真核生物中广泛存在的一种长约20到24个核苷酸的内源性非编码单链RNA分子。miRNA通过RNA诱导沉默复合体(RISC)与靶基因的3'非翻译区(3' UTR)相结合,导致靶基因mRNA降解或者抑制其翻译,从而调节基因转录后的表达。miRNA在调控基因表达、细胞周期、生物体发育时序等方面起重要作用。

\slide{miRNA的特点}miRNA不具有开放阅读框,不编码蛋白质,表达具有时序性和组织特异性,进化上具有高度的保守性。在植物、动物和真菌中发现的miRNAs只在特定的组织和发育阶段表达。miRNA的组织特异性和时序性,决定组织和细胞的功能特异性,表明miRNA在细胞生长和发育过程的调节过程中起多种作用。

\question{miRNA的生成过程。}
\slide{miRNA生成过程}编码基因在核内产生长度为300~1000nt的初级转录本(primary transcript),即初始miRNA(pri-miRNA),被双链RNA特异的核糖核酸酶Drosha切割成长度为70~90nt、具有茎环二级结构(发卡结构)的单链前体miRNA(pre-miRNA)。这些发夹结构的pre-miRNA通过核输出蛋白Exportin 5机制转运到细胞质,被第二个双链RNA特异的核糖核酸酶Dicer及其辅因子加工形成20~24nt的成熟miRNA及其互补体。miRNA前体在各个物种间具有高度的进化保守性,茎部保守性最强,环部可以容许更多的突变位点存在。

miRNA与其靶基因间是多对多的关系:一个miRNA可以调控多个靶基因,一个基因也可以受多个miRNA的调控。这种复杂的调节网络既可以通过一个miRNA来调控多个基因的表达,也可以通过几个miRNAs的组合来精细调控某个基因的表达。在动物中,一个miRNA通常可以调控数十个基因。miRNA的作用机制取决于miRNA与靶mRNA的互补程度,包括完全互补型和不完全互补型\slide{miRNA与靶mRNA的互补类型}。miRNA与靶mRNA完全互补(或者几乎完全互补)导致靶基因mRNA降解(在植物中比较常见),通过这种机制起作用的miRNAs的结合位点通常都在mRNA的编码区或开放阅读框中;不完全互补导致靶基因mRNA的翻译受到抑制,使用这种机制的miRNA结合位点通常在mRNA的3'端非翻译区。。
\question{[课后思考]为什么仅仅完全、不完全互补就会导致完全不同的结果?}

miRNA通过作用于相应靶基因mRNA完成生物学功能,如个体发育的调控、细胞分化和组织发育等。据推测脊椎动物基因组有多达1000个不同的miRNAs,调控至少30\%以上的基因表达。miRNA的异常与疾病发生发展具有相关性。最近的研究发现,miRNA表达与多种癌症相关,大约50\%得到注释的miRNAs在基因组上定位于与肿瘤相关的脆性位点(fragile site)。这说明miRNAs在肿瘤发生过程中起至关重要的作用,这些miRNAs所起的作用类似于抑癌基因和癌基因的功能。

miRNA在细胞分化,生物发育及疾病发生发展过程中发挥巨大作用,越来越多地引起研究人员的关注。随着对于miRNA作用机理的进一步的深入研究,以及利用最新的例如miRNA芯片、miRNA-seq等高通量的技术手段对于miRNA和疾病之间的关系进行研究,将会使人们对于高等真核生物基因表达调控的网络理解提高到一个新的水平。这也将使miRNA可能成为疾病诊断的新的生物学标记,还可能使得这一分子成为药靶,或是模拟这一分子进行新药研发,这将可能会给人类疾病的治疗提供一种新的手段。

miRNA分析主要包括miRNA预测和miRNA靶基因预测两方面。

miRNA主要通过cDNA克隆测序和计算预测两种方法获得。早期克隆测序直接、可靠,但很难克隆出在不同时期表达或只在特定组织或细胞系中表达的miRNA,由于它的固有局限性,也很难捕获表达丰度较低的miRNA。最近几年发展起来的miRNA-seq等高通量技术使得通过实验预测miRNA得到了一定程度的改观。随着miRNA研究的发展,生物信息学预测miRNA的方法成为一条重要的辅助途径,其优势是不受miRNA表达的时间和组织特异性以及表达水平的影响。

常用的miRNA预测方法主要有5种\slide{miRNA预测方法}:
\begin{enumerate}
	\item 同源片段搜索方法。将已知miRNA或pre-miRNA序列在自身或其他相近基因组中用比对算法搜索同源序列,结合序列二级结构特征进行筛选。
	\item 基于比较基因组学的预测方法。依据进化过程中的保守性在多物种中搜索潜在的miRNA。
	\item 基于序列和结构特征打分的预测方法。根据已知miRNA序列和结构的特征对全基因组范围中能形成茎环结构的片段进行筛选,是发现非同源、物种特异性miRNA的方法。
	\item 结合作用靶标的预测方法。依据miRNA与其靶基因序列间的碱基互补配对的保守性的特点预测miRNA。
	\item 基于机器学习的预测方法。通过对阳性miRNA和阴性miRNA数据集的训练来构建区分两者的分类器,根据所得分类器对未知序列进行预测。其中支持向量机(SVM,Support Vector Machine)方法是目前miRNA分类和预测最常用的机器学习方法。
\end{enumerate}
常用的miRNA预测软件有MiRscan、MiPred、miRFinder等。

miRNA通过与靶基因mRNA的3' UTR不精确互补配对使靶mRNA的翻译受到抑制,二者相互作用以miRNA:mRNA二聚体结构形式存在。miRNA序列5'端的2-8nt为种子区域\slide{种子区域},在miRNA靶基因预测中起主导作用。种子区域具有保守性,与靶mRNA序列能较好地互补配对结合,且在不同物种中靶序列也是保守的,这些特征是靶基因预测方法的重要依据。miRNA靶基因预测方法主要有2类\slide{miRNA靶基因预测方法与工具}:
\begin{enumerate}
	\item 基于种子区域互补和保守性的规则预测,常用软件有miRanda、TargetScan等。
	\item 基于机器学习方法训练参数进行靶基因预测,常用软件有PicTar、miTarget等。
\end{enumerate}

\slide{miRNA数据库资源}miRBase是集miRNA序列、注释信息和预测的靶基因数据为一体的数据库,是目前存储miRNA信息最主要的公共数据库之一。TarBase数据库是存储已被实验验证的miRNA与靶基因间关系的数据库。miRGen是整合了miRNA靶基因数据、基因组注释信息以及位置关系的综合数据库。更多数据库可以参看\href{http://zh.wikipedia.org/wiki/\%E5\%BE\%AERNA\%E4\%B8\%8E\%E5\%BE\%AERNA\%E9\%9D\%B6\%E6\%95\%B0\%E6\%8D\%AE\%E5\%BA\%93}{微RNA与微RNA靶数据库(维基百科)}。

\section{lncRNA简介}\checkpoint{5, 00:30--00:35 (-2.2-)}
\question{垃圾DNA真的是垃圾吗?(以学校等社会机构进行类比)}
目前已知在人类基因组中,lncRNA基因的数目已经达到了13249。据估计,人类基因组中lncRNA基因的总数可能在15000条以上。在基因组范围上,对已知lncRNA进行的研究表明\slide{lncRNA的特点}:大多数lncRNA是被RNA聚合酶\Rmnum{2}所转录的,有5'帽子和3'端的poly(A)尾巴,主要富集在细胞核;与蛋白质编码基因相比,lncRNA的长度偏短、外显子数目偏少,在不同物种间的保守性差,稳定性偏低,表达水平很低,而且表达具有细胞、组织、发育、疾病等时空特异性。lncRNA以RNA分子形式在表观遗传学水平、转录水平和转录后水平上调控基因的表达,参与基因转录、剪接、翻译、修饰和印迹等重要的生物学过程。lncRNA的异常表达与众多疾病的发生发展相关,如肿瘤、阿尔兹海默病和心血管疾病等。

虽然在基因表达调控过程中具有重要作用,且与肿瘤、阿尔兹海默病等疾病密切相关,但迄今为止,仅有数量有限的lncRNA得到了比较细致的研究,对lncRNA的大规模分析更是处于起步阶段。\slide{lncRNA数据库资源}lncRNA的相关数据库可以参看\href{http://zh.wikipedia.org/wiki/\%E9\%95\%BF\%E9\%93\%BE\%E9\%9D\%9E\%E7\%BC\%96\%E7\%A0\%81RNA\%E6\%95\%B0\%E6\%8D\%AE\%E5\%BA\%93}{长链非编码RNA数据库(维基百科)}。
\question{[课后思考]查阅lncRNA与疾病关系的相关资料。}

\section{查找数据库与分析工具}\checkpoint{5, 00:35-00:40 (-2.2-)}
每一个研究领域或研究对象都涉及众多的数据库与分析工具,如何快速查找到它们也是生物信息学工作者的必备技能之一。实际工作中可以采用以下几种途径:
\question{如何查找需要的数据库和工具?}
\begin{itemize}
	\item 借鉴相关文献中使用的数据库与工具。
	\item 向特定领域的专家请教。
	\item \textit{Nucleic Acids Research}每年的第一期为数据库专刊。
	\item 维基百科等总结性网站。
	\item \href{http://elements.eaglegenomics.com/}{The Elements of Bioinformatics}。
	\item 使用Google等搜索引擎搜索。
\end{itemize}

\section{总结与答疑}\checkpoint{10, 00:40-00:50 (-2.2\$)}
本次课需要掌握的知识点与技能:
\begin{itemize}
	\item 知识点:
		\begin{itemize}
			\item 基因识别——原核和真核的基因结构,基因识别方法。
			\item mRNA选择性剪接——选择性剪接的类型,数据资源。
			\item miRNA——miRNA的特点,miRNA预测方法与工具,miRNA靶基因预测方法与工具。
		\end{itemize}
	\item 技能:
		\begin{itemize}
			\item 查找数据库——时效性。
			\item 查找分析工具——适用范围。
		\end{itemize}
\end{itemize}

\question{[课后思考]总结自己遇到的文本型问题(如:根据学号整合其姓名与成绩),经典问题将在Galaxy操作演示中予以解决。}

\end{document}
