\documentclass[11pt,a4paper,twoside]{book}

\documentclass[11pt,a4paper,landscape,twoside]{book}

\usepackage{fontspec}
\setmainfont{Times New Roman}
\setsansfont{Arial}
\setmonofont{Courier New}

\usepackage[BoldFont,SlantFont,CJKchecksingle,CJKnumber]{xeCJK}
\setCJKmainfont[BoldFont={Adobe Heiti Std},ItalicFont={Adobe Kaiti Std}]{Adobe Song Std}
\setCJKsansfont{Adobe Heiti Std}
\setCJKmonofont{Adobe Fangsong Std}
\punctstyle{hangmobanjiao}

\defaultfontfeatures{Mapping=tex-text}
\usepackage{xunicode}
\usepackage{xltxtra}

\XeTeXlinebreaklocale "zh"
\XeTeXlinebreakskip = 0pt plus 1pt minus 0.1pt

%\usepackage{indentfirst}
\makeatletter
\let\@afterindentfalse\@afterindenttrue
\@afterindenttrue
\makeatother
\setlength{\parindent}{2em}

\linespread{1.2}

\usepackage[top=1.5cm,bottom=1.5cm,left=1.5cm,right=5cm,marginparwidth=4.7cm,marginparsep=0.3cm]{geometry}

\usepackage{titlesec}
\titleformat{\chapter}{\centering\LARGE\bfseries}{第 \thechapter 章}{1em}{}
\titlespacing*{\section}{0pt}{0.2\baselineskip}{0.2\baselineskip}

\usepackage{fancyhdr}
\pagestyle{fancy}
\renewcommand{\chaptermark}[1]{\markboth{\small 第 \thechapter 章\quad #1}{}}
\renewcommand{\sectionmark}[1]{\markright{\small \thesection \quad #1}{}}
\fancyhf{}
\fancyhead[ER]{\leftmark}
\fancyhead[OL]{\rightmark}
\fancyhead[EL,OR]{$\cdot$ \thepage \ $\cdot$}
\renewcommand{\headrulewidth}{0.5pt}

\usepackage{xcolor}
\usepackage{graphicx}
\graphicspath{{figures/}}
\usepackage[xetex,bookmarksnumbered=true,bookmarksopen=true,pdfborder=1,breaklinks,colorlinks,linkcolor=blue,urlcolor=blue,citecolor=blue]{hyperref}

\renewcommand{\today}{\number\year 年 \number\month 月 \number\day 日}
\renewcommand{\contentsname}{目录}
\renewcommand{\listfigurename}{插图目录}
\renewcommand{\listtablename}{表格目录}
\renewcommand{\figurename}{图}
\renewcommand{\tablename}{表}
\renewcommand{\bibname}{参考文献}

\renewcommand{\figureautorefname}{图}
\renewcommand{\tableautorefname}{表}
\renewcommand{\footnoteautorefname}{脚注}

\usepackage{booktabs,tabu}

\newtheorem{example}{例}[chapter]

%调整表格行高
\renewcommand{\arraystretch}{0.8}

%调整列表间及其上下的间距
%\usepackage{mdwlist}
\usepackage{enumitem}
\setlist{nosep}

% auto adjust the marginals
\usepackage{marginfix}


%可以添加多姿多彩的边注
\usepackage{todonotes}
\newcommand{\checkpoint}[1]{\todo[linecolor=green!70!white,backgroundcolor=blue!20!white,bordercolor=red,noline,size=\large]{#1}}
\newcommand{\question}[1]{\todo[inline,backgroundcolor=yellow!50!gray]{\textbf{提问:}#1}}
\newcommand{\slide}[1]{\todo[color=green!40,noline]{#1}}
%\newcommand{\slide}[1]{\todo[color=green!40]{#1}}

%在正文和边注间添加分割线
\usepackage{lipsum}
\usepackage{eso-pic}
\usepackage{ifthen}
\usepackage{tikz}

\def\bottommargin{\paperheight - \topmargin - \textheight - \headheight - \headsep - 1in - \voffset}
\def\toptotalheight{\paperheight - \topmargin - \headheight - \headsep - 1in - \voffset}
\def\leftlength{\evensidemargin - 0.5*\marginparsep + 1in + \hoffset}
\def\rightlength{\paperwidth - \evensidemargin + 0.5*\marginparsep - 1in - \hoffset} 

\makeatletter
\newcommand{\nomarginbar}{\let\ESO@HookIIBG\@empty}
\makeatother

\newcommand{\thisisfullsize}{\path (0,0) --  (\paperwidth,\paperheight);}

\newcommand\LeftBar{%
  \put(0,0){%
    \parbox[b][\paperheight]{\paperwidth}{%
      \vfill
      \centering
      \begin{tikzpicture}
        \thisisfullsize
        \draw[line width=1pt] (\leftlength,\bottommargin) -- (\leftlength,\toptotalheight);
      \end{tikzpicture}
      \vfill
}}}

\newcommand\RightBar{ 
  \put(0,0){
    \parbox[b][\paperheight]{\paperwidth}{
      \vfill
      \centering
      \begin{tikzpicture}
        \thisisfullsize
        \draw[line width=1pt] (\rightlength,\bottommargin) -- (\rightlength,\toptotalheight);
      \end{tikzpicture}
      \vfill
}}}

%%% Use this in two-side documents
\AtBeginShipout{
  \ifthenelse{\isodd{\value{page}}}
  {\AddToShipoutPictureBG*{\LeftBar}
  }
  {\AddToShipoutPictureBG*{\RightBar}
  }
}

% %%% Use this in one-side documents
% \AtBeginShipout{%
%   \AddToShipoutPictureBG*{\RightBar}%
% }

%%% Use this anyway (to take care of the first page of the document)
\AtBeginDocument{
\AddToShipoutPictureBG*{\RightBar}
}


%设置颜色的快捷命令
\newcommand{\red}{\textcolor{red}}
\newcommand{\gray}{\textcolor{gray}}
\newcommand{\black}{\textcolor{black}}

%罗马数字
\makeatletter
\newcommand{\rmnum}[1]{\romannumeral #1}
\newcommand{\Rmnum}[1]{\expandafter\@slowromancap\romannumeral #1@}
\makeatother


\begin{document}

\setcounter{chapter}{-1}
\chapter{课堂纪律与自我介绍}


\noindent
\shadowbox{\parbox{\textwidth}{
\textbf{了解:}
\begin{enumerate}
  \item 基本的课堂纪律:不迟到、不早退,不缺勤、不走神;
  \item 交流信息的方式:电话、短信、邮件、面谈;
  \item 共享资料的方法:U盘、邮箱、网盘;
  \item 设置密码的原则:唯一、复杂、勤换;
  \item 管理密码的软件:KeePassX、LastPass。
\end{enumerate}
}}


\section{课堂纪律}

《史记·孙子吴起列传》中有这么一句话:\red{“约束不明,申令不熟,将之罪也;既已明而不如法者,吏士之罪也。”}意思是说:“纪律弄不清楚,号令不熟悉,这是将领的过错;已经讲得清清楚楚,却不遵照号令行事,那就是军官和士兵的过错了。”从这句话中至少可以看出两层含义:一是“国有国法,家有家规”,三百六十行行行都有自己的职业操守;二是“无规矩不成方圆”,人人都按规矩行事,社会自然就和谐了。

向孙武学习,在正式上课之前先明确一下作为学生应该遵守的课堂纪律:
\begin{itemize}
  \item 只有正式上课前的请假有效。
  \item 提前5分钟到教室,严禁迟到。
  \item 上课期间手机关机或调成震动。
  \item 上课期间离开教室先举手示意。
  \item 课上有疑问的话先举手后提问。
  \item 上课期间严禁交头接耳,大声喧哗。
  \item 随机点名,缺勤扣分如下:1、3、6。
  \item 缺勤三次或三次以上者,平时成绩为0。
\end{itemize}

对上述规范有异议的话,可以以匿名方式把自己的想法传达给我。如果对授课内容、方式等有何建议,可以通过各种方式\footnote{交流信息的方式有哪些?}反馈给我。

\section{自我介绍}
伊现富(Yi Xianfu),1986年生人,本科毕业于山东大学生命科学学院,修习生物科学专业;之后保送到中国科学院上海生命科学研究院读研究生,主要从事人类复杂疾病相关的生物信息学研究。

我常用的邮箱有两个:\href{mailto:yixfbio@gmail.com}{yixfbio@gmail.com},主要用于工作中的科研业务交流;\href{mailto:yixf1986@gmail.com}{yixf1986@gmail.com},主要用于生活中的闲杂琐事交流。联系电话:15620610763。个人博客:\href{http://yixf.name}{http://yixf.name},内容五花八门,其中有和生物信息学相关的一些资料,感兴趣的可以去看看。网络上的昵称主要以“yixf”为主。

为了方便进行信息交流与资源共享\footnote{共享资料的方法有哪些?},我专门注册了一个126的邮箱,账号:\textcolor{red}{bioinfo\_TIJMU@126.com},密码:\textcolor{red}{\texttt{C\&563f\&nzx!s}};申请了一个百度云网盘,账号:\textcolor{red}{bioinfo\_TIJMU@126.com},密码:\textcolor{red}{\texttt{566\&Us3Rp6\#C}}。授课讲义、幻灯片、视频等资料都会存储在网盘中,有需要的可以自行去下载。

\section{授课规律}
每次课课前5分钟播放与授课内容相关的视频;课堂中不点名,但随机提问;授课内容以讲义为主、幻灯片为辅,两者相互补充;讲义中包含生物信息学的知识点与必备技能,掺杂有相关的参考资料与课外阅读材料;幻灯片多是讲义内容的图表化,图多字少,因此以讲解为主;每次课开始有回顾和引言,最后有总结和答疑。

每一章最后列出复习思考题,涉及相关知识点和技能;在网盘中共享讲义、幻灯片、视频等所有授课资料。

%\noindent
%\hrulefill
\vspace{1cm}
\hrule height 3pt

\noindent
{\Large \bfseries \Coffeecup 课后思考}
\begin{enumerate}
  \item 与他人交流信息的方式有哪些?(电话、短信、邮件、面谈、……)
  \item 与同学/同事/朋友共享资料的方法有哪些?(U盘、邮箱、网盘、……)
  \item 从哪些方面可以提高密码的强健度?(唯一、复杂、勤换、……)
  \item 如果方便安全地管理众多的密码?(KeePassX、LastPass、……)
\end{enumerate}

%\noindent
%\hrulefill
\vspace{1cm}
\hrule height 3pt

\noindent
{\Large \bfseries \HandPencilLeft 参考资料}

\columnratio{0.372}
\begin{paracol}{2}
孙子武者,齐人也。以兵法见於吴王阖庐。阖庐曰:“子之十三篇,吾尽观之矣,可以小试勒兵乎?”对曰:“可。”阖庐曰:“可试以妇人乎?”曰:“可。”於是许之,出宫中美女,得百八十人。孙子分为二队,以王之宠姬二人各为队长,皆令持戟。令之曰:“汝知而心与左右手背乎?”妇人曰:“知之。”孙子曰:“前,则视心;左,视左手;右,视右手;后,即视背。”妇人曰:“诺。”约束既布,乃设鈇钺,即三令五申之。於是鼓之右,妇人大笑。孙子曰:“\red{约束不明,申令不熟,将之罪也。}”复三令五申而鼓之左,妇人复大笑。孙子曰:“\red{约束不明,申令不熟,将之罪也;既已明而不如法者,吏士之罪也。}”乃欲斩左古队长。吴王从台上观,见且斩爱姬,大骇。趣使使下令曰:“寡人已知将军能用兵矣。寡人非此二姬,食不甘味,愿勿斩也。”孙子曰:“臣既已受命为将,将在军,君命有所不受。”遂斩队长二人以徇。用其次为队长,於是复鼓之。妇人左右前后跪起皆中规矩绳墨,无敢出声。於是孙子使使报王曰:“兵既整齐,王可试下观之,唯王所欲用之,虽赴水火犹可也。”吴王曰:“将军罢休就舍,寡人不愿下观。”孙子曰:“王徒好其言,不能用其实。”於是阖庐知孙子能用兵,卒以为将。西破彊楚,入郢,北威齐晋,显名诸侯,孙子与有力焉。(《史记·孙子吴起列传》)
\switchcolumn
孙子名武,是齐国人。因为他精通兵法受到吴王阖庐的接见。阖庐说:“您的十三篇兵书我都看过了,可用来小规模地试着指挥军队吗?”孙子回答说:“可以。”阖庐说:“可以用妇女试验吗?”回答说:“可以。”于是阖庐答应他试验,叫出宫中美女,共约百八十人。孙子把她们分为两队,让吴王阖庐最宠爱的两位侍妾分别担任各队队长,让所有的美女都拿一支戟。然后命令她们说:“你们知道自己的心、左右手和背吗?”妇人们回答说:“知道。”孙子说:“我说向前,你们就看心口所对的方向;我说向左,你们就看左手所对的方向;我说向右,你们就看右手所对的方向;我说向后,你们就看背所对的方向。”妇人们答道:“是。”号令宣布完毕,于是摆好斧铖等刑具,旋即又把已经宣布的号令多次重复地交待清楚。就击鼓发令,叫她们向右,妇人们都哈哈大笑。孙子说:“\red{纪律还不清楚,号令不熟悉,这是将领的过错。}”又多次重复地交待清楚,然后击鼓发令让她们向左,妇人们又都哈哈大笑。孙子说:“\red{纪律弄不清楚,号令不熟悉,这是将领的过错;现在既然讲得清清楚楚,却不遵照号令行事,那就是军官和士兵的过错了。}”于是就要杀左、右两队的队长。吴王正在台上观看,见孙子将要杀自己的爱妾,大吃一惊。急忙派使臣传达命令说:“我已经知道将军善用兵了,我要没了这两个侍妾,吃起东西来也不香甜,希望你不要杀她们吧。”孙子回答说:“我已经接受命令为将,将在军队里,国君的命令有的可以不接受。”于是杀了两个队长示众。然后按顺序任用两队第二人为队长,于是再击鼓发令,妇人们不论是向左向右、向前向后、跪倒、站起都符合号令、纪律的要求,再没有人敢出声。于是孙子派使臣向吴王报告说:“队伍已经操练整齐,大王可以下台来验察她们的演习,任凭大王怎样使用她们,即使叫她们赴汤蹈火也办得到啊。”吴王回答说:“让将军停止演练,回宾馆休息。我不愿下去察看了。”孙子感叹地说:“大王只是欣赏我的军事理论,却不能让我付诸实践。”从此,吴王阖庐知道孙子果真善于用兵,终于任命他做了将军。后业吴国向西打败了强大的楚国,攻克郢都,向北威震齐国和晋国,在诸侯各国名声赫赫,这其间,孙子不仅参与,而且出了很大的力啊。
\end{paracol}


\end{document}
